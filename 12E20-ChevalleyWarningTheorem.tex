\documentclass[12pt]{article}
\usepackage{pmmeta}
\pmcanonicalname{ChevalleyWarningTheorem}
\pmcreated{2013-03-22 17:46:52}
\pmmodified{2013-03-22 17:46:52}
\pmowner{kshum}{5987}
\pmmodifier{kshum}{5987}
\pmtitle{Chevalley-Warning Theorem}
\pmrecord{6}{40240}
\pmprivacy{1}
\pmauthor{kshum}{5987}
\pmtype{Theorem}
\pmcomment{trigger rebuild}
\pmclassification{msc}{12E20}

\endmetadata

% this is the default PlanetMath preamble.  as your knowledge
% of TeX increases, you will probably want to edit this, but
% it should be fine as is for beginners.

% almost certainly you want these
\usepackage{amssymb}
\usepackage{amsmath}
\usepackage{amsfonts}

% used for TeXing text within eps files
%\usepackage{psfrag}
% need this for including graphics (\includegraphics)
%\usepackage{graphicx}
% for neatly defining theorems and propositions
%\usepackage{amsthm}
% making logically defined graphics
%%%\usepackage{xypic}

% there are many more packages, add them here as you need them

% define commands here

\begin{document}
Let $\mathbb{F}_q$ be the finite field of $q$ elements with
characteristic $p$. Let $f_i(x_1,\ldots, x_n)$, $i=1,2,\ldots, r$,
be polynomial of $n$ variables over $\mathbb{F}_q$. If $n >
\sum_{i=1}^r \deg(f_i)$, then the number of solutions over
$\mathbb{F}_q$ to the system of equations
\begin{align*}
f_1(x_1,x_2,\ldots, x_n) &=0\\
f_2(x_1,x_2,\ldots, x_n) &=0\\
\vdots & \\
f_r(x_1,x_2,\ldots, x_n) &=0
\end{align*}
is divisible by $p$. In particular, if none of the polynomials
$f_1$, $f_2,\ldots, f_r$ have constant term, then there are at least
$p$ solutions.
%%%%%
%%%%%
\end{document}
