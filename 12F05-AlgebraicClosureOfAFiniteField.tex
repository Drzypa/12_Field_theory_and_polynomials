\documentclass[12pt]{article}
\usepackage{pmmeta}
\pmcanonicalname{AlgebraicClosureOfAFiniteField}
\pmcreated{2013-03-22 16:40:51}
\pmmodified{2013-03-22 16:40:51}
\pmowner{Algeboy}{12884}
\pmmodifier{Algeboy}{12884}
\pmtitle{algebraic closure of a finite field}
\pmrecord{5}{38889}
\pmprivacy{1}
\pmauthor{Algeboy}{12884}
\pmtype{Derivation}
\pmcomment{trigger rebuild}
\pmclassification{msc}{12F05}
\pmrelated{FiniteField}
\pmrelated{FiniteFieldCannotBeAlgebraicallyClosed}

\endmetadata

\usepackage{latexsym}
\usepackage{amssymb}
\usepackage{amsmath}
\usepackage{amsfonts}
\usepackage{amsthm}

%%\usepackage{xypic}

%-----------------------------------------------------

%       Standard theoremlike environments.

%       Stolen directly from AMSLaTeX sample

%-----------------------------------------------------

%% \theoremstyle{plain} %% This is the default

\newtheorem{thm}{Theorem}

\newtheorem{coro}[thm]{Corollary}

\newtheorem{lem}[thm]{Lemma}

\newtheorem{lemma}[thm]{Lemma}

\newtheorem{prop}[thm]{Proposition}

\newtheorem{conjecture}[thm]{Conjecture}

\newtheorem{conj}[thm]{Conjecture}

\newtheorem{defn}[thm]{Definition}

\newtheorem{remark}[thm]{Remark}

\newtheorem{ex}[thm]{Example}



%\countstyle[equation]{thm}



%--------------------------------------------------

%       Item references.

%--------------------------------------------------


\newcommand{\exref}[1]{Example-\ref{#1}}

\newcommand{\thmref}[1]{Theorem-\ref{#1}}

\newcommand{\defref}[1]{Definition-\ref{#1}}

\newcommand{\eqnref}[1]{(\ref{#1})}

\newcommand{\secref}[1]{Section-\ref{#1}}

\newcommand{\lemref}[1]{Lemma-\ref{#1}}

\newcommand{\propref}[1]{Prop\-o\-si\-tion-\ref{#1}}

\newcommand{\corref}[1]{Cor\-ol\-lary-\ref{#1}}

\newcommand{\figref}[1]{Fig\-ure-\ref{#1}}

\newcommand{\conjref}[1]{Conjecture-\ref{#1}}


% Normal subgroup or equal.

\providecommand{\normaleq}{\unlhd}

% Normal subgroup.

\providecommand{\normal}{\lhd}

\providecommand{\rnormal}{\rhd}
% Divides, does not divide.

\providecommand{\divides}{\mid}

\providecommand{\ndivides}{\nmid}


\providecommand{\union}{\cup}

\providecommand{\bigunion}{\bigcup}

\providecommand{\intersect}{\cap}

\providecommand{\bigintersect}{\bigcap}










\begin{document}
Fix a prime $p$ in $\mathbb{Z}$.  Then the Galois fields $GF(p^e)$ denotes the
finite field of order $p^e$, $e\geq 1$.  This can be concretely constructed as 
the splitting field of the polynomials $x^{p^e}-x$ over $\mathbb{Z}_p$.  In so doing we
have $GF(p^{e})\subseteq GF(p^{f})$ whenever $e|f$.  In particular, we have an
infinite chain:
\[GF(p^{1!})\subseteq GF(p^{2!})\subseteq GF(p^{3!})\subseteq\cdots \subseteq
    GF(p^{n!})\subseteq\cdots.\]
So we define $\displaystyle GF(p^{\infty})=\bigunion_{n=1}^\infty GF(p^{n!})$.

\begin{thm}
$GF(p^{\infty})$ is an algebraically closed field of characteristic $p$.
Furthermore, $GF(p^e)$ is a contained in $GF(p^{\infty})$ for all $e\geq 1$.
Finally, $GF(p^\infty)$ is the algebraic closure of $GF(p^e)$ for any $e\geq 1$.
\end{thm}
\begin{proof}
Given elements $x,y\in GF(p^\infty)$ then there exists some $n$ such that
$x,y\in GF(p^{n!})$.  So $x+y$ and $xy$ are contained in $GF(p^{n!})$ and also
in $GF(p^\infty)$.  The properties of a field are thus inherited and we have
that $GF(p^\infty)$ is a field.  Furthermore, for any $e\geq 1$, $GF(p^e)$ is
contained in $GF(p^{e!})$ as $e|e!$, and so $GF(p^e)$ is contained in $GF(p^\infty)$.

Now given $p(x)$ a polynomial over $GF(p^\infty)$ then there exists some $n$ 
such that $p(x)$ is a polynomial over $GF(p^{n!})$.  As the splitting field 
of $p(x)$ is a finite extension of $GF(p^{n!})$, so it is a finite field 
$GF(p^{e})$ for some $e$, and hence contained in $GF(p^\infty)$.  Therefore 
$GF(p^\infty)$ is algebraically closed.
\end{proof}

We say $GF(p^\infty)$ is \emph{the} algebraic closure indicating that up to field 
isomorphisms, there is only one algebraic closure of a field.  The actual objects 
and constructions may vary.

\begin{coro}
The algebraic closure of a finite field is countable.
\end{coro}
\begin{proof}
By construction the algebraic closure is a countable union of finite sets so
it is countable.
\end{proof}

\bibliographystyle{amsplain}
\providecommand{\bysame}{\leavevmode\hbox to3em{\hrulefill}\thinspace}
\providecommand{\MR}{\relax\ifhmode\unskip\space\fi MR }
% \MRhref is called by the amsart/book/proc definition of \MR.
\providecommand{\MRhref}[2]{%
\href{http://www.ams.org/mathscinet-getitem?mr=#1}{#2}
}
\providecommand{\href}[2]{#2}
\begin{thebibliography}{10}

\bibitem{McDonald}
McDonald, Bernard R.,
\emph{Finite rings with identity}, Pure and Applied Mathematics, Vol. 28,
Marcel Dekker Inc., New York, 1974, p. 48.

\end{thebibliography}





%%%%%
%%%%%
\end{document}
