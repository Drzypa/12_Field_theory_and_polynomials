\documentclass[12pt]{article}
\usepackage{pmmeta}
\pmcanonicalname{ResultantalternativeTreatment}
\pmcreated{2013-03-22 12:32:52}
\pmmodified{2013-03-22 12:32:52}
\pmowner{Mathprof}{13753}
\pmmodifier{Mathprof}{13753}
\pmtitle{resultant (alternative treatment)}
\pmrecord{6}{32792}
\pmprivacy{1}
\pmauthor{Mathprof}{13753}
\pmtype{Definition}
\pmcomment{trigger rebuild}
\pmclassification{msc}{12E05}
\pmrelated{Discriminant}

\usepackage{amsmath}
\usepackage{amsfonts}
\usepackage{amssymb}

\newcommand{\dn}{\delta^{(n)}}
\newcommand{\kfield}{\mathbb{K}}
\newcommand{\Res}{\mathrm{Res}}
 
\newcommand{\ha}{A}
\newcommand{\hb}{B}

 
\newcommand{\reals}{\mathbb{R}}
\newcommand{\natnums}{\mathbb{N}}
\newcommand{\cnums}{\mathbb{C}}
\newcommand{\znums}{\mathbb{Z}}

\newcommand{\lp}{\left(}
\newcommand{\rp}{\right)}
\newcommand{\lb}{\left[}
\newcommand{\rb}{\right]}

\newcommand{\supth}{^{\text{th}}}


\newtheorem{proposition}{Proposition}
\begin{document}
\paragraph{Summary.}
The {\em resultant} of two polynomials is a number, calculated from
the coefficients of those polynomials, that vanishes if and only if
the two polynomials share a common root.  Conversely, the resultant is
non-zero if and only if the two polynomials are mutually prime.

\paragraph{Definition.}
Let $\kfield$ be a field and let
\begin{align*}
p(x) &= a_0 x^n + a_1 x^{n-1} + \ldots + a_n,\\
q(x) &= b_0 x^m + b_1 x^{m-1} + \ldots + b_m  
\end{align*}
be two polynomials over $\kfield$ of degree $n$ and $m$, respectively.
We define $\Res[p,q]\in\kfield$, the resultant of $p(x)$ and $q(x)$,
to be the determinant of a $n+m$ square matrix with columns 1 to $m$
formed by shifted sequences consisting of the coefficients of $p(x)$,
and columns $m+1$ to $n+m$ formed by shifted sequences consisting of
coefficients of $q(x)$, i.e.

$$\Res[p,q] = \left | 
  \begin{array}{cccccccc}
    a_0 & 0 &  \ldots &  0 & b_0 & 0 & \ldots & 0 \\
    a_1 & a_0  & \ldots & 0 & b_1 & b_0 & \ldots & 0 \\
    a_2 & a_1 & \ldots  & 0 & b_2 & b_1 & \ldots & 0\\
    \vdots & \vdots  & \ddots & \vdots  &\vdots  &\vdots 
    &\ddots & \vdots \\
    0 & 0 &  \ldots  & a_{n-1} & 0 & 0 &\ldots & b_{m-1} \\
    0 & 0 &  \ldots  & a_n & 0 & 0 &\ldots & b_m \\
  \end{array}
\right|
$$

% {\small
% $$\left | 
%   \begin{array}{ccccccccccc}
%     a_0 & a_1 & \ldots & a_{m-2} & a_{m-1} & a_m & \ldots & a_{n} & 0
%     & \ldots & 0 \\ 
%     0 & a_0 &  \ldots & a_{m-3} & a_{m-2} & a_{m-1} &  \ldots &
%     a_{n-1} & a_{n} & 
%     \ldots & 0 \\ 
%     \vdots  & \vdots & \ddots & \vdots&\vdots &\vdots & \ddots &
%     \vdots & \vdots & 
%     \ddots & \vdots \\
%     0 & 0 &  \ldots & a_0 & a_1 & a_2 & \ldots & a_{n-m+2} & a_{n-m+3} &
%     \ldots & a_n \\ 
%     b_0 & b_1 &\ldots &b_{m-2} & b_{m-1} & b_m & \ldots & 0 & 0&\ldots & 0 \\ 
%     0 & b_0 &  \ldots &b_{m-3} & b_{m-2} & b_{m-1} & \ldots & 0 & 0&
%     \ldots & 0 \\ 
%     \vdots  & \vdots & \ddots & \vdots& \vdots & \vdots & \ddots & \vdots &
%     \vdots & \ddots & \vdots \\
%     0 & 0  &  \ldots & b_0 & b_1 & b_2 & \ldots & b_{n-m+2} &
%     b_{n-m+3}  & \ldots & 0 \\ 
%     \vdots  & \vdots & \ddots & \vdots& \vdots & \vdots & \ddots & \vdots &
%     \vdots & \ddots & \vdots \\
%     0 & 0  &  \ldots & 0 & 0 & 0 & \ldots & b_2 &
%     b_3  & \ldots & b_m \\ 
%     \end{array}\right|$$}

\begin{proposition}
  The resultant of two polynomials is non-zero if and only if the
  polynomials are  relatively prime.
\end{proposition}
{\em Proof.} Let $p(x), q(x)\in \kfield[x]$ be two arbitrary
polynomials of degree $n$ and $m$, respectively.  The polynomials are
relatively prime if and only if every polynomial --- including the
unit polynomial 1 --- can be formed as a linear combination of $p(x)$
and $q(x)$.  Let
\begin{align*}
r(x) &= c_0 x^{m-1} + c_1 x^{m-2} + \ldots + c_{m-1},\\
s(x) &= d_0 x^{n-1} + b_1 x^{n-2} + \ldots + d_{n-1}
\end{align*}
be polynomials of degree
$m-1$ and $n-1$, respectively.  The coefficients of the linear
combination
$r(x) p(x) + s(x) q(x)$ are given by the following matrix--vector
multiplication:

$$\begin{bmatrix}
    a_0 & 0 &  \ldots &  0 & b_0 & 0 & \ldots & 0 \\
    a_1 & a_0  & \ldots & 0 & b_1 & b_0 & \ldots & 0 \\
    a_2 & a_1 & \ldots  & 0 & b_2 & b_1 & \ldots & 0\\
    \vdots & \vdots  & \ddots & \vdots  &\vdots  &\vdots 
    &\ddots & \vdots \\
    0 & 0 &  \ldots  & a_{n-1} & 0 & 0 &\ldots & b_{m-1} \\
    0 & 0 &  \ldots  & a_n & 0 & 0 &\ldots & b_m \\
  \end{bmatrix}
  \begin{bmatrix}
    c_0 \\ c_1 \\ c_2 \\ \vdots \\ c_{m-1} \\ d_0 \\ d_1 \\ d_2 \\
    \vdots \\ d_{n-1}
  \end{bmatrix}
$$
In consequence of the preceding remarks, $p(x)$ and $q(x)$ are
relatively prime if and only if the matrix above is non-singular,
i.e. the resultant is non-vanishing.  Q.E.D.

\paragraph{Alternative Characterization.}
The following Proposition describes the resultant of two polynomials
in terms of the polynomials' roots.  Indeed this property uniquely
characterizes the resultant, as can be seen by carefully studying the
appended proof.
\begin{proposition}
  Let $p(x), q(x)$ be as above and let $x_1,\ldots,x_n$ and
  $y_1,\ldots,y_m$ be their respective roots in the algebraic closure
  of $\kfield$.  Then,
  $$\Res[p,q] = a_0^m\, b_0^n \prod_{i=1}^n \prod_{j=1}^m (x_i-y_j)$$
\end{proposition}
{\em Proof.}  The multilinearity property of determinants implies that
$$\Res[p,q] = a_0^m\, b_0^n
 \left | 
  \begin{array}{cccccccc}
    1 & 0 &  \ldots &  0 & 1 & 0 & \ldots & 0 \\
    \ha_1 & 1  & \ldots & 0 & \hb_1 & 1 & \ldots & 0 \\
    \ha_2 & \ha_1 & \ldots  & 0 & \hb_2 & \hb_1 & \ldots & 0\\
    \vdots & \vdots  & \ddots & \vdots  &\vdots  &\vdots 
    &\ddots & \vdots \\
    0 & 0 &  \ldots  & \ha_{n-1} & 0 & 0 &\ldots & \hb_{m-1} \\
    0 & 0 &  \ldots  & \ha_n & 0 & 0 &\ldots & \hb_m \\
  \end{array}
\right|$$
where
\begin{align*}
A_i &= \frac{a_i}{a_0},\quad i=1,\ldots n,\\
B_j &= \frac{b_j}{b_0},\quad j=1,\ldots m.
\end{align*}
It therefore suffices to prove the proposition for monic polynomials.
Without loss of generality we can also assume that the roots in
question are algebraically independent.

Thus, let $X_1,\ldots,X_n, Y_1,\ldots,Y_m$ be indeterminates and set
\begin{align*}
F(X_1,\ldots,X_n, Y_1,\ldots,Y_m) &= \prod_{i=1}^n \prod_{j=1}^m
(X_i-Y_j)\\
P(x)&=(x-X_1)\ldots (x-X_n),\\
Q(x) &= (x-Y_1)\ldots (x-Y_m),\\
G(X_1,\ldots,X_n, Y_1,\ldots,Y_m) &= \Res[P,Q]
\end{align*}
Now by Proposition 1, $G$ vanishes if we replace any of the
$Y_1,\ldots,Y_m$ by any of $X_1,\ldots,X_n$ and hence $F$ divides $G$.

Next, consider the main diagonal of the matrix whose determinant gives
$\Res[P,Q]$. The first $m$ entries of the diagonal are equal to $1$,
and the next $n$ entries are equal to $(-1)^m Y_1\ldots Y_m$.  It
follows that the expansion of $G$ contains a term of the form
$(-1)^{mn} Y_1^n\ldots Y_m^n$.  However, the expansion of $F$ contains
exactly the same term, and therefore $F=G$.  Q.E.D.
%%%%%
%%%%%
\end{document}
