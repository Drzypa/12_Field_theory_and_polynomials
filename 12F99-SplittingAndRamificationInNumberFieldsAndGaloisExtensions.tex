\documentclass[12pt]{article}
\usepackage{pmmeta}
\pmcanonicalname{SplittingAndRamificationInNumberFieldsAndGaloisExtensions}
\pmcreated{2013-03-22 15:05:29}
\pmmodified{2013-03-22 15:05:29}
\pmowner{alozano}{2414}
\pmmodifier{alozano}{2414}
\pmtitle{splitting and ramification in number fields and Galois extensions}
\pmrecord{11}{36818}
\pmprivacy{1}
\pmauthor{alozano}{2414}
\pmtype{Definition}
\pmcomment{trigger rebuild}
\pmclassification{msc}{12F99}
\pmclassification{msc}{13B02}
\pmclassification{msc}{11S15}
\pmsynonym{completely split}{SplittingAndRamificationInNumberFieldsAndGaloisExtensions}
\pmsynonym{strongly ramified}{SplittingAndRamificationInNumberFieldsAndGaloisExtensions}
\pmsynonym{wild ramification}{SplittingAndRamificationInNumberFieldsAndGaloisExtensions}
\pmrelated{Ramify}
\pmrelated{InertialDegree}
\pmrelated{CalculatingTheSplittingOfPrimes}
\pmrelated{PrimeIdealDecompositionInQuadraticExtensionsOfMathbbQ}
\pmrelated{PrimeIdealDecompositionInCyclotomicExtensionsOfMathbbQ}
\pmdefines{totally ramified}
\pmdefines{totally split}
\pmdefines{wildly ramified}
\pmdefines{tamely ramified}

\endmetadata

% this is the default PlanetMath preamble.  as your knowledge
% of TeX increases, you will probably want to edit this, but
% it should be fine as is for beginners.

% almost certainly you want these
\usepackage{amssymb}
\usepackage{amsmath}
\usepackage{amsthm}
\usepackage{amsfonts}

% used for TeXing text within eps files
%\usepackage{psfrag}
% need this for including graphics (\includegraphics)
%\usepackage{graphicx}
% for neatly defining theorems and propositions
%\usepackage{amsthm}
% making logically defined graphics
%%%\usepackage{xypic}

% there are many more packages, add them here as you need them

% define commands here

\newtheorem{thm}{Theorem}
\newtheorem{defn}{Definition}
\newtheorem{prop}{Proposition}
\newtheorem{lemma}{Lemma}
\newtheorem{cor}{Corollary}

% Some sets
\newcommand{\Nats}{\mathbb{N}}
\newcommand{\Ints}{\mathbb{Z}}
\newcommand{\Reals}{\mathbb{R}}
\newcommand{\Complex}{\mathbb{C}}
\newcommand{\Rats}{\mathbb{Q}}
\newcommand{\p}{{\mathfrak{p}}}
\newcommand{\A}{{\mathfrak{A}}}
\renewcommand{\P}{{\mathfrak{P}}}
\newcommand{\Gal}{\operatorname{Gal}}
\newcommand{\intK}{\mathcal{O}_K}
\newcommand{\intF}{\mathcal{O}_F}
\begin{document}
Let $F/K$ be an extension of number fields and let $\mathcal{O}_F$ and $\mathcal{O}_K$ be their respective rings of integers. The ring of integers of a number field is a Dedekind domain, and these enjoy the property that every ideal $\A$ factors uniquely as a finite product of prime ideals (see the entry \PMlinkname{fractional ideal}{FractionalIdeal}). Let $\p$ be a prime ideal of $\intK$. Then $\p \intF$ is an ideal of $\intF$. Let us assume that the prime ideal factorization of $\p \intF$ into primes of $\intF$ is as follows:
\begin{eqnarray}
\label{eq1} \p \intF=\prod_{i=1}^r {\P_i}^{e_i}
\end{eqnarray}
We say that the primes $\P_i$ lie above $\p$ and $\P_i|\p$ (divides). The exponent $e_i$ (commonly denoted as $e(\P_i|\p)$) is the ramification index of $\P_i$ over $\p$. Notice that for each prime ideal $\P_i$, the quotient ring $\intF/\P_i$ is a finite field extension of the finite field $\intK/\p$ (also called the residue field). The degree of this extension is called the inertial degree of $\P_i$ over $\p$ and it is usually denoted by:
$$f(\P_i|\p)=[\intF/\P_i:\intK/\p].$$

Notice that as it is pointed out in the entry ``\PMlinkname{inertial degree}{InertialDegree}'', the ramification index and the inertial degree are related by the formula:
\begin{eqnarray}
\label{eq2}
\sum_{i=1}^r e(\P_i|\p)f(\P_i|\p)=[F:K]
\end{eqnarray}
where $r$ is the number of prime ideals lying above $\p$ (as in Eq. (\ref{eq1})). See the theorem below for an improvement of Eq. (\ref{eq2}) in the case when $F/K$ is Galois.
\begin{defn}
Let $F,K$ and $\P_i,\p$ be as above.
\begin{enumerate}
\item If $e_i>1$ for some $i$, then we say that $\P_i$ is {\bf ramified} over $\p$ and $\p$ ramifies in $F/K$. If $e_i=1$ for all $i$ then we say that $\p$ is unramified in $F/K$.
\item If there is a unique prime ideal $\P$ lying above $\p$ (so $r=1$) and $f(\P|\p)=1$ then we say that $\p$ is {\bf totally ramified} in $F/K$. In this case $e(\P|\p)=[F:K]$.
\item On the other hand, if $e(\P_i|\p)=f(\P_i|\p)=1$ for all $i$, we say that $\p$ is {\bf totally split} (or splits completely) in $F/K$. Notice that there are exactly $r=[F:K]$ prime ideals of $\intF$ lying above $\p$.
\item  Let $p$ be the characteristic of the residue field $\intK/\p$. If $e_i=e(\P_i|\p)>1$ and $e_i$ and $p$ are relatively prime, then we say that $\P_i$ is {\bf tamely ramified}. If $p|e_i$ then we say that $\P_i$ is {\bf strongly ramified} (or wildly ramified). 
\end{enumerate}
\end{defn}

When the extension $F/K$ is a Galois extension then Eq. (\ref{eq2}) is quite more simple:

\begin{thm}
Assume that $F/K$ is a Galois extension of number fields. Then all the ramification indices $e_i=e(\P_i|\p)$ are equal to the same number $e$, all the inertial degrees $f_i=f(\P_i|\p)$ are equal to the same number $f$ and the ideal $\p \intF$ factors as:
$$\p\intF = \prod_{i=1}^r \P_i^e=(\P_1\cdot\P_2\cdot\ldots\cdot\P_r)^e$$
Moreover:
$$e\cdot f\cdot r=[F:K].$$
\end{thm}
%%%%%
%%%%%
\end{document}
