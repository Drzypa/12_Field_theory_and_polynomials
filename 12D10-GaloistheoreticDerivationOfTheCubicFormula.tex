\documentclass[12pt]{article}
\usepackage{pmmeta}
\pmcanonicalname{GaloistheoreticDerivationOfTheCubicFormula}
\pmcreated{2013-03-22 12:10:43}
\pmmodified{2013-03-22 12:10:43}
\pmowner{djao}{24}
\pmmodifier{djao}{24}
\pmtitle{Galois-theoretic derivation of the cubic formula}
\pmrecord{9}{31438}
\pmprivacy{1}
\pmauthor{djao}{24}
\pmtype{Proof}
\pmcomment{trigger rebuild}
\pmclassification{msc}{12D10}
\pmrelated{GaloisTheoreticDerivationOfTheQuarticFormula}

\endmetadata

% this is the default PlanetMath preamble.  as your knowledge
% of TeX increases, you will probably want to edit this, but
% it should be fine as is for beginners.

% almost certainly you want these
\usepackage{amssymb}
\usepackage{amsmath}
\usepackage{amsfonts}

% used for TeXing text within eps files
%\usepackage{psfrag}
% need this for including graphics (\includegraphics)
%\usepackage{graphicx}
% for neatly defining theorems and propositions
%\usepackage{amsthm}
% making logically defined graphics
\usepackage[matrix,arrow]{xypic} 

% there are many more packages, add them here as you need them

% define commands here
\newcommand{\C}{\mathbb{C}}
\newcommand{\Gal}{\operatorname{Gal}}
\begin{document}
We are trying to find the roots $r_1, r_2, r_3$ of the polynomial $x^3
+ ax^2 + bx + c = 0$. From the equation
$$
(x-r_1)(x-r_2)(x-r_3) = x^3 + ax^2 + bx + c
$$
we see that
\begin{eqnarray*}
a & = & -(r_1 + r_2 + r_3) \\
b & = & r_1 r_2 + r_1 r_3 + r_2 r_3 \\
c & = & -r_1 r_2 r_3
\end{eqnarray*}
The goal is to explicitly construct a radical tower over the field
$k = \C(a,b,c)$ that contains the three roots $r_1, r_2, r_3$.

Let $L = \C(r_1,r_2,r_3)$. By Galois theory we know that
$\Gal(L/\C(a,b,c)) = S_3$. Let $K \subset L$ be the fixed field of
$A_3 \subset S_3$. We have a tower of field extensions
$$
\xymatrix{
\makebox[1em][l]{$L = \C(r_1,r_2,r_3)$} \ar@{-}[d]_{A_3} \\
\makebox[1em][l]{$K =\ ?$} \ar@{-}[d]_{S_3/A_3} \\
\makebox[1em][l]{$k = \C(a,b,c)$}
}
$$
which we know from Galois theory is radical. We use Galois theory to find $K$ and exhibit radical generators for these extensions.

Let $\sigma := (123)$ be a generator of $\Gal(L/K) = A_3$. Let $\omega = e^{2 \pi i/3} \in \C \subset L$ be a primitive cube root of unity. Since $\omega$ has norm 1, Hilbert's Theorem 90 tells us that $\omega = y/\sigma(y)$ for some $y \in L$. Galois theory (or Kummer theory) then tells us that $L = K(y)$ and $y^3 \in K$, thus exhibiting $L$ as a radical extension of $K$.

The proof of Hilbert's Theorem 90 provides a procedure for finding $y$, which is as follows: choose any $x \in L$, form the quantity
$$
\omega x + \omega^2 \sigma(x) + \omega^3 \sigma^2(x);
$$
then this quantity automatically yields a suitable value for $y$ provided that it is nonzero. In particular, choosing $x = r_2$ yields
$$
y = r_1 + \omega r_2 + \omega^2 r_3.
$$
and we have $L = K(y)$ with $y^3 \in K$. Moreover, since $\tau := (23)$ does not fix $y^3$, it follows that $y^3 \notin k$, and this, combined with $[K:k] = 2$, shows that $K = k(y^3)$.

Set $z := \tau(y) = r_1 + \omega^2 r_2 + \omega r_3$. Applying the same technique to the extension $K/k$, we find that $K = k(y^3 - z^3)$ with $(y^3-z^3)^2 \in k$, and this exhibits $K$ as a radical extension of $k$.

To get explicit formulas, start with $y^3 + z^3$ and $y^3 z^3$, which are fixed by $S_3$ and thus guaranteed to be in $k$. Using the reduction algorithm for symmetric polynomials, we find
\begin{eqnarray*}
y^3 + z^3 & = & -2a^3 + 9ab - 27c \\
y^3 z^3 & = & (a^2 - 3b)^3
\end{eqnarray*}
Solving this system for $y$ and $z$ yields
\begin{eqnarray*}
y & = & \left(\frac{-2a^3 + 9ab - 27c + \sqrt{(2a^3-9ab+27c)^2 + 4(-a^2+3b)^3}}{2}\right)^{1/3} \\
z & = & \left(\frac{-2a^3 + 9ab - 27c - \sqrt{(2a^3-9ab+27c)^2 + 4(-a^2+3b)^3}}{2}\right)^{1/3}
\end{eqnarray*}
Now we solve the linear system
\begin{eqnarray*}
a & = & -(r_1+r_2+r_3) \\
y & = & r_1 + \omega r_2 + \omega^2 r_3 \\
z & = & r_1 + \omega^2 r_2 + \omega r_3
\end{eqnarray*}
and we get
\begin{eqnarray*}
r_1 & = & \frac{1}{3} (-a + y + z) \\
r_2 & = & \frac{1}{3} (-a + \omega^2 y + \omega z) \\
r_3 & = & \frac{1}{3} (-a + \omega y + \omega^2 z)
\end{eqnarray*}
which expresses $r_1, r_2, r_3$ as radical expressions of $a,b,c$ by way of the previously obtained expressions for $y$ and $z$, and completes the derivation of the cubic formula.
%%%%%
%%%%%
%%%%%
\end{document}
