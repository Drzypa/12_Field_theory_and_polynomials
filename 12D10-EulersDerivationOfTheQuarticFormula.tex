\documentclass[12pt]{article}
\usepackage{pmmeta}
\pmcanonicalname{EulersDerivationOfTheQuarticFormula}
\pmcreated{2013-03-22 17:51:58}
\pmmodified{2013-03-22 17:51:58}
\pmowner{pahio}{2872}
\pmmodifier{pahio}{2872}
\pmtitle{Euler's derivation of the quartic formula}
\pmrecord{10}{40344}
\pmprivacy{1}
\pmauthor{pahio}{2872}
\pmtype{Theorem}
\pmcomment{trigger rebuild}
\pmclassification{msc}{12D10}
\pmsynonym{quartic formula by Euler}{EulersDerivationOfTheQuarticFormula}
\pmrelated{TchirnhausTransformations}
\pmrelated{CasusIrreducibilis}
\pmrelated{ZeroRuleOfProduct}
\pmrelated{ErnstLindelof}
\pmrelated{KalleVaisala}
\pmrelated{BiquadraticEquation2}
\pmrelated{SymmetricQuarticEquation}

% this is the default PlanetMath preamble.  as your knowledge
% of TeX increases, you will probably want to edit this, but
% it should be fine as is for beginners.

% almost certainly you want these
\usepackage{amssymb}
\usepackage{amsmath}
\usepackage{amsfonts}

% used for TeXing text within eps files
%\usepackage{psfrag}
% need this for including graphics (\includegraphics)
%\usepackage{graphicx}
% for neatly defining theorems and propositions
 \usepackage{amsthm}
% making logically defined graphics
%%%\usepackage{xypic}

% there are many more packages, add them here as you need them

% define commands here

\theoremstyle{definition}
\newtheorem*{thmplain}{Theorem}

\begin{document}
Let us consider the quartic equation
\begin{align}
y^4+py^2+qy+r = 0,
\end{align}
where $p,\,q,\,r$ are arbitrary known complex numbers.\, We substitute in the equation
\begin{align}
y := u+v+w.
\end{align}
We get firstly\\
$y^2 = (u^2+v^2+w^2)+2(vw+wu+uv),$\\
$y^4 = (u^2+v^2+w^2)^2+4(u^2+v^2+w^2)(vw+wu+uv)+4(v^2w^2+w^2u^2+u^2v^2)+8uvw(u+v+w).$

Thus (1) attains the form 
$$4(v^2w^2+w^2u^2+u^2v^2)+(u^2+v^2+w^2)^2+p(u^2+v^2+w^2)+r \qquad\;$$
$$+(vw+wu+uv)[4(u^2+v^2+w^2)+2p]+(u+v+w)[8uvw+q] = 0.$$
When $u,\,v,\,w$ are determined so that
\begin{align}
u^2+v^2+w^2 = -\frac{p}{2},
\end{align}
\begin{align}
uvw = -\frac{q}{8},
\end{align}
the expressions in the brackets vanish and our equation shrinks to the form
\begin{align}
v^2w^2+w^2u^2+u^2v^2 = \frac{p^2-4r}{16}.
\end{align}
Squaring (4) gives
\begin{align}
u^2v^2w^2 = \frac{q^2}{64}.
\end{align}
The left hand sides of (3), (5) and (6) are the elementary symmetric polynomials of $u^2$, $v^2$, $w^2$, whence these three squares are the roots $z_1$, $z_2$, $z_3$ of the so-called cubic resolvent equation
\begin{align}
z^3+\frac{p}{2}z^2+\frac{p^2-4r}{16}z-\frac{q^2}{64} = 0.
\end{align}
Therefore we may write
$$u = \pm\sqrt{z_1}, \quad v = \pm\sqrt{z_2}, \quad w = \pm\sqrt{z_3}.$$
All 8 sign combinations of those square roots satisfy the equations (3), (5), (6).  In order to satisfy also (4) the signs must be chosen suitably.\, If\, $u_0,\,v_0,\,w_0$ is some suitable combination of the values of the square roots, then all possible combinations are
$$u_0,\,v_0,\,w_0;\quad u_0,\,-v_0,\,-w_0;\quad -u_0,\,v_0,\,-w_0;\quad -u_0,\,-v_0,\,w_0.$$ 

Accordingly, we have the 

\textbf{Theorem} (Euler 1739).\, The roots of the equation (1) are
\begin{align}
\begin{cases}
y_1 = \;\;u_0+v_0+w_0,\\
y_2 = \;\;u_0-v_0-w_0,\\
y_3 = -u_0+v_0-w_0,\\
y_4 = -u_0-v_0+w_0,
\end{cases}
\end{align}
where $u_0,\,v_0,\,w_0$ are square roots of the roots of the cubic resolvent (7).\, The signs of the square roots must be chosen such that 
$$u_0v_0w_0 = -\frac{q}{8}.\\$$


The equations (8) imply an important formula
\begin{align*}
(y_1-y_2)(y_1-y_3)(y_1-y_4)(y_2-y_3)(y_2-y_4)(y_3-y_4) = & -2^6(v_0^2-w_0^2)(w_0^2-u_0^2)(u_0^2-v_0^2)\\ =
& -64(z_2-z_3)(z_3-z_1)(z_1-z_2),
\end{align*}
which yields the

\textbf{Corollary.}\, A quartic equation has a multiple root always and only when its cubic resolvent has such one.


\begin{thebibliography}{9}
\bibitem{J} {\sc Ernst Lindel\"of}: {\em Johdatus korkeampaan analyysiin}. Fourth edition. Werner S\"oderstr\"om Osakeyhti\"o, Porvoo ja Helsinki (1956).
\bibitem{K.V.} {\sc K. V\"ais\"al\"a}: {\em Lukuteorian ja korkeamman algebran alkeet}.\, Tiedekirjasto No. 17.\quad  Kustannusosakeyhti\"o Otava, Helsinki (1950).

\end{thebibliography}
%%%%%
%%%%%
\end{document}
