\documentclass[12pt]{article}
\usepackage{pmmeta}
\pmcanonicalname{FundamentalTheoremOfAlgebraResult}
\pmcreated{2013-03-22 14:22:01}
\pmmodified{2013-03-22 14:22:01}
\pmowner{rspuzio}{6075}
\pmmodifier{rspuzio}{6075}
\pmtitle{fundamental theorem of algebra result}
\pmrecord{7}{35851}
\pmprivacy{1}
\pmauthor{rspuzio}{6075}
\pmtype{Theorem}
\pmcomment{trigger rebuild}
\pmclassification{msc}{12D99}
\pmclassification{msc}{30A99}

\endmetadata

% this is the default PlanetMath preamble.  as your knowledge
% of TeX increases, you will probably want to edit this, but
% it should be fine as is for beginners.

% almost certainly you want these
\usepackage{amssymb}
\usepackage{amsmath}
\usepackage{amsfonts}

% used for TeXing text within eps files
%\usepackage{psfrag}
% need this for including graphics (\includegraphics)
%\usepackage{graphicx}
% for neatly defining theorems and propositions
%\usepackage{amsthm}
% making logically defined graphics
%%%\usepackage{xypic}

% there are many more packages, add them here as you need them

% define commands here
\newcommand{\Lindent}{0.4in}
\newenvironment{Lalgorithm}[4]{
\textbf{Algorithm} \textsc{#1}\texttt{(#2)}\newline
\textit{Input}: #3\newline
\textit{Output}: #4

}{}
\newenvironment{Lfloatalgorithm}[6][h]{
\begin{figure}[#1]
\caption{#2}
\begin{Lalgorithm}{#3}{#4}{#5}{#6}
}{
\end{Lalgorithm}
\end{figure}
}
\newcommand{\Lgets}{\ensuremath{\gets}}
\newcommand{\Lgroup}[1]{\textbf{begin}\\\hspace*{\Lindent}\parbox{\textwidth}{#1}\\\textbf{end}}

\newcommand{\Lif}[2]{\textbf{if} #1 \textbf{then}\\\hspace*{\Lindent}\parbox{\textwidth}{#2}}

\newcommand{\Lwhile}[2]{\textbf{while} #1 \\\hspace*{\Lindent}\parbox{\textwidth}{#2}}

\newcommand{\Lelse}[1]{\textbf{else}\\\hspace*{\Lindent}\parbox{\textwidth}{#1}}
\newcommand{\Lelseif}[2]{\textbf{else if} #1 \textbf{then}\\\hspace*{\Lindent}\parbox{\textwidth}{#2}}
\begin{document}
This leads to the following theorem:

Given a polynomial $p(x)=a_nx^n+a_{n-1}x^{n-1}+\ldots+a_1x+a_0 $ of degree $n\geq 1$ where $a_i\in \mathbb{C}$, there are exactly $n$ roots in $\mathbb{C}$ to the equation $p(x)=0$ if we count multiple roots.

\emph{Proof}
The non-constant polynomial $a_1x-a_0$ has one root, $x=a_0/a_1$.
Next, assume that a polynomial of degree $n-1$ has $n-1$ roots.

The polynomial of degree $n$ has then by the fundamental theorem of algebra a root $z_n$. With polynomial division we find the unique polynomial $q(x)$ such that $p(x)=(x-z_n)q(x)$. The original equation has then $1 + (n-1)=n $ roots.
By induction, every non-constant polynomial of degree $n$ has exactly $n$ roots.

For example, $x^4=0$ has four roots, $x_1=x_2=x_3=x_4=0$.
%%%%%
%%%%%
\end{document}
