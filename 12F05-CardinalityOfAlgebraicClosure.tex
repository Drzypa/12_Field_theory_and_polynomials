\documentclass[12pt]{article}
\usepackage{pmmeta}
\pmcanonicalname{CardinalityOfAlgebraicClosure}
\pmcreated{2013-03-22 16:28:54}
\pmmodified{2013-03-22 16:28:54}
\pmowner{rspuzio}{6075}
\pmmodifier{rspuzio}{6075}
\pmtitle{cardinality of algebraic closure}
\pmrecord{22}{38649}
\pmprivacy{1}
\pmauthor{rspuzio}{6075}
\pmtype{Theorem}
\pmcomment{trigger rebuild}
\pmclassification{msc}{12F05}

% this is the default PlanetMath preamble.  as your knowledge
% of TeX increases, you will probably want to edit this, but
% it should be fine as is for beginners.

% almost certainly you want these
\usepackage{amssymb}
\usepackage{amsmath}
\usepackage{amsfonts}

% used for TeXing text within eps files
%\usepackage{psfrag}
% need this for including graphics (\includegraphics)
%\usepackage{graphicx}
% for neatly defining theorems and propositions
\usepackage{amsthm}
% making logically defined graphics
%%%\usepackage{xypic}

% there are many more packages, add them here as you need them

% define commands here

\newtheorem{theorem}{Theorem}
\begin{document}
\begin{theorem}
If a field is finite, then its algebraic closure is countably infinite.
\end{theorem}

\begin{proof}
Because a finite field cannot be algebraically closed, the algebraic closure 
of a finite field must be infinite.  Hence, it only remains to show that the
algebraic closure is countable.  Every element of the algebraic closure is 
the root of some polynomial.  Furthermore, every polynomial
has a finite number of roots (the number is bounded by its degree) and there
are a countable number of polynomials whose coefficients belong to a given
finite set.  Since the union of a countable family of finite sets is 
countable, the number of elements of the algebraic closure is countable.
\end{proof}

\begin{theorem}
If a field is infinite, then its algebraic closure has the same cardinality
as the original field.
\end{theorem}

\begin{proof}
Since a field is isomorphic to a subset of its algebraic closure, it follows 
that the cardinality of the closure is at least the cardinality of the original field.  The number of polynomials of degree $n$ with coefficients in
a given set is the same as the number of $n$- tuplets of elements of $S$,
which is the cardinality of the set raised to the $n$-th power.  Since an
infinite cardinal raised to an finite power equals itself, the number of
polynomials of a given degree equals the the cardinality the original field.
Since the cardinality of the union of a countable number of sets each of
which has the same infinite number of elements equals the common cardinality 
of the sets, the total number of polynomials with coefficients in the field
equals the cardinality of the field.  Since every element of the algebraic
closure of a field is the root of some polynomial with elements of the
field for coefficients and a polynomial has a finite number of roots, it
follows that the cardinality of the algebraic closure is bounded by the
cardinality of the original field.
\end{proof}

\begin{theorem}
For every transfinite cardinal number $N$, there exists an algebrically
closed field with exactly $N$ elements.
\end{theorem}

\begin{proof}
Let $F$ be the field of rational functions with integer coefficients in 
variables $x_i$, where the index $i$ ranges over an index set $I$ 
whose cardinality is $N$.  We claim that the cardinality of $F$ is $N$. 
The cardinality is at least $N$ becasue we have the $N$ rational functions
$x_i$, so it only remains to show that the cardinality is not greater
than $N$.  To do this, we first show that the number of polynomials in
the $x_i$ with integer coefficients equals $N$.  A polynomial is 
determined by a finite set of coefficients and a finite set of monomials.
The number of possible sets of coefficients is the number of finite 
tuplets of integers, which is $\aleph_0$.  Since a monomial may be determined 
by a mapping of a finite set into the set $\{ x_i \mid i \in I \}$, the number 
of possible monomials of degree $n$ is bounded by $N^n$.  Since $N$ is
transfinite and $n$ is finite, we have $N^n = N$.  Thus the number of 
possible monomials is bounded by $N \aleph_0 = N$.  So the number of 
polynomials is bounded by the product of $\aleph_0$ and $N$, which is
$N$ and the number of rational functions is bounded by $N^2$, which
equals $N$.
\end{proof}

%%%%%
%%%%%
\end{document}
