\documentclass[12pt]{article}
\usepackage{pmmeta}
\pmcanonicalname{TranslationAutomorphismOfAPolynomialRing}
\pmcreated{2013-03-22 14:16:13}
\pmmodified{2013-03-22 14:16:13}
\pmowner{archibal}{4430}
\pmmodifier{archibal}{4430}
\pmtitle{translation automorphism of a polynomial ring}
\pmrecord{4}{35720}
\pmprivacy{1}
\pmauthor{archibal}{4430}
\pmtype{Example}
\pmcomment{trigger rebuild}
\pmclassification{msc}{12E05}
\pmclassification{msc}{11C08}
\pmclassification{msc}{13P05}
\pmrelated{IsomorphismSwappingZeroAndUnity}

% this is the default PlanetMath preamble.  as your knowledge
% of TeX increases, you will probably want to edit this, but
% it should be fine as is for beginners.

% almost certainly you want these
\usepackage{amssymb}
\usepackage{amsmath}
\usepackage{amsfonts}

% used for TeXing text within eps files
%\usepackage{psfrag}
% need this for including graphics (\includegraphics)
%\usepackage{graphicx}
% for neatly defining theorems and propositions
%\usepackage{amsthm}
% making logically defined graphics
%%%\usepackage{xypic}

% there are many more packages, add them here as you need them

% define commands here

\newtheorem{theorem}{Theorem}
\newtheorem{defn}{Definition}
\newtheorem{prop}{Proposition}
\newtheorem{lemma}{Lemma}
\newtheorem{cor}{Corollary}
\begin{document}
Let $R$ be a commutative ring, let $R[X]$ be the polynomial ring over $R$, and let $a$ be an element of $R$.  Then we can define a homomorphism $\tau_a$ of $R[X]$ by constructing the evaluation homomorphism from $R[X]$ to $R[X]$ taking $r\in R$ to itself and taking $X$ to $X+a$.  

To see that $\tau_a$ is an automorphism, observe that $\tau_{-a}\circ\tau_a$ is the identity on $R\subset R[X]$ and takes $X$ to $X$, so by the uniqueness of the evaluation homomorphism, $\tau_{-a}\circ\tau_{a}$ is the identity.
%%%%%
%%%%%
\end{document}
