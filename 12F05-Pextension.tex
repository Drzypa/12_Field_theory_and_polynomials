\documentclass[12pt]{article}
\usepackage{pmmeta}
\pmcanonicalname{Pextension}
\pmcreated{2013-03-22 15:02:56}
\pmmodified{2013-03-22 15:02:56}
\pmowner{alozano}{2414}
\pmmodifier{alozano}{2414}
\pmtitle{$p$-extension}
\pmrecord{5}{36764}
\pmprivacy{1}
\pmauthor{alozano}{2414}
\pmtype{Definition}
\pmcomment{trigger rebuild}
\pmclassification{msc}{12F05}
\pmsynonym{p-extension}{Pextension}
%\pmkeywords{field extension}
\pmrelated{PGroup4}
\pmrelated{UnramifiedExtensionsAndClassNumberDivisibility}
\pmrelated{PushDownTheoremOnClassNumbers}
\pmrelated{ClassNumberDivisibilityInPExtensions}
\pmrelated{QuadraticExtension}

% this is the default PlanetMath preamble.  as your knowledge
% of TeX increases, you will probably want to edit this, but
% it should be fine as is for beginners.

% almost certainly you want these
\usepackage{amssymb}
\usepackage{amsmath}
\usepackage{amsthm}
\usepackage{amsfonts}

% used for TeXing text within eps files
%\usepackage{psfrag}
% need this for including graphics (\includegraphics)
%\usepackage{graphicx}
% for neatly defining theorems and propositions
%\usepackage{amsthm}
% making logically defined graphics
%%%\usepackage{xypic}

% there are many more packages, add them here as you need them

% define commands here

\newtheorem{thm}{Theorem}
\newtheorem{defn}{Definition}
\newtheorem{prop}{Proposition}
\newtheorem{lemma}{Lemma}
\newtheorem{cor}{Corollary}
\newtheorem{exa}{Example}

% Some sets
\newcommand{\Nats}{\mathbb{N}}
\newcommand{\Ints}{\mathbb{Z}}
\newcommand{\Reals}{\mathbb{R}}
\newcommand{\Complex}{\mathbb{C}}
\newcommand{\Rats}{\mathbb{Q}}
\begin{document}
\begin{defn}
Let $p$ be a prime number. A Galois extension of fields $E/F$, with $G=\operatorname{Gal}(E/F)$, is said to be a $p$-extension if $G$ is a $p$-group.  
\end{defn}

\begin{exa}
Let $d$ be a square-free integer. Then the field extension $\Rats(\sqrt{d})/\Rats$ is a $2$-extension.
\end{exa}

\begin{exa}
Let $p>2$ be a prime and, for any $n$, let $\zeta_{p^n}$ be a primitive $p^n$th root of unity. The cyclotomic extension:
$$\Rats(\zeta_{p^n})/\Rats(\zeta_p)$$
is a $p$-extension. Indeed:
$$G_n=\operatorname{Gal}(\Rats(\zeta_{p^n})/\Rats)\cong (\Ints/p^n\Ints)^\times$$
Thus, $|G_n|=\varphi(p^n)=p^{(n-1)}(p-1)$ and $|G_1|=\varphi(p)=p-1$, where $\varphi$ is the Euler phi function. Therefore the extension above is of degree $p^{(n-1)}$.
\end{exa}
%%%%%
%%%%%
\end{document}
