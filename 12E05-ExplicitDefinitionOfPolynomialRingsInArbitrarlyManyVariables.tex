\documentclass[12pt]{article}
\usepackage{pmmeta}
\pmcanonicalname{ExplicitDefinitionOfPolynomialRingsInArbitrarlyManyVariables}
\pmcreated{2013-03-22 19:18:10}
\pmmodified{2013-03-22 19:18:10}
\pmowner{joking}{16130}
\pmmodifier{joking}{16130}
\pmtitle{explicit definition of polynomial rings in arbitrarly many variables}
\pmrecord{8}{42238}
\pmprivacy{1}
\pmauthor{joking}{16130}
\pmtype{Definition}
\pmcomment{trigger rebuild}
\pmclassification{msc}{12E05}
\pmclassification{msc}{13P05}
\pmclassification{msc}{11C08}

% this is the default PlanetMath preamble.  as your knowledge
% of TeX increases, you will probably want to edit this, but
% it should be fine as is for beginners.

% almost certainly you want these
\usepackage{amssymb}
\usepackage{amsmath}
\usepackage{amsfonts}

% used for TeXing text within eps files
%\usepackage{psfrag}
% need this for including graphics (\includegraphics)
%\usepackage{graphicx}
% for neatly defining theorems and propositions
%\usepackage{amsthm}
% making logically defined graphics
%%%\usepackage{xypic}

% there are many more packages, add them here as you need them

% define commands here

\newcommand{\X}{\mathbb{X}}
\newcommand{\F}{\mathcal{F}}

\begin{document}
Let $R$ be a ring and let $\X$ be any set (possibly empty). We wish to give an explicit and formal definition of the polynomial ring $R[\X]$.

We start with the set 
$$\F(\X)=\{f:\X\to\mathbb{N}\ |\ f(x)=0\mbox{ for almost all }x\}.$$
If $\X=\{X_1,\ldots,X_n\}$ then the elements of $\F(\X)$ can be interpreted as monomials
$$X_1^{\alpha_1}\cdots X_n^{\alpha_n}.$$
Now define
$$R[\X]=\{F:\F(\X)\to R\ |\ F(f)=0\mbox{ for almost all }f\}.$$
The addition in $R[\X]$ is defined as pointwise addition.

Now we will define multiplication. First note that we have a multiplication on $\F(\X)$. For any $f,g:\X\to\mathbb{N}$ put
$$(fg)(x)=f(x)+g(x).$$
This is the same as multiplying $x^a\cdot x^b=x^{a+b}$.

Now for any $f\in\F(\X)$ define
$$M(h)=\{(f,g)\in\F(\X)^{2}\ |\ h=fg\},$$
Now if $F,G\in R[\X]$ then we define the multiplication
$$FG:\F(\X)\to R$$
by putting
$$(FG)(h)=\sum_{(f,g))\in M(h)}F(f)G(g).$$
Note that all of this well-defined, since both $F$ and $G$ vanish almost everywhere.

It can be shown that $R[\X]$ with these operations is a ring, even an $R$-algebra. This algebra is commutative if and only if $R$ is. Furthermore we have an algebra homomorphism
$$E:R\to R[\X]$$
which is defined as follows: for any $r\in R$ let $F_r:\F(\X)\to R$ be the function such that if $f:\X\to\mathbb{N}$ is such that $f(x)=0$ for any $x\in\X$, then put $F_r(f)=r$ and for any other function $f\in\F(\X)$ put $F_r(f)=0$. Then
$$E(r)=F_r$$
is our function, which is a monomorphism. Furthermore if $R$ is unital with the identity $1$, then
$$E(1)$$
is the identity in $R[\X]$. Anyway we can always interpret $R$ as a subset of $R[\X]$ if put $r=F_r$ for $r\in R$.

Note, that if $\X=\emptyset$, then $R[\emptyset]$ is still defined and $E:R\to R[\X]$ is an isomorphism of rings (it is ,,onto''). Actually these two conditions are equivalent.

Also note, that $\X$ itself can be interpreted as a subset of $R[\X]$. Indeed, for any $x\in\X$ define
$$f_x:\X\to\mathbb{N}$$
by $f_x(x)=1$ and $f_x(y)=0$ for any $y\neq x$. Then define
$$F_x:\F(\X)\to R$$
by putting $F_x(f_x)=1$ and $F_x(f)=0$ for any $f\neq f_x$. It can be easily seen that $F_x=F_y$ if and only if $x=y$. Thus we will use convention $x=F_x$. 

With these notations (i.e. $R,\X\subseteq R[\X]$) we have that elements of $R[\X]$ are exactly polynomials in the set of variables $\X$ with coefficients in $R$.
%%%%%
%%%%%
\end{document}
