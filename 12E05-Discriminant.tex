\documentclass[12pt]{article}
\usepackage{pmmeta}
\pmcanonicalname{Discriminant}
\pmcreated{2013-03-22 12:31:12}
\pmmodified{2013-03-22 12:31:12}
\pmowner{rspuzio}{6075}
\pmmodifier{rspuzio}{6075}
\pmtitle{discriminant}
\pmrecord{17}{32759}
\pmprivacy{1}
\pmauthor{rspuzio}{6075}
\pmtype{Definition}
\pmcomment{trigger rebuild}
\pmclassification{msc}{12E05}
\pmsynonym{polynomial discriminant}{Discriminant}
\pmrelated{Resolvent}
\pmrelated{DiscriminantOfANumberField}
\pmrelated{ModularDiscriminant}
\pmrelated{JInvariant}
\pmrelated{DiscriminantOfAlgebraicNumber}

\usepackage{amsmath}
\usepackage{amsfonts}
\usepackage{amssymb}
\usepackage{amsthm}

\newcommand{\dn}{\delta^{(n)}}
\newcommand{\kfield}{\mathbb{K}}
\newcommand{\fchar}{\mathrm{char}}
 
\newcommand{\reals}{\mathbb{R}}
\newcommand{\natnums}{\mathbb{N}}
\newcommand{\cnums}{\mathbb{C}}
\newcommand{\znums}{\mathbb{Z}}

\newcommand{\lp}{\left(}
\newcommand{\rp}{\right)}
\newcommand{\lb}{\left[}
\newcommand{\rb}{\right]}

\newcommand{\supth}{^{\text{th}}}


\newtheorem{proposition}{Proposition}
\begin{document}
\paragraph{Summary.}
The \emph{discriminant} of a given polynomial is a number, calculated
from the coefficients of that polynomial, that vanishes if and only if
that polynomial has one or more multiple roots.  Using the
discriminant we can test for the presence of multiple roots, without
having to actually calculate the roots of the polynomial in question.  

There are other ways to do this of course; one can look at the formal derivative of the polynomial (it will be coprime to the original polynomial if and only if that original had no multiple roots).  But the discriminant turns out to be valuable in a number of other contexts.  For example, we will see that the discriminant of $X^2+bX+c$ is $b^2-4c$; the quadratic formula states that the roots are $-b/2 \pm \sqrt{b^2-4c}/2$, so that the discriminant also determines whether the roots of this polynomial are real or not.  In higher degrees, its role is more complicated.  

There are other uses of the word ``discriminant'' that are closely related to this one.
If $\mathbb{Q}(\alpha)$ is a number field, then the \PMlinkid{discriminant}{2895} of $\mathbb{Q}(\alpha)$ is the discriminant of the minimal polynomial of $\alpha$.  For more general extensions of number fields, one must use a different definition of discriminant generalizing this one. If we have an elliptic curve over the rational numbers defined by the equation $y^2 = x^3 + Ax +B$, then its modular discriminant is the discriminant of the cubic polynomial on the right-hand side.  For more on both these facts, see \cite{marcus} on number fields and \cite{silv} on elliptic curves.

\paragraph{Definition.}
The discriminant of order $n\in\natnums$ is the polynomial, denoted
here \footnote{ The discriminant of a polynomial $p$ is oftentimes
also denoted as ``$\mathop{\rm disc} (p)$''}
by $\dn = \dn(a_1,\ldots,a_n)$, characterized by the following
relation:
\begin{equation}
\dn(s_1,s_2,\ldots,s_n) =
\prod_{i=1}^n \prod_{j=i+1}^n (x_i-x_j)^2,  
\end{equation}
where 
\[
s_k= s_k(x_1,\ldots,x_n),\quad k=1,\ldots,n
\]
is the
$k\supth$ elementary symmetric polynomial.  

The above relation is a defining one, because the right-hand side of
(1) is, evidently, a symmetric polynomial, and because the algebra of
symmetric polynomials is freely generated by the basic symmetric
polynomials, i.e. every symmetric polynomial arises in a unique
fashion as a polynomial of $s^1,\ldots,s^n$.

\begin{proposition}
The discriminant $d$ of a polynomial may be expressed with the resultant $R$ of the polynomial and its first derivative:
$$d \;=\; (-1)^{\frac{n(n-1)}{2}}R/a_n$$\\
\end{proposition}

\begin{proposition}
  Up to sign, the discriminant is given by the determinant of a
  $2n\!-\!1$ square matrix with columns 1 to $n\!-\!1$ formed by shifting the
  sequence \,$1,\,a_1,\,\ldots,\,a_n$,\, and columns $n$ to $2n\!-\!1$ formed by
  shifting the sequence\, $n,\,(n\!-\!1)a_1,\;\ldots,\;a_{n-1}$,\, i.e.
 
  {\small
  \begin{equation}
    \label{eq:detform}
    \dn = \left | 
      \begin{array}{ccccccccc}
        1   & 0   & \ldots & 0 & n   & 0   & \ldots & 0 &0\\
        a_1 & 1   & \ldots & 0 & (n-1)\,a_1 & n   & \ldots & 0&0 \\
        \vdots & \vdots & \ddots & \vdots & \vdots & \vdots & \ddots &
        \vdots & \vdots\\ 
        a_{n-2} & a_{n-3} & \ldots & 1 & 2\,a_{n-2} & 3\,a_{n-3} &
        \ldots & n & 0\\   
        a_{n-1}  & a_{n-2} & \ldots & a_1 & a_{n-1} & 2\,a_{n-2} &
        \ldots & (n-1) a_1 & n\\
        a_{n}  & a_{n-1} & \ldots & a_2  & 0 & a_{n-1} & \ldots &
        (n-2) a_2  & (n-1) a_1\\
        \vdots  & \vdots & \ddots & \vdots & \vdots & \vdots & \ddots
        & \vdots & \vdots\\
        0 & 0  & \ldots & a_{n-1}  &0  &0
        &\ldots & a_{n-1}& 2\,a_{n-2} \\
        0 & 0 &  \ldots  & a_n & 0 & 0 &\ldots &0 & a_{n-1} 
      \end{array}\right|    
  \end{equation}}
\end{proposition}

\paragraph{Multiple root test.}  Let $\kfield$ be a field, let $x$ denote an
indeterminate, and let
\[
p = x^n + a_1 x^{n-1} + \ldots + a_{n-1} x + a_n,\quad
a_i\in\kfield
\]
be a monic polynomial over $\kfield$.  We define
$\delta[p]$, the discriminant of $p$, by setting
\[
\delta[p] = \dn\lp a_1,\ldots, a_n\rp.
\]
The discriminant of a non-monic polynomial is defined homogenizing the
above definition, i.e by setting
\[
\delta[a p] = a^{2n-2} \delta[p],\quad a\in\kfield.
\]


\begin{proposition}
%  Suppose that the characteristic of the field is zero, or that
%  $\fchar(\kfield)$ does not divide the degree of $p$.  Then, 
  The discriminant vanishes if and only if $p$ has multiple roots in
  its splitting field.  
\end{proposition}
\begin{proof}
It isn't hard to show that a polynomial has multiple roots if and only
if that polynomial and its derivative share a common root.  The
desired conclusion now follows by observing that
the determinant formula in equation \eqref{eq:detform} gives the
resolvent of a polynomial and its derivative.  This resolvent vanishes
if and only if the polynomial in question has a multiple root.
\end{proof}

\paragraph{Some Examples.}
Here are the first few discriminants.
\begin{align*}
\delta^{(1)}  &=1\\
\delta^{(2)}  &=a_1^2 - 4\,  a_2\\
\delta^{(3)}  &=18\,  a_1 a_2 a_3 + a_1^2 a_2^2 - 4\,  a_2^3 - 4\,
a_1^3 a_3 - 27 a_3^2 \\
\delta^{(4)} &= 
a_1^2a_2^2a_3^2 - 4\,a_2^3a_3^2 - 4\,a_1^3a_3^3 + 
    18\,a_1a_2a_3^3 - 27\,a_3^4\\
& - 4\,a_1^2a_2^3a_4 +    16\,a_2^4a_4 + 
 18\,a_1^3a_2a_3a_4 - 
    80\,a_1a_2^2a_3a_4\\
& - 6\,a_1^2a_3^2a_4 +    144\,a_2a_3^2a_4   - 27\,a_1^4a_4^2 + 
    144\,a_1^2a_2a_4^2 \\
&- 128\,a_2^2a_4^2 - 
    192\,a_1a_3a_4^2 + 256\,a_4^3
\end{align*}

Here is the matrix used to calculate $\delta^{(4)}$:
\[
\delta^{(4)} = 
\left | 
  \begin{array}{ccccccc}
    1   & 0   & 0   & 4    & 0    & 0  & 0 \\
    a_1 & 1   & 0   & 3a_1 & 4    & 0  & 0 \\
    a_2 & a_1 & 1   & 2a_2 & 3a_1 & 4  & 0 \\
    a_3 & a_2 & a_1 &  a_3 & 2a_2 &3a_1& 4 \\
    a_4 & a_3 & a_2 &   0  &  a_3 &2a_2&3a_1\\
      0 & a_4 & a_3 &    0 &    0 & a_3&2a_2\\
      0 &   0 & a_4 &    0 &    0 &   0& a_3\\
    \end{array}\right|
\]

\begin{thebibliography}{99}


\bibitem{marcus} Daniel A. Marcus, {\em Number Fields}, Springer, New York.

\bibitem{silv} Joseph H. Silverman, {\em The Arithmetic of Elliptic Curves}. Springer-Verlag, New York, 1986.

\end{thebibliography}

See also the \PMlinkname{bibliography for number theory}{BibliographyForNumberTheory} and the \PMlinkname{bibliography for algebraic geometry}{BibliographyForAlgebraicGeometry}.
%%%%%
%%%%%
\end{document}
