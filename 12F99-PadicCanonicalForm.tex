\documentclass[12pt]{article}
\usepackage{pmmeta}
\pmcanonicalname{PadicCanonicalForm}
\pmcreated{2013-03-22 14:13:37}
\pmmodified{2013-03-22 14:13:37}
\pmowner{pahio}{2872}
\pmmodifier{pahio}{2872}
\pmtitle{p-adic canonical form}
\pmrecord{66}{35665}
\pmprivacy{1}
\pmauthor{pahio}{2872}
\pmtype{Example}
\pmcomment{trigger rebuild}
\pmclassification{msc}{12F99}
\pmrelated{IntegralElement}
\pmrelated{UltrametricTriangleInequality}
\pmrelated{NonIsomorphicCompletionsOfMathbbQ}
\pmrelated{IdealsOfADiscreteValuationRingArePowersOfItsMaximalIdeal}
\pmdefines{proper p-adic number}
\pmdefines{dyadic number}
\pmdefines{dyadic point}
\pmdefines{2-adic fractional number}
\pmdefines{2-adic integer}
\pmdefines{2-adic valuation}

\endmetadata

% this is the default PlanetMath preamble.  as your knowledge
% of TeX increases, you will probably want to edit this, but
% it should be fine as is for beginners.

% almost certainly you want these
\usepackage{amssymb}
\usepackage{amsmath}
\usepackage{amsfonts}

% used for TeXing text within eps files
%\usepackage{psfrag}
% need this for including graphics (\includegraphics)
%\usepackage{graphicx}
% for neatly defining theorems and propositions
%\usepackage{amsthm}
% making logically defined graphics
%%%\usepackage{xypic}

% there are many more packages, add them here as you need them

% define commands here
\begin{document}
Every non-zero $p$-adic number ($p$ is a positive rational prime number) can be uniquely written in {\em canonical form}, formally as a Laurent series, 
    $$\xi = a_{-m}p^{-m}+a_{-m+1}p^{-m+1}+\cdots+a_0+a_1p+a_2p^2+\cdots$$
where\, $m \in \mathbb{N}$,\, $0 \leqq a_k \leqq p-1$\, for all $k$'s, and at least one of the integers $a_k$ is positive.\, In addition, we can write:\, 
$0 = 0+0p+0p^2+\cdots$

The field $\mathbb {Q}_p$ of the $p$-adic numbers is the completion of the field $\mathbb {Q}$ with respect to its \PMlinkname{$p$-adic valuation}{PAdicValuation}; thus $\mathbb {Q}$ may be thought the subfield (prime subfield) of $\mathbb {Q}_p$.\, We can call the elements of\, $\mathbb{Q}_p\!\setminus\!\mathbb{Q}$\, the {\em proper $p$-adic numbers}.

If, e.g.,\, $p = 2$,\, we have the 2-{\em adic} or, according to G. W. Leibniz, {\em dyadic numbers}, for which every $a_k$ is 0 or 1.\, In this case we can write the sum expression for $\xi$ in the reverse \PMlinkescapetext{order} and use the ordinary \PMlinkname{positional}{Base3} \PMlinkescapetext{binary} (i.e., {\em dyadic}) \PMlinkname{figure system}{Base3}.\, Then, for example, we have the rational numbers
   $$-1 = ...111111,$$
   $$1 = ...0001,$$ 
   $$6.5 = ...000110.1,$$
   $$\frac{1}{5}= ...00110011001101.$$
(You may check the first by adding 1, and the last by multiplying by\, 5 = ...000101.) 
All 2-adic rational numbers have periodic binary \PMlinkname{expansion}{DecimalExpansion}.\, Similarly as the \PMlinkname{decimal}{DecimalExpansion} (according to Leibniz: {\em decadic}) expansions of irrational real numbers are aperiodic, the proper 2-adic numbers also have aperiodic binary expansion, for example the 2-{\em adic fractional number}
   $$\alpha = ...1000010001001011.10111.$$

The {\em 2-adic fractional numbers} have some bits ``1'' after the {\em dyadic point} ``.'' (in continental Europe: comma ``$,$''), the {\em 2-adic integers} have none.\, The 2-adic integers form a subring of the 2-adic field $\mathbb {Q}_2$ such that $\mathbb{Q}_2$ is the quotient field of this ring.

Every such 2-adic integer $\varepsilon$ whose last bit is ``1'', as\, $-3/7 = ...11011011011$, is a unit of this ring, because the division\, $1\colon\!\varepsilon$\, clearly gives as quotient a \PMlinkescapetext{similar} integer (by the way, the divisions of the binary expansions in practice go from right to left and are very comfortable!). 

Those integers ending in a ``0'' are non-units of the ring, and they apparently form the only maximal ideal in the ring (which thus is a local ring).\, This is a principal ideal $\mathfrak{p}$, the generator of which may be taken\, $...00010 = 10$ (i.e., two).\, Indeed, two is the only prime number of the ring, but it has infinitely many associates, a kind of copies, namely all expansions of the form\, $...10 = \varepsilon\cdot 10$.\, The only non-trivial ideals in the ring of 2-adic integers are\, $\mathfrak{p},\, \mathfrak{p}^2,\, \mathfrak{p}^3,\, \ldots$\, They have only 0 as common element.

All 2-adic non-zero integers are of the form $\varepsilon\cdot 2^n$ where\, $n = 0,\,1,\,2,\,\ldots$.\, The values\, $n = -1,\,-2,\,-3,\,\ldots$\, here would give non-integral, i.e. fractional 2-adic numbers.

If in the binary \PMlinkescapetext{representation} of an arbitrary 2-adic number, the last non-zero digit ``1'' corresponds to the power $2^n$, then the {\em 2-adic valuation} of the 2-adic number $\xi$ is given by
    $$|\xi|_2 = 2^{-n}.$$
%%%%%
%%%%%
\end{document}
