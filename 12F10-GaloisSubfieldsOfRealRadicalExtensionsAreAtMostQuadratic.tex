\documentclass[12pt]{article}
\usepackage{pmmeta}
\pmcanonicalname{GaloisSubfieldsOfRealRadicalExtensionsAreAtMostQuadratic}
\pmcreated{2013-03-22 17:43:05}
\pmmodified{2013-03-22 17:43:05}
\pmowner{rm50}{10146}
\pmmodifier{rm50}{10146}
\pmtitle{Galois subfields of real radical extensions are at most quadratic}
\pmrecord{7}{40163}
\pmprivacy{1}
\pmauthor{rm50}{10146}
\pmtype{Theorem}
\pmcomment{trigger rebuild}
\pmclassification{msc}{12F10}
\pmclassification{msc}{12F05}

% this is the default PlanetMath preamble.  as your knowledge
% of TeX increases, you will probably want to edit this, but
% it should be fine as is for beginners.

% almost certainly you want these
\usepackage{amssymb}
\usepackage{amsmath}
\usepackage{amsfonts}

% used for TeXing text within eps files
%\usepackage{psfrag}
% need this for including graphics (\includegraphics)
%\usepackage{graphicx}
% for neatly defining theorems and propositions
\usepackage{amsthm}
% making logically defined graphics
%%\usepackage{xypic}

% there are many more packages, add them here as you need them

% define commands here
\newcommand{\Reals}{\mathbb{R}}
\DeclareMathOperator{\Gal}{Gal}
\newtheorem{thm}{Theorem}

\begin{document}
\begin{thm} Suppose $F\subset L\subset K=F(\sqrt[n]\alpha)\subset\Reals$ are fields with $\alpha\in F$ and $L$ Galois over $F$. Then $[L:F]\leq 2$.
\end{thm}

\textbf{Proof. }
Let $\zeta_n$ be a primitive $n^{\mathrm{th}}$ root of unity, and define $F'=F(\zeta_n)$, $L'=L(\zeta_n)$, and $K'=K(\zeta_n)=F'(\sqrt[n]\alpha)$.
\[\xymatrix @R1pc@C.3pc{
& K'=K(\zeta_n)=F'(\sqrt[n]\alpha) \ar@{-}[dl]\ar@{-}[dr] & & \\
K=F(\sqrt[n]\alpha) \ar@{-}[dr] & & L'=L(\zeta_n) \ar@{-}[dl]\ar@{-}[dr] & \\
& L \ar@{-}[dr] & & F'=F(\zeta_n) \ar@{-}[dl] \\
& & F &
}
\]
Now, $L'/F'$ is Galois since $L/F$ is. But $K'$ is a Kummer extension of $F'$, so has cyclic Galois group and thus $L'/F'$ has cyclic Galois group as well (being a quotient of $\Gal(K'/F')$). Thus $L'$ is a Kummer extension of $F'$, so that $L'=F'(\sqrt[n]{\beta})$ for some $\beta\in F'$. It follows that $L=F(\sqrt[n]{\beta})$. But since $L$ is Galois over $F$, it follows that $n\leq 2$ (since otherwise in order to be Galois, $L$ would have to contain the non-real $n^{\mathrm{th}}$ roots of unity).

%%%%%
%%%%%
\end{document}
