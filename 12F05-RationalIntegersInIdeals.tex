\documentclass[12pt]{article}
\usepackage{pmmeta}
\pmcanonicalname{RationalIntegersInIdeals}
\pmcreated{2013-03-22 19:08:47}
\pmmodified{2013-03-22 19:08:47}
\pmowner{pahio}{2872}
\pmmodifier{pahio}{2872}
\pmtitle{rational integers in ideals}
\pmrecord{6}{42049}
\pmprivacy{1}
\pmauthor{pahio}{2872}
\pmtype{Theorem}
\pmcomment{trigger rebuild}
\pmclassification{msc}{12F05}
\pmclassification{msc}{06B10}
\pmclassification{msc}{11R04}
\pmrelated{CharacteristicPolynomialOfAlgebraicNumber}
\pmrelated{IdealNorm}

\endmetadata

% this is the default PlanetMath preamble.  as your knowledge
% of TeX increases, you will probably want to edit this, but
% it should be fine as is for beginners.

% almost certainly you want these
\usepackage{amssymb}
\usepackage{amsmath}
\usepackage{amsfonts}

% used for TeXing text within eps files
%\usepackage{psfrag}
% need this for including graphics (\includegraphics)
%\usepackage{graphicx}
% for neatly defining theorems and propositions
 \usepackage{amsthm}
% making logically defined graphics
%%%\usepackage{xypic}

% there are many more packages, add them here as you need them

% define commands here

\theoremstyle{definition}
\newtheorem*{thmplain}{Theorem}

\begin{document}
Any non-zero ideal of an algebraic number field $K$, i.e. of the maximal order $\mathcal{O}_K$ of $K$, contains positive rational integers.\\

\emph{Proof.}\, Let\, $\mathfrak{a} \neq (0)$\, be any ideal of $\mathcal{O}_K$.\, Take a nonzero element $\alpha$ of 
$\mathfrak{a}$.\, The \PMlinkname{norm}{NormInNumberField} of $\alpha$ is the product
$$\mbox{N}(\alpha) \;=\; \alpha^{(1)}\underbrace{\alpha^{(2)}\cdots\alpha^{(n)}}_{\gamma}$$
where $n$ is the degree of the number field and $\alpha^{(1)},\,\alpha^{(2)},\,\cdots,\,\alpha^{(n)}$ is the \PMlinkescapetext{complete} set of the \PMlinkid{$K$-conjugates}{12046} of\, $\alpha = \alpha^{(1)}$.\, The number
$$\gamma = \frac{\mbox{N}(\alpha)}{\alpha}$$
belongs to the field $K$ and it is an algebraic integer, since $\alpha^{(2)},\,\cdots,\,\alpha^{(n)}$ are, as algebraic conjugates of $\alpha$, also algebraic integers.\, Thus\, $\gamma \in \mathcal{O}_K$.\, Consequently, the non-zero integer
$$\mbox{N}(\alpha) \;=\; \alpha\gamma$$
belongs to the ideal $\mathfrak{a}$, and similarly its opposite number.\, So, $\mathfrak{a}$ contains positive integers, in fact infinitely many.

%%%%%
%%%%%
\end{document}
