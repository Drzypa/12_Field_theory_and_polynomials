\documentclass[12pt]{article}
\usepackage{pmmeta}
\pmcanonicalname{FieldAdjunction}
\pmcreated{2015-02-21 15:39:45}
\pmmodified{2015-02-21 15:39:45}
\pmowner{pahio}{2872}
\pmmodifier{pahio}{2872}
\pmtitle{field adjunction}
\pmrecord{16}{35875}
\pmprivacy{1}
\pmauthor{pahio}{2872}
\pmtype{Definition}
\pmcomment{trigger rebuild}
\pmclassification{msc}{12F99}
\pmsynonym{simple extension}{FieldAdjunction}
%\pmkeywords{adjunction}
\pmrelated{GroundFieldsAndRings}
\pmrelated{Forcing}
\pmrelated{PolynomialRingOverFieldIsEuclideanDomain}
\pmrelated{AConditionOfAlgebraicExtension}

% this is the default PlanetMath preamble.  as your knowledge
% of TeX increases, you will probably want to edit this, but
% it should be fine as is for beginners.

% almost certainly you want these
\usepackage{amssymb}
\usepackage{amsmath}
\usepackage{amsfonts}

% used for TeXing text within eps files
%\usepackage{psfrag}
% need this for including graphics (\includegraphics)
%\usepackage{graphicx}
% for neatly defining theorems and propositions
 \usepackage{amsthm}
% making logically defined graphics
%%%\usepackage{xypic}

% there are many more packages, add them here as you need them

% define commands here
\theoremstyle{definition}
\newtheorem*{thmplain}{Theorem}
\begin{document}
Let $K$ be a field and $E$ an extension field of $K$.\, If $\alpha 
\in E$, then the smallest subfield of $E$, that contains $K$ and 
$\alpha$, is denoted by  $K(\alpha)$.\, We say that $K(\alpha)$ is 
obtained from the field $K$ by {\em adjoining} the element $\alpha$ 
to $K$ via {\em field adjunction}.\\

\textbf{Theorem.}\, $K(\alpha)$ is identical with the quotient field $Q$ of $K[\alpha]$.

{\it Proof.}  (1)  Because $K[\alpha]$ is an integral domain (as a subring of the field $E$), all possible quotients of the elements of $K[\alpha]$ belong to $E$. So we have
  $$K\cup\{\alpha\} \subseteq K[\alpha] \subseteq Q \subseteq E,$$
and because $K(\alpha)$ was the smallest, then \,$K(\alpha) \subseteq Q.$

(2)  $K(\alpha)$ is a subring of $E$ containing $K$ and $\alpha$, therefore also the whole ring $K[\alpha]$, that is, \,$K[\alpha] \subseteq K(\alpha)$. \,And because $K(\alpha)$ is a field, it must contain all possible quotients of the elements of $K[\alpha]$, i.e., \,$Q \subseteq K(\alpha)$.\\


In \PMlinkescapetext{addition} to the adjunction of one single element, we can adjoin to $K$ an arbitrary subset $S$ of $E$:\, the resulting field $K(S)$ is the smallest of such subfields of $E$, i.e. the intersection of such subfields of $E$, that contain both $K$ and $S$.\, We say that $K(S)$ is obtained from $K$ by adjoining the set $S$ to it.\, Naturally,
                    $$K \subseteq K(S) \subseteq E.$$
The field $K(S)$ contains all elements of $K$ and $S$, and being a field, also all such elements that can be formed via addition, subtraction, multiplication and division from the elements of $K$ and $S$.\, But such elements constitute a field, which therefore must be equal with $K(S)$.\, Accordingly, we have the

\textbf{Theorem.}\, $K(S)$ constitutes of all rational expressions formed of the elements of the field $K$ with the elements of the set $S$.\\


\textbf{Notes.}

1. $K(S)$ is the union of all fields $K(T)$ where $T$ is a finite subset of $S$.\\
2. $K(S_1 \cup S_2) = K(S_1)(S_2)$.\\
3. If, especially, $S$ also is a subfield of $E$, then one may denote\, $K(S) = KS$.

\begin{thebibliography}{9}
\bibitem{vdW}{\sc B. L. van der Waerden:} {\em Algebra. Erster Teil}.\, Siebte Auflage der {\em Modernen Algebra}. Springer-Verlag; Berlin, Heidelberg, New York (1966).
\end{thebibliography}



%%%%%
%%%%%
\end{document}
