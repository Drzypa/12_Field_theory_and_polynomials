\documentclass[12pt]{article}
\usepackage{pmmeta}
\pmcanonicalname{CharacterizationsOfIntegral}
\pmcreated{2013-03-22 14:56:54}
\pmmodified{2013-03-22 14:56:54}
\pmowner{pahio}{2872}
\pmmodifier{pahio}{2872}
\pmtitle{characterizations of integral}
\pmrecord{11}{36642}
\pmprivacy{1}
\pmauthor{pahio}{2872}
\pmtype{Theorem}
\pmcomment{trigger rebuild}
\pmclassification{msc}{12E99}
\pmclassification{msc}{13B21}
\pmsynonym{characterisations of integral}{CharacterizationsOfIntegral}
%\pmkeywords{integral over a ring}

% this is the default PlanetMath preamble.  as your knowledge
% of TeX increases, you will probably want to edit this, but
% it should be fine as is for beginners.

% almost certainly you want these
\usepackage{amssymb}
\usepackage{amsmath}
\usepackage{amsfonts}

% used for TeXing text within eps files
%\usepackage{psfrag}
% need this for including graphics (\includegraphics)
%\usepackage{graphicx}
% for neatly defining theorems and propositions
 \usepackage{amsthm}
% making logically defined graphics
%%%\usepackage{xypic}

% there are many more packages, add them here as you need them

% define commands here
\theoremstyle{definition}
\newtheorem*{thmplain}{Theorem}
\begin{document}
\begin{thmplain}
Let $R$ be a subring of a field $K$,\, $1\in R$\, and let $\alpha$ be a non-zero element of $K$.\, The following conditions are equivalent:
\begin{enumerate}
 \item $\alpha$ is integral over $R$.
 \item $\alpha$ belongs to $R[\alpha^{-1}]$.
 \item $\alpha$ is unit of $R[\alpha^{-1}]$.
 \item $\alpha^{-1}R[\alpha^{-1}] = R[\alpha^{-1}]$.
\end{enumerate}
\end{thmplain}

{\em Proof.} \,Supposing the first condition \PMlinkescapetext{means} that an equation
          $$\alpha^n+a_1\alpha^{n-1}+\ldots+a_{n-1}\alpha+a_n = 0,$$
with $a_j$'s belonging to $R$, holds.\, Dividing both \PMlinkescapetext{sides} by $\alpha^{n-1}$ gives 
          $$\alpha = -a_1-a_2\alpha^{-1}-\ldots-a_n\alpha^{-n+1}.$$
One sees that $\alpha$ belongs to the ring $R[\alpha^{-1}]$ even being a unit of this (of course \,$\alpha^{-1}\in R[\alpha^{-1}]$).\, Therefore also the principal ideal $\alpha^{-1}R[\alpha^{-1}]$ of the ring $R[\alpha^{-1}]$ coincides with this ring.\, Conversely, the last circumstance implies that $\alpha$ is integral over $R$.

\begin{thebibliography}{7}
\bibitem{artin} Emil Artin: {\em \PMlinkescapetext{Theory of Algebraic Numbers}}.\, Lecture notes.\, Mathematisches Institut, G\"ottingen (1959).
\end{thebibliography}
%%%%%
%%%%%
\end{document}
