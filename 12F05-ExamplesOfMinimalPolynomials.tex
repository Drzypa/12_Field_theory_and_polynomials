\documentclass[12pt]{article}
\usepackage{pmmeta}
\pmcanonicalname{ExamplesOfMinimalPolynomials}
\pmcreated{2013-03-22 16:55:18}
\pmmodified{2013-03-22 16:55:18}
\pmowner{Wkbj79}{1863}
\pmmodifier{Wkbj79}{1863}
\pmtitle{examples of minimal polynomials}
\pmrecord{13}{39184}
\pmprivacy{1}
\pmauthor{Wkbj79}{1863}
\pmtype{Example}
\pmcomment{trigger rebuild}
\pmclassification{msc}{12F05}
\pmclassification{msc}{12E05}
\pmrelated{IrreduciblePolynomialsObtainedFromBiquadraticFields}

\endmetadata

\usepackage{amssymb}
\usepackage{amsmath}
\usepackage{amsfonts}

\usepackage{psfrag}
\usepackage{graphicx}
\usepackage{amsthm}
%%\usepackage{xypic}

\begin{document}
Note that $\sqrt[4]{2}$ is algebraic over the fields $\mathbb{Q}$ and $\mathbb{Q}(\sqrt{2})$.  The minimal polynomials for $\sqrt[4]{2}$ over these fields are $x^4-2$ and $x^2-\sqrt{2}$, respectively.  Note that $x^4-2$ is irreducible over $\mathbb{Q}$ by using Eisenstein's criterion and \PMlinkname{Gauss's lemma}{GausssLemmaII} (see \PMlinkname{this entry}{AlternativeProofThatSqrt2IsIrrational} for more details), and $x^2-\sqrt{2}$ is irreducible over $\mathbb{Q}(\sqrt{2})$ since it is a quadratic polynomial and neither of its roots ($\sqrt[4]{2}$ and $-\sqrt[4]{2}$) are in $\mathbb{Q}(\sqrt{2})$.

A common method for constructing minimal polynomials for numbers that are expressible over $\mathbb{Q}$ is ``backwards \PMlinkescapetext{algebra}'':  The number can be set equal to $x$, and the equation can be algebraically manipulated until a monic polynomial in $\mathbb{Q}[x]$ is equal to 0.  Finally, if the monic polynomial is not irreducible, then it can be factored into irreducible polynomials $\mathbb{Q}[x]$, and the original number will be a root of one of these.  A very \PMlinkescapetext{simple} example is $\sqrt[4]{2}$:

\begin{center}
$\begin{array}{rl}
x & =\sqrt[4]{2} \\
x^4 & =2 \\
x^4-2 & =0 \end{array}$
\end{center}

This method will be further demonstrated with three more examples:  One for $\displaystyle \frac{1+\sqrt{5}}{2}$, one for $1+\omega_5$ where $\omega_5$ is a fifth root of unity, and one for $\sqrt[3]{2}+\sqrt[3]{3}$.

\begin{center}
$\begin{array}{rl}
x & =\displaystyle \frac{1+\sqrt{5}}{2} \\
2x & =1+\sqrt{5} \\
2x-1 & =\sqrt{5} \\
(2x-1)^2 & =5 \\
4x^2-4x+1 & =5 \\
4x^2-4x-4 & =0 \\
x^2-x-1 & =0 \end{array}$

$\begin{array}{rl}
x & =1+\omega_5 \\
x-1 & =\omega_5 \\
(x-1)^5 & =1 \\
x^5-5x^4+10x^3-10x^2+5x-1 & =1 \\
x^5-5x^4+10x^3-10x^2+5x-2 & =0 \end{array}$

$\begin{array}{rl}
x & =\sqrt[3]{2}+\sqrt[3]{3} \\
x^3 & =2+3\sqrt[3]{2^2 \cdot 3}+3\sqrt[3]{2 \cdot 3^2}+3 \\
x^3-5 & =3\sqrt[3]{6}(\sqrt[3]{2}+\sqrt[3]{3}) \\
x^3-5 & =3\sqrt[3]{6} \, x \\
(x^3-5)^3 & =27 \cdot 6x^3 \\
x^9-3 \cdot 5x^6+3 \cdot 25x^3-125 & =162x^3 \\
x^9-15x^6-87x^3-125 & =0 \end{array}$
\end{center}

Since $x^2-x-1$ is a quadratic and has no roots in $\mathbb{Q}$, it is irreducible over $\mathbb{Q}$.  Thus, it is the minimal polynomial over $\mathbb{Q}$ for $\displaystyle \frac{1+\sqrt{5}}{2}$.

On the other hand, $x^5-5x^4+10x^3-10x^2+5x-2$ factors over $\mathbb{Q}$ as $(x-2)(x^4-3x^3+4x^2-2x+1)$.  Since $1+\omega_5$ is not a root of $x-2$, it must be a root of $x^4-3x^3+4x^2-2x+1$.  Moreover, this polynomial must be irreducible.  This fact can be proven in the following manner:  Let $m(x)$ be the minimal polynomial for $1+\omega_5$ over $\mathbb{Q}$.  Since $\mathbb{Q}(1+\omega_5)=\mathbb{Q}(\omega_5)$, $\deg m(x)=[\mathbb{Q}(1+\omega_5)\!:\!\mathbb{Q}]=[\mathbb{Q}(\omega_5)\!:\!\mathbb{Q}]=\varphi(5)=4=\deg (x^4-3x^3+4x^2-2x+1)$.  (Here $\varphi$ denotes the Euler totient function.)  Since $m(x)$ divides $x^4-3x^3+4x^2-2x+1$ and they have the same degree, it follows that $m(x)=x^4-3x^3+4x^2-2x+1$.

It turns out that $x^9-15x^6-87x^3-125$ is irreducible over $\mathbb{Q}$.  (This can be proven in a \PMlinkescapetext{similar} manner as above.  Note that $[\mathbb{Q}(\sqrt[3]{2}+\sqrt[3]{3})\!:\!\mathbb{Q}]=9$.)  Thus, it is the minimal polynomial over $\mathbb{Q}$ for $\sqrt[3]{2}+\sqrt[3]{3}$.
%%%%%
%%%%%
\end{document}
