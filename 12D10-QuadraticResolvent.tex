\documentclass[12pt]{article}
\usepackage{pmmeta}
\pmcanonicalname{QuadraticResolvent}
\pmcreated{2014-11-27 16:05:30}
\pmmodified{2014-11-27 16:05:30}
\pmowner{pahio}{2872}
\pmmodifier{pahio}{2872}
\pmtitle{quadratic resolvent}
\pmrecord{9}{40338}
\pmprivacy{1}
\pmauthor{pahio}{2872}
\pmtype{Definition}
\pmcomment{trigger rebuild}
\pmclassification{msc}{12D10}
\pmsynonym{quadratic resolvent equation}{QuadraticResolvent}
%\pmkeywords{resolvent}
\pmrelated{CardanosFormulae}
\pmrelated{TchirnhausTransformations}
\pmdefines{cubic resolvent}
\pmdefines{resolvent equation}

% this is the default PlanetMath preamble.  as your knowledge
% of TeX increases, you will probably want to edit this, but
% it should be fine as is for beginners.

% almost certainly you want these
\usepackage{amssymb}
\usepackage{amsmath}
\usepackage{amsfonts}

% used for TeXing text within eps files
%\usepackage{psfrag}
% need this for including graphics (\includegraphics)
%\usepackage{graphicx}
% for neatly defining theorems and propositions
 \usepackage{amsthm}
% making logically defined graphics
%%%\usepackage{xypic}

% there are many more packages, add them here as you need them

% define commands here

\theoremstyle{definition}
\newtheorem*{thmplain}{Theorem}

\begin{document}
The {\em quadratic resolvent} of the cubic equation
\begin{align}
y^3+py+q = 0,
\end{align}
where $p$ and $q$ are known complex numbers, is the auxiliary equation
$$z^2+qz-\left(\frac{p}{3}\right)^3 = 0$$
determining as its \PMlinkname{roots}{Equation}
$$z_1 = u^3, \qquad z_2 = v^3$$
the numbers $u$ and $v$ whose sum \,$y = u+v$\, satisfies the
equation (1).\, See 
\PMlinkname{example of solving a cubic equation}{exampleofsolvingacubicequation}.

Analogically, a quartic equation has a {\em cubic resolvent} (resolvent cubic) equation.
%%%%%
%%%%%
\end{document}
