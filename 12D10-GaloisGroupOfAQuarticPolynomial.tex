\documentclass[12pt]{article}
\usepackage{pmmeta}
\pmcanonicalname{GaloisGroupOfAQuarticPolynomial}
\pmcreated{2013-03-22 17:41:35}
\pmmodified{2013-03-22 17:41:35}
\pmowner{rm50}{10146}
\pmmodifier{rm50}{10146}
\pmtitle{Galois group of a quartic polynomial}
\pmrecord{7}{40134}
\pmprivacy{1}
\pmauthor{rm50}{10146}
\pmtype{Topic}
\pmcomment{trigger rebuild}
\pmclassification{msc}{12D10}

\endmetadata

% this is the default PlanetMath preamble.  as your knowledge
% of TeX increases, you will probably want to edit this, but
% it should be fine as is for beginners.

% almost certainly you want these
\usepackage{amssymb}
\usepackage{amsmath}
\usepackage{amsfonts}

% used for TeXing text within eps files
%\usepackage{psfrag}
% need this for including graphics (\includegraphics)
%\usepackage{graphicx}
% for neatly defining theorems and propositions
%\usepackage{amsthm}
% making logically defined graphics
%%%\usepackage{xypic}

% there are many more packages, add them here as you need them

% define commands here
\newcommand{\Order}[1]{\lvert #1 \rvert}
\newcommand{\Rats}{\mathbb{Q}}
\newcommand{\Ints}{\mathbb{Z}}
\newcommand{\subgroup}{\leq}
\begin{document}
\PMlinkescapeword{addition}
\PMlinkescapeword{class}
Consider a general (monic) quartic polynomial over $\Rats$
\[f(x)=x^4 + ax^3+bx^2+cx+d\]
and denote the Galois group of $f(x)$ by $G$.

The Galois group $G$ is isomorphic to a subgroup of $S_4$ (see the article on the Galois group of a cubic polynomial for a discussion of this question). 

If the quartic splits into a linear factor and an irreducible cubic, then its Galois group is simply the Galois group of the cubic portion and thus is isomorphic to a subgroup of $S_3$ (embedded in $S_4$) - again, see the article on the Galois group of a cubic polynomial.

If it factors as two irreducible quadratics, then the splitting field of $f(x)$ is the compositum of $\Rats(\sqrt{D_1})$ and $\Rats(\sqrt{D_2})$, where $D_1$ and $D_2$ are the discriminants of the two quadratics. This is either a biquadratic extension and thus has Galois group isomorphic to $V_4$, or else $D_1D_2$ is a square, and $\Rats(\sqrt{D_1},\sqrt{D_2})=\Rats(\sqrt{D_1})$ and the Galois group  is isomorphic to $\Ints/2\Ints$.

This leaves us with the most interesting case, where $f(x)$ is irreducible. In this case, the Galois group acts transitively on the roots of $f(x)$, so it must be isomorphic to a \PMlinkname{transitive}{GroupAction} subgroup of $S_4$. The transitive subgroups of $S_4$ are
\begin{align*}
S_4&\\
A_4&\\
D_8&\cong\{e,\ (1234),\ (13)(24),\ (1432),\ (12)(34),\ (14)(23),\ (13),\ (24)\}\text{ and its conjugates}\\
V_4&\cong\{e,\ (12)(34),\ (13)(24),\ (14)(23)\}\\
\Ints/4\Ints &\cong \{e,\ (1234),\ (13)(24),\ (1432)\}\text{ and its conjugates}
\end{align*}
We will see that each of these transitive subgroups actually appears as the Galois group of some class of irreducible quartics.

The resolvent cubic of $f(x)$ is
\[C(x) = x^3 - 2b x^2 + (b^2+ac-4d) x + (c^2+a^2d - abc)\]
and has roots
\begin{gather*}
r_1=(\alpha_1+\alpha_2)(\alpha_3+\alpha_4)\\
r_2=(\alpha_1+\alpha_3)(\alpha_2+\alpha_4)\\
r_3=(\alpha_1+\alpha_4)(\alpha_2+\alpha_3)
\end{gather*}
But then a short computation shows that the discriminant $D$ of $C(x)$ is the same as the discriminant of $f(x)$. Also, since $r_1,r_2,r_3\in\Rats(\alpha_1,\alpha_2,\alpha_3,\alpha_4)$, it follows that the splitting field of $C(x)$ is a subfield of the splitting field of $f(x)$ and thus that the Galois group of $C(x)$ is a quotient of the Galois group of $f(x)$. There are four cases:

\begin{itemize}

\item If $C(x)$ is irreducible, and $D$ is not a rational square, then $G$ does not fix $D$ and thus is not contained in $A_4$. But in this case, where $D$ is not a square, the Galois group of $C(x)$ is $S_3$, which has order $6$. The only subgroup of $S_4$ not contained in $A_4$ with order a multiple of $6$ (and thus capable of having a subgroup of index $6$) is $S_4$ itself, so in this case $G\cong S_4$.

\item If $C(x)$ is irreducible but $D$ is a rational square, then $G$ fixes $D$, so $G\subgroup A_4$. In addition, the Galois group of $C(x)$ is $A_3$, so $3$ divides the order of a transitive subgroup of $A_4$, which means that $G\cong A_4$ itself.

\item If $C(x)$ is reducible, suppose first that it splits completely in $\Rats$. Then each of $r_1,r_2,r_3\in \Rats$ and thus each element of $G$ fixes each $r_i$. Thus $G\cong V_4$.

\item Finally, if $C(x)$ splits into a linear factor and an irreducible quadratic, then one of the $r_i$, say $r_2$, is in $\Rats$. Then $G$ fixes $r_2=(\alpha_1+\alpha_3)(\alpha_2+\alpha_4)$ but not $r_1$ or $r_3$. The only possibilities from among the transitive groups are then that $G\cong D_8$ or $G\cong \Ints/4\Ints$. In this case, the discriminant of the quadratic is not a rational square, but it is a rational square times $D$.

Now, $G\cap A_4$ fixes $\Rats(\sqrt{D})$, since $G$ fixes $\sqrt{D}$ up to sign and $A_4$ restricts our attention to even permutations. But $\Order{G:G\cap A_4}=2$, so the fixed field of $G\cap A_4$ has dimension $2$ over $\Rats$ and thus is exactly $\Rats(\sqrt{D})$. If $G\cong D_8$, then $G\cap A_4\cong V_4$, while if $G\cong \Ints/4\Ints$, then $G\cap A_4\cong \Ints/2\Ints$; in the first case only, $G\cap A_4$ acts transitively on the roots of $f(x)$. Thus $G\cap A_4\cong V_4$ if and only if $f(x)$ is irreducible over $\Rats(\sqrt{D})$.
\end{itemize}

So, in summary, for $f(x)$ irreducible, we have the following:
\begin{center}
\begin{tabular}{lc}
Condition & Galois group\\
\hline
$C(x)$ irreducible, $D$ not a rational square & $S_4$\\
$C(x)$ irreducible, $D$ a rational square & $A_4$\\
$C(x)$ splits completely & $V_4$\\
$C(x)$ factors as linear times irreducible quadratic, $f(x)$ irreducible over $\Rats(\sqrt{D})$ & $D_8$\\
$C(x)$ factors as linear times irreducible quadratic, $f(x)$ reducible over $\Rats(\sqrt{D})$ & $\Ints/4\Ints$
\end{tabular}
\end{center}

\begin{thebibliography}{10}
\bibitem{dummit}
D.S.~Dummit, R.M.~Foote, \emph{Abstract Algebra}, Wiley and Sons, 2004.
\end{thebibliography}
%%%%%
%%%%%
\end{document}
