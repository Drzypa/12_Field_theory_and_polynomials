\documentclass[12pt]{article}
\usepackage{pmmeta}
\pmcanonicalname{PrimitiveElementTheorem}
\pmcreated{2013-03-22 11:45:48}
\pmmodified{2013-03-22 11:45:48}
\pmowner{alozano}{2414}
\pmmodifier{alozano}{2414}
\pmtitle{primitive element theorem}
\pmrecord{18}{30214}
\pmprivacy{1}
\pmauthor{alozano}{2414}
\pmtype{Theorem}
\pmcomment{trigger rebuild}
\pmclassification{msc}{12F05}
\pmclassification{msc}{65-01}
%\pmkeywords{number theory}
\pmrelated{SimpleFieldExtension}
\pmrelated{PrimitiveElementOfBiquadraticField2}
\pmrelated{PrimitiveElementOfBiquadraticField}

% this is the default PlanetMath preamble.  as your knowledge
% of TeX increases, you will probably want to edit this, but
% it should be fine as is for beginners.

% almost certainly you want these
\usepackage{amssymb}
\usepackage{amsmath}
\usepackage{amsfonts}

% used for TeXing text within eps files
%\usepackage{psfrag}
% need this for including graphics (\includegraphics)
%\usepackage{graphicx}
% for neatly defining theorems and propositions
\usepackage{amsthm}
% making logically defined graphics
%%%%%\usepackage{xypic}

% there are many more packages, add them here as you need them

% define commands here

\newtheorem{theorem}{Theorem}
\newtheorem{defn}{Definition}
\newtheorem{prop}{Proposition}
\newtheorem{lemma}{Lemma}
\newtheorem{cor}{Corollary}
\begin{document}
\begin{theorem}
Let $F$ and $K$ be arbitrary fields, and let $K$ be an extension of $F$ of finite degree.  Then there exists an element $\alpha\in K$ such that $K=F(\alpha)$ if and only if there are finitely many fields $L$ with $F\subseteq L\subseteq K$.
\end{theorem}

Note that this implies that every finite separable extension is not only finitely generated, it is generated by a single element.  

Let $X$ be an indeterminate.  Then $\mathbb{Q}(X,i)$ is not generated over $\mathbb{Q}$ by a single element (and there are infinitely many intermediate fields $\mathbb{Q}(X,i)/L/\mathbb{Q}$).  To see this, suppose it is generated by an element $\alpha$. Then clearly $\alpha$ must be transcendental, or it would generate an extension of finite degree.  But if $\alpha$ is transcendental, we know it is isomorphic to $\mathbb{Q}(X)$, and this field is not isomorphic to $\mathbb{Q}(X,i)$: for example, the polynomial $Y^2+1$ has no roots in the first but it has two roots in the second.  It is also clear that it is not sufficient for every element of $K$ to be algebraic over $F$: we know that the algebraic closure of $\mathbb{Q}$ has infinite degree over $\mathbb{Q}$, but if $\alpha$ is algebraic over $\mathbb{Q}$ then $[\mathbb{Q}(\alpha):\mathbb{Q}]$ will be finite.  

This theorem has the corollary:
\begin{cor}
Let $F$ be a field, and let $[F(\beta,\gamma):F]$ be finite and separable.  Then there exists $\alpha \in F(\beta,\gamma)$ such that $F(\beta,\gamma)=F(\alpha)$. In fact, we can always take $\alpha$ to be an $F$-\PMlinkname{linear combination}{LinearCombination} of $\beta$ and $\gamma$. 
\end{cor}

To see this (in the case of characteristic $0$), we need only show that there are finitely many intermediate fields.  But any intermediate field is contained in the splitting field of the minimal polynomials of $\beta$ and $\gamma$, which is Galois with finite Galois group. The explicit form of $\alpha$ comes from the proof of the theorem.

For more detail on this theorem and its proof see, for example, \emph{Field and Galois Theory}, by Patrick Morandi (Springer Graduate Texts in Mathematics 167, 1996).
%%%%%
%%%%%
%%%%%
%%%%%
\end{document}
