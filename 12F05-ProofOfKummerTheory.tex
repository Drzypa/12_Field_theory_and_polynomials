\documentclass[12pt]{article}
\usepackage{pmmeta}
\pmcanonicalname{ProofOfKummerTheory}
\pmcreated{2013-03-22 18:42:07}
\pmmodified{2013-03-22 18:42:07}
\pmowner{rm50}{10146}
\pmmodifier{rm50}{10146}
\pmtitle{proof of Kummer theory}
\pmrecord{5}{41464}
\pmprivacy{1}
\pmauthor{rm50}{10146}
\pmtype{Proof}
\pmcomment{trigger rebuild}
\pmclassification{msc}{12F05}

% this is the default PlanetMath preamble.  as your knowledge
% of TeX increases, you will probably want to edit this, but
% it should be fine as is for beginners.

% almost certainly you want these
\usepackage{amssymb}
\usepackage{amsmath}
\usepackage{amsfonts}

% used for TeXing text within eps files
%\usepackage{psfrag}
% need this for including graphics (\includegraphics)
%\usepackage{graphicx}
% for neatly defining theorems and propositions
\usepackage{amsthm}
% making logically defined graphics
%%%\usepackage{xypic}

% there are many more packages, add them here as you need them

% define commands here
\DeclareMathOperator{\Gal}{Gal}
\DeclareMathOperator{\N}{N}
%
%% \theoremstyle{plain} %% This is the default
\newtheorem{thm}{Theorem}
\newtheorem{cor}[thm]{Corollary}
\newtheorem{lem}[thm]{Lemma}
\newtheorem{prop}[thm]{Proposition}

\begin{document}
\begin{proof} Let $\zeta\in K$ be a primitive $n^{\mathrm{th}}$ root of unity, and denote by $\boldsymbol{\mu}_n$ the subgroup of $K^{\star}$ generated by $\zeta$.

(1) Let $L=K(\sqrt[n]{a})$; then $L/K$ is Galois since $K$ contains all $n^{\mathrm{th}}$ roots of unity and thus is a splitting field for $x^n-a$, which is separable since $n\neq 0$ in $K$. Thus the elements of $\Gal(L/K)$ permute the roots of $x^n-a$, which are
\[\sqrt[n]{a},\ \zeta \sqrt[n]{a},\ \zeta^2 \sqrt[n]{a},\ \dotsc,\ \zeta^{n-1}\sqrt[n]{a}\]
and thus for $\sigma\in \Gal(L/K)$, we have $\sigma(\sqrt[n]{a}) = \zeta_{\sigma} \sqrt[n]{a}$ for some $\zeta_{\sigma}\in\boldsymbol{\mu}_n$.
Define a map
\[p:\Gal(L/K)\to \boldsymbol{\mu}_n:\sigma\mapsto\zeta_{\sigma}\]
We will show that $p$ is an injective homomorphism, which proves the result.

Since $\boldsymbol{\mu}_n\subset K$, each $n^{\mathrm{th}}$ root of unity is fixed by $\Gal(L/K)$. Then for $\sigma,\tau\in\Gal(L/K)$,
\[
  \zeta_{\sigma\tau}\sqrt[n]{a} =\sigma\tau(\sqrt[n]{a}) = \sigma(\zeta_{\tau}\sqrt[n]{a}) 
                          = \zeta_{\tau}(\sigma(\sqrt[n]{a}))
                          = \zeta_{\sigma}\zeta_{\tau}\sqrt[n]{a}
\]
so that $\zeta_{\sigma\tau} = \zeta_{\sigma}\zeta_{\tau}$ and $p$ is a homomorphism. The kernel of the map consists of all elements of $\Gal(L/K)$ which fix $\sqrt[n]{a}$, so that $p$ is injective and we are done.

(2) Note that $\N_{L/K}(\zeta) = 1$ since $\zeta$ is a root of $x^n-1$, so that by Hilbert's Theorem 90, 
\[\zeta = \sigma(u)/u,\quad\text{for some }u\in L\]
But then $\sigma(u) = \zeta u$ so that $\sigma(u^n) = \sigma(u)^n = \zeta^n u^n = u^n$ and $a=u^n\in K$ since it is fixed by a generator of $\Gal(L/K)$. Then clearly $K(u)$ is a splitting field of $x^n-a$, and the elements of $\Gal(L/K)$ send $u$ into distinct elements of $K(u)$. Thus $K(u)$ admits at least $n$ automorphisms over $K$, so that $[K(u):K]\geq n = [L:K]$. But $K(u)\subset L$, so $K(\sqrt[n]{a})=K(u)=L$.
\end{proof}

\begin{thebibliography}{10}
\bibitem{bib:df}
Dummit,~D.,~Foote,~R.M., \emph{Abstract Algebra, Third Edition}, Wiley, 2004.
\bibitem{bib:kap}
Kaplansky,~I., \emph{Fields and Rings}, University of Chicago Press, 1969.
\end{thebibliography}
%%%%%
%%%%%
\end{document}
