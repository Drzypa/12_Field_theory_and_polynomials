\documentclass[12pt]{article}
\usepackage{pmmeta}
\pmcanonicalname{ExampleOfNonperfectField}
\pmcreated{2013-03-22 13:08:31}
\pmmodified{2013-03-22 13:08:31}
\pmowner{CWoo}{3771}
\pmmodifier{CWoo}{3771}
\pmtitle{example of nonperfect field}
\pmrecord{7}{33580}
\pmprivacy{1}
\pmauthor{CWoo}{3771}
\pmtype{Example}
\pmcomment{trigger rebuild}
\pmclassification{msc}{12F10}

% this is the default PlanetMath preamble.  as your knowledge
% of TeX increases, you will probably want to edit this, but
% it should be fine as is for beginners.

% almost certainly you want these
\usepackage{amssymb}
\usepackage{amsmath}
\usepackage{amsfonts}

% used for TeXing text within eps files
%\usepackage{psfrag}
% need this for including graphics (\includegraphics)
%\usepackage{graphicx}
% for neatly defining theorems and propositions
%\usepackage{amsthm}
% making logically defined graphics
%%%\usepackage{xypic}

% there are many more packages, add them here as you need them

% define commands here
\begin{document}
In this entry, we exhibit an example of a field that is not a perfect field.

Let $F = \mathbb{F}_p(t)$, where $\mathbb{F}_p$ is the field with $p$ elements and $t$ transcendental over $\mathbb{F}_p$. The splitting field $E$ of the irreducible polynomial $f = x^p - t$ is not separable over $F$. Indeed, if $\alpha$ is an element of $E$ such that $\alpha^p = t$, we have
\[ x^p - t = x^p - \alpha^p = (x-\alpha)^p, \]
which shows that $f$ has one root of multiplicity $p$.
%%%%%
%%%%%
\end{document}
