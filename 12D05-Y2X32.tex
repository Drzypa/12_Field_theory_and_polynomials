\documentclass[12pt]{article}
\usepackage{pmmeta}
\pmcanonicalname{Y2X32}
\pmcreated{2013-03-22 14:52:05}
\pmmodified{2013-03-22 14:52:05}
\pmowner{CWoo}{3771}
\pmmodifier{CWoo}{3771}
\pmtitle{$y^2= x^3-2$}
\pmrecord{10}{36544}
\pmprivacy{1}
\pmauthor{CWoo}{3771}
\pmtype{Application}
\pmcomment{trigger rebuild}
\pmclassification{msc}{12D05}
\pmclassification{msc}{11R04}
\pmsynonym{$y^2+2=x^3$}{Y2X32}
\pmsynonym{finding integer solutions to $y^2+2=x^3$}{Y2X32}
\pmrelated{UFD}

\usepackage{graphicx}
%%%\usepackage{xypic} 
\usepackage{bbm}
\newcommand{\Z}{\mathbbmss{Z}}
\newcommand{\C}{\mathbbmss{C}}
\newcommand{\R}{\mathbbmss{R}}
\newcommand{\Q}{\mathbbmss{Q}}
\newcommand{\mathbb}[1]{\mathbbmss{#1}}
\newcommand{\figura}[1]{\begin{center}\includegraphics{#1}\end{center}}
\newcommand{\figuraex}[2]{\begin{center}\includegraphics[#2]{#1}\end{center}}
\newtheorem{dfn}{Definition}
\begin{document}
We want to solve the equation $y^2=x^3 - 2$ over the integers.

By writing $y^2+2=x^3$ we can factor on $\Z[\sqrt{-2}]$ as
\[(y-i\sqrt{2})(y+i\sqrt{2})=x^3.
\]

Using congruences modulo $8$, one can show that both $x,y$ must be odd, and it can also be shown that $(y-i\sqrt{2})$ and $(y+i\sqrt{2})$ are relatively prime (if it were not the case, any divisor would have even norm, which is not possible).

Therefore, by unique factorization, and using that the only \PMlinkname{units}{UnitsOfQuadraticFields} on $\Z[\sqrt{-2}]$ are $1,-1$, we have that each factor must be a cube.

So let us write
\[
(y+i\sqrt{2}) = (a+bi\sqrt{2})^3 = (a^3 - 6ab^2) + i(3a^2b-2b^3)\sqrt{2}
\]

Then $y=a^3 - 6ab^2$ and $1=3a^2b-2b^3=b(3a^2-2b^2)$. These two equations imply $b=\pm 1$ and thus $a=\pm 1$, from where the only possible solutions are $x=3, y=\pm 5$.

\begin{thebibliography}{9}
\bibitem{esm}
  Esmonde, Ram Murty; \emph{Problems in Algebraic Number Theory}. Springer.
\end{thebibliography}
%%%%%
%%%%%
\end{document}
