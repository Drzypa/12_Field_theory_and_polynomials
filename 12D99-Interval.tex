\documentclass[12pt]{article}
\usepackage{pmmeta}
\pmcanonicalname{Interval}
\pmcreated{2013-03-22 13:44:58}
\pmmodified{2013-03-22 13:44:58}
\pmowner{PrimeFan}{13766}
\pmmodifier{PrimeFan}{13766}
\pmtitle{interval}
\pmrecord{16}{34446}
\pmprivacy{1}
\pmauthor{PrimeFan}{13766}
\pmtype{Definition}
\pmcomment{trigger rebuild}
\pmclassification{msc}{12D99}
\pmclassification{msc}{26-00}
\pmclassification{msc}{54C30}
\pmrelated{OpenSetsInMathbbRnContainsAnOpenRectangle}
\pmrelated{LineSegment}
\pmrelated{CircularSegment}
\pmdefines{open interval}
\pmdefines{closed interval}
\pmdefines{half-open interval}
\pmdefines{right half-open interval}
\pmdefines{left-half-open interval}
\pmdefines{segment}

% this is the default PlanetMath preamble.  as your knowledge
% of TeX increases, you will probably want to edit this, but
% it should be fine as is for beginners.

% almost certainly you want these
\usepackage{amssymb}
\usepackage{amsmath}
\usepackage{amsfonts}

% used for TeXing text within eps files
%\usepackage{psfrag}
% need this for including graphics (\includegraphics)
\usepackage{graphicx}
% for neatly defining theorems and propositions
%\usepackage{amsthm}
% making logically defined graphics
%%%\usepackage{xypic}

% there are many more packages, add them here as you need them

% define commands here

\newcommand{\sR}[0]{\mathbb{R}}
\newcommand{\sC}[0]{\mathbb{C}}
\newcommand{\sN}[0]{\mathbb{N}}
\newcommand{\sZ}[0]{\mathbb{Z}}

\newcommand*{\norm}[1]{\lVert #1 \rVert}
\newcommand*{\abs}[1]{| #1 |}
\begin{document}
Loosely speaking, an \emph{interval} is a part of the real numbers
that start at one number and stops at another number. For instance, 
all numbers greater that $1$ and smaller than $2$ form in interval.
Another interval is formed by numbers greater or equal to $1$ and 
smaller than $2$. Thus, when talking about intervals, it is necessary
to specify whether the endpoints are part of the interval or not. 
There are then four types of intervals with three different names: 
open, closed and half-open. 
Let us next define these precisely. 

\begin{enumerate}
\item The open interval contains neither of the endpoints. 
If $a$ and $b$ are real numbers, then the \emph{open interval} of 
numbers between $a$ and $b$ is written as $(a,b)$ and
$$(a,b)=\{ x\in \sR \mid a<x<b\}.$$ 
\item The closed interval contains both endpoints.
If $a$ and $b$ are real numbers, then the \emph{closed interval} 
is written as $[a,b]$ and
$$[a,b]=\{ x\in \sR \mid a\le x\le b\}.$$ 
\item A half-open interval contains only one of 
the endpoints. 
If $a$ and $b$ are real numbers, the \emph{half-open intervals}
$(a,b]$ and $[a,b)$ are defined as
\begin{eqnarray*}
 (a,b] &=& \{ x\in \sR \mid a<x\le b\},\\
 \,\![a,b) &=&  \{ x\in \sR \mid a\le x< b\}. 
\end{eqnarray*}
\end{enumerate}

Note that this definition includes the empty set as an interval by, for example, taking the interval $(a,a)$ for any $a$.

An interval is a subset $S$ of a totally ordered set $T$ with the property that whenever $x$ and $y$ are in $S$ and $x < z < y$ then $z$ is in $S$. Applied to the real numbers, this encompasses open, closed, half-open, half-infinite, infinite, empty, and one-point intervals. All the various different types of interval in $\mathbb{R}$ have this in common. Intervals in $\mathbb{R}$ are connected under the usual topology.

There is a standard way of graphically representing intervals
on the real line using filled and empty circles. This is illustrated in
the below figures:

\begin{figure}[!htb]
\begin{center}
\includegraphics{intervals.1.eps}
\end{center}
%\caption{..}
\end{figure}

\begin{figure}[!htb]
\begin{center}
\includegraphics{intervals.2.eps}
\end{center}
%\caption{..}
\end{figure}

\begin{figure}[!htb]
\begin{center}
\includegraphics{intervals.3.eps}
\end{center}
%\caption{..}
\end{figure}

\begin{figure}[!htb]
\begin{center}
\includegraphics{intervals.4.eps}
\end{center}
%\caption{..}
\end{figure}

The logic is here that a empty circle represent a point not belonging 
to the interval, while a filled circle represents a point belonging
to the interval. For example, the first interval is an open interval.

\subsubsection*{Infinite intervals}
If we allow either (or both) of $a$ and $b$ to be infinite, then we
define
\begin{eqnarray*}
(a,\infty) &=& \{ x\in \sR \mid x> a\}, \\
\,\![a,\infty) &=& \{ x\in \sR \mid x\ge  a\}, \\
(-\infty, a) &=& \{ x\in \sR \mid x< a\}, \\
(-\infty, a] &=& \{ x\in \sR \mid x\le  a\}, \\
(-\infty, \infty) &=& \sR.
\end{eqnarray*}

The graphical representation of infinite intervals is as follows:

\begin{figure}[!htb]
\begin{center}
\includegraphics{intervals.5.eps}
\end{center}
%\caption{..}
\end{figure}


\begin{figure}[!htb]
\begin{center}
\includegraphics{intervals.6.eps}
\end{center}
%\caption{..}
\end{figure}

\begin{figure}[!htb]
\begin{center}
\includegraphics{intervals.7.eps}
\end{center}
%\caption{..}
\end{figure}

\begin{figure}[!htb]
\begin{center}
\includegraphics{intervals.8.eps}
\end{center}
%\caption{..}
\end{figure}

\begin{figure}[!htb]
\begin{center}
\includegraphics{intervals.9.eps}
\end{center}
%\caption{..}
\end{figure}

\subsubsection*{Note on naming and notation}
In \cite{rudin, rudin_real}, an open interval is always called
a \emph{segment}, and a closed interval is called simply an interval.
However, the above naming with open, closed, and half-open interval seems
to be more widely adopted. 
See e.g. \cite{adams, beta, silverman}. To distinguish between $[a,b)$ and
$(a,b]$, the former is sometimes called a \emph{right half-open interval } and
the latter a \emph{left half-open interval} \cite{igari}. 
The notation $(a,b)$, $[a,b)$, $(a,b]$, $[a,b]$ seems to be standard. However,  
some authors (especially from the French school) use notation
$]a,b[$, $[a,b[$, $]a,b]$, $[a,b]$ instead of the above (in the same \PMlinkescapetext{order}). Bourbaki, for example, uses this notation.

{\it This entry contains content adapted from the Wikipedia article \PMlinkexternal{Interval (mathematics)}{http://en.wikipedia.org/wiki/Interval_(mathematics)} as of November 10, 2006.}

\begin{thebibliography}{9}
 \bibitem{rudin}
 W. Rudin, \emph{Principles of Mathematical Analysis}, McGraw-Hill Inc., 1976.
 \bibitem{rudin_real}
 W. Rudin, \emph{Real and complex analysis}, 3rd ed., McGraw-Hill Inc., 1987.
\bibitem{adams} R. Adams, \emph{Calculus, a complete course},
 Addison-Wesley Publishers Ltd., 3rd ed., 1995.
\bibitem{beta} L. R\r{a}de, B. Westergren,
 \emph{Mathematics Handbook for Science and Engineering},
 Studentlitteratur, 1995.
 \bibitem{silverman}
R.A. Silverman, \emph{Introductory Complex Analysis}, 
Dover Publications, 1972.
\bibitem{igari} S. Igari, \emph{Real analysis - With an introduction to Wavelet Theory}, American Mathematical Society, 1998.
\end{thebibliography}

The metapost code for the figures can be found
\PMlinkexternal{here}{http://aux.planetmath.org/files/objects/4446/}.
%%%%%
%%%%%
\end{document}
