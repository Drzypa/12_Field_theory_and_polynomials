\documentclass[12pt]{article}
\usepackage{pmmeta}
\pmcanonicalname{TheRamificationIndexAndTheInertialDegreeAreMultiplicativeInTowers}
\pmcreated{2013-03-22 15:06:34}
\pmmodified{2013-03-22 15:06:34}
\pmowner{alozano}{2414}
\pmmodifier{alozano}{2414}
\pmtitle{the ramification index and the inertial degree are multiplicative in towers}
\pmrecord{5}{36842}
\pmprivacy{1}
\pmauthor{alozano}{2414}
\pmtype{Theorem}
\pmcomment{trigger rebuild}
\pmclassification{msc}{12F99}
\pmclassification{msc}{13B02}
\pmclassification{msc}{11S15}
%\pmkeywords{towers of number fields}
%\pmkeywords{ramification}
%\pmkeywords{inertia}
\pmrelated{Ramify}
\pmrelated{InertialDegree}

\endmetadata

% this is the default PlanetMath preamble.  as your knowledge
% of TeX increases, you will probably want to edit this, but
% it should be fine as is for beginners.

% almost certainly you want these
\usepackage{amssymb}
\usepackage{amsmath}
\usepackage{amsthm}
\usepackage{amsfonts}

% used for TeXing text within eps files
%\usepackage{psfrag}
% need this for including graphics (\includegraphics)
%\usepackage{graphicx}
% for neatly defining theorems and propositions
%\usepackage{amsthm}
% making logically defined graphics
%%\usepackage{xypic}

% there are many more packages, add them here as you need them

% define commands here

\newtheorem*{thm}{Theorem}
\newtheorem{defn}{Definition}
\newtheorem{prop}{Proposition}
\newtheorem{lemma}{Lemma}
\newtheorem{cor}{Corollary}

% Some sets
\newcommand{\Nats}{\mathbb{N}}
\newcommand{\Ints}{\mathbb{Z}}
\newcommand{\Reals}{\mathbb{R}}
\newcommand{\Complex}{\mathbb{C}}
\newcommand{\Rats}{\mathbb{Q}}
\newcommand{\p}{{\mathfrak{p}}}
\newcommand{\A}{{\mathfrak{A}}}
\renewcommand{\P}{{\mathfrak{P}}}
\newcommand{\Pcal}{\mathcal{P}}
\newcommand{\Gal}{\operatorname{Gal}}
\newcommand{\intK}{\mathcal{O}_K}
\newcommand{\intF}{\mathcal{O}_F}
\newcommand{\intE}{\mathcal{O}_E}
\begin{document}
\begin{thm}
Let $E,\ F$ and $K$ be number fields in a tower:
$$K\subseteq F \subseteq E$$
and let $\intE,\ \intF$ and $\intK$ be their rings of integers respectively. Suppose $\p$ is a prime ideal of $\intK$ and let $\P$ be a prime ideal of $\intF$ lying above $\p$, and $\Pcal$ is a prime ideal of $\intE$ lying above $\P$. 

\begin{center}
$\xymatrix{
{E} \ar@{-}[d] & {\intE} \ar@{-}[d] & {\Pcal} \ar@{-}[d] \\
{F} \ar@{-}[d] & {\intF} \ar@{-}[d] & {\P} \ar@{-}[d]\\
K & \intK & \p }$
\end{center}

Then the indices of the extensions, the ramification indices and inertial degrees satisfy:
\begin{eqnarray}
[E:K] &=& [E:F]\cdot [F:K],\\
\nonumber & &\\
e(\Pcal|\p) &=& e(\Pcal|\P)\cdot e(\P|\p),\\
\nonumber & &\\
f(\Pcal|\p) &=& f(\Pcal|\P)\cdot f(\P|\p).
\end{eqnarray}
\end{thm}
%%%%%
%%%%%
\end{document}
