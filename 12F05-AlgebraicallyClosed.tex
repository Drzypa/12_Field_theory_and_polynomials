\documentclass[12pt]{article}
\usepackage{pmmeta}
\pmcanonicalname{AlgebraicallyClosed}
\pmcreated{2013-03-22 12:12:06}
\pmmodified{2013-03-22 12:12:06}
\pmowner{djao}{24}
\pmmodifier{djao}{24}
\pmtitle{algebraically closed}
\pmrecord{10}{31509}
\pmprivacy{1}
\pmauthor{djao}{24}
\pmtype{Definition}
\pmcomment{trigger rebuild}
\pmclassification{msc}{12F05}
\pmdefines{algebraic closure}

% this is the default PlanetMath preamble.  as your knowledge
% of TeX increases, you will probably want to edit this, but
% it should be fine as is for beginners.

% almost certainly you want these
\usepackage{amssymb}
\usepackage{amsmath}
\usepackage{amsfonts}

% used for TeXing text within eps files
%\usepackage{psfrag}
% need this for including graphics (\includegraphics)
%\usepackage{graphicx}
% for neatly defining theorems and propositions
%\usepackage{amsthm}
% making logically defined graphics
%%%%\usepackage{xypic} 

% there are many more packages, add them here as you need them

% define commands here
\begin{document}
A field $K$ is \emph{algebraically closed} if every non-constant polynomial in $K[X]$ has a root in $K$.

An extension field $L$ of $K$ is an \emph{algebraic closure} of $K$ if $L$ is algebraically closed and every element of $L$ is algebraic over $K$.  Using the axiom of choice, one can show that any field has an algebraic closure.  Moreover, any two algebraic closures of a field are isomorphic as fields, but not necessarily canonically isomorphic.

%%%%%
%%%%%
%%%%%
\end{document}
