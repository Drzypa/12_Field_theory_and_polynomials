\documentclass[12pt]{article}
\usepackage{pmmeta}
\pmcanonicalname{InverseNumber}
\pmcreated{2013-03-22 14:53:46}
\pmmodified{2013-03-22 14:53:46}
\pmowner{pahio}{2872}
\pmmodifier{pahio}{2872}
\pmtitle{inverse number}
\pmrecord{12}{36578}
\pmprivacy{1}
\pmauthor{pahio}{2872}
\pmtype{Definition}
\pmcomment{trigger rebuild}
\pmclassification{msc}{12E99}
\pmclassification{msc}{00A05}
\pmsynonym{inverse}{InverseNumber}
\pmsynonym{reciprocal}{InverseNumber}
\pmrelated{ConditionOfOrthogonality}
\pmrelated{InverseFormingInProportionToGroupOperation}
\pmdefines{reciprocal number}

% this is the default PlanetMath preamble.  as your knowledge
% of TeX increases, you will probably want to edit this, but
% it should be fine as is for beginners.

% almost certainly you want these
\usepackage{amssymb}
\usepackage{amsmath}
\usepackage{amsfonts}

% used for TeXing text within eps files
%\usepackage{psfrag}
% need this for including graphics (\includegraphics)
%\usepackage{graphicx}
% for neatly defining theorems and propositions
%\usepackage{amsthm}
% making logically defined graphics
%%%\usepackage{xypic}

% there are many more packages, add them here as you need them

% define commands here
\begin{document}
The {\em inverse number} or {\em reciprocal number} of a non-zero real or complex number $a$ may be denoted by $a^{-1}$, and it \PMlinkescapetext{means} the quotient $\frac{1}{a}$ (so, it is really the $-1^\mathrm{th}$ power of $a$). 

\begin{itemize}
 \item Two numbers are inverse numbers of each other if and only if their product is equal to 1 (cf. opposite inverses).
 \item If $a$ ($\neq 0$) is given in a quotient form $\frac{b}{c}$, then its inverse number is simply
             $$\left(\frac{b}{c}\right)^{-1} = \frac{c}{b}.$$
 \item Forming the inverse number is also a multiplicative function, i.e.
         $$(bc)^{-1} = b^{-1}c^{-1}$$
(to be more precise, it is an automorphism of the multiplicative group of $\mathbb{R}$ resp. $\mathbb{C}$).
\end{itemize}
%%%%%
%%%%%
\end{document}
