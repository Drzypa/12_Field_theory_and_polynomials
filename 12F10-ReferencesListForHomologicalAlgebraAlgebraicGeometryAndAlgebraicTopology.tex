\documentclass[12pt]{article}
\usepackage{pmmeta}
\pmcanonicalname{ReferencesListForHomologicalAlgebraAlgebraicGeometryAndAlgebraicTopology}
\pmcreated{2013-03-22 18:22:21}
\pmmodified{2013-03-22 18:22:21}
\pmowner{bci1}{20947}
\pmmodifier{bci1}{20947}
\pmtitle{references list for homological algebra, algebraic geometry and algebraic topology}
\pmrecord{15}{41014}
\pmprivacy{1}
\pmauthor{bci1}{20947}
\pmtype{Bibliography}
\pmcomment{trigger rebuild}
\pmclassification{msc}{12F10}
\pmclassification{msc}{14A22}
\pmclassification{msc}{14A20}
\pmclassification{msc}{18-00}
\pmclassification{msc}{03G20}
\pmclassification{msc}{03G12}
\pmclassification{msc}{11-00}
\pmclassification{msc}{14-01}
\pmclassification{msc}{03G30}
\pmclassification{msc}{00A15}
\pmsynonym{references on algebraic geometry}{ReferencesListForHomologicalAlgebraAlgebraicGeometryAndAlgebraicTopology}
%\pmkeywords{algebraic geometry references}
%\pmkeywords{algebraic varieties}
%\pmkeywords{functorial morphisms and schemas}
%\pmkeywords{algebraic logic}
%\pmkeywords{categorical logic}
%\pmkeywords{topoi and generalized toposes}
\pmrelated{AlexanderGrothendiecksAvailableSeminarsAndBooks}

% this is the default PlanetMath preamble.  as your knowledge
% of TeX increases, you will probably want to edit this, but
% it should be fine as is for beginners.

% almost certainly you want these
\usepackage{amssymb}
\usepackage{amsmath}
\usepackage{amsfonts}

% used for TeXing text within eps files
%\usepackage{psfrag}
% need this for including graphics (\includegraphics)
%\usepackage{graphicx}
% for neatly defining theorems and propositions
%\usepackage{amsthm}
% making logically defined graphics
%%%\usepackage{xypic}

% there are many more packages, add them here as you need them

% define commands here
\usepackage{amsmath, amssymb, amsfonts, amsthm, amscd, latexsym}
%%\usepackage{xypic}
\usepackage[mathscr]{eucal}

\setlength{\textwidth}{6.5in}
%\setlength{\textwidth}{16cm}
\setlength{\textheight}{9.0in}
%\setlength{\textheight}{24cm}

\hoffset=-.75in     %%ps format
%\hoffset=-1.0in     %%hp format
\voffset=-.4in

\theoremstyle{plain}
\newtheorem{lemma}{Lemma}[section]
\newtheorem{proposition}{Proposition}[section]
\newtheorem{theorem}{Theorem}[section]
\newtheorem{corollary}{Corollary}[section]

\theoremstyle{definition}
\newtheorem{definition}{Definition}[section]
\newtheorem{example}{Example}[section]
%\theoremstyle{remark}
\newtheorem{remark}{Remark}[section]
\newtheorem*{notation}{Notation}
\newtheorem*{claim}{Claim}

\renewcommand{\thefootnote}{\ensuremath{\fnsymbol{footnote%%@
}}}
\numberwithin{equation}{section}

\newcommand{\Ad}{{\rm Ad}}
\newcommand{\Aut}{{\rm Aut}}
\newcommand{\Cl}{{\rm Cl}}
\newcommand{\Co}{{\rm Co}}
\newcommand{\DES}{{\rm DES}}
\newcommand{\Diff}{{\rm Diff}}
\newcommand{\Dom}{{\rm Dom}}
\newcommand{\Hol}{{\rm Hol}}
\newcommand{\Mon}{{\rm Mon}}
\newcommand{\Hom}{{\rm Hom}}
\newcommand{\Ker}{{\rm Ker}}
\newcommand{\Ind}{{\rm Ind}}
\newcommand{\IM}{{\rm Im}}
\newcommand{\Is}{{\rm Is}}
\newcommand{\ID}{{\rm id}}
\newcommand{\GL}{{\rm GL}}
\newcommand{\Iso}{{\rm Iso}}
\newcommand{\Sem}{{\rm Sem}}
\newcommand{\St}{{\rm St}}
\newcommand{\Sym}{{\rm Sym}}
\newcommand{\SU}{{\rm SU}}
\newcommand{\Tor}{{\rm Tor}}
\newcommand{\U}{{\rm U}}

\newcommand{\A}{\mathcal A}
\newcommand{\Ce}{\mathcal C}
\newcommand{\D}{\mathcal D}
\newcommand{\E}{\mathcal E}
\newcommand{\F}{\mathcal F}
\newcommand{\G}{\mathcal G}
\newcommand{\Q}{\mathcal Q}
\newcommand{\R}{\mathcal R}
\newcommand{\cS}{\mathcal S}
\newcommand{\cU}{\mathcal U}
\newcommand{\W}{\mathcal W}

\newcommand{\bA}{\mathbb{A}}
\newcommand{\bB}{\mathbb{B}}
\newcommand{\bC}{\mathbb{C}}
\newcommand{\bD}{\mathbb{D}}
\newcommand{\bE}{\mathbb{E}}
\newcommand{\bF}{\mathbb{F}}
\newcommand{\bG}{\mathbb{G}}
\newcommand{\bK}{\mathbb{K}}
\newcommand{\bM}{\mathbb{M}}
\newcommand{\bN}{\mathbb{N}}
\newcommand{\bO}{\mathbb{O}}
\newcommand{\bP}{\mathbb{P}}
\newcommand{\bR}{\mathbb{R}}
\newcommand{\bV}{\mathbb{V}}
\newcommand{\bZ}{\mathbb{Z}}

\newcommand{\bfE}{\mathbf{E}}
\newcommand{\bfX}{\mathbf{X}}
\newcommand{\bfY}{\mathbf{Y}}
\newcommand{\bfZ}{\mathbf{Z}}

\renewcommand{\O}{\Omega}
\renewcommand{\o}{\omega}
\newcommand{\vp}{\varphi}
\newcommand{\vep}{\varepsilon}

\newcommand{\diag}{{\rm diag}}
\newcommand{\grp}{{\mathbb G}}
\newcommand{\dgrp}{{\mathbb D}}
\newcommand{\desp}{{\mathbb D^{\rm{es}}}}
\newcommand{\Geod}{{\rm Geod}}
\newcommand{\geod}{{\rm geod}}
\newcommand{\hgr}{{\mathbb H}}
\newcommand{\mgr}{{\mathbb M}}
\newcommand{\ob}{{\rm Ob}}
\newcommand{\obg}{{\rm Ob(\mathbb G)}}
\newcommand{\obgp}{{\rm Ob(\mathbb G')}}
\newcommand{\obh}{{\rm Ob(\mathbb H)}}
\newcommand{\Osmooth}{{\Omega^{\infty}(X,*)}}
\newcommand{\ghomotop}{{\rho_2^{\square}}}
\newcommand{\gcalp}{{\mathbb G(\mathcal P)}}

\newcommand{\rf}{{R_{\mathcal F}}}
\newcommand{\glob}{{\rm glob}}
\newcommand{\loc}{{\rm loc}}
\newcommand{\TOP}{{\rm TOP}}

\newcommand{\wti}{\widetilde}
\newcommand{\what}{\widehat}

\renewcommand{\a}{\alpha}
\newcommand{\be}{\beta}
\newcommand{\ga}{\gamma}
\newcommand{\Ga}{\Gamma}
\newcommand{\de}{\delta}
\newcommand{\del}{\partial}
\newcommand{\ka}{\kappa}
\newcommand{\si}{\sigma}
\newcommand{\ta}{\tau}
\newcommand{\med}{\medbreak}
\newcommand{\medn}{\medbreak \noindent}
\newcommand{\bign}{\bigbreak \noindent}
\newcommand{\lra}{{\longrightarrow}}
\newcommand{\ra}{{\rightarrow}}
\newcommand{\rat}{{\rightarrowtail}}
\newcommand{\oset}[1]{\overset {#1}{\ra}}
\newcommand{\osetl}[1]{\overset {#1}{\lra}}
\newcommand{\hr}{{\hookrightarrow}}
\begin{document}
\section{A Reference (partial) List for Homological Algebra, Algebraic Geometry and Algebraic Topology:}

\begin{thebibliography}{99}

\bibitem{Alex1}
Alexander Grothendieck. 1971, Rev\^{e}tements \'Etales et Groupe Fondamental (SGA1),
chapter VI: Cat\'egories fibr\'ees et descente, \emph{Lecture Notes in Math.}
\textbf{224}, Springer--Verlag: Berlin.

\bibitem{Alex2}
Alexander Grothendieck. 1957, Sur quelque point d-alg\'{e}bre homologique. , \emph{Tohoku Math. J.}, \textbf{9:} 119-121.

\bibitem{Alex3}
Alexander Grothendieck and J. Dieudonn\'{e}.: 1960, El\'{e}ments de geometrie alg\'{e}brique., \emph{Publ. Inst. des Hautes Etudes de Science}, \textbf{4}.

\bibitem{ALEXsem1}
Alexander Grothendieck et al.,1971. Séminaire de Géométrie Algébrique du Bois-Marie, Vol. 1--7, Berlin: Springer-Verlag.

\bibitem{ALEXsem2}
Alexander Grothendieck. 1962. Séminaires en Géométrie Algébrique du Bois-Marie, Vol. 2 - Cohomologie Locale des Faisceaux Cohèrents et Théorèmes de Lefschetz Locaux et Globaux. , pp.287. (with an additional contributed expos\'e by Mme. Michele Raynaud). 
\PMlinkexternal{Typewritten manuscript available in French}{http://modular.fas.harvard.edu/sga/sga/2/index.html};
\PMlinkexternal{see also a brief summary in English}{http://planetmath.org/?op=getobj&from=books&id=78}

\bibitem{ALEX57}
Alexander Grothendieck. 1957, Sur Quelques Points d'algèbre homologique, {\em Tohoku Mathematics Journal}, 9, 119--221.
 
\bibitem{Alex60to61sga1}
Alexander Grothendieck et al. \emph{S\'eminaires en G\'eometrie Alg\`ebrique- 4}, Tome 1, Expos\'e 1 
(or the Appendix to Expos\'ee 1, by `N. Bourbaki' for more detail and a large number of results.
AG4 is \PMlinkexternal{freely available}{http://modular.fas.harvard.edu/sga/sga/pdf/index.html} in French;
also available here is an extensive 
\PMlinkexternal{Abstract in English}{http://planetmath.org/?op=getobj&from=books&id=158}.

\bibitem{Alex84}
Alexander Grothendieck, 1984. ``Esquisse d'un Programme'', (1984 manuscript), 
{\em finally published in ``Geometric Galois Actions''}, L. Schneps, P. Lochak, eds., 
{\em London Math. Soc. Lecture Notes} {\bf 242}, Cambridge University Press, 1997, pp.5-48;
English transl., ibid., pp. 243-283. MR 99c:14034 .

\bibitem{Alex81}
Alexander Grothendieck, ``La longue marche in à travers la théorie de Galois'' 
\emph{= ``The Long March Towards/Across the Theory of Galois''}, 1981 manuscript, University of Montpellier preprint series 1996, edited by J. Malgoire. 

\bibitem{LS94}
Leila Schneps. 1994. 
\PMlinkexternal{The Grothendieck Theory of Dessins d'Enfants}{http://planetmath.org/?op=getobj&from=books&id=163}.
(London Mathematical Society Lecture Note Series), Cambridge University Press, 376 pp.

\bibitem{DHSL2k}
David Harbater and Leila Schneps. 2000.
\PMlinkexternal{Fundamental groups of moduli and the Grothendieck-Teichmüller group}{http://www.ams.org/tran/2000-352-07/S0002-9947-00-02347-3/home.html}, \emph{Trans. Amer. Math. Soc}. 352 (2000), 3117-3148. 
MSC: Primary 11R32, 14E20, 14H10; Secondary 20F29, 20F34, 32G15.

\bibitem{JPS1964}
J. P. Serre. 1964. {\em Cohomologie Galoisienne}, Springer-Verlag: Berlin.

\bibitem{JLV1965}
J. L. Verdier. 1965. {\em Alg\`ebre homologiques et Cat\'egories deriv\'ees}. North Holland Publ. Cie.

\bibitem{GJ2k7}
G. Janelidze, 2007, \PMlinkexternal{Descent and Galois Theory}{http://planetmath.org/?op=getobj&from=books&id=168}, Outline of a three-lecture series presented in Belgium, in June 2007. 

\bibitem{YL91}
Yves Laszlo. \PMlinkexternal{Descent Cohomologique}{http://planetmath.org/?op=getobj&from=books&id=167},
Monograph, Preprint: ``Cohmology Descent theory''; pp.43, (orig. in French).

\bibitem{JSM1993}
J.S. Milne. 1993, 
\PMlinkexternal{``Algebraic Geometry- Lecture notes for Math 631''}{http://planetmath.org/?op=getobj&from=lec&id=22}
, taught at the University of Michigan, Fall 1993. 
 
\bibitem{AH2k8a}
Allen Hatcher. 2008a,
\PMlinkexternal{\emph{Spectral Sequences in Algebraic Topology}}{http://planetmath.org/?op=getobj&from=books&id=95}
\PMlinkexternal{Ch.1 (J.P.) Serre Spectral Sequence}{http://www.math.cornell.edu/~hatcher/SSAT/SSch1.pdf}

\bibitem{AH2k8b}
Allen Hatcher. 2008b,
{Allen Hatcher's Book Projects}{http://www.math.cornell.edu/~hatcher/}

\bibitem{AH2k8b}
Allen Hatcher. 2008c, \PMlinkexternal{Vector Bundles and K-Theory}{http://www.math.cornell.edu/~hatcher/VBKT/VB.pdf},
(v. also \PMlinkexternal{Extended Abstract}{http://planetmath.org/?op=getobj&from=books&id=169})

\end{thebibliography}
%%%%%
%%%%%
\end{document}
