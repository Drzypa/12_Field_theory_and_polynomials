\documentclass[12pt]{article}
\usepackage{pmmeta}
\pmcanonicalname{CharacterizationOfField}
\pmcreated{2013-03-22 13:57:03}
\pmmodified{2013-03-22 13:57:03}
\pmowner{alozano}{2414}
\pmmodifier{alozano}{2414}
\pmtitle{characterization of field}
\pmrecord{7}{34715}
\pmprivacy{1}
\pmauthor{alozano}{2414}
\pmtype{Theorem}
\pmcomment{trigger rebuild}
\pmclassification{msc}{12E99}
\pmsynonym{a field only has two ideals}{CharacterizationOfField}
\pmrelated{Field}
\pmrelated{Ring}
\pmrelated{Ideal}

% this is the default PlanetMath preamble.  as your knowledge
% of TeX increases, you will probably want to edit this, but
% it should be fine as is for beginners.

% almost certainly you want these
\usepackage{amssymb}
\usepackage{amsmath}
\usepackage{amsthm}
\usepackage{amsfonts}

% used for TeXing text within eps files
%\usepackage{psfrag}
% need this for including graphics (\includegraphics)
%\usepackage{graphicx}
% for neatly defining theorems and propositions
%\usepackage{amsthm}
% making logically defined graphics
%%%\usepackage{xypic}

% there are many more packages, add them here as you need them

% define commands here

\newtheorem{thm}{Theorem}
\newtheorem{defn}{Definition}
\newtheorem{prop}{Proposition}
\newtheorem{lemma}{Lemma}
\newtheorem{cor}{Corollary}

% Some sets
\newcommand{\Nats}{\mathbb{N}}
\newcommand{\Ints}{\mathbb{Z}}
\newcommand{\Reals}{\mathbb{R}}
\newcommand{\Complex}{\mathbb{C}}
\newcommand{\Rats}{\mathbb{Q}}
\begin{document}
\begin{prop}
Let $\mathcal{R}\neq 0$ be a commutative ring with identity. The ring $\mathcal{R}$ (as above) is a field if and only if
$\mathcal{R}$ has exactly two ideals: $(0),\mathcal{R}$.
\end{prop}

\begin{proof}
{\bf ($\Rightarrow$)} Suppose $\mathcal{R}$ is a field and let
$\mathcal{A}$ be a non-zero ideal of $\mathcal{R}$. Then there
exists $r\in \mathcal{A}\subseteq \mathcal{R}$ with $r\neq 0$.
Since $\mathcal{R}$ is a field and $r$ is a non-zero element,
there exists $s\in \mathcal{R}$ such that
$$s\cdot r =1 \in \mathcal{R}$$
Moreover, $\mathcal{A}$ is an ideal, $r\in \mathcal{A}, s\in
\mathcal{S}$, so $s\cdot r =1 \in \mathcal{A}$. Hence
$\mathcal{A}=\mathcal{R}$. We have proved that the only ideals of
$\mathcal{R}$ are $(0)$ and $\mathcal{R}$ as desired.

{\bf ($\Leftarrow$)} Suppose the ring $\mathcal{R}$ has only two
ideals, namely $(0),\mathcal{R}$. Let $a\in \mathcal{R}$ be a
non-zero element; we would like to prove the existence of a
multiplicative inverse for $a$ in $\mathcal{R}$. Define the
following set:
$$\mathcal{A}=(a)=\{r\in\mathcal{R} \mid r=s\cdot a, \text{ for
some } s\in\mathcal{R}\}$$ This is clearly an ideal, the ideal
generated by the element $a$. Moreover, this ideal is not the zero
ideal because $a\in \mathcal{A}$ and $a$ was assumed to be
non-zero. Thus, since there are only two ideals, we conclude
$\mathcal{A}=\mathcal{R}$. Therefore $1\in
\mathcal{A}=\mathcal{R}$ so there exists an element $s\in
\mathcal{R}$ such that
$$s\cdot a=1 \in \mathcal{R}$$
Hence for all non-zero $a\in \mathcal{R}$, $a$ has a multiplicative inverse in $\mathcal{R}$, so $\mathcal{R}$ is, in fact, a field.
\end{proof}
%%%%%
%%%%%
\end{document}
