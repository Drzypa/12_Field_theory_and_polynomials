\documentclass[12pt]{article}
\usepackage{pmmeta}
\pmcanonicalname{NonconstantElementOfRationalFunctionField}
\pmcreated{2013-03-22 15:02:50}
\pmmodified{2013-03-22 15:02:50}
\pmowner{pahio}{2872}
\pmmodifier{pahio}{2872}
\pmtitle{non-constant element of rational function field}
\pmrecord{17}{36762}
\pmprivacy{1}
\pmauthor{pahio}{2872}
\pmtype{Theorem}
\pmcomment{trigger rebuild}
\pmclassification{msc}{12F99}
\pmsynonym{field of rational functions}{NonconstantElementOfRationalFunctionField}
\pmsynonym{rational function field}{NonconstantElementOfRationalFunctionField}

\endmetadata

% this is the default PlanetMath preamble.  as your knowledge
% of TeX increases, you will probably want to edit this, but
% it should be fine as is for beginners.

% almost certainly you want these
\usepackage{amssymb}
\usepackage{amsmath}
\usepackage{amsfonts}

% used for TeXing text within eps files
%\usepackage{psfrag}
% need this for including graphics (\includegraphics)
%\usepackage{graphicx}
% for neatly defining theorems and propositions
  \usepackage{amsthm}
% making logically defined graphics
%%%\usepackage{xypic}

% there are many more packages, add them here as you need them

% define commands here
\theoremstyle{definition}
\newtheorem*{thmplain}{Theorem}
\begin{document}
Let $K$ be a field. \, Every \PMlinkname{simple}{SimpleFieldExtension} transcendent field extension $K(\alpha)/K$ may be represented by the extension $K(X)/K$, where $K(X)$ is the field of fractions of the polynomial ring $K[X]$ in one indeterminate $X$. \,The elements of $K(X)$ are rational functions, i.e. rational expressions
\begin{align}
\varrho = \frac{f(X)}{g(X)}
\end{align}
with $f(X)$ and $g(X)$ polynomials in $K[X]$.

\begin{thmplain}
\,\,Let the non-constant rational function (1) be reduced to lowest terms and let the greater of the degrees of its numerator and denominator be $n$. \,This element $\varrho$ is transcendental with respect to the base field $K$. \,The field extension $K(X)/K(\varrho)$ is algebraic and of degree $n$.
\end{thmplain}

{\em Proof.}\, The element $X$ satisfies the equation
\begin{align}
                \varrho\,g(X)\!-\!f(X) = 0,
\end{align}
the coefficients of which are in the field $K(\varrho)$, actually in the ring $K[\varrho]$.\, If all these coefficients were zero, we could take one non-zero coefficient $b_\nu$ in $g(X)$ and the coefficient $a_\nu$ of the same power of $X$ in $f(X)$, and then we would have especially\, $\varrho b_\nu\!-a_\nu = 0$;\, this would mean that\, $\varrho = \frac{a_\nu}{b_\nu}$ = constant, contrary to the supposition.\, Thus at least one coefficient in (2) differs from zero, and we conclude that $X$ is algebraic with respect to $K(\varrho)$.\, If $K(\varrho)$ were algebraic with respect to $K$, then also $X$ should be algebraic with respect to $K$.\, This is not true, and therefore we see that $K(\varrho)$ is transcendental, Q.E.D.

Further, $X$ is a zero of the $n^\mathrm{th}$ degree polynomial
               $$h(Y) = \varrho\,g(Y)\!-\!f(Y)$$
of the ring $K(\varrho)[Y]$, actually of the ring $K[\varrho][Y]$, i.e. of $K[\varrho$,\,Y].\, The polynomial is irreducible in this ring, since otherwise it would have there two factors, and because $h(Y)$ is linear in $\varrho$, the other factor should depend only on $Y$; but there can not be such a factor, for the polynomials $f(Z)$ and $g(Z)$ are relatively prime.\, The conclusion is that $X$ is an algebraic element over $K(\varrho)$ of degree $n$ and therefore also
     $$(K(X):K(\varrho)) = n,$$
Q.E.D.

\begin{thebibliography}{9}
\bibitem{Waerden} B. L. van der Waerden: {\em Algebra}.\, Siebte Auflage der {\em Modernen Algebra}.\, Erster Teil. \\--- Springer-Verlag. Berlin, Heidelberg (1966).
\end{thebibliography}
%%%%%
%%%%%
\end{document}
