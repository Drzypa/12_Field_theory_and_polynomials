\documentclass[12pt]{article}
\usepackage{pmmeta}
\pmcanonicalname{ZariskiLemma}
\pmcreated{2013-03-22 17:18:11}
\pmmodified{2013-03-22 17:18:11}
\pmowner{polarbear}{3475}
\pmmodifier{polarbear}{3475}
\pmtitle{Zariski lemma}
\pmrecord{7}{39650}
\pmprivacy{1}
\pmauthor{polarbear}{3475}
\pmtype{Derivation}
\pmcomment{trigger rebuild}
\pmclassification{msc}{12F05}
\pmclassification{msc}{11J85}

\endmetadata

% this is the default PlanetMath preamble.  as your knowledge
% of TeX increases, you will probably want to edit this, but
% it should be fine as is for beginners.

% almost certainly you want these
\usepackage{amssymb}
\usepackage{amsmath}
\usepackage{amsfonts}

% used for TeXing text within eps files
\usepackage{psfrag}
% need this for including graphics (\includegraphics)
%\usepackage{graphicx}
% for neatly defining theorems and propositions
\usepackage{amsthm}
% making logically defined graphics
%%%\usepackage{xypic}

% there are many more packages, add them here as you need them

% define commands here
\newtheorem{prop*}{Proposition}
\newtheorem{lem*}{Lemma}
\begin{document}
\begin{prop*} Let $R\subseteq S\subseteq T$ be commutative rings. If $R$ is noetherian, and T finitely generated as an $R$-algebra and as an $S$-module, then $S$ is finitely generated as an $R$-algebra.
\end{prop*}
\begin{lem*}[Zariski's lemma]
 Let $(L:K)$ be a field extension and $a_1,\ldots,a_n\in L$ be such that $K(a_1,\ldots,a_n)=K[a_1,\ldots,a_n]$. Then the elements $a_1,\ldots,a_n$ are algebraic over $K$.\end{lem*}
\begin{proof}
The case $n=1$ is clear. Now suppose $n>1$ and not all $a_i,1\leq i\leq n$ are algebraic over $K$.\\ Wlog we may assume $a_1,\ldots,a_n$ are algebraically independent and each element $a_{r+1},\ldots,a_n$ is algebraic over $D:=K(a_1,\ldots,a_r)$. Hence $K[a_1,\ldots,a_n]$ is a finite algebraic extension of $D$ and therefore is a finitely generated $D$-module.\newline
 The above proposition applied to $K\subseteq D\subseteq K[a_1,\ldots,a_n]$ shows that $D$ is finitely generated as a $K$-algebra, i.e $D=K[d_1,\ldots,d_n]$.\newline


 Let $d_i=\frac{p_i(a_1,\ldots,a_n)}{q_i(a_1,\ldots,a_n)}$, where $p_i,q_i\in K[x_1,\ldots,x_n]$.\\ Now $a_1,\ldots,a_n$ are algebraically independent so that $K[a_1,\ldots,a_n]\cong K[x_1,\ldots,x_n]$, which is a \PMlinkname{UFD}{UFD}.\newline
 Let $h$ be a prime divisor of $q_1\cdots q_r+1$. Since $q$ is relatively prime to each of $q_i$, the element ${q(a_1,\ldots,a_n)}^{-1} \in D$ cannot be in $K[d_1,\ldots,d_n]$. We obtain a contradiction.
\end{proof}
%%%%%
%%%%%
\end{document}
