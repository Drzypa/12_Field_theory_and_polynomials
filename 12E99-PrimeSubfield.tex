\documentclass[12pt]{article}
\usepackage{pmmeta}
\pmcanonicalname{PrimeSubfield}
\pmcreated{2013-03-22 12:37:47}
\pmmodified{2013-03-22 12:37:47}
\pmowner{djao}{24}
\pmmodifier{djao}{24}
\pmtitle{prime subfield}
\pmrecord{4}{32892}
\pmprivacy{1}
\pmauthor{djao}{24}
\pmtype{Definition}
\pmcomment{trigger rebuild}
\pmclassification{msc}{12E99}

% this is the default PlanetMath preamble.  as your knowledge
% of TeX increases, you will probably want to edit this, but
% it should be fine as is for beginners.

% almost certainly you want these
\usepackage{amssymb}
\usepackage{amsmath}
\usepackage{amsfonts}

% used for TeXing text within eps files
%\usepackage{psfrag}
% need this for including graphics (\includegraphics)
%\usepackage{graphicx}
% for neatly defining theorems and propositions
%\usepackage{amsthm}
% making logically defined graphics
%%%\usepackage{xypic} 

% there are many more packages, add them here as you need them

% define commands here
\begin{document}
The {\em prime subfield} of a field $F$ is the intersection of all subfields of $F$, or equivalently the smallest subfield of $F$. It can also be constructed by taking the quotient field of the additive subgroup of $F$ generated by the multiplicative identity $1$.

If $F$ has characteristic $p$ where $p > 0$ is a prime, then the prime subfield of $F$ is isomorphic to the field $\mathbb{Z}/p\mathbb{Z}$ of integers mod $p$. When $F$ has characteristic zero, the prime subfield of $F$ is isomorphic to the field $\mathbb{Q}$ of rational numbers.
%%%%%
%%%%%
\end{document}
