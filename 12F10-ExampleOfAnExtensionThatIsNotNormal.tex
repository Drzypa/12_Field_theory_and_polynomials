\documentclass[12pt]{article}
\usepackage{pmmeta}
\pmcanonicalname{ExampleOfAnExtensionThatIsNotNormal}
\pmcreated{2013-03-22 16:00:28}
\pmmodified{2013-03-22 16:00:28}
\pmowner{Wkbj79}{1863}
\pmmodifier{Wkbj79}{1863}
\pmtitle{example of an extension that is not normal}
\pmrecord{6}{38038}
\pmprivacy{1}
\pmauthor{Wkbj79}{1863}
\pmtype{Example}
\pmcomment{trigger rebuild}
\pmclassification{msc}{12F10}

% this is the default PlanetMath preamble.  as your knowledge
% of TeX increases, you will probably want to edit this, but
% it should be fine as is for beginners.

% almost certainly you want these
\usepackage{amssymb}
\usepackage{amsmath}
\usepackage{amsfonts}

% used for TeXing text within eps files
%\usepackage{psfrag}
% need this for including graphics (\includegraphics)
%\usepackage{graphicx}
% for neatly defining theorems and propositions
%\usepackage{amsthm}
% making logically defined graphics
%%%\usepackage{xypic}

% there are many more packages, add them here as you need them

% define commands here

\begin{document}
In this entry, $\sqrt[3]{2}$ indicates the real cube root of $2$.

Consider the extension $\mathbb{Q}(\sqrt[3]{2})/\mathbb{Q}$.  The minimal polynomial for $\sqrt[3]{2}$ over $\mathbb{Q}$ is $x^3-2$.  This polynomial factors in $\mathbb{Q}(\sqrt[3]{2})$ as $x^3-2=(x-\sqrt[3]{2})(x^2+x\sqrt[3]{2}+\sqrt[3]{4})$.  Let $f(x)=x^2+x\sqrt[3]{2}+\sqrt[3]{4}$.  Note that $\operatorname{disc}(f(x))=(\sqrt[3]{2})^2-4\sqrt[3]{4}=\sqrt[3]{4}-4\sqrt[3]{4}=-3\sqrt[3]{4}<0$.  Thus, $f(x)$ has no real roots.  Therefore, $f(x)$ has no roots in $\mathbb{Q}(\sqrt[3]{2})$ since $\mathbb{Q}(\sqrt[3]{2}) \subseteq \mathbb{R}$.  Hence, $x^3-2$ has a root in $\mathbb{Q}(\sqrt[3]{2})$ but does not split in $\mathbb{Q}(\sqrt[3]{2})$.  It follows that the extension $\mathbb{Q}(\sqrt[3]{2})/\mathbb{Q}$ is not normal.
%%%%%
%%%%%
\end{document}
