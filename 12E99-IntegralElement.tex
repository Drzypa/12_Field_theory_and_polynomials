\documentclass[12pt]{article}
\usepackage{pmmeta}
\pmcanonicalname{IntegralElement}
\pmcreated{2013-03-22 14:15:56}
\pmmodified{2013-03-22 14:15:56}
\pmowner{pahio}{2872}
\pmmodifier{pahio}{2872}
\pmtitle{integral element}
\pmrecord{31}{35715}
\pmprivacy{1}
\pmauthor{pahio}{2872}
\pmtype{Definition}
\pmcomment{trigger rebuild}
\pmclassification{msc}{12E99}
\pmrelated{PAdicCanonicalForm}
\pmrelated{PAdicValuation}
\pmrelated{KummersCongruence}

\endmetadata

% this is the default PlanetMath preamble.  as your knowledge
% of TeX increases, you will probably want to edit this, but
% it should be fine as is for beginners.

% almost certainly you want these
\usepackage{amssymb}
\usepackage{amsmath}
\usepackage{amsfonts}

% used for TeXing text within eps files
%\usepackage{psfrag}
% need this for including graphics (\includegraphics)
%\usepackage{graphicx}
% for neatly defining theorems and propositions
%\usepackage{amsthm}
% making logically defined graphics
%%%\usepackage{xypic}

% there are many more packages, add them here as you need them

% define commands here
\begin{document}
An element $a$ of a field $K$ is an {\em integral element of the field} $K$, iff    
                         $$|a| \leq 1$$
for every non-archimedean valuation\, $|\cdot|$\, of this field.

The set $\mathcal{O}$ of all integral elements of $K$ is a subring (in fact, an integral domain) of $K$, because it is the intersection of all valuation rings in $K$.

\textbf{Examples}
\begin{enumerate}

 \item $K = \mathbb{Q}$.\, The only non-archimedean valuations of $\mathbb{Q}$ are the $p$-adic valuations\, $|\cdot|_p$\, (where $p$ is a rational prime) and the trivial valuation (all values are 1 except the value of 0).\, The valuation ring $\mathcal{O}_p$ of\, $|\cdot|_p$\, consists of all so-called {\em p-integral rational numbers} whose denominators are not divisible by $p$.\, The valuation ring of the trivial valuation is, generally, the whole field.\, Thus, $\mathcal{O}$ is, by definition, the intersection of the $\mathcal{O}_p$'s for all $p$;\, this is the set of rationals whose denominators are not divisible by any prime, which is exactly the set $\mathbb{Z}$ of ordinary integers.

 \item If $K$ is a finite field, it has only the trivial valuation.\, In fact, if $|\cdot|$ is a valuation and $a$ any non-zero element of $K$, then there is a positive integer $m$ such that\, $a^m = 1$,\, and we have\, $|a|^m = |a^m| = |1| = 1$,\, and therefore\, $|a| = 1$.\, Thus, $|\cdot|$ is trivial and\, $\mathcal{O} = K$.\, This means that all elements of the field are integral elements.

 \item If $K$ is the field $\mathbb{Q}_p$ of the \PMlinkname{$p$-adic numbers}{NonIsomorphicCompletionsOfMathbbQ}, it has only one non-trivial valuation, the $p$-adic valuation, and now the ring $\mathcal{O}$ is its valuation ring, which is the ring of \PMlinkname{{\em $p$-adic integers}}{PAdicIntegers};\, this is visualized in the 2-adic ({\em dyadic}) case in the article ``$p$-adic canonical form''. 

\end{enumerate}
%%%%%
%%%%%
\end{document}
