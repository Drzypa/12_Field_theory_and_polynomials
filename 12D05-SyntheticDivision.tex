\documentclass[12pt]{article}
\usepackage{pmmeta}
\pmcanonicalname{SyntheticDivision}
\pmcreated{2013-03-22 14:07:27}
\pmmodified{2013-03-22 14:07:27}
\pmowner{mathwizard}{128}
\pmmodifier{mathwizard}{128}
\pmtitle{synthetic division}
\pmrecord{7}{35531}
\pmprivacy{1}
\pmauthor{mathwizard}{128}
\pmtype{Definition}
\pmcomment{trigger rebuild}
\pmclassification{msc}{12D05}

\endmetadata

\usepackage{amssymb}
\usepackage{amsmath}
\usepackage{amsfonts}
\begin{document}
\emph{Synthetic division} is simply a shorthand for long division, 
when one of the factors is linear.  
So when an equation such as $$y = 2x^2 + 10x + 12$$ 
is given and a factor of $x+3$ is known, we can do the following:

\begin{tabular}{lcrrr}
$\phantom{-3}$ &\vline& $2$ & $10$ & $12$\\
   &\vline&   &    &   \\
   \cline{2-5}
\end{tabular}

First, we can use $-3$ since $-3$ is a solution if $x+3=0$.

\begin{tabular}{lcrrr}
$-3$ &\vline& $2$ & $10$ & $12$\\
   &\vline&   &    &   \\
   \cline{2-5}
\end{tabular}

Carry down the first coefficient:

\begin{tabular}{lcrrr}
$-3$ &\vline& $2$ & $10$ & $12$\\
   &\vline&   &    &   \\
   \cline{2-5}
   && $2$  &    &   \\
\end{tabular}

Multiply $2$ by $-3$:

\begin{tabular}{lcrrr}
$-3$ &\vline& $2$ & $10$ & $12$\\
   &\vline&   & $-6$   &   \\
   \cline{2-5}
   && $2$  &    &   \\
\end{tabular}

Add up $10$ and $-6$:

\begin{tabular}{lcrrr}
$-3$ &\vline& $2$ & $10$ & $12$\\
   &\vline&   & $-6$   &   \\
   \cline{2-5}
   && $2$  & $4$   &   \\
\end{tabular}

Multiply $4$ by $-3$:

\begin{tabular}{lcrrr}
$-3$ &\vline& $2$ & $10$ & $12$\\
   &\vline&   & $-6$   & $-12$  \\
   \cline{2-5}
   && $2$  & $4$   & $0$  \\
\end{tabular}

This means that, by factoring the equation by $x+3$, we will get another factor as $2x+4$.

Synthetic is simply an variation of the long division --- a version that's manipulated and simplified to shorten the calculations.
%%%%%
%%%%%
\end{document}
