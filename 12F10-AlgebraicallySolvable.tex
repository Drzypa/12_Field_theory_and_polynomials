\documentclass[12pt]{article}
\usepackage{pmmeta}
\pmcanonicalname{AlgebraicallySolvable}
\pmcreated{2015-04-15 13:48:08}
\pmmodified{2015-04-15 13:48:08}
\pmowner{pahio}{2872}
\pmmodifier{pahio}{2872}
\pmtitle{algebraically solvable}
\pmrecord{9}{40361}
\pmprivacy{1}
\pmauthor{pahio}{2872}
\pmtype{Definition}
\pmcomment{trigger rebuild}
\pmclassification{msc}{12F10}
\pmsynonym{algebraic solvability}{AlgebraicallySolvable}
\pmsynonym{solvable algebraically}{AlgebraicallySolvable}
\pmrelated{RadicalExtension}
\pmrelated{KalleVaisala}

% this is the default PlanetMath preamble.  as your knowledge
% of TeX increases, you will probably want to edit this, but
% it should be fine as is for beginners.

% almost certainly you want these
\usepackage{amssymb}
\usepackage{amsmath}
\usepackage{amsfonts}

% used for TeXing text within eps files
%\usepackage{psfrag}
% need this for including graphics (\includegraphics)
%\usepackage{graphicx}
% for neatly defining theorems and propositions
 \usepackage{amsthm}
% making logically defined graphics
%%%\usepackage{xypic}

% there are many more packages, add them here as you need them

% define commands here

\theoremstyle{definition}
\newtheorem*{thmplain}{Theorem}

\begin{document}
An equation
\begin{align}
x^n+a_1x^{n-1}+\ldots+a_n = 0,
\end{align}
with coefficients $a_j$ in a field $K$, is {\it algebraically 
solvable}, if some of its \PMlinkname{roots}{Equation} may 
be expressed with the elements of $K$ by using rational 
operations (addition, subtraction, multiplication, division) 
and root extractions.\, I.e., a root of (1) is in a field\, 
$K(\xi_1,\,\xi_2,\,\ldots,\,\xi_m)$\, which is obtained of 
$K$ by \PMlinkname{adjoining}{FieldAdjunction} to it in 
succession certain suitable radicals 
$\xi_1,\,\xi_2,\,\ldots,\,\xi_m$.\, Each radical may 
\PMlinkescapetext{contain} under the root sign one or more of 
the previous radicals,
\begin{align*}
\begin{cases}
\xi_1 = \sqrt[p_1]{r_1},\\
\xi_2 = \sqrt[p_2]{r_2(\xi_1)},\\
\xi_3 = \sqrt[p_3]{r_3(\xi_1,\,\xi_2)},\\
\cdots\qquad\cdots\\
\xi_m = \sqrt[p_m]{r_m(\xi_1,\,\xi_2,\,\ldots,\,\xi_{m-1})},
\end{cases}
\end{align*}
where generally\, $r_k(\xi_1,\,\xi_2,\,\ldots,\,\xi_{k-1})$\, is an element of the field 
$K(\xi_1,\,\xi_2,\,\ldots,\,\xi_{k-1})$\, but no $p_k$'th power of an element of this field.\, Because of the formula
      $$\sqrt[jk]{r} = \sqrt[j]{\sqrt[k]{r}}$$
one can, without hurting the generality, suppose that the \PMlinkname{indices}{Root} $p_1,\,p_2,\,\ldots,\,p_m$ are prime numbers.\\

\textbf{Example.}\, Cardano's formulae show that all roots of the cubic equation\; $y^3+py+q = 0$\; are in the algebraic number field which is obtained by adjoining to the field\, $\mathbb{Q}(p,\,q)$\, successively the radicals
$$\xi_1 = \sqrt{\left(\frac{q}{2}\right)^2\!+\!\left(\frac{p}{3}\right)^3}, \qquad 
\xi_2 = \sqrt[3]{-\frac{q}{2}\!+\!\xi_1}, \qquad \xi_3 = \sqrt{-3}.$$
In fact, as we consider also the equation (4), the roots may be expressed as
\begin{align*}
\begin{cases}
\displaystyle y_1 = \xi_2-\frac{p}{3\xi_2}\\
\displaystyle y_2 = \frac{-1\!+\!\xi_3}{2}\cdot\xi_2-\frac{-1\!-\!\xi_3}{2}\cdot\!\frac{p}{3\xi_2}\\
\displaystyle y_3 = \frac{-1\!-\!\xi_3}{2}\cdot\xi_2-\frac{-1\!+\!\xi_3}{2}\cdot\!\frac{p}{3\xi_2}
\end{cases}
\end{align*}

\begin{thebibliography}{9}
\bibitem{K.V.}{\sc K. V\"ais\"al\"a:} {\em Lukuteorian ja korkeamman algebran alkeet}. \,Tiedekirjasto No. 17. \, Kustannusosakeyhti\"o Otava, Helsinki (1950).
\end{thebibliography}
%%%%%
%%%%%
\end{document}
