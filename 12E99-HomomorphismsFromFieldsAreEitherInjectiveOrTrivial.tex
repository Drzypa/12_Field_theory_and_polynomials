\documentclass[12pt]{article}
\usepackage{pmmeta}
\pmcanonicalname{HomomorphismsFromFieldsAreEitherInjectiveOrTrivial}
\pmcreated{2013-03-22 14:39:07}
\pmmodified{2013-03-22 14:39:07}
\pmowner{mathcam}{2727}
\pmmodifier{mathcam}{2727}
\pmtitle{homomorphisms from fields are either injective or trivial}
\pmrecord{4}{36243}
\pmprivacy{1}
\pmauthor{mathcam}{2727}
\pmtype{Corollary}
\pmcomment{trigger rebuild}
\pmclassification{msc}{12E99}

\endmetadata

% this is the default PlanetMath preamble.  as your knowledge
% of TeX increases, you will probably want to edit this, but
% it should be fine as is for beginners.

% almost certainly you want these
\usepackage{amssymb}
\usepackage{amsmath}
\usepackage{amsfonts}
\usepackage{amsthm}

% used for TeXing text within eps files
%\usepackage{psfrag}
% need this for including graphics (\includegraphics)
%\usepackage{graphicx}
% for neatly defining theorems and propositions
%\usepackage{amsthm}
% making logically defined graphics
%%%\usepackage{xypic}

% there are many more packages, add them here as you need them

% define commands here

\newcommand{\mc}{\mathcal}
\newcommand{\mb}{\mathbb}
\newcommand{\mf}{\mathfrak}
\newcommand{\ol}{\overline}
\newcommand{\ra}{\rightarrow}
\newcommand{\la}{\leftarrow}
\newcommand{\La}{\Leftarrow}
\newcommand{\Ra}{\Rightarrow}
\newcommand{\nor}{\vartriangleleft}
\newcommand{\Gal}{\text{Gal}}
\newcommand{\GL}{\text{GL}}
\newcommand{\Z}{\mb{Z}}
\newcommand{\R}{\mb{R}}
\newcommand{\Q}{\mb{Q}}
\newcommand{\C}{\mb{C}}
\newcommand{\<}{\langle}
\renewcommand{\>}{\rangle}
\begin{document}
Suppose $F$ is a field, $R$ is a ring, and $\phi\colon F\rightarrow R$ is a homomorphism of rings.  Then $\phi$ is either trivial or injective.

\begin{proof}
We use the fact that kernels of ring homomorphism are ideals.  Since $F$ is a field, by the above result, we have that the kernel of $\phi$ is an ideal of the field $F$ and hence either empty or all of $F$.  If the kernel is empty, then since a ring homomorphism is injective iff the kernel is trivial, we get that $\phi$ is injective.  If the kernel is all of $F$, then $\phi$ is the zero map from $F$ to $R$.
\end{proof}

Finally, it is clear that both of these possibilities are in fact achieved:
\begin{itemize}
\item The map $\phi:\Q\ra \Q$ given by $\phi(n)=0$ is trivial (has all of $\Q$ as a kernel)
\item The inclusion $\Q\ra \Q[x]$ is injective (i.e. the kernel is trivial).
\end{itemize}
%%%%%
%%%%%
\end{document}
