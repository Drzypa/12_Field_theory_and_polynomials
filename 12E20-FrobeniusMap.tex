\documentclass[12pt]{article}
\usepackage{pmmeta}
\pmcanonicalname{FrobeniusMap}
\pmcreated{2013-03-22 12:34:52}
\pmmodified{2013-03-22 12:34:52}
\pmowner{djao}{24}
\pmmodifier{djao}{24}
\pmtitle{Frobenius map}
\pmrecord{6}{32830}
\pmprivacy{1}
\pmauthor{djao}{24}
\pmtype{Definition}
\pmcomment{trigger rebuild}
\pmclassification{msc}{12E20}
\pmclassification{msc}{11T99}
\pmrelated{FrobeniusAutomorphism}

% this is the default PlanetMath preamble.  as your knowledge
% of TeX increases, you will probably want to edit this, but
% it should be fine as is for beginners.

% almost certainly you want these
\usepackage{amssymb}
\usepackage{amsmath}
\usepackage{amsfonts}

% used for TeXing text within eps files
%\usepackage{psfrag}
% need this for including graphics (\includegraphics)
%\usepackage{graphicx}
% for neatly defining theorems and propositions
%\usepackage{amsthm}
% making logically defined graphics
%%%\usepackage{xypic} 

% there are many more packages, add them here as you need them

% define commands here
\newcommand{\Frob}{\operatorname{Frob}}
\begin{document}
Let $K$ be any field of characteristic $p > 0$, and suppose $K$ contains the finite field $\mathbb{F}_q$ of size $q$, where $q = p^r$. The $q^{\rm th}$ power Frobenius map on $K$ is the map $\Frob_q: K \longrightarrow K$ defined by $\Frob_q(x) := x^q$.

If $K$ is perfect, then $\Frob_q$ is an automorphism of $K$ which fixes $\mathbb{F}_q$, and accordingly is a member of the Galois group $\operatorname{Gal}(K/\mathbb{F}_q)$.
%%%%%
%%%%%
\end{document}
