\documentclass[12pt]{article}
\usepackage{pmmeta}
\pmcanonicalname{ExampleOfSolvingACubicEquation}
\pmcreated{2015-11-18 14:15:18}
\pmmodified{2015-11-18 14:15:18}
\pmowner{pahio}{2872}
\pmmodifier{pahio}{2872}
\pmtitle{example of solving a cubic equation}
\pmrecord{14}{40335}
\pmprivacy{1}
\pmauthor{pahio}{2872}
\pmtype{Example}
\pmcomment{trigger rebuild}
\pmclassification{msc}{12D10}
\pmsynonym{example of using Cardano's formulas}{ExampleOfSolvingACubicEquation}
\pmrelated{PolynomialEquationOfOddDegree}
\pmrelated{ConjugatedRootsOfEquation2}

\endmetadata

% this is the default PlanetMath preamble.  as your knowledge
% of TeX increases, you will probably want to edit this, but
% it should be fine as is for beginners.

% almost certainly you want these
\usepackage{amssymb}
\usepackage{amsmath}
\usepackage{amsfonts}

% used for TeXing text within eps files
%\usepackage{psfrag}
% need this for including graphics (\includegraphics)
%\usepackage{graphicx}
% for neatly defining theorems and propositions
 \usepackage{amsthm}
% making logically defined graphics
%%%\usepackage{xypic}

% there are many more packages, add them here as you need them

% define commands here

\theoremstyle{definition}
\newtheorem*{thmplain}{Theorem}

\begin{document}
Let us use Cardano's formulae for solving algebraically the cubic equation 
\begin{align}
x^3\!+\!3x^2\!-\!1 \;=\; 0.
\end{align}
First apply the \PMlinkname{Tschirnhaus transformation}{CardanosDerivationOfTheCubicFormula}\, $x := y\!-\!1$\, for removing the quadratic term; from\, $(y-1)^3+3(y-1)^2-1 = 0$\, we get the simplified equation
\begin{align}
y^3\!+\!3y\!-\!2 \;=\; 0.
\end{align}
We now suppose that\, $y := u\!+\!v$.\, Substituting this into (2) and rewriting the equation in the form
$$(u^3\!+\!v^3\!-\!2)+3(uv\!+\!1)(u\!+\!v) \;=\; 0,$$
one can determine $u$ and $v$ such that\, $u^3\!+\!v^3\!-\!2 = 0$\, and\, $uv\!+\!1 = 0$,\, i.e. 
\begin{align*}
\begin{cases}
u^3\!+\!v^3 \;=\; 2,\\
u^3v^3 \;=\; -1.
\end{cases}
\end{align*}
Using the properties of quadratic equation, we infer that $u^3$ and $v^3$ are the roots of the resolvent equation
$$z^2\!-\!2z\!-\!1 \;=\; 0.$$
Therefore, $u$ and $v$ satisfy the binomial equations
\begin{align}
u^3 \;=\; 1\!+\!\sqrt{2}, \qquad v^3 \;=\; 1\!-\!\sqrt{2},
\end{align}
respectively.\, If we choose the real radicals \,$u = u_0 = \sqrt[3]{1\!+\!\sqrt{2}}$\, and\, 
$v = v_0 = \sqrt[3]{1\!-\!\sqrt{2}}$,\, the other solutions $u,\,v$ of (3) are
\begin{align}
\zeta u_0,\;\;\zeta^2v_0; \qquad \zeta^2u_0,\;\;\zeta v_0,
\end{align}
where\; $\zeta = \frac{-1+i\sqrt{3}}{2}, 
\quad \zeta^2 = \frac{-1-i\sqrt{3}}{2}$\;
are the primitive third roots of unity.\,  One must combine the pairs\, $(u,\,v)$\, of (4) so that
$$uv \;=\; \sqrt[3]{u^3v^3} \;=\; -1.$$
Accordingly, all three roots of the cubic equation (2) are
\begin{align*}
\begin{cases}
y_1 \;=\; u_0\!+\!v_0 \;=\; \sqrt[3]{1\!+\!\sqrt{2}}\!+\!\sqrt[3]{1\!-\!\sqrt{2}},\\ 
y_2 \;=\; \zeta u_0\!+\!\zeta^2v_0 \;=\;  
  \frac{-1+i\sqrt{3}}{2}\sqrt[3]{1\!+\!\sqrt{2}}\!+\!\frac{-1-i\sqrt{3}}{2}\sqrt[3]{1\!-\!\sqrt{2}},\\
y_3 \;=\; \zeta^2u_0\!+\!\zeta v_0 \;=\;  
  \frac{-1-i\sqrt{3}}{2}\sqrt[3]{1\!+\!\sqrt{2}}\!+\!\frac{-1+i\sqrt{3}}{2}\sqrt[3]{1\!-\!\sqrt{2}}.\\
\end{cases}
\end{align*}
The roots of the original equation (1) are gotten via the used substitution equation\, $x := y-1$, i.e. adding $-1$ to the values of $y$.\, If we also separate the \PMlinkname{real}{RealPart} and imaginary parts, we have the following solution of (1):
\begin{align*}
\begin{cases}
x_1 \;=\; -1\!+\!\sqrt[3]{1\!+\!\sqrt{2}}\!+\!\sqrt[3]{1\!-\!\sqrt{2}},\\ 
x_2 \;=\; -1\!-\!\frac{1}{2}\!\left(\sqrt[3]{1\!+\!\sqrt{2}}\!+\!\sqrt[3]{1\!-\!\sqrt{2}}\right)\!
+i\frac{\sqrt{3}}{2}\!\left(\sqrt[3]{1\!+\!\sqrt{2}}\!-\!\sqrt[3]{1\!-\!\sqrt{2}}\right),\\
x_3 \;=\; -1\!-\!\frac{1}{2}\!\left(\sqrt[3]{1\!+\!\sqrt{2}}\!+\!\sqrt[3]{1\!-\!\sqrt{2}}\right)\!
-i\frac{\sqrt{3}}{2}\!\left(\sqrt[3]{1\!+\!\sqrt{2}}\!-\!\sqrt[3]{1\!-\!\sqrt{2}}\right).
\end{cases}
\end{align*}
One of the roots is a real number, but the other two are \PMlinkescapetext{imaginary} (i.e. non-real) complex conjugates of each other.
%%%%%
%%%%%
\end{document}
