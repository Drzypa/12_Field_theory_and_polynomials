\documentclass[12pt]{article}
\usepackage{pmmeta}
\pmcanonicalname{QuadraticClosure}
\pmcreated{2013-03-22 15:42:43}
\pmmodified{2013-03-22 15:42:43}
\pmowner{CWoo}{3771}
\pmmodifier{CWoo}{3771}
\pmtitle{quadratic closure}
\pmrecord{8}{37659}
\pmprivacy{1}
\pmauthor{CWoo}{3771}
\pmtype{Definition}
\pmcomment{trigger rebuild}
\pmclassification{msc}{12F05}
\pmdefines{quadratically closed}

\usepackage{amssymb,amscd}
\usepackage{amsmath}
\usepackage{amsfonts}

% used for TeXing text within eps files
%\usepackage{psfrag}
% need this for including graphics (\includegraphics)
%\usepackage{graphicx}
% for neatly defining theorems and propositions
%\usepackage{amsthm}
% making logically defined graphics
%%%\usepackage{xypic}

% define commands here
\begin{document}
A field $K$ is said to be \emph{quadratically closed} if it has no quadratic extensions.  In other words, every element of $K$ is a square.  Two obvious examples are $\mathbb{C}$ and $\mathbb{F}_2$.

A field $K$ is said to be a \emph{quadratic closure} of another field $k$ if
\begin{enumerate}
\item $K$ is quadratically closed, and 
\item among all quadratically closed subfields of the algebraic closure $\overline{k}$ of $k$, $K$ is the smallest one.
\end{enumerate}

By the second condition, a quadratic closure of a field is unique up to field isomorphism.  So we say \emph{the} quadratic closure of a field $k$, and we denote it by $\widetilde{k}$ Alternatively, the second condition on $K$ can be replaced by the following:

\begin{quote}
$K$ is the smallest field extension over $k$ such that, if $L$ is any field extension over $k$ obtained by a finite number of quadratic extensions starting with $k$, then $L$ is a subfield of $K$.
\end{quote}

\textbf{Examples.}  
\begin{itemize}
\item $\mathbb{C}=\widetilde{\mathbb{R}}$.  
\item If $\mathbb{E}$ is the Euclidean field in $\mathbb{R}$, then the quadratic extension $\mathbb{E}(\sqrt{-1})$ is the quadratic closure $\widetilde{\mathbb{Q}}$ of the rational numbers $\mathbb{Q}$.
\item If $k=\mathbb{F}_5$, consider the chain of fields
$$k\le k(\sqrt{2})\le k(\sqrt[4]{2})\le \cdots \le k(\sqrt[2^n]{2})\le \cdots $$
Take the union of all these fields to obtain a field $K$.  Then it can be shown that $K=\widetilde{k}$.
\end{itemize}
%%%%%
%%%%%
\end{document}
