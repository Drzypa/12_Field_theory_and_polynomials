\documentclass[12pt]{article}
\usepackage{pmmeta}
\pmcanonicalname{AConditionOfAlgebraicExtension}
\pmcreated{2013-03-22 17:53:34}
\pmmodified{2013-03-22 17:53:34}
\pmowner{pahio}{2872}
\pmmodifier{pahio}{2872}
\pmtitle{a condition of algebraic extension}
\pmrecord{7}{40379}
\pmprivacy{1}
\pmauthor{pahio}{2872}
\pmtype{Theorem}
\pmcomment{trigger rebuild}
\pmclassification{msc}{12F05}
\pmrelated{RingAdjunction}
\pmrelated{FieldAdjunction}
\pmrelated{Overring}
\pmrelated{AConditionOfSimpleExtension}
\pmrelated{SteinitzTheoremOnFiniteFieldExtension}

% this is the default PlanetMath preamble.  as your knowledge
% of TeX increases, you will probably want to edit this, but
% it should be fine as is for beginners.

% almost certainly you want these
\usepackage{amssymb}
\usepackage{amsmath}
\usepackage{amsfonts}

% used for TeXing text within eps files
%\usepackage{psfrag}
% need this for including graphics (\includegraphics)
%\usepackage{graphicx}
% for neatly defining theorems and propositions
 \usepackage{amsthm}
% making logically defined graphics
%%%\usepackage{xypic}

% there are many more packages, add them here as you need them

% define commands here

\theoremstyle{definition}
\newtheorem*{thmplain}{Theorem}

\begin{document}
\textbf{Theorem.}\, A field extension $L/K$ is \PMlinkname{algebraic}{AlgebraicExtension} if and only if any subring of the extension field $L$ containing the base field $K$ is a field.\\

{\em Proof.}\, Assume first that $L/K$ is algebraic.\, Let $R$ be a subring of $L$ containing $K$.\, For any non-zero element $r$ of $R$, naturally\, $K[r] \subseteq R$,\, and since $r$ is an algebraic element over $K$, the ring $K[r]$ coincides with the field $K(r)$.\, Therefore we have\, $r^{-1} \in K[r] \subseteq R$,\, and $R$ must be a field.

Assume then that each subring of $L$ which contains $K$ is a field.\, Let $a$ be any non-zero element of $L$.\, Accordingly, the subring $K[a]$ of $L$ contains $K$ and is a field.\, So we have\, $a^{-1} \in K[a]$.\, This means that there is a polynomial $f(x)$ in the polynomial ring $K[x]$ such that\, $a^{-1} = f(a)$.\, Because\, $af(a)-1 = 0$,\, the element $a$ is a zero of the polynomial $xf(x)-1$ of $K[x]$, i.e. is algebraic over $K$.\, Thus every element of $L$ is algebraic over $K$.

\begin{thebibliography}{9}
\bibitem{D.B.}{\sc David M. Burton}: {\em A first course in rings and ideals}. Addison-Wesley Publishing Company. Reading, Menlo Park, London, Don Mills (1970).
\end{thebibliography}

%%%%%
%%%%%
\end{document}
