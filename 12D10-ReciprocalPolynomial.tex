\documentclass[12pt]{article}
\usepackage{pmmeta}
\pmcanonicalname{ReciprocalPolynomial}
\pmcreated{2013-03-22 13:36:33}
\pmmodified{2013-03-22 13:36:33}
\pmowner{rspuzio}{6075}
\pmmodifier{rspuzio}{6075}
\pmtitle{reciprocal polynomial}
\pmrecord{12}{34239}
\pmprivacy{1}
\pmauthor{rspuzio}{6075}
\pmtype{Definition}
\pmcomment{trigger rebuild}
\pmclassification{msc}{12D10}
\pmrelated{CharacteristicPolynomialOfASymplecticMatrixIsAReciprocalPolynomial}

\endmetadata

% this is the default PlanetMath preamble.  as your knowledge
% of TeX increases, you will probably want to edit this, but
% it should be fine as is for beginners.

% almost certainly you want these
\usepackage{amssymb}
\usepackage{amsmath}
\usepackage{amsfonts}

% used for TeXing text within eps files
%\usepackage{psfrag}
% need this for including graphics (\includegraphics)
%\usepackage{graphicx}
% for neatly defining theorems and propositions
%\usepackage{amsthm}
% making logically defined graphics
%%%\usepackage{xypic}

% there are many more packages, add them here as you need them

% define commands here
\begin{document}
\PMlinkescapeword{name}
\PMlinkescapeword{degree}

{\bf Definition} \cite{eves}
Let $p:\mathbb{C}\to\mathbb{C}$
be a polynomial of degree $n$ with complex (or real)
coefficients. Then $p$  is a \emph{reciprocal polynomial} if
$$
   p(z) = \pm z^n p(1/z)
$$
for all $z\in \mathbb{C}$.

Examples of reciprocal polynomials are Gaussian polynomials, as well as the characteristic polynomials of orthogonal matrices (including the identity matrix as a special case), symplectic matrices, \PMlinkname{involution matrices}{LinearInvolution}, and the Pascal matrices \cite{higham}. 

It is clear that if $z$ is a zero for a reciprocal polynomial, then
$1/z$ is also a zero. This property motivates the name. This
means that the spectra of matrices of above type is symmetric
with respect to the unit circle in $\mathbb{C}$; if $\lambda\in \mathbb{C}$ is an
eigenvalue, so is $1/\lambda$. 

The sum, difference, and product of two reciprocal polynomials is again a reciprocal polynomial.  Hence, reciprocal polynomials form an algebra over the complex numbers.

\begin{thebibliography}{9}
\bibitem {eves} H. Eves,
 \emph{Elementary Matrix Theory},
 Dover publications, 1980.
\bibitem{higham} N.J. Higham, \emph{Accuracy and Stability of Numerical Algorithms}, 
2nd ed., SIAM, 2002. 
\end{thebibliography}
%%%%%
%%%%%
\end{document}
