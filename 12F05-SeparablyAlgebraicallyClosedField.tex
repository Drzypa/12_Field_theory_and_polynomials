\documentclass[12pt]{article}
\usepackage{pmmeta}
\pmcanonicalname{SeparablyAlgebraicallyClosedField}
\pmcreated{2013-03-22 15:58:30}
\pmmodified{2013-03-22 15:58:30}
\pmowner{polarbear}{3475}
\pmmodifier{polarbear}{3475}
\pmtitle{separably algebraically closed field}
\pmrecord{6}{37991}
\pmprivacy{1}
\pmauthor{polarbear}{3475}
\pmtype{Definition}
\pmcomment{trigger rebuild}
\pmclassification{msc}{12F05}
\pmdefines{separably algebraically closed}

% this is the default PlanetMath preamble.  as your knowledge
% of TeX increases, you will probably want to edit this, but
% it should be fine as is for beginners.

% almost certainly you want these
\usepackage{amssymb}
\usepackage{amsmath}
\usepackage{amsfonts}

% used for TeXing text within eps files
%\usepackage{psfrag}
% need this for including graphics (\includegraphics)
%\usepackage{graphicx}
% for neatly defining theorems and propositions
%\usepackage{amsthm}
% making logically defined graphics
%%%\usepackage{xypic}

% there are many more packages, add them here as you need them

% define commands here

\begin{document}
A field $K$ is called \emph{separably algebraically closed} if every separable element of the algebraic closure of $K$ belongs to $K$.\newline
 In the case when $K$ has characteristic 0, it is separably algebraically closed if and only if it is algebraically closed.\newline If $K$ has positive characteristic $p$, $K$ is separably algebraically closed if and only if its algebraic closure is a purely inseparable extension of $K$.


%%%%%
%%%%%
\end{document}
