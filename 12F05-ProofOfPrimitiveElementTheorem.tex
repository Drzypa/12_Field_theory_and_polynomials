\documentclass[12pt]{article}
\usepackage{pmmeta}
\pmcanonicalname{ProofOfPrimitiveElementTheorem}
\pmcreated{2013-03-22 14:16:27}
\pmmodified{2013-03-22 14:16:27}
\pmowner{alozano}{2414}
\pmmodifier{alozano}{2414}
\pmtitle{proof of primitive element theorem}
\pmrecord{8}{35725}
\pmprivacy{1}
\pmauthor{alozano}{2414}
\pmtype{Proof}
\pmcomment{trigger rebuild}
\pmclassification{msc}{12F05}

\endmetadata

% this is the default PlanetMath preamble.  as your knowledge
% of TeX increases, you will probably want to edit this, but
% it should be fine as is for beginners.

% almost certainly you want these
\usepackage{amssymb}
\usepackage{amsmath}
\usepackage{amsthm}
\usepackage{amsfonts}

% used for TeXing text within eps files
%\usepackage{psfrag}
% need this for including graphics (\includegraphics)
%\usepackage{graphicx}
% for neatly defining theorems and propositions
%\usepackage{amsthm}
% making logically defined graphics
%%%\usepackage{xypic}

% there are many more packages, add them here as you need them

% define commands here

\newtheorem*{theorem}{Theorem}
\newtheorem{defn}{Definition}
\newtheorem{prop}{Proposition}
\newtheorem{lemma}{Lemma}
\newtheorem{cor}{Corollary}

\DeclareMathOperator{\Gal}{Gal}
\begin{document}
\begin{theorem}
Let $F$ and $K$ be arbitrary fields, and let $K$ be an extension of $F$ of finite degree. Then there exists an element $\alpha\in K$ such that $K=F(\alpha)$ if and only if there are finitely many fields $L$ with $F\subseteq L\subseteq K$.
\end{theorem}

\begin{proof}
Let $F$ and $K$ be fields, and let $[K:F]=n$ be finite. 

Suppose first that $K=F(\alpha)$. Since $K/F$ is finite, $\alpha$ is algebraic over $F$. Let $m(x)$ be the minimal polynomial of $\alpha$ over $F$. Now, let $L$ be an intermediary field with $F\subseteq L\subseteq K$ and let $m'(x)$ be the minimal polynomial of $\alpha$ over $L$. Also, let $L'$ be the field generated by the coefficients of the polynomial $m'(x)$. Thus, the minimal polynomial of $\alpha$ over $L'$ is still $m'(x)$ and $L'\subseteq L$. By the properties of the minimal polynomial, and since $m(\alpha)=0$, we have a divisibility $m'(x)|m(x)$, and so:
$$[K:L]=\deg(m'(x))=[K:L'].$$ 
Since we know that $L'\subseteq L$, this implies that $L'=L$. Thus, this shows that each intermediary subfield $F\subseteq L \subseteq K$ corresponds with the field of definition of a (monic) factor of $m(x)$. Since the polynomial $m(x)$ has only finitely many monic factors, we conclude that there can be only finitely many subfields of $K$ containing $F$.


Now suppose conversely that there are only finitely many such intermediary fields $L$.  If $F$ is a finite field, then so is $K$, and we have an explicit description of all such possibilities; all such extensions are generated by a single element.  So assume $F$ (and therefore $K$) are infinite.  Let $\alpha_1, \alpha_2, \ldots, \alpha_n$ be a basis for $K$ over $F$.  Then $K=F(\alpha_1, \ldots, \alpha_n)$.  So if we can show that any field extension generated by two elements is also generated by one element, we will be done: simply apply the result to the last two elements $\alpha_{j-1}$ and $\alpha_j$ repeatedly until only one is left.

So assume $K=F(\beta,\gamma)$.  Consider the set of elements $\{\beta+a\gamma\}$ for $a\in F^{\times}$.  By assumption, this set is infinite, but there are only finitely many fields intermediate between $K$ and $F$; so two values must generate the same extension $L$ of $F$, say $\beta+a\gamma$ and $\beta+b\gamma$.  This field $L$ contains
\[
\frac{(\beta+a\gamma)-(\beta+b\gamma)}{a-b} = \gamma
\] 
and
\[
\frac{(\beta+a\gamma)/a-(\beta+b\gamma)/b}{1/a-1/b} = \beta
\] 
and so letting $\alpha = \beta+a\gamma$, we see that 
\[
F(\alpha)=L=F(\beta,\gamma)=K.
\]
\end{proof}
%%%%%
%%%%%
\end{document}
