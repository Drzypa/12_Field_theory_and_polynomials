\documentclass[12pt]{article}
\usepackage{pmmeta}
\pmcanonicalname{DeMoivreIdentityProofOf}
\pmcreated{2013-03-22 14:34:08}
\pmmodified{2013-03-22 14:34:08}
\pmowner{rspuzio}{6075}
\pmmodifier{rspuzio}{6075}
\pmtitle{de Moivre identity, proof of}
\pmrecord{10}{36126}
\pmprivacy{1}
\pmauthor{rspuzio}{6075}
\pmtype{Proof}
\pmcomment{trigger rebuild}
\pmclassification{msc}{12E10}
\pmsynonym{proof of de Moivre's formula}{DeMoivreIdentityProofOf}
\pmsynonym{proof of de Moivre's theorem}{DeMoivreIdentityProofOf}
\pmrelated{AngleSumIdentity}

\endmetadata

% this is the default PlanetMath preamble.  as your knowledge
% of TeX increases, you will probably want to edit this, but
% it should be fine as is for beginners.

% almost certainly you want these
\usepackage{amssymb}
\usepackage{amsmath}
\usepackage{amsfonts}

% used for TeXing text within eps files
%\usepackage{psfrag}
% need this for including graphics (\includegraphics)
%\usepackage{graphicx}
% for neatly defining theorems and propositions
%\usepackage{amsthm}
% making logically defined graphics
%%%\usepackage{xypic}

% there are many more packages, add them here as you need them

% define commands here
\begin{document}
\PMlinkescapeword{identity}
\PMlinkescapeword{side}
\PMlinkescapeword{sides}

To prove the de Moivre identity, we will first prove by induction on $n$ 
that the identity holds for all natural numbers.  

For the case $n=0$, observe that
\[
\cos (0\theta) + i \sin (0\theta) = 1 + i 0 = \left(\cos (\theta) + i \sin (\theta)\right)^0.
\]

Assume that the identity holds for a certain value of $n$:
\[
\cos (n \theta) + i \sin (n \theta) = \left( \cos (\theta) + i \sin (\theta) \right)^n.
\]
Multiply both sides of this identity by $\cos (\theta) + i \sin (\theta)$ 
and expand the left side to obtain
\begin{align*}
\cos (\theta) \cos (n \theta) - \sin (\theta) \sin (n \theta) + i \cos (\theta) \sin (n \theta) + i \sin (\theta) \cos (n \theta)
&= \left( \cos (\theta) + i \sin (\theta) \right) \left( \cos (n \theta) + i \sin (n \theta) \right) \\
&= \left( \cos (\theta) + i \sin (\theta) \right)^{n+1}.
\end{align*}
By the angle sum identities,
\begin{align*}
\cos (\theta) \cos (n \theta) - \sin (\theta) \sin (n \theta) &= \cos (n \theta + \theta) \\
\cos (\theta) \sin (n \theta) + \sin (\theta) \cos (n \theta) &= \sin (n \theta + \theta)
\end{align*}
Therefore,
\[
\cos ((n + 1) \theta) + i \sin ((n + 1) \theta) = \left( \cos (\theta) + i \sin (\theta) \right)^{n+1}.
\]

Hence by induction de Moivre's identity holds for all natural $n$.

Now let $-n$ be any negative integer.  Then using the fact that $\cos$ is an even and $\sin$ an odd function, we obtain that
\begin{align*}
\cos (-n \theta) + i \sin (-n \theta) 
&= \cos (n \theta) - i \sin (n \theta) \\
&= \frac{\cos (n \theta) - i \sin (n \theta)}{\cos^2 (n \theta) + \sin^2 (n \theta)} \\
&= \frac{1}{\cos (n \theta) + i \sin (n \theta)} \cdot
   \frac{\cos (n \theta) - i \sin (n \theta)}{\cos (n \theta) - i \sin (n \theta)} \\
&= \frac{1}{\cos (n \theta) + i \sin (n \theta)},
\end{align*}
the denominator of which is $\left( \cos (n \theta) + i \sin ( n \theta) \right)^n$.  Hence 
\[
\cos (-n \theta) + i \sin (-n \theta) = \left( \cos (\theta) + i \sin (\theta) \right)^{-n}.
\]
%%%%%
%%%%%
\end{document}
