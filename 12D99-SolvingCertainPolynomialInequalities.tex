\documentclass[12pt]{article}
\usepackage{pmmeta}
\pmcanonicalname{SolvingCertainPolynomialInequalities}
\pmcreated{2013-03-22 16:57:23}
\pmmodified{2013-03-22 16:57:23}
\pmowner{CWoo}{3771}
\pmmodifier{CWoo}{3771}
\pmtitle{solving certain polynomial inequalities}
\pmrecord{20}{39226}
\pmprivacy{1}
\pmauthor{CWoo}{3771}
\pmtype{Feature}
\pmcomment{trigger rebuild}
\pmclassification{msc}{12D99}
\pmrelated{QuadraticInequality}

\endmetadata

\usepackage{amssymb,amscd}
\usepackage{amsmath}
\usepackage{amsfonts}
\usepackage{pstricks}

% used for TeXing text within eps files
%\usepackage{psfrag}
% need this for including graphics (\includegraphics)
%\usepackage{graphicx}
% for neatly defining theorems and propositions
\usepackage{amsthm}
% making logically defined graphics
%%\usepackage{xypic}
\usepackage{pst-plot}
\usepackage{psfrag}

% define commands here
\newtheorem{prop}{Proposition}
\newtheorem{thm}{Theorem}
\newtheorem{ex}{Example}
\newcommand{\real}{\mathbb{R}}
\begin{document}
In this article, we discuss inequality of the form $p(x)\ge 0$ or $p(x)>0$, where $p(x)$ is a polynomial with real coefficients such that $p(x)$ can be expressed as product of linear factors:
$$p(x)=(x-a_1)(x-a_2)\cdots (x-a_n)$$
where $a_i$ are real numbers (in the \PMlinkescapetext{language} of field theory, this means that $p(x)$ splits in the field of real numbers).

When we plot the polynomial $y=p(x)$, whenever it crosses the $x$-axis, the crossing point is a root of $p(x)$.  On either side of the crossing point, the values of $p(x)$ may be negative or positive.  The way to solve the inequality $p(x)>0$ or $p(x)\ge 0$ is to look at how $p(x)$ crosses the $x$-axis, and to realize that when $x$ is very large, $p(x)$ will be positive.  This idea can be illustrated in the following figure:
\begin{center}
\begin{pspicture}(-7,-2)(7,3)
\psset{unit=0.8cm}
\rput[l](-7,0){.}
\rput[r](7,0){.}
\rput[a](3,-2){.}
\rput[b](-3.7,3.1){.}
\psline{<->}(-7,0)(7,0)
\pscurve{<->}(-6,-1)(-4,3)(-3,3)(0,0)(1,1)(3,-2)(6,1)
\rput[b](6.74,-0.4){$x$}
\rput[b](-1.25,2){$p(x)$}
\rput[b](6.5,0.25){$+$}
\rput[b](3.5,0.25){$-$}
\rput[b](1,0.25){$+$}
\rput[b](0,0.25){$-$}
\rput[b](-3.5,0.25){$+$}
\rput[b](-6.5,0.25){$-$}
\rput[b](5.75,-0.5){$a_1$}
\rput[b](2.4,-0.5){$a_2$}
\rput[b](0,-0.5){$a_3$}
\rput[b](-5.25,-0.5){$a_4$}
\end{pspicture}
\end{center}
If we start from the far right, the curve (graph of $p(x)$) is above the horizontal axis, and so the values of $p(x)$ are positive there.  As it approaches $a_1$, its values decrease until it crosses $a_1$ and dips below the horizontal axis.  The values of $p(x)$ are now negative.  As it continues to travel along to the left, its values increase until it passes over another crossing point, $a_2$, and the values become positive as soon as it passes $a_2$.  When $p(x)$ reaches $a_3$ however, it merely touches the $x$-axis (at $a_3$) and then rises again.  So on either side of $a_3$ the values of $p(x)$ are positive.  Nevertheless, an analysis of how $p(x)$ crosses the $x$-axis is enough to give us some clue on how to solve inequalities of the form $p(x)\ge 0$ or $p(x)>0$.

With this idea in mind, the steps are devised when solving inequalities of this type:
\begin{enumerate}
\item arrange $a_i$ so that they are in the ascending order: $a_1\ge a_2\ge \cdots \ge a_n$
\item plot $a_i$ on the number line (the real axis), so each $a_i$ is now a point on the line
\item label above the interval $i_1:=(a_1,\infty)$ to the right of $a_1$ positive
\item go to point $a_2$
\item if $a_2\ne a_1$, label above the interval $i_2:=(a_2,a_1)$ adjacent to $i_1$ negative
\item if $a_2=a_1$, label above $a_2$ negative
\item go to $a_3$, and iterate the labeling processes 4-6
\item stop when $(-\infty,a_n)$ is labeled.
\item if we are trying to solve $p(x)\ge 0$, then all intervals that are labeled positive, including the end points, are solutions to the inequality
\item if we are solving $p(x)>0$, then all intervals labeled positive, excluding the end points, are solutions to the inequality.
\end{enumerate}

\textbf{Remark}.  After all the labeling is done, there should a total of $n+1$ labels, one over each interval, including the null intervals (the points).

To see how this works, let us look at some actual examples.
\begin{itemize}
\item Solve $(x-2)x(x+3)\ge 0$.
\begin{enumerate}
\item Plot $2$, $0$, $-3$ on the number line. 
\item The intervals separated by these points are $(2,\infty),(0,2),(-3,0),(-\infty,-3)$.
\item Since no two points are the same, the intervals that are labeled positive are $(2,\infty)$ and $(-3,0)$.
\item The solutions to the inequality are $[2,\infty) \cup [-3,0]$, the square brackets signify that the end points are included in the solutions.
\end{enumerate}
\begin{center}
\begin{pspicture}(-7,-2)(7,2)
\psset{unit=0.8cm}
\psline{<->}(-7,0)(7,0)
\psline[linewidth=1.5pt]{->}(2,0)(7,0)
\psline[linewidth=1.5pt](-3,0)(0,0)
\psdots[dotscale=1.5](2,0)(0,0)(-3,0)
\rput[b](2,-0.75){$2$}
\rput[b](0,-0.75){$0$}
\rput[b](-3,-0.75){$-3$}
\rput[b](5,0.25){$+$}
\rput[b](1,0.25){$-$}
\rput[b](-1.5,0.25){$+$}
\rput[b](-5,0.25){$-$}
\end{pspicture}
\end{center}
\item Solve $(x-8)^7>0$.
\begin{enumerate}
\item Plot $8$ on the number line. 
\item The intervals separated by these points are $(8,\infty),(-\infty,8)$, since $8$ is repeated $7$ times.
\item Start with labeling $(8,\infty)$ positive (the first label)
\item The point $8$ is then labeled alternately $-(2), +(3), -(4), +(5), -(6), +(7)$.
\item The last label goes to $(-\infty,8)$, which is negative.
\item Therefore, the solution set is $(8,\infty)$ (excluding $8$).
\end{enumerate}
\begin{center}
\begin{pspicture}(-7,-2)(7,2)
\psset{unit=0.8cm}
\psline{<->}(-7,0)(7,0)
\psline[linewidth=1.5pt]{->}(1,0)(7,0)
\psdots[dotstyle=o,dotscale=1.5](1,0)
\rput[b](1,-0.75){$8$}
\rput[b](3,0.25){$+$}
\rput[b](1,0.25){$-$}
\rput[b](1,0.5){$+$}
\rput[b](1,0.75){$-$}
\rput[b](1,1){$+$}
\rput[b](1,1.25){$-$}
\rput[b](1,1.5){$+$}
\rput[b](-1,0.25){$-$}
\end{pspicture}
\end{center}
\item Solve $(x-1)(x+1)^4\ge 0$.
\begin{enumerate}
\item Plot $1,-1$ on the number line. 
\item The intervals separated by these points are $(1,\infty),(-1,1)$, and $(-\infty,-1)$, since $-1$ is repeated $4$ times.
\item Start with labeling $(1,\infty)$ positive (the first label), followed by $(-1,1)$ as negative.
\item The point $-1$ is then labeled alternately $+(3), -(4), +(5)$.
\item The last label goes to $(-\infty,-1)$, which is negative.
\item Therefore, the solution set is $[1,\infty) \cup \lbrace -1\rbrace$.
\end{enumerate}
\begin{center}
\begin{pspicture}(-7,-2)(7,2)
\psset{unit=0.8cm}
\psline{<->}(-7,0)(7,0)
\psline[linewidth=1.5pt]{->}(0,0)(7,0)
\psdots[dotscale=1.5](-2,0)(0,0)
\rput[b](-2,-0.75){$-1$}
\rput[b](0,-0.75){$1$}
\rput[b](2,0.25){$+$}
\rput[b](-1,0.25){$-$}
\rput[b](-2,0.25){$+$}
\rput[b](-2,0.5){$-$}
\rput[b](-2,0.75){$+$}
\rput[b](-4,0.25){$-$}
\end{pspicture}
\end{center}
\end{itemize}

\textbf{Remarks}.  
\begin{enumerate}
\item
In the last two examples, we observe that a simplification can be made when solving the inequality: whenever we have repeating roots ($a_i=a_{i+1}$).  Depending on the parity of the number of repeating roots, we have two situations:
\begin{itemize}
\item If the number $n_i$ of repeating root, say $a_i$, is odd, then solving inequality involving $(x-a_i)^{n_i}$ is the same as solving the same inequality with $(x-a_i)^{n_i}$ replaced by $(x-a_i)$.  In other words, their solution sets are the same.  For example, 
\begin{center} solving $(x-8)^7>0$ is the same as solving $(x-8)>0$. \end{center}
\item If the number $n_i$ of repeating root $a_i$ is even, then we look at whether the inequality is strict or not.
\begin{itemize}
\item If the inequality is strict, then we can completely eliminate $(x-a_i)^{n_i}$ from the inequality without altering the solution set.  For example, \begin{center} solving $(x-1)(x+1)^4>0$ is the same as solving $(x-1)>0$. \end{center}
\item Otherwise, we need to remember the roots themselves as solutions.  Therefore, the solution set for $(x-1)(x+1)^4\ge 0$ is the same as the solution set of $(x-1)\ge 0$ together with the root $-4$.
\end{itemize}
\end{itemize}
\item
Using the rules above, we may also solve inequalities $p(x)<0$ or $p(x)\le 0$.  The solution set for $p(x)<0$ is the complement of the solution set for $p(x)\ge 0$, and the solution set for $p(x)\le 0$ is the complement of the solution set for $p(x)>0$.
\end{enumerate}
%%%%%
%%%%%
\end{document}
