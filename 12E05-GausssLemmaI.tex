\documentclass[12pt]{article}
\usepackage{pmmeta}
\pmcanonicalname{GausssLemmaI}
\pmcreated{2013-03-22 13:07:49}
\pmmodified{2013-03-22 13:07:49}
\pmowner{bshanks}{153}
\pmmodifier{bshanks}{153}
\pmtitle{Gauss's lemma I}
\pmrecord{17}{33566}
\pmprivacy{1}
\pmauthor{bshanks}{153}
\pmtype{Theorem}
\pmcomment{trigger rebuild}
\pmclassification{msc}{12E05}
\pmrelated{GausssLemmaII}

\endmetadata

% this is the default PlanetMath preamble.  as your knowledge
% of TeX increases, you will probably want to edit this, but
% it should be fine as is for beginners.

% almost certainly you want these
\usepackage{amssymb}
\usepackage{amsmath}
\usepackage{amsfonts}

% used for TeXing text within eps files
%\usepackage{psfrag}
% need this for including graphics (\includegraphics)
%\usepackage{graphicx}
% for neatly defining theorems and propositions
%\usepackage{amsthm}
% making logically defined graphics
%%%\usepackage{xypic}

% there are many more packages, add them here as you need them

% define commands here
\begin{document}
\PMlinkescapeword{primitive}

There are a few different things that are sometimes called ``Gauss's Lemma''. See also Gauss's Lemma II.
\\
\\\emph{Gauss's Lemma I:} If $R$ is a UFD and $f(x)$ and $g(x)$ are both primitive polynomials in $R[x]$, so is $f(x) g(x)$. 
\\
\\ \emph{Proof:} 
Suppose $f(x) g(x)$ not primitive. We will show either $f(x)$ or $g(x)$ isn't as well. $f(x) g(x)$ not primitive means that there exists some non-unit $d$ in $R$ that divides all the coefficients of $f(x) g(x)$. Let $p$ be an irreducible factor of $d$, which exists and is a prime element because $R$ is a UFD. We consider the quotient ring of $R$ by the principal ideal $pR$ generated by $p$, which is a prime ideal since $p$ is a prime element. The canonical projection $R \to R/pR$ induces a surjective ring homomorphism $\theta:R[X]\to(R/pR)[X]$, whose kernel consists of all polynomials all of whose coefficients are divisible by~$p$; these polynomials are therefore not primitive.

Since $pR$ is a prime ideal, $R/pR$ is an integral domain, so $(R/pR)[x]$ is also an integral domain. By hypothesis $\theta$ sends the product $f(x)g(x)$ to $0\in(R/pR)[X]$, which is therefore the product of $\theta(f(x))$ and $\theta(g(x))$, and one of these two factors in $(R/pR)[x]$ must be zero. But that means that $f(x)$ or $g(x)$ is in the kernel of $\theta$, and therefore not primitive.
%%%%%
%%%%%
\end{document}
