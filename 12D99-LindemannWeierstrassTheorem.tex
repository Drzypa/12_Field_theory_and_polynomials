\documentclass[12pt]{article}
\usepackage{pmmeta}
\pmcanonicalname{LindemannWeierstrassTheorem}
\pmcreated{2013-03-22 14:19:22}
\pmmodified{2013-03-22 14:19:22}
\pmowner{CWoo}{3771}
\pmmodifier{CWoo}{3771}
\pmtitle{Lindemann-Weierstrass theorem}
\pmrecord{11}{35791}
\pmprivacy{1}
\pmauthor{CWoo}{3771}
\pmtype{Theorem}
\pmcomment{trigger rebuild}
\pmclassification{msc}{12D99}
\pmclassification{msc}{11J85}
\pmsynonym{Lindemann's theorem}{LindemannWeierstrassTheorem}
\pmrelated{SchanuelsConjecutre}
\pmrelated{GelfondsTheorem}
\pmrelated{Irrational}
\pmrelated{EIsTranscendental}

\endmetadata

% this is the default PlanetMath preamble.  as your knowledge
% of TeX increases, you will probably want to edit this, but
% it should be fine as is for beginners.

% almost certainly you want these
\usepackage{amssymb}
\usepackage{amsmath}
\usepackage{amsfonts}

% used for TeXing text within eps files
%\usepackage{psfrag}
% need this for including graphics (\includegraphics)
%\usepackage{graphicx}
% for neatly defining theorems and propositions
%\usepackage{amsthm}
% making logically defined graphics
%%%\usepackage{xypic}

% there are many more packages, add them here as you need them

% define commands here
\begin{document}
If $\alpha_1,\ldots,\alpha_n$ are linearly independent algebraic numbers over $\mathbb{Q}$, then $e^{\alpha_1},\ldots,e^{\alpha_n}$ are algebraically independent over $\mathbb{Q}$.

An equivalent version of the theorem \PMlinkescapetext{states} that if $\alpha_1,\ldots,\alpha_n$ are distinct algebraic numbers over $\mathbb{Q}$, then $e^{\alpha_1},\ldots,e^{\alpha_n}$ are linearly independent over $\mathbb{Q}$.

Some immediate consequences of this theorem:
\begin{itemize}
\item
If $\alpha$ is a non-zero algebraic number over $\mathbb{Q}$, then $e^{\alpha}$ is transcendental over $\mathbb{Q}$.
\item
$e$ is transcendental over $\mathbb{Q}$. 
\item
$\pi$ is transcendental over $\mathbb{Q}$.  As a result, it is impossible to ``square the circle''!
\end{itemize}

It is easy to see that $\pi$ is transcendental over $\mathbb{Q}(e)$ iff $e$ is transcendental over $\mathbb{Q}(\pi)$ iff $\pi$ and $e$ are algebraically independent.  However, whether $\pi$ and $e$ are algebraically independent is still an open question today.

Schanuel's conjecture is a generalization of the Lindemann-Weierstrass theorem.  If Schanuel's conjecture were proven to be true, then the algebraic independence of $e$ and $\pi$ over $\mathbb{Q}$ can be shown.
%%%%%
%%%%%
\end{document}
