\documentclass[12pt]{article}
\usepackage{pmmeta}
\pmcanonicalname{TheoremsOnSumsOfSquares}
\pmcreated{2013-03-22 15:06:04}
\pmmodified{2013-03-22 15:06:04}
\pmowner{CWoo}{3771}
\pmmodifier{CWoo}{3771}
\pmtitle{theorems on sums of squares}
\pmrecord{14}{36830}
\pmprivacy{1}
\pmauthor{CWoo}{3771}
\pmtype{Theorem}
\pmcomment{trigger rebuild}
\pmclassification{msc}{12D15}
\pmclassification{msc}{16D60}
\pmclassification{msc}{15A63}
\pmclassification{msc}{11E25}
\pmsynonym{Pfister theorem}{TheoremsOnSumsOfSquares}
\pmrelated{MazursStructureTheorem}
\pmrelated{SumsOfTwoSquares}
\pmrelated{StufeOfAField}
\pmrelated{CayleyDicksonConstruction}
\pmrelated{CompositionAlgebra}
\pmrelated{Octonion}
\pmdefines{Hurwitz theorem}
\pmdefines{Pfister's theorem}
\pmdefines{Hilbert's 17th Problem}

% this is the default PlanetMath preamble.  as your knowledge
% of TeX increases, you will probably want to edit this, but
% it should be fine as is for beginners.

% almost certainly you want these
\usepackage{amssymb,amscd}
\usepackage{amsmath}
\usepackage{amsfonts}

% used for TeXing text within eps files
%\usepackage{psfrag}
% need this for including graphics (\includegraphics)
%\usepackage{graphicx}
% for neatly defining theorems and propositions
\usepackage{amsthm}
% making logically defined graphics
%%%\usepackage{xypic}

% there are many more packages, add them here as you need them

% define commands here
\theoremstyle{theorem}
\newtheorem*{thm}{Theorem}
\begin{document}
\begin{thm}[\PMlinkescapetext{\textbf{Hurwitz Theorem}}]  Let $F$ be a field with
characteristic not $2$.  The sum of squares identity of the form
$$(x_1^2+\cdots+x_n^2)(y_1^2+\cdots+y_n^2)=z_1^2+\cdots+z_n^2$$
where each $z_k$ is bilinear over $x_i$ and $y_j$ (with coefficients
in $F$), is possible iff $n=1,2,4,8$.
\end{thm}

\textbf{Remarks}.
\begin{enumerate}
\item When the ground field is $\mathbb{R}$, this theorem is equivalent to the fact that the only normed real division alternative algebra is one of $\mathbb{R}$, $\mathbb{C}$, $\mathbb{H}$, $\mathbb{O}$, as one observes that the sums of squares can be interpreted as the square of the norm defined for each of the above algebras.
\item An equivalent characterization is that the above four mentioned algebras are the only real composition algebras.
\end{enumerate}

A generalization of the above is the following:

\begin{thm}[{\textbf{Pfister's Theorem}}]  Let $F$ be a field of
characteristic not $2$.  The sum of squares identity of the form
$$(x_1^2+\cdots+x_n^2)(y_1^2+\cdots+y_n^2)=z_1^2+\cdots+z_n^2$$
where each $z_k$ is a rational function of $x_i$ and $y_j$ (element
of $F(x_1,\ldots,x_n,y_1,\ldots,y_n)$), is possible iff $n$ is a
power of $2$.
\end{thm}

\textbf{Remark}.  The form of Pfister's theorem is stated in a way
so as to mirror the form of Hurwitz theorem.  In fact, Pfister
proved the following:  if $F$ is a field and $n$ is a power of 2,
then there exists a sum of squares identity of the form
$$(x_1^2+\cdots+x_n^2)(y_1^2+\cdots+y_n^2)=z_1^2+\cdots+z_n^2$$ such
that each $z_k$ is a rational function of the $x_i$ and a linear
function of the $y_j$, or that
$$z_k=\sum_{j=1}^{n}r_{kj}y_j\qquad\mbox{where }r_{kj}\in
F(x_1,\ldots,x_n).$$  Conversely, if $n$ is not a power of $2$, then
there exists a field $F$ such that the above sum of square identity
does not hold for \emph{any} $z_i\in
F(x_1,\ldots,x_n,y_1,\ldots,y_n)$.  Notice that $z_i$ is no longer
required to be a linear function of the $y_j$ anymore.

When $F$ is the field of reals $\mathbb{R}$, we have the following
generalization, also due to Pfister:

\begin{thm}
If $f\in\mathbb{R}(X_1,\ldots,X_n)$ is positive semidefinite, then
$f$ can be written as a sum of $2^n$ squares.
\end{thm}

The above theorem is very closely related to Hilbert's 17th Problem:

{\textbf{Hilbert's 17th Problem.}} \emph{Whether it is possible, to
write a positive semidefinite rational function in $n$
indeterminates over the reals, as a sum of squares of rational
functions in $n$ indeterminates over the reals?}

The answer is yes, and it was proved by Emil Artin in 1927.
Additionally, Artin showed that the answer is also yes if the reals
were replaced by the rationals.


\begin{thebibliography}{8}
\bibitem{hurwitz} A. Hurwitz, {\em \"{U}ber die Komposition der quadratishen Formen von beliebig vielen Variabeln}, Nachrichten von der K\"{o}niglichen Gesellschaft der Wissenschaften in G\"{o}ttingen (1898).
\bibitem{pfister} A. Pfister, {\em Zur Darstellung definiter Funktionen als Summe von Quadraten}, Inventiones Mathematicae (1967).
\bibitem{rajwade} A. R. Rajwade, {\em Squares}, Cambridge University Press (1993).
\bibitem{conway} J. Conway, D. A. Smith, {\em On Quaternions and Octonions}, A K Peters, LTD. (2002).
\end{thebibliography}
%%%%%
%%%%%
\end{document}
