\documentclass[12pt]{article}
\usepackage{pmmeta}
\pmcanonicalname{LimitOfNthRootOfN}
\pmcreated{2014-09-28 13:20:59}
\pmmodified{2014-09-28 13:20:59}
\pmowner{pahio}{2872}
\pmmodifier{pahio}{2872}
\pmtitle{limit of nth root of n}
\pmrecord{8}{40112}
\pmprivacy{1}
\pmauthor{pahio}{2872}
\pmtype{Example}
\pmcomment{trigger rebuild}
\pmclassification{msc}{12D99}
\pmclassification{msc}{30-00}
\pmsynonym{sequence of nth roots of n}{LimitOfNthRootOfN}

\endmetadata

% this is the default PlanetMath preamble.  as your knowledge
% of TeX increases, you will probably want to edit this, but
% it should be fine as is for beginners.

% almost certainly you want these
\usepackage{amssymb}
\usepackage{amsmath}
\usepackage{amsfonts}

% used for TeXing text within eps files
%\usepackage{psfrag}
% need this for including graphics (\includegraphics)
%\usepackage{graphicx}
% for neatly defining theorems and propositions
 \usepackage{amsthm}
% making logically defined graphics
%%%\usepackage{xypic}

% there are many more packages, add them here as you need them

% define commands here

\theoremstyle{definition}
\newtheorem*{thmplain}{Theorem}

\begin{document}
The \PMlinkname{$n$th root}{NthRoot} of $n$ tends to 1 as $n$ tends to infinity, i.e. the real number sequence
$$\sqrt[1]{1},\, \sqrt[2]{2},\,\sqrt[3]{3},\,\ldots,\,\sqrt[n]{n},\,\ldots$$
converges to the limit
\begin{align}
\lim_{n\to\infty}\sqrt[n]{n} = 1.
\end{align}

{\it Proof.}\, If we denote\, $\sqrt[n]{n} := 1+\delta_n$, we may write by the binomial theorem that
$$n = (1+\delta_n)^n = 1+{n\choose1}\delta_n+{n\choose2}\delta_n^2+\ldots+{n\choose n}\delta_n^n.$$
This implies, since all \PMlinkescapetext{terms of the right} hand side are positive (when\, $n > 1$), that
$$n > {n\choose2}\delta_n^2 = \frac{n(n\!-\!1)}{2!}\delta_n^2, 
\qquad \delta_n^2 < \frac{2}{n-1}, 
\qquad 0 < \delta_n < \sqrt{\frac{2}{n-1}},$$
whence\, $\displaystyle\lim_{n\to\infty}\delta_n = 0$.\, Accordingly, 
$$\lim_{n\to\infty}\sqrt[n]{n} = \lim_{n\to\infty}(1+\delta_n) = 1,$$
Q.E.D.\\

\textbf{Note.}\, (1) follows also from the corollary 3 in the entry growth of exponential function.


             

%%%%%
%%%%%
\end{document}
