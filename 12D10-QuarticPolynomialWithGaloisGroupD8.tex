\documentclass[12pt]{article}
\usepackage{pmmeta}
\pmcanonicalname{QuarticPolynomialWithGaloisGroupD8}
\pmcreated{2013-03-22 17:44:09}
\pmmodified{2013-03-22 17:44:09}
\pmowner{rm50}{10146}
\pmmodifier{rm50}{10146}
\pmtitle{quartic polynomial with Galois group $D_8$}
\pmrecord{6}{40185}
\pmprivacy{1}
\pmauthor{rm50}{10146}
\pmtype{Example}
\pmcomment{trigger rebuild}
\pmclassification{msc}{12D10}

% this is the default PlanetMath preamble.  as your knowledge
% of TeX increases, you will probably want to edit this, but
% it should be fine as is for beginners.

% almost certainly you want these
\usepackage{amssymb}
\usepackage{amsmath}
\usepackage{amsfonts}

% used for TeXing text within eps files
%\usepackage{psfrag}
% need this for including graphics (\includegraphics)
%\usepackage{graphicx}
% for neatly defining theorems and propositions
%\usepackage{amsthm}
% making logically defined graphics
%%%\usepackage{xypic}

% there are many more packages, add them here as you need them

% define commands here
\DeclareMathOperator{\Gal}{Gal}
\newcommand{\Rats}{\mathbb{Q}}
\begin{document}
The polynomial $f(x)=x^4-2x^2-2$ is Eisenstein at $2$ and thus irreducible over $\Rats$. Solving $f(x)$ as a quadratic in $x^2$, we see that the roots of $f(x)$ are
\begin{center}
\begin{tabular}{ccc}
$\alpha_1 = \sqrt{1+\sqrt{3}}$ &\qquad & $\alpha_3 = -\sqrt{1+\sqrt{3}}$ \\
$\alpha_2 = \sqrt{1-\sqrt{3}}$ &\qquad & $\alpha_4 = -\sqrt{1-\sqrt{3}}$
\end{tabular}
\end{center}

Note that the discriminant of $f(x)$ is $-4608=-2^9\cdot 3^2$, and that its resolvent cubic is
\[x^3+4x^2+12x=x(x^2+4x+12)=0\]
which factors over $\Rats$ into a linear and an irreducible quadratic. Additionally, $f(x)$ remains irreducible over $\Rats(\sqrt{-4608})=\Rats(\sqrt{-2})$, since none of the roots of $f(x)$ lie in this field and the discriminant of $f(x)$, regarded as a quadratic in $x^2$, does not lie in this field either, so $f(x)$ cannot factor as a product of two quadratics. So according to the article on the Galois group of a quartic polynomial, $f(x)$ should indeed have Galois group isomorphic to $D_8$. We show that this is the case by explicitly examining the structure of its splitting field. 

Let $K$ be the splitting field of $f(x)$ over $\Rats$, and let $G=\Gal(K/\Rats)$.

Let $K_1=\Rats(\alpha_1) = \Rats(\alpha_3)$ and $K_2 = \Rats(\alpha_2)=\Rats(\alpha_4)$. Clearly $K$ contains both $K_1$ and $K_2$ and thus contains $K_1K_2=\Rats(\alpha_1,\alpha_2)$. But obviously $f(x)$ splits in $K_1K_2$, so that $K=K_1K_2$. We next determine the degree of $K$ over $\Rats$.

Note that $K_1\neq K_2$ since $K_1$ is a real field while $K_2$ is not. Thus $K_1\cap K_2\subsetneq K_1, K_2$. Clearly $[K_1:\Rats]=[K_2:\Rats]=4$, so $[K_1\cap K_2:\Rats]\leq 2$. But
\[\sqrt{3}=\left(\sqrt{1+\sqrt{3}}\right)^2-1 = -\left(\sqrt{1-\sqrt{3}}\right)^2+1\]
so $\sqrt{3}\in K_1\cap K_2$. Hence $K_1\cap K_2=\Rats(\sqrt{3})$; call this field $F$.

Since $K_1\neq K_2$, we also have $K=K_1K_2\neq K_1$ and $K=K_1K_2\neq K_2$; thus $K$ is a quadratic extension of each and $[K:F]=4$.

Putting these results together, we see that
\[[K:\Rats]=[K:F][F:\Rats]=8\]
so that $G$ has order $8$.

Now, neither $K_1$ nor $K_2$ is Galois over $\Rats$ (since the Galois closure of either one is $K$), so that the subgroup of $G$ fixing (say) $K_1$ is a nonnormal subgroup of $G$. Thus $G$ must be nonabelian, so must be isomorphic to either $D_8$ or $Q_8$ (the quaternions). But the subgroups of $G$ corresponding to $K_1$ and $K_2$ are distinct subgroups of order $2$ in $G$, and $Q_8$ has only one subgroup of order $2$. Thus $G\cong D_8$. (Alternatively, note that all subgroups of $Q_8$ are normal, so $G\cong D_8$ since it has a nonnormal subgroup).
%%%%%
%%%%%
\end{document}
