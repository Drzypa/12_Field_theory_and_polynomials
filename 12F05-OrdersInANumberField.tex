\documentclass[12pt]{article}
\usepackage{pmmeta}
\pmcanonicalname{OrdersInANumberField}
\pmcreated{2013-03-22 16:52:46}
\pmmodified{2013-03-22 16:52:46}
\pmowner{pahio}{2872}
\pmmodifier{pahio}{2872}
\pmtitle{orders in a number field}
\pmrecord{17}{39132}
\pmprivacy{1}
\pmauthor{pahio}{2872}
\pmtype{Topic}
\pmcomment{trigger rebuild}
\pmclassification{msc}{12F05}
\pmclassification{msc}{11R04}
\pmclassification{msc}{06B10}
%\pmkeywords{order}
\pmrelated{Module}
\pmdefines{module}
\pmdefines{complete}
\pmdefines{order of a number field}
\pmdefines{principal order}
\pmdefines{maximal order}

% this is the default PlanetMath preamble.  as your knowledge
% of TeX increases, you will probably want to edit this, but
% it should be fine as is for beginners.

% almost certainly you want these
\usepackage{amssymb}
\usepackage{amsmath}
\usepackage{amsfonts}

% used for TeXing text within eps files
%\usepackage{psfrag}
% need this for including graphics (\includegraphics)
%\usepackage{graphicx}
% for neatly defining theorems and propositions
 \usepackage{amsthm}
% making logically defined graphics
%%%\usepackage{xypic}

% there are many more packages, add them here as you need them

% define commands here

\theoremstyle{definition}
\newtheorem*{thmplain}{Theorem}

\begin{document}
\PMlinkescapeword{coefficient ring}

If\, $\mu_1,\,\ldots,\,\mu_m$\, are elements of an algebraic number field $K$, then the subset 
$$M = 
 \{n_1\mu_1+\ldots+n_m\mu_m\in K\,\vdots\;\; n_i\in\mathbb{Z}\;\;\forall i\}$$
of $K$ is a $\mathbb{Z}$-module, called a {\em module in} $K$.\, If the module contains as many over $\mathbb{Z}$ linearly independent elements as is the \PMlinkname{degree}{NumberField} of $K$ over $\mathbb{Q}$, then the module is {\em complete}.

If a complete module in $K$ \PMlinkescapetext{contains} the unity 1 of $K$ and is a ring, it is called an {\em order} (in German: {\em Ordnung}) in the field $K$.\\

A number $\alpha$ of the algebraic number field $K$ is called a {\em coefficient of the module} $M$, if\, $\alpha M \subseteq M$.\, 

\textbf{Theorem 1.}\; The set $\mathcal{L}_M$ of all coefficients of a complete module $M$ is an order in the field.\, Conversely, every order $\mathcal{L}$ in the number field $K$ is a coefficient ring of some module.

\textbf{Theorem 2.}\; If $\alpha$ belongs to an order in the field, then the coefficients of the \PMlinkname{characteristic equation}{CharacteristicEquation} of $\alpha$ and thus the coefficients of the minimal polynomial of $\alpha$ are rational integers.

Theorem 2 means that any order is contained in the ring of integers of the algebraic number field $K$.\, Thus this ring $\mathcal{O}_K$, being itself an order, is the greatest order; $\mathcal{O}_K$ is called the {\em maximal order} or the {\em principal order} (in German: {\em Hauptordnung}).\, The set of the orders is partially ordered by the set inclusion.

\textbf{Example.}\, In the field $\mathbb{Q}(\sqrt{2})$, the coefficient ring of the module $M$ generated by $2$ and $\frac{\sqrt{2}}{2}$ is the module $\mathcal{L}_M$ generated by $1$ and $2\sqrt{2}$.\, The maximal order of the field is generated by $1$ and $\sqrt{2}$.

\begin{thebibliography}{9}
\bibitem{BS}{\sc S. Borewicz \& I. Safarevic}: {\em Zahlentheorie}.\, Birkh\"auser Verlag. Basel und Stuttgart (1966).
\end{thebibliography}

%%%%%
%%%%%
\end{document}
