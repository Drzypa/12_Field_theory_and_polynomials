\documentclass[12pt]{article}
\usepackage{pmmeta}
\pmcanonicalname{TheoremsOnContinuation}
\pmcreated{2013-03-22 17:59:51}
\pmmodified{2013-03-22 17:59:51}
\pmowner{pahio}{2872}
\pmmodifier{pahio}{2872}
\pmtitle{theorems on continuation}
\pmrecord{10}{40510}
\pmprivacy{1}
\pmauthor{pahio}{2872}
\pmtype{Theorem}
\pmcomment{trigger rebuild}
\pmclassification{msc}{12J20}
\pmclassification{msc}{13A18}
\pmclassification{msc}{13F30}
\pmclassification{msc}{11R99}
\pmsynonym{theorems on continuations of exponents}{TheoremsOnContinuation}

% this is the default PlanetMath preamble.  as your knowledge
% of TeX increases, you will probably want to edit this, but
% it should be fine as is for beginners.

% almost certainly you want these
\usepackage{amssymb}
\usepackage{amsmath}
\usepackage{amsfonts}

% used for TeXing text within eps files
%\usepackage{psfrag}
% need this for including graphics (\includegraphics)
%\usepackage{graphicx}
% for neatly defining theorems and propositions
 \usepackage{amsthm}
% making logically defined graphics
%%%\usepackage{xypic}

% there are many more packages, add them here as you need them

% define commands here

\theoremstyle{definition}
\newtheorem*{thmplain}{Theorem}

\begin{document}
\PMlinkescapeword{exponent} \PMlinkescapeword{exponents}

\textbf{Theorem 1.}\, When $\nu_0$ is an exponent valuation of the field $k$ and $K/k$ is a finite field extension, $\nu_0$ has a continuation to the extension field $K$.

\textbf{Theorem 2.}\, If the \PMlinkname{degree}{ExtensionField} of the field extension $K/k$ is $n$ and $\nu_0$ is an arbitrary \PMlinkname{exponent}{ExponentValuation2} of $k$, then $\nu_0$ has at most $n$ continuations to the extension field $K$.

\textbf{Theorem 3.}\, Let $\nu_0$ be an exponent valuation of the field $k$ and $\mathfrak{o}$ the ring of the exponent $\nu_0$.\, Let $K/k$ be a finite extension and $\mathfrak{O}$ the integral closure of $\mathfrak{o}$ in $K$.\, If\, $\nu_1,\,\ldots,\,\nu_m$ are all different continuations of $\nu_0$ to the field $K$ and $\mathfrak{O}_1,\,\ldots,\,\mathfrak{O}_m$ \PMlinkname{their rings}{RingOfExponent}, then
$$\mathfrak{O} = \bigcap_{i=1}^m\mathfrak{O}_i.$$



The proofs of those theorems are found in [1], which is available also in Russian (original), English and French.\\

\textbf{Corollary.}\, The ring $\mathfrak{O}$ (of theorem 3) is a UFD.\, The exponents of $K$, which are determined by the pairwise coprime prime elements of $\mathfrak{O}$, coincide with the continuations $\nu_1,\,\ldots,\,\nu_m$ of $\nu_0$.\, If $\pi_1,\,\ldots,\,\pi_m$ are the pairwise coprime prime elements of $\mathfrak{O}$ such that\; $\nu_i(\pi_1) = 1$\, for all\, $i$'s and if the prime element $p$ of the ring $\mathfrak{o}$ has the \PMlinkescapetext{representation}
$$p = \varepsilon\pi_1^{e_1}\cdots\pi_m^{e_m}$$
with $\varepsilon$ a unit of $\mathfrak{O}$, then $e_i$ is the ramification index of the exponent $\nu_i$ with respect to $\nu_0$ ($i = 1,\,\ldots,\,m$).



\begin{thebibliography}{9}
\bibitem{BS}{\sc S. Borewicz \& I. Safarevic}: {\em Zahlentheorie}.\, Birkh\"auser Verlag. Basel und Stuttgart (1966).
\end{thebibliography}

%%%%%
%%%%%
\end{document}
