\documentclass[12pt]{article}
\usepackage{pmmeta}
\pmcanonicalname{ComplexNumber}
\pmcreated{2013-03-22 11:52:35}
\pmmodified{2013-03-22 11:52:35}
\pmowner{djao}{24}
\pmmodifier{djao}{24}
\pmtitle{complex number}
\pmrecord{9}{30471}
\pmprivacy{1}
\pmauthor{djao}{24}
\pmtype{Definition}
\pmcomment{trigger rebuild}
\pmclassification{msc}{12D99}
\pmclassification{msc}{30-00}
\pmclassification{msc}{32-00}
\pmclassification{msc}{46L05}
\pmclassification{msc}{18B40}
\pmclassification{msc}{46M20}
\pmclassification{msc}{17B37}
\pmclassification{msc}{22A22}
\pmclassification{msc}{81R50}
\pmclassification{msc}{22D25}
\pmsynonym{$\mathbb{C}$}{ComplexNumber}
\pmrelated{Complex}

\usepackage{amssymb}
\usepackage{amsmath}
\usepackage{amsfonts}
\usepackage{graphicx}
%%%%\usepackage{xypic}
\begin{document}
The ring of complex numbers $\mathbb{C}$ is defined to be the quotient ring of the polynomial ring $\mathbb{R}[X]$ in one variable over the reals by the principal ideal $(X^2+1)$. For $a,b \in \mathbb{R}$, the equivalence class of $a+bX$ in $\mathbb{C}$ is usually denoted $a+bi$, and one has $i^2 = -1$.

The complex numbers form an algebraically closed field. There is a standard metric on the complex numbers, defined by
$$
d(a_1+b_1 i, a_2+b_2 i) := \sqrt{(a_2-a_1)^2 + (b_2-b_1)^2}.
$$
%%%%%
%%%%%
%%%%%
%%%%%
\end{document}
