\documentclass[12pt]{article}
\usepackage{pmmeta}
\pmcanonicalname{IndeterminateForm}
\pmcreated{2013-03-22 12:28:19}
\pmmodified{2013-03-22 12:28:19}
\pmowner{akrowne}{2}
\pmmodifier{akrowne}{2}
\pmtitle{indeterminate form}
\pmrecord{7}{32658}
\pmprivacy{1}
\pmauthor{akrowne}{2}
\pmtype{Definition}
\pmcomment{trigger rebuild}
\pmclassification{msc}{12D99}
\pmsynonym{indeterminate value}{IndeterminateForm}
\pmrelated{LHpitalsRule}
\pmrelated{ImproperLimits}
\pmrelated{EmptyProduct}

\endmetadata

\usepackage{amssymb}
\usepackage{amsmath}
\usepackage{amsfonts}

%\usepackage{psfrag}
%\usepackage{graphicx}
%%%\usepackage{xypic}
\begin{document}
The expression

$$ \frac{0}{0} $$

is known as the \emph{indeterminate form}.  The motivation for this name is that there are no rules for comparing the value of $\frac{0}{0}$ to the other real numbers.  Note that, for example, $\frac{1}{0}$ is \emph{not} indeterminate, since we can justifiably associate it with $+\infty$, which \emph{does} compare with the rest of the real numbers (in particular, it is defined to be greater than all of them.)

\section{Other Indeterminate Forms}

Although $\frac{0}{0}$ is often called ``the'' indeterminate form, there are many others.  Some of these are:

\begin{enumerate} 

\item $ \frac{\infty}{\infty} $, for the same motivating reasons as $\frac{0}{0}$.

\item $ 0^0 $; which is the result of much impassioned debate (especially since $0!$ is defined to be 1, counter-intuitively, but not unreasonably).

\item $1^{\infty}$; notably because of the derivation of $e$: 

$$ \lim_{n \to \infty} \left( 1+\frac{1}{n} \right)^n = e $$

A direct substitution would yield $1^\infty$.

\end{enumerate}
%%%%%
%%%%%
\end{document}
