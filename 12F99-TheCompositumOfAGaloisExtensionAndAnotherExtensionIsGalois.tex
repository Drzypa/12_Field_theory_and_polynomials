\documentclass[12pt]{article}
\usepackage{pmmeta}
\pmcanonicalname{TheCompositumOfAGaloisExtensionAndAnotherExtensionIsGalois}
\pmcreated{2013-03-22 15:04:13}
\pmmodified{2013-03-22 15:04:13}
\pmowner{alozano}{2414}
\pmmodifier{alozano}{2414}
\pmtitle{the compositum of a Galois extension and another extension is Galois}
\pmrecord{6}{36791}
\pmprivacy{1}
\pmauthor{alozano}{2414}
\pmtype{Theorem}
\pmcomment{trigger rebuild}
\pmclassification{msc}{12F99}
\pmclassification{msc}{11R32}
%\pmkeywords{compositum}
%\pmkeywords{composite field}
%\pmkeywords{Galois extension}
\pmrelated{FundamentalTheoremOfGaloisTheory}
\pmrelated{GaloisExtension}
\pmrelated{ExampleOfNormalExtension}
\pmrelated{ClassNumberDivisibilityInExtensions}
\pmrelated{GaloisGroupOfTheCompositumOfTwoGaloisExtensions}
\pmrelated{ExtensionsWithoutUnramifiedSubextensionsAndClassNumberDivisibility}

\endmetadata

% this is the default PlanetMath preamble.  as your knowledge
% of TeX increases, you will probably want to edit this, but
% it should be fine as is for beginners.

% almost certainly you want these
\usepackage{amssymb}
\usepackage{amsmath}
\usepackage{amsthm}
\usepackage{amsfonts}

% used for TeXing text within eps files
%\usepackage{psfrag}
% need this for including graphics (\includegraphics)
%\usepackage{graphicx}
% for neatly defining theorems and propositions
%\usepackage{amsthm}
% making logically defined graphics
%%%\usepackage{xypic}

% there are many more packages, add them here as you need them

% define commands here

\newtheorem{thm}{Theorem}
\newtheorem{defn}{Definition}
\newtheorem{prop}{Proposition}
\newtheorem{lemma}{Lemma}
\newtheorem{cor}{Corollary}
\newtheorem{rem}{Remark}

% Some sets
\newcommand{\Nats}{\mathbb{N}}
\newcommand{\Ints}{\mathbb{Z}}
\newcommand{\Reals}{\mathbb{R}}
\newcommand{\Complex}{\mathbb{C}}
\newcommand{\Rats}{\mathbb{Q}}
\newcommand{\Gal}{\operatorname{Gal}}
\begin{document}
\begin{thm}
Let $E/K$ be a Galois extension of fields, let $F/K$ be an arbitrary extension and  assume that $E$ and $F$ are both subfields of some other larger field $T$. The compositum of $E$ and $F$ is here denoted by $EF$. Then:
\begin{enumerate}
\item $EF$ is a Galois extension of $F$ and $E$ is Galois over $E\cap F$;\\

\item Let $H=\Gal(EF/F)$. The restriction map:
\begin{eqnarray*}
H=\Gal(EF/F) & \longrightarrow & \Gal(E/E\cap F)\\
\sigma & \longrightarrow & \sigma |_{E}
\end{eqnarray*}
is an isomorphism, where $\sigma |_{E}$ denotes the restriction of $\sigma$ to $E$. 
\end{enumerate}
\end{thm}

\begin{rem}
Notice, however, that if $E/F$ and $F/K$ are both Galois extensions, the extension $E/K$ need not be Galois. See example of normal extension for a counterexample.
\end{rem}
%%%%%
%%%%%
\end{document}
