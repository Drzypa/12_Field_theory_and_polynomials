\documentclass[12pt]{article}
\usepackage{pmmeta}
\pmcanonicalname{RamificationOfArchimedeanPlaces}
\pmcreated{2013-03-22 15:07:19}
\pmmodified{2013-03-22 15:07:19}
\pmowner{alozano}{2414}
\pmmodifier{alozano}{2414}
\pmtitle{ramification of archimedean places}
\pmrecord{9}{36861}
\pmprivacy{1}
\pmauthor{alozano}{2414}
\pmtype{Definition}
\pmcomment{trigger rebuild}
\pmclassification{msc}{12F99}
\pmclassification{msc}{13B02}
\pmclassification{msc}{11S15}
\pmsynonym{finite place}{RamificationOfArchimedeanPlaces}
\pmsynonym{infinite place}{RamificationOfArchimedeanPlaces}
\pmrelated{DecompositionGroup}
\pmrelated{PlaceOfField}
\pmrelated{RealAndComplexEmbeddings}
\pmrelated{PlaceAsExtensionOfHomomorphism}
\pmdefines{decomposition and inertia group for archimedean places}
\pmdefines{archimedean place}
\pmdefines{non-archimedean place}

% this is the default PlanetMath preamble.  as your knowledge
% of TeX increases, you will probably want to edit this, but
% it should be fine as is for beginners.

% almost certainly you want these
\usepackage{amssymb}
\usepackage{amsmath}
\usepackage{amsthm}
\usepackage{amsfonts}

% used for TeXing text within eps files
%\usepackage{psfrag}
% need this for including graphics (\includegraphics)
%\usepackage{graphicx}
% for neatly defining theorems and propositions
%\usepackage{amsthm}
% making logically defined graphics
%%%\usepackage{xypic}

% there are many more packages, add them here as you need them

% define commands here

\newtheorem{thm}{Theorem}
\newtheorem{defn}{Definition}
\newtheorem{prop}{Proposition}
\newtheorem{lemma}{Lemma}
\newtheorem{cor}{Corollary}

% Some sets
\newcommand{\Nats}{\mathbb{N}}
\newcommand{\Ints}{\mathbb{Z}}
\newcommand{\Reals}{\mathbb{R}}
\newcommand{\Complex}{\mathbb{C}}
\newcommand{\Rats}{\mathbb{Q}}
\newcommand{\Gal}{\operatorname{Gal}}
\begin{document}
Throughout this entry, if $\alpha$ is a complex number, we denote the complex conjugate of $\alpha$ by $\overline{\alpha}$.

\begin{defn}
Let $K$ be a number field.
\begin{enumerate}
\item An {\bf archimedean place} of $K$ is either a real embedding $\phi\colon K \to \Reals$ or a pair of complex-conjugate embeddings $(\psi,\overline{\psi})$, with $\overline{\psi}\neq \psi$ and $\psi\colon K\to \Complex$. The archimedean places are sometimes called the infinite places (cf. place of field).\\

\item The {\bf non-archimedean places} of $K$ are the prime ideals in $\mathcal{O}_K$, the ring of integers of $K$ (see \PMlinkname{non-archimedean valuation}{Valuation}). The non-archimedean places are sometimes called the finite places.
\end{enumerate}
\end{defn}

Notice that any archimedean place $\phi\colon K\to \Complex$ can be extended to an embedding $\hat{\phi}\colon \overline{\Rats} \to \Complex$, where $\overline{\Rats}$ is a fixed algebraic closure of $\Rats$ (in order to prove this, one uses the fact that $\Complex$ is algebraically closed and also Zorn's Lemma). See also \PMlinkname{this entry}{PlaceAsExtensionOfHomomorphism}. In particular, if $F$ is a finite extension of $K$ then $\phi$ can be extended to an archimidean place $\hat{\phi}\colon F \to \Complex$ of $F$.

Next, we define the decomposition and inertia group associated to archimedean places. For the case of non-archimedean places (i.e. prime ideals) see the entries decomposition group and ramification.

Let $F/K$ be a finite Galois extension of number fields and let $\phi$ be a (real or a pair of complex) archimedean place of $K$. Let $\phi_1$ and $\phi_2$ be two archimedean places of $F$ which extend $\phi$. Notice that, since $F/K$ is Galois, the image of $\phi_1$ and $\phi_2$ are equal, in other words:
$$\phi_1(F)=\phi_2(F)\subset \Complex.$$
Hence, the composition $\phi_1^{-1}\circ \phi_2$ is an automorphism of $F$ (here $\phi_1^{-1}$ denotes the inverse map of $\phi_1$, restricted to $\phi_1(F)$). Thus, $\phi_1^{-1}\circ \phi_2 =\sigma \in \Gal(F/K)$ and
$$\phi_2 = \phi_1\circ \sigma$$
so $\phi_1$ and $\phi_2$ differ by an element of the Galois group. Similarly, if $(\psi_1,\overline{\psi_1})$ and $(\psi_2,\overline{\psi_2})$ are complex embeddings which extend $\phi$, then there is $\sigma\in\Gal(F/K)$ such that
$$(\psi_2,\overline{\psi_2})=(\psi_1,\overline{\psi_1})\circ \sigma$$
meaning that either $\psi_2 = \psi_1\circ \sigma$ (and thus $\overline{\psi_2} = \overline{\psi_1}\circ \sigma$) or $\overline{\psi_2} = \psi_1\circ \sigma$ (and thus $\psi_2 = \overline{\psi_1}\circ \sigma$). We are ready now to make the definitions.

\begin{defn}
Let $F/K$ be a Galois extension of number fields and let $w$ be an archimedean place of $F$ lying above a place $v$ of $K$. The decomposition and inertia subgroups  for the pair $w|v$ are equal and are defined by:
$$D(w|v)=T(w|v)=\{ \sigma \in \Gal(F/K) : w\circ \sigma = w\}.$$
Let $e=e(w|v)=|T(w|v)|$ be the size of the inertia subgroup. If $e>1$ then we say that the {\bf archimedean place $v$ is ramified} in the extension $F/K$. 
\end{defn}

The ramification in the archimedean case is much simpler than the non-archimedean analogue. One readily proves the following proposition:
\begin{prop}
The inertia subgroup $T(w|v)$ is nontrivial only when $v$ is real, $w=(\psi,\overline{\psi})$ is a complex archimedean place of $F$ and $\sigma$ is the ``complex conjugation'' map which has order $2$. Therefore $e(w|v)=1$ or $2$ and ramification of archimedean places occurs if and only if there is a complex place of $F$ lying above a real place of $K$.
\end{prop}
\begin{proof}
Suppose first that $w=\phi\colon F\to \Reals$ is a real embedding. Then $\phi$ is injective and $\phi\circ \sigma = \phi$ implies that $\sigma$ is the identity automorphism and $T(w|v)$ would be trivial. So let us assume that $w=(\psi,\overline{\psi})$ is a complex archimedean place and let $\sigma\in \Gal(F/K)$ such that 
$$(\psi,\overline{\psi})=(\psi,\overline{\psi})\circ \sigma.$$
Therefore, either $\psi=\psi\circ \sigma$ (which implies that $\sigma$ is the identity by the injectivity of $\psi$) or $\psi=\overline{\psi}\circ \sigma$. The latter implies that $\sigma=\overline{\psi^{-1}}\circ \psi$, which is simply complex conjugation:
$$\overline{\psi^{-1}}\circ \psi(k)=\overline{\psi^{-1}(\psi(k))}=\overline{k}.$$
Finally, since $w$ is an extension of $v$, the equation $w\circ \sigma = w$ restricts to $\overline{v}=v$, thus $v$ must be real.
\end{proof}
\begin{cor}
Suppose $L/K$ is an extension of number fields and assume that $K$ is a \PMlinkname{totally imaginary}{TotallyRealAndImaginaryFields} number field. Then the extension $L/K$ is unramified at all archimedean places.
\end{cor}
\begin{proof}
Since $K$ is totally imaginary none of the embeddings of $K$ are real. By the proposition, only real places can ramify. 
\end{proof}
%%%%%
%%%%%
\end{document}
