\documentclass[12pt]{article}
\usepackage{pmmeta}
\pmcanonicalname{ExamplesForHenselsLemma}
\pmcreated{2013-03-22 15:08:36}
\pmmodified{2013-03-22 15:08:36}
\pmowner{alozano}{2414}
\pmmodifier{alozano}{2414}
\pmtitle{examples for Hensel's lemma}
\pmrecord{6}{36888}
\pmprivacy{1}
\pmauthor{alozano}{2414}
\pmtype{Example}
\pmcomment{trigger rebuild}
\pmclassification{msc}{12J99}
\pmclassification{msc}{11S99}
\pmclassification{msc}{13H99}
%\pmkeywords{2-adic roots}

% this is the default PlanetMath preamble.  as your knowledge
% of TeX increases, you will probably want to edit this, but
% it should be fine as is for beginners.

% almost certainly you want these
\usepackage{amssymb}
\usepackage{amsmath}
\usepackage{amsthm}
\usepackage{amsfonts}

% used for TeXing text within eps files
%\usepackage{psfrag}
% need this for including graphics (\includegraphics)
%\usepackage{graphicx}
% for neatly defining theorems and propositions
%\usepackage{amsthm}
% making logically defined graphics
%%%\usepackage{xypic}

% there are many more packages, add them here as you need them

% define commands here

\newtheorem{thm}{Theorem}
\newtheorem{defn}{Definition}
\newtheorem{prop}{Proposition}
\newtheorem{lemma}{Lemma}
\newtheorem{cor}{Corollary}

\theoremstyle{definition}
\newtheorem{exa}{Example}

% Some sets
\newcommand{\Nats}{\mathbb{N}}
\newcommand{\Ints}{\mathbb{Z}}
\newcommand{\Reals}{\mathbb{R}}
\newcommand{\Complex}{\mathbb{C}}
\newcommand{\Rats}{\mathbb{Q}}
\newcommand{\Gal}{\operatorname{Gal}}
\newcommand{\Cl}{\operatorname{Cl}}
\begin{document}
\begin{exa}
Let $p$ be a prime number greater than $2$. Are there solutions to $x^2+7=0$ in the field $\Rats_p$ (the \PMlinkname{$p$-adic numbers}{PAdicIntegers})? If there are, $-7$ must be a quadratic residue modulo $p$. Thus, let $p$ be a prime such that 
$$\left( \frac{-7}{p} \right)=1$$
where $( \frac{\cdot}{p} )$ is the Legendre symbol. Hence, there exist $\alpha_0\in \Ints$ such that $\alpha_0^2\equiv -7 \mod p$. We claim that $x^2+7=0$ has a solution in $\Rats_p$ if and only if $-7$ is a quadratic residue modulo $p$. Indeed, if we let $f(x)=x^2+7$ (so $f'(x)=2x$), the element $\alpha_0\in \Ints_p$ satisfies the conditions of the (trivial case of) Hensel's lemma. Therefore there exist a root $\alpha\in \Rats_p$ of $x^2+7=0$. \\
\end{exa}

\begin{exa}
Let $p=2$. Are there any solutions to $x^2+7=0$ in $\Rats_2$? Notice that if we let $f(x)=x^2+7$, then $f'(x)=2x$ and for any $\alpha_0\in \Ints_2$, the number $f'(\alpha_0)=2\alpha_0$ is congruent to $0$ modulo $2$. Thus, we cannot use the trivial case of Hensel's lemma.

Let $\alpha_0=1\in \Ints_2$. Notice that $f(1)=8$ and $f'(1)=2$. Thus
$$|8|_2<|2^2|_2$$
and the general case of Hensel's lemma applies. Hence, there exist a $2$-adic solution to $x^2+7=0$. The following is the \PMlinkname{$2$-adic canonical form}{PAdicCanonicalForm} for one of the solutions:
$$\alpha=1+1\cdot 2^3+1\cdot 2^4 +\ldots=\ldots 11001$$
\end{exa}
%%%%%
%%%%%
\end{document}
