\documentclass[12pt]{article}
\usepackage{pmmeta}
\pmcanonicalname{VariantOfCardanosDerivation}
\pmcreated{2013-03-22 13:38:43}
\pmmodified{2013-03-22 13:38:43}
\pmowner{mathcam}{2727}
\pmmodifier{mathcam}{2727}
\pmtitle{variant of Cardano's derivation}
\pmrecord{9}{34296}
\pmprivacy{1}
\pmauthor{mathcam}{2727}
\pmtype{Proof}
\pmcomment{trigger rebuild}
\pmclassification{msc}{12D10}

% this is the default PlanetMath preamble.  as your knowledge
% of TeX increases, you will probably want to edit this, but
% it should be fine as is for beginners.

% almost certainly you want these
\usepackage{amssymb}
\usepackage{amsmath}
\usepackage{amsfonts}
\usepackage{amsthm}

% used for TeXing text within eps files
%\usepackage{psfrag}
% need this for including graphics (\includegraphics)
%\usepackage{graphicx}
% for neatly defining theorems and propositions
%\usepackage{amsthm}
% making logically defined graphics
%%%\usepackage{xypic}

% there are many more packages, add them here as you need them

% define commands here

\newcommand{\mc}{\mathcal}
\newcommand{\mb}{\mathbb}
\newcommand{\mf}{\mathfrak}
\newcommand{\ol}{\overline}
\newcommand{\ra}{\rightarrow}
\newcommand{\la}{\leftarrow}
\newcommand{\La}{\Leftarrow}
\newcommand{\Ra}{\Rightarrow}
\newcommand{\nor}{\vartriangleleft}
\newcommand{\Gal}{\text{Gal}}
\newcommand{\GL}{\text{GL}}
\newcommand{\Z}{\mb{Z}}
\newcommand{\R}{\mb{R}}
\newcommand{\Q}{\mb{Q}}
\newcommand{\C}{\mb{C}}
\newcommand{\<}{\langle}
\renewcommand{\>}{\rangle}
\begin{document}
\PMlinkescapeword{identities}
By a linear change of variable, a cubic polynomial over ${\mathbb C}$ can
be given the form $x^3+3bx+c$. To find the zeros of this cubic in the
form of surds in $b$ and $c$, make the substitution $x=y^{1/3}+z^{1/3},$ thus replacing one unknown with two, and then write down identities which are suggested by the resulting equation in two unknowns. Specifically, we get
\begin{align}
y+3(y^{1/3}+z^{1/3})y^{1/3}z^{1/3}+z+3b(y^{1/3}+z^{1/3})+c=0.
\end{align}
This will be true if
\begin{align}
y+z+c=0 \\
3y^{1/3}z^{1/3}+3b=0,
\end{align}
which in turn requires
\begin{align}
yz=-b^3.
\end{align}
The pair of equations (2) and (4) is a quadratic system in $y$ and $z$,
readily solved. But notice that (3) puts a restriction on a certain
choice of cube roots.
%%%%%
%%%%%
\end{document}
