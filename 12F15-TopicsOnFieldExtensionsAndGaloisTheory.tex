\documentclass[12pt]{article}
\usepackage{pmmeta}
\pmcanonicalname{TopicsOnFieldExtensionsAndGaloisTheory}
\pmcreated{2013-03-22 17:43:11}
\pmmodified{2013-03-22 17:43:11}
\pmowner{rm50}{10146}
\pmmodifier{rm50}{10146}
\pmtitle{topics on field extensions and Galois theory}
\pmrecord{13}{40165}
\pmprivacy{1}
\pmauthor{rm50}{10146}
\pmtype{Topic}
\pmcomment{trigger rebuild}
\pmclassification{msc}{12F15}
\pmclassification{msc}{12F10}
\pmclassification{msc}{12F05}

\endmetadata

% this is the default PlanetMath preamble.  as your knowledge
% of TeX increases, you will probably want to edit this, but
% it should be fine as is for beginners.

% almost certainly you want these
\usepackage{amssymb}
\usepackage{amsmath}
\usepackage{amsfonts}

% used for TeXing text within eps files
%\usepackage{psfrag}
% need this for including graphics (\includegraphics)
%\usepackage{graphicx}
% for neatly defining theorems and propositions
%\usepackage{amsthm}
% making logically defined graphics
%%%\usepackage{xypic}

% there are many more packages, add them here as you need them

% define commands here

\begin{document}
\subsubsection*{Definitions}
\begin{enumerate}
\item extension field
\item simple field extension
\item \PMlinkname{degree}{Degree}
\item field adjunction
\item composite field
\item finite extension
\item algebraic extension
\item algebraic closure
\item abelian number field
\item abelian extension
\item biquadratic field
\item biquadratic extension
\item quadratic closure
\item cyclic extension
\item pure cubic field
\item number field
\item algebraically solvable
\item norm and trace of algebraic number
\item perfect field
\item cyclotomic field
\item cyclotomic extension
\item \PMlinkname{$p$-extension}{PExtension}
\item radical
\item radical extension
\item solvable by radicals
\item expressible
\item normal extension
\item normal closure
\item separable extension
\item separable closure
\item splitting field
\item fixed field of a set of automorphisms
\item conjugate fields
\item Galois extension
\item Galois closure
\item Galois group
\item absolute Galois group
\item Galois connection
\item inverse Galois problem
\item function field
\item ray class field
\end{enumerate}

\subsubsection*{Field theory in relation to Euclidean geometry}
\begin{enumerate}
\item Euclidean field
\item constructible numbers
\item motivation of definition of constructible numbers
\item compass and straightedge construction
\item theorem on constructible numbers
\item theorem on constructible angles
\item criterion for constructibility of regular polygon
\item classical problems of constructibility
\end{enumerate}

\subsubsection*{Galois theory}
\begin{enumerate}
\item fundamental theorem of Galois theory
\item Galois group of the compositum of two Galois extensions
\item \PMlinkname{Galois subfields of real radical extensions}{GaloisSubfieldsOfRealRootExtensionsAreAtMostQuadratic}
\item \PMlinkname{compositum of a Galois extension and another extension}{CompositumOfAGaloisExtensionAndAnotherExtensionIsGalois}
\item \PMlinkname{criterion for solvability of a polynomial by radicals}{GaloisCriterionForSolvabilityOfAPolynomialByRadicals}
\item Hilbert Theorem 90
\item Shafarevich's theorem (finite solvable groups as Galois groups)
\item \PMlinkname{number fields with a particular finite abelian Galois group}{GaloisGroupsOfFiniteAbelianExtensionsOfMathbbQ}
\end{enumerate}

\subsubsection*{Applications of Galois theory}
\begin{enumerate}
\item Galois group of a biquadratic extension
\item cubic formula
\item quartic formula
\item Galois group of a cubic polynomial
\item Galois group of a quartic polynomial
\item Galois-theoretic derivation of the cubic formula
\item proof of fundamental theorem of algebra
\item casus irreducibilis
\item Cardano's formulae 
\item a condition of algebraic extension
\item primitive element theorem (Steinitz)
\end{enumerate}

%%%%%
%%%%%
\end{document}
