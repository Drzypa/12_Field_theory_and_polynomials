\documentclass[12pt]{article}
\usepackage{pmmeta}
\pmcanonicalname{FactorizationOfPrimitivePolynomial}
\pmcreated{2013-03-22 19:20:30}
\pmmodified{2013-03-22 19:20:30}
\pmowner{pahio}{2872}
\pmmodifier{pahio}{2872}
\pmtitle{factorization of primitive polynomial}
\pmrecord{4}{42290}
\pmprivacy{1}
\pmauthor{pahio}{2872}
\pmtype{Algorithm}
\pmcomment{trigger rebuild}
\pmclassification{msc}{12D99}
\pmclassification{msc}{26C05}
\pmsynonym{primitive factors of primitive polynomial}{FactorizationOfPrimitivePolynomial}
\pmrelated{EliminationOfUnknown}

% this is the default PlanetMath preamble.  as your knowledge
% of TeX increases, you will probably want to edit this, but
% it should be fine as is for beginners.

% almost certainly you want these
\usepackage{amssymb}
\usepackage{amsmath}
\usepackage{amsfonts}

% used for TeXing text within eps files
%\usepackage{psfrag}
% need this for including graphics (\includegraphics)
%\usepackage{graphicx}
% for neatly defining theorems and propositions
 \usepackage{amsthm}
% making logically defined graphics
%%%\usepackage{xypic}

% there are many more packages, add them here as you need them

% define commands here

\theoremstyle{definition}
\newtheorem*{thmplain}{Theorem}

\begin{document}
\PMlinkescapeword{factor}
As an application of the \PMlinkname{parent entry}{EliminationOfUnknown} we take the factorization of a primitive polynomial of $\mathbb{Z}[x]$ into \PMlinkname{primitive}{PrimitivePolynomial} prime factors.\, We shall see that the procedure may be done by performing a finite number of tests.\\

Let
$$a(x) \;=:\; a_nx^n\!+\!a_{n-1}x^{n-1}\!+\ldots+\!a_0$$
be a primitive polynomial in $\mathbb{Z}[x]$.

By the rational root theorem and the factor theorem, one finds all first-degree prime factors $x\!-\!a$ and thus all primitive prime factors of the polynomial $a(x)$.

If $a(x)$ has a primitive quadratic factor, then it has also a factor
\begin{align}
x^2\!+\!px\!+\!q
\end{align}
where $p$ and $q$ are  rationals (and conversely).\, For settling the existence of such a factor we treat $p$ and $q$ as unknowns and perform the long division
$$a(x):(x^2\!+\!px\!+\!q).$$
It gives finally the remainder \,$b(p,\,q)\,x+c(p,\,q)$ where $b(p,\,q)$ and $c(p,\,q)$ belong to $\mathbb{Z}[p,\,q]$.\, According to the \PMlinkname{parent entry}{EliminationOfUnknown} we bring the system
\begin{align*}
\begin{cases}
b(p,\,q) \;=\; 0\\
c(p,\,q) \;=\; 0
\end{cases}
\end{align*}
to the form
\begin{align*}
\begin{cases}
\bar{b}(q) \;=\; 0\\
\bar{c}(p,\,q) \;=\; 0
\end{cases}
\end{align*}
and then can determine the possible rational solutions \,$(p,\,q)$\, of the system via a finite number of tests.\, Hence 
 we find the possible quadratic factors (1) having rational coefficients.\, Such a factor is converted into a primitive one when it is multiplied by the gcd of the denominators of $p$ and $q$.

Determining a possible cubic factor $x^3\!+\!px^2\!+\!qx\!+\!r$ with rational coefficients requires examination of a remainder of the form
$$b(p,\,q,\,r)\,x^2+c(p,\,q,\,r)\,x+d(p,\,q,\,r).$$
In the needed system
\begin{align*}
\begin{cases}
b(p,\,q,\,r) \;=\; 0\\
c(p,\,q,\,r) \;=\; 0\\
d(p,\,q,\,r) \;=\; 0
\end{cases}
\end{align*}
we have to perform two eliminations.\, Then we can act as above and find a primitive cubic factor of $a(x)$.\, Similarly also the primitive factors of higher degree.\, All in all, one needs only look for factors of degree $\le \frac{n}{2}$.

\begin{thebibliography}{9}
\bibitem{K.V.}{\sc K. V\"ais\"al\"a:} {\em Lukuteorian ja korkeamman algebran alkeet}. \,Tiedekirjasto No. 17. \, Kustannusosakeyhti\"o Otava, Helsinki (1950).
\end{thebibliography}
%%%%%
%%%%%
\end{document}
