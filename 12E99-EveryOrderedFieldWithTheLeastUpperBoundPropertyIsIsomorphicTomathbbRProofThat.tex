\documentclass[12pt]{article}
\usepackage{pmmeta}
\pmcanonicalname{EveryOrderedFieldWithTheLeastUpperBoundPropertyIsIsomorphicTomathbbRProofThat}
\pmcreated{2013-03-22 14:10:51}
\pmmodified{2013-03-22 14:10:51}
\pmowner{mps}{409}
\pmmodifier{mps}{409}
\pmtitle{every ordered field with the least upper bound property is isomorphic to $\mathbb{R}$, proof that}
\pmrecord{9}{35607}
\pmprivacy{1}
\pmauthor{mps}{409}
\pmtype{Proof}
\pmcomment{trigger rebuild}
\pmclassification{msc}{12E99}
\pmclassification{msc}{54C30}
\pmclassification{msc}{26-00}
\pmclassification{msc}{12D99}

\endmetadata

% this is the default PlanetMath preamble.  as your knowledge
% of TeX increases, you will probably want to edit this, but
% it should be fine as is for beginners.

% almost certainly you want these
\usepackage{amssymb}
\usepackage{amsmath}
\usepackage{amsfonts}

% used for TeXing text within eps files
%\usepackage{psfrag}
% need this for including graphics (\includegraphics)
%\usepackage{graphicx}
% for neatly defining theorems and propositions
%\usepackage{amsthm}
% making logically defined graphics
%%%\usepackage{xypic}

% there are many more packages, add them here as you need them

% define commands here
\newcommand{\embed}{\hookrightarrow}
\begin{document}
Let $F$ be an ordered field with the least upper bound property.  By the 
order properties of $F$, $0 < 1_F$ and by an induction argument $0 < n\cdot 1_F$ 
for any positive integer $n$.  Hence the characteristic of the field $F$ is 
zero, implying that there is an order-preserving embedding $j\colon \mathbb{Q}\to F$.

We would like to extend this map to an embedding of $\mathbb{R}$ into $F$.  Let $r\in\mathbb{R}$
and let $D_r = \{ q \in\mathbb{Q}\colon q < r \}$ be the associated Dedekind cut.  Since $D_r$ is 
nonempty and bounded above in $\mathbb{Q}$, it follows that the set $j(D_r)$ is nonempty and
bounded above in $F$.  Applying the least upper bound property of $F$, define a function
$\widetilde{\jmath}\colon\mathbb{R}\to F$ by
\[
\widetilde{\jmath}(r) = \sup\left(j(D_r)\right).
\]
One can check that $\widetilde{\jmath}$ is an order-preserving field homomorphism.  
By replacing $F$ with the isomorphic field $F\setminus\widetilde{\jmath}(\mathbb{R})\cup\mathbb{R}$,
we may assume that $\mathbb{R}\subset F$.

We claim that in fact $\mathbb{R}=F$.  To see this, first recall that 
\PMlinkname{since $F$ is a partially ordered group with the least upper 
bound property, $F$ has the Archimedean property}{DistributivityInPoGroups}.
So for any $f\in F$, there exists some positive integer $n$ such that $-n < f < n$.
Hence the set
$D'_f = \{ r\in\mathbb{R} \colon r < f \}\subset\mathbb{R}$ is nonempty and
bounded above, implying that $f' = \sup_{\mathbb{R}} D'_f$ lies in $\mathbb{R}$.
Now observe that applying the least upper bound axiom in $F$ gives us that
$f = \sup_F D'_f$.  Since $f'$ is an upper bound of $D'_f$ in $F$, it follows
that $f\le f'$.

Seeking a contradiction, suppose $f<f'$.  By the Archimedean property,
there is some positive integer $n$ such that $f<f'-n^{-1}<f'$.  Because
$f'=\sup_{\mathbb{R}} D'_f$, we obtain $f'-n^{-1}<f$, which implies the 
contradiction
$f<f$.  Therefore $f=f'$, and so $f\in\mathbb{R}$.  This completes the proof.

\PMlinkescapeword{bounded}
\PMlinkescapeword{completes}
\PMlinkescapeword{embedding}
\PMlinkescapeword{isomorphic}
\PMlinkescapeword{satisfies}

%%%%%
%%%%%
\end{document}
