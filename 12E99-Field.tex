\documentclass[12pt]{article}
\usepackage{pmmeta}
\pmcanonicalname{Field}
\pmcreated{2013-03-22 11:48:43}
\pmmodified{2013-03-22 11:48:43}
\pmowner{djao}{24}
\pmmodifier{djao}{24}
\pmtitle{field}
\pmrecord{9}{30355}
\pmprivacy{1}
\pmauthor{djao}{24}
\pmtype{Definition}
\pmcomment{trigger rebuild}
\pmclassification{msc}{12E99}
\pmclassification{msc}{03A05}

\usepackage{amssymb}
\usepackage{amsmath}
\usepackage{amsfonts}
\usepackage{graphicx}
%%%%\usepackage{xypic}
\begin{document}
A \emph{field} is a set $F$ together with two binary operations on $F$, called addition and multiplication, and denoted $+$ and $\cdot$, satisfying the following properties, for all $a,b,c \in F$:

\begin{enumerate}
\item $a + (b+c) = (a+b)+ c$ (associativity of addition)
\item $a+b = b+a$ (commutativity of addition)
\item $a+0 = a$ for some element $0 \in F$ (existence of zero element)
\item $a+(-a) = 0$ for some element $-a \in F$ (existence of additive inverses)
\item $a\cdot (b\cdot c) = (a\cdot b)\cdot c$ (associativity of multiplication)
\item $a\cdot b = b\cdot a$ (commutativity of multiplication)
\item $a\cdot 1 = a$ for some element $1 \in F$, with $1 \neq 0$ (existence of unity element)
\item If $a \neq 0$, then $a \cdot a^{-1} = 1$ for some element $a^{-1} \in F$ (existence of multiplicative inverses)
\item $a\cdot (b+c) = (a\cdot b) + (a\cdot c)$ (distributive property)
\end{enumerate}

Equivalently, a field is a commutative ring $F$ with identity such that:
\begin{itemize}
\item $1 \neq 0$
\item If $a \in F$, and $a \neq 0$, then there exists $b \in F$ with $a \cdot b = 1$.
\end{itemize}

%%%%%
%%%%%
%%%%%
%%%%%
\end{document}
