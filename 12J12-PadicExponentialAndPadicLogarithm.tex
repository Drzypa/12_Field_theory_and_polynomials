\documentclass[12pt]{article}
\usepackage{pmmeta}
\pmcanonicalname{PadicExponentialAndPadicLogarithm}
\pmcreated{2013-03-22 15:13:50}
\pmmodified{2013-03-22 15:13:50}
\pmowner{alozano}{2414}
\pmmodifier{alozano}{2414}
\pmtitle{p-adic exponential and p-adic logarithm}
\pmrecord{6}{37000}
\pmprivacy{1}
\pmauthor{alozano}{2414}
\pmtype{Definition}
\pmcomment{trigger rebuild}
\pmclassification{msc}{12J12}
\pmclassification{msc}{11S99}
\pmclassification{msc}{11S80}
\pmsynonym{$p$-adic exponential}{PadicExponentialAndPadicLogarithm}
\pmsynonym{$p$-adic logarithm}{PadicExponentialAndPadicLogarithm}
\pmrelated{PAdicRegulator}
\pmrelated{PAdicAnalytic}
\pmrelated{GeneralPower}
\pmdefines{general $p$-adic power}

% this is the default PlanetMath preamble.  as your knowledge
% of TeX increases, you will probably want to edit this, but
% it should be fine as is for beginners.

% almost certainly you want these
\usepackage{amssymb}
\usepackage{amsmath}
\usepackage{amsthm}
\usepackage{amsfonts}

% used for TeXing text within eps files
%\usepackage{psfrag}
% need this for including graphics (\includegraphics)
%\usepackage{graphicx}
% for neatly defining theorems and propositions
%\usepackage{amsthm}
% making logically defined graphics
%%%\usepackage{xypic}

% there are many more packages, add them here as you need them

% define commands here

\newtheorem{thm}{Theorem}
\newtheorem{defn}{Definition}
\newtheorem*{prop}{Proposition}
\newtheorem{lemma}{Lemma}
\newtheorem{cor}{Corollary}

\theoremstyle{definition}
\newtheorem{exa}{Example}

% Some sets
\newcommand{\Nats}{\mathbb{N}}
\newcommand{\Ints}{\mathbb{Z}}
\newcommand{\Reals}{\mathbb{R}}
\newcommand{\Complex}{\mathbb{C}}
\newcommand{\Rats}{\mathbb{Q}}
\newcommand{\Gal}{\operatorname{Gal}}
\newcommand{\Cl}{\operatorname{Cl}}
\begin{document}
Let $p$ be a prime number and let $\Complex_p$ be the field of \PMlinkname{complex $p$-adic numbers}{ComplexPAdicNumbers5}.

\begin{defn}
The $p$-adic exponential is a function $\exp_p\colon R \to \Complex_p$ defined by
$$\exp_p(s)=\sum_{n=0}^\infty \frac{s^n}{n!}$$
where $$R=\{ s\in \Complex_p : |s|_p<\frac{1}{p^{1/(p-1)}}\}.$$ 
\end{defn}

The domain of $\exp_p$ is restricted because the radius of convergence of the series $\sum_{n=0}^\infty z^n/n!$ over $\Complex_p$ is precisely $r=p^{-1/(p-1)}$. Recall that, for $z\in \Rats_p$, we define 
$$|z|_p=\frac{1}{p^{\nu_p(z)}}$$
where $\nu_p(z)$ is the largest exponent $\nu$ such that $p^\nu$ divides $z$. For example, if $p\geq 3$, then $\exp_p$ is defined over $p\Ints_p$. However, $e=\exp_p(1)$ is never defined, but $\exp_p(p)$ is well-defined over $\Complex_p$ (when $p=2$, the number $e^4\in \Complex_2$ because $|4|_2=0.25<0.5=r$).

\begin{defn}
The $p$-adic logarithm is a function $\log_p\colon S\to \Complex_p$ defined by
$$\log_p(1+s)=\sum_{n=1}^\infty (-1)^{n+1}\frac{s^n}{n}$$
where
$$S=\{ s\in \Complex_p : |s|_p<1\}.$$
We extend the $p$-adic logarithm to the entire $p$-adic complex field $\Complex_p$ as follows. One can show that:
$$\Complex_p=\{ p^t\cdot w\cdot u: t\in \Rats,\ w\in W,\ u\in U\}=p^{\Rats}\times W \times U$$
where $W$ is the group of all roots of unity of order prime to $p$ in $\Complex_p^\times$ and $U$ is the open circle of radius centered at $z=1$:
$$U=\{ s\in \Complex_p : |s-1|_p < 1\}.$$
We define $\log_p\colon \Complex_p \to \Complex_p$ by:
$$\log_p(s)=log_p(u)$$
where $s=p^r\cdot w \cdot u$, with $w\in W$ and $u\in U$.
\end{defn}

\begin{prop}[Properties of $\exp_p$ and $\log_p$]
With $\exp_p$ and $\log_p$ defined as above:
\begin{enumerate}
\item If $\exp_p(s)$ and $\exp_p(t)$ are defined then $\exp_p(s+t)=\exp_p(s)\exp_p(t)$.
\item $\log_p(s)=0$ if and only if $s$ is a rational power of $p$ times a root of unity.
\item $\log_p(xy)=\log_p(x)+\log_p(y)$, for all $x$ and $y$.
\item If $|s|_p<p^{-1/(p-1)}$ then 
$$\exp_p(\log_p(1+s))=1+s,\quad \log_p(\exp_p(s))=s.$$
\end{enumerate}

\end{prop}

In a similar way one defines the general $p$-adic power by:
$$s^z=\exp_p(z\log_p(s))$$
where it makes sense.
%%%%%
%%%%%
\end{document}
