\documentclass[12pt]{article}
\usepackage{pmmeta}
\pmcanonicalname{FieldHomomorphism}
\pmcreated{2013-03-22 13:54:54}
\pmmodified{2013-03-22 13:54:54}
\pmowner{alozano}{2414}
\pmmodifier{alozano}{2414}
\pmtitle{field homomorphism}
\pmrecord{9}{34670}
\pmprivacy{1}
\pmauthor{alozano}{2414}
\pmtype{Definition}
\pmcomment{trigger rebuild}
\pmclassification{msc}{12E99}
\pmsynonym{field monomorphism}{FieldHomomorphism}
%\pmkeywords{field}
%\pmkeywords{map}
\pmrelated{RingHomomorphism}
\pmdefines{field homomorphism}
\pmdefines{field isomorphism}

% this is the default PlanetMath preamble.  as your knowledge
% of TeX increases, you will probably want to edit this, but
% it should be fine as is for beginners.

% almost certainly you want these
\usepackage{amssymb}
\usepackage{amsmath}
\usepackage{amsthm}
\usepackage{amsfonts}

% used for TeXing text within eps files
%\usepackage{psfrag}
% need this for including graphics (\includegraphics)
%\usepackage{graphicx}
% for neatly defining theorems and propositions
%\usepackage{amsthm}
% making logically defined graphics
%%%\usepackage{xypic}

% there are many more packages, add them here as you need them

% define commands here

\newtheorem{thm}{Theorem}
\newtheorem*{defn}{Definition}
\newtheorem{prop}{Proposition}
\newtheorem*{lemma}{Lemma}
\newtheorem{cor}{Corollary}

% Some sets
\newcommand{\Nats}{\mathbb{N}}
\newcommand{\Ints}{\mathbb{Z}}
\newcommand{\Reals}{\mathbb{R}}
\newcommand{\Complex}{\mathbb{C}}
\newcommand{\Rats}{\mathbb{Q}}
\begin{document}
Let $F$ and $K$ be fields. 
\begin{defn}
A {\em field homomorphism} is a function $\psi\colon F \to K$ such that:
\begin{enumerate}
\item $\psi(a+b) = \psi(a)+\psi(b)$ for all $a,b \in F$
\item $\psi(a\cdot b) = \psi(a) \cdot \psi(b)$ for all $a,b \in F$
\item $\psi(1)=1,\quad \psi(0)=0$
\end{enumerate}
If $\psi$ is injective and surjective, then we say that $\psi$ is a \emph{field isomorphism}.
\end{defn}

\begin{lemma}
Let $\psi\colon F\to K$ be a field homomorphism. Then $\psi$ is
injective.
\end{lemma}
\begin{proof}
Indeed, if $\psi$ is a field homomorphism, in particular it is a
ring homomorphism. Note that the kernel of a ring homomorphism is
an ideal and a field $F$ only has two ideals, namely $\{0\}, F$.
Moreover, by the definition of field homomorphism, $\psi(1)=1$,
hence $1$ is not in the kernel of the map, so the kernel must be
equal to $\{0\}$.
\end{proof}


{\bf Remark}: For this reason the terms ``field homomorphism'' and
``field monomorphism'' are synonymous. Also note that if $\psi$ is
a field monomorphism, then
$$\psi(F)\cong F, \quad \psi(F)\subseteq K$$
so there is a ``copy'' of $F$ in $K$. In other words, if
$$\psi\colon F\to K$$ is a field homomorphism then there exist a
subfield $H$ of $K$ such that $H\cong F$. Conversely, suppose
there exists $H\subset K$ with $H$ isomorphic to $F$. Then there
is an isomorphism $$\chi \colon F \to H$$ and we also have the
inclusion homomorphism $$\iota\colon H \hookrightarrow K$$ Thus
the composition
$$\iota \circ \chi\colon F \to K$$
is a field homomorphism.

{\bf Remark}: Let $\psi : F \to K$ be a field homomorphism. We claim that the characteristic of $F$ and $K$ must be the same. Indeed, since $\psi(1_F)=1_K$ and $\psi(0_F)=0_K$ then $\psi(n\cdot 1_F)=n\cdot 1_K$ for all natural numbers $n$. If the characteristic of $F$ is $p>0$ then $0=\psi(p\cdot 1)=p\cdot 1$ in $K$, and so the characteristic of $K$ is also $p$. If the characteristic of $F$ is $0$, then the characteristic of $K$ must be $0$ as well. For if $p\cdot 1=0$ in $K$ then $\psi(p\cdot 1)=0$, and since $\psi$ is injective by the lemma, we would have $p\cdot 1=0$ in $F$ as well.
%%%%%
%%%%%
\end{document}
