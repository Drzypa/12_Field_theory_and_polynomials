\documentclass[12pt]{article}
\usepackage{pmmeta}
\pmcanonicalname{Multiplicity}
\pmcreated{2013-03-22 14:24:18}
\pmmodified{2013-03-22 14:24:18}
\pmowner{pahio}{2872}
\pmmodifier{pahio}{2872}
\pmtitle{multiplicity}
\pmrecord{14}{35905}
\pmprivacy{1}
\pmauthor{pahio}{2872}
\pmtype{Definition}
\pmcomment{trigger rebuild}
\pmclassification{msc}{12D10}
\pmsynonym{order of the zero}{Multiplicity}
\pmrelated{OrderOfVanishing}
\pmrelated{DerivativeOfPolynomial}
\pmdefines{zero of order}
\pmdefines{multiple zero}
\pmdefines{simple zero}
\pmdefines{simple}

% this is the default PlanetMath preamble.  as your knowledge
% of TeX increases, you will probably want to edit this, but
% it should be fine as is for beginners.

% almost certainly you want these
\usepackage{amssymb}
\usepackage{amsmath}
\usepackage{amsfonts}

% used for TeXing text within eps files
%\usepackage{psfrag}
% need this for including graphics (\includegraphics)
%\usepackage{graphicx}
% for neatly defining theorems and propositions
%\usepackage{amsthm}
% making logically defined graphics
%%%\usepackage{xypic}

% there are many more packages, add them here as you need them

% define commands here
\begin{document}
If a polynomial $f(x)$ in $\mathbb{C}[x]$ is divisible by $(x-a)^m$ but not by $(x-a)^{m+1}$ ($a$ is some complex number, \,$m \in \mathbb{Z}_+$), we say that \, $x = a$ \, is a\, {\em zero of the polynomial with multiplicity $m$} or alternatively a {\em zero of order $m$}.

Generalization of the multiplicity to \PMlinkname{real}{RealFunction} and complex functions (by rspuzio): \,If the function $f$ is continuous on some open set $D$ and \,$f(a) = 0$\, for some \,$a \in D$, then the zero of $f$ at $a$ is said to be of multiplicity $m$ if $\frac{f(z)}{(z\!-\!a)^m}$ is continuous in $D$ but $\frac{f(z)}{(z-a)^{m+1}}$ is not.

If\, $m \geqq 2$, we speak of a {\em multiple zero}; if\, $m = 1$, we speak of a {\em simple zero}.\, If\, $m = 0$, then actually the number $a$ is not a zero of $f(x)$, i.e.\, $f(a) \neq 0$. 

Some properties (from which 2, 3 and 4 concern only polynomials):
\begin{enumerate}

\item The zero $a$ of a polynomial $f(x)$ with multiplicity $m$ is a zero of the \PMlinkescapetext{derivative} $f'(x)$ with multiplicity $m\!-\!1$.
\item The zeros of the polynomial $\gcd(f(x), f'(x))$ are same as the multiple zeros of $f(x)$.
\item The quotient $\displaystyle\frac{f(x)}{\gcd(f(x), f'(x))}$ has the same zeros as $f(x)$ but they all are \PMlinkescapetext{simple}.
\item The zeros of any irreducible polynomial are \PMlinkescapetext{simple}.
\end{enumerate}
%%%%%
%%%%%
\end{document}
