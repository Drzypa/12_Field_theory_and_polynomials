\documentclass[12pt]{article}
\usepackage{pmmeta}
\pmcanonicalname{TotallyRealAndImaginaryFields}
\pmcreated{2013-03-22 13:55:02}
\pmmodified{2013-03-22 13:55:02}
\pmowner{alozano}{2414}
\pmmodifier{alozano}{2414}
\pmtitle{totally real and imaginary fields}
\pmrecord{8}{34673}
\pmprivacy{1}
\pmauthor{alozano}{2414}
\pmtype{Definition}
\pmcomment{trigger rebuild}
\pmclassification{msc}{12D99}
\pmsynonym{complex multiplication field}{TotallyRealAndImaginaryFields}
%\pmkeywords{totally}
%\pmkeywords{real}
%\pmkeywords{imaginary}
%\pmkeywords{complex multiplication}
\pmrelated{RealAndComplexEmbeddings}
\pmrelated{TotallyImaginaryExamplesOfTotallyReal}
\pmrelated{ExamplesOfRamificationOfArchimedeanPlaces}
\pmdefines{totally real field}
\pmdefines{totally imaginary field}
\pmdefines{CM-field}
\pmdefines{maximal real subfield}

% this is the default PlanetMath preamble.  as your knowledge
% of TeX increases, you will probably want to edit this, but
% it should be fine as is for beginners.

% almost certainly you want these
\usepackage{amssymb}
\usepackage{amsmath}
\usepackage{amsthm}
\usepackage{amsfonts}

% used for TeXing text within eps files
%\usepackage{psfrag}
% need this for including graphics (\includegraphics)
%\usepackage{graphicx}
% for neatly defining theorems and propositions
%\usepackage{amsthm}
% making logically defined graphics
%%%\usepackage{xypic}

% there are many more packages, add them here as you need them

% define commands here

\newtheorem{thm}{Theorem}
\newtheorem{defn}{Definition}
\newtheorem{prop}{Proposition}
\newtheorem{lemma}{Lemma}
\newtheorem{cor}{Corollary}

% Some sets
\newcommand{\Nats}{\mathbb{N}}
\newcommand{\Ints}{\mathbb{Z}}
\newcommand{\Reals}{\mathbb{R}}
\newcommand{\Complex}{\mathbb{C}}
\newcommand{\Rats}{\mathbb{Q}}
\begin{document}
For this entry, we follow the notation of the entry real and
complex embeddings.

Let $K$ be a subfield of the complex numbers, $\Complex$, and let
$\Sigma_K$ be the set of all embeddings of $K$ in $\Complex$.

\begin{defn}
With $K$ as above:
\begin{enumerate}
\item $K$ is a \emph{totally real field} if all embeddings
$\psi\in \Sigma_K$ are real embeddings.

\item $K$ is a \emph{totally imaginary field} if all embeddings
$\psi\in\Sigma_K$ are (non-real) complex embeddings.

\item $K$ is a \emph{CM-field} or \emph{complex multiplication
field} if $K$ is a totally imaginary quadratic extension of a totally real
field. 
\end{enumerate}
\end{defn}

Note that, for example, one can obtain a CM-field $K$ from a totally real number field $F$ by adjoining the square root of a number all of whose
conjugates are negative.

Note: A complex number $\omega$ is real if and only if
$\bar{\omega}$, the complex conjugate of $\omega$, equals
$\omega$:
$$\omega\in \Reals \Leftrightarrow \omega=\bar{\omega}$$
Thus, a field $K$ which is fixed \emph{pointwise} by complex
conjugation is real (i.e. strictly contained in $\Reals$). However, $K$ might not be {\it totally real}. For example, let $\alpha$ be the unique real third root of $2$. Then $\Rats(\alpha)$ is real but not totally real. \\

Given a field $L$, the subfield of
$L$ fixed pointwise by complex conjugation is called the
\emph{maximal real subfield of} $L$.

For examples (of $(1),(2)$ and $(3)$), see examples of totally real fields.
%%%%%
%%%%%
\end{document}
