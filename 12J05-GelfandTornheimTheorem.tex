\documentclass[12pt]{article}
\usepackage{pmmeta}
\pmcanonicalname{GelfandTornheimTheorem}
\pmcreated{2013-03-22 14:11:49}
\pmmodified{2013-03-22 14:11:49}
\pmowner{pahio}{2872}
\pmmodifier{pahio}{2872}
\pmtitle{Gelfand--Tornheim theorem}
\pmrecord{40}{35628}
\pmprivacy{1}
\pmauthor{pahio}{2872}
\pmtype{Theorem}
\pmcomment{trigger rebuild}
\pmclassification{msc}{12J05}
\pmsynonym{Gelfand-Tornheim theorem}{GelfandTornheimTheorem}
%\pmkeywords{real numbers}
%\pmkeywords{complex numbers}
\pmrelated{ExtensionOfKrullValuation}
\pmrelated{TopicEntryOnRealNumbers}
\pmrelated{BanachAlgebra}
\pmrelated{NormedAlgebra}
\pmrelated{ArchimedeanOrderedFieldsAreReal}
\pmdefines{normed field}

\endmetadata

% this is the default PlanetMath preamble.  as your knowledge
% of TeX increases, you will probably want to edit this, but
% it should be fine as is for beginners.

% almost certainly you want these
\usepackage{amssymb}
\usepackage{amsmath}
\usepackage{amsfonts}

% used for TeXing text within eps files
%\usepackage{psfrag}
% need this for including graphics (\includegraphics)
%\usepackage{graphicx}
% for neatly defining theorems and propositions
 \usepackage{amsthm}
% making logically defined graphics
%%%\usepackage{xypic}

% there are many more packages, add them here as you need them

% define commands here
\theoremstyle{definition}
\newtheorem*{thmplain}{Theorem}
\begin{document}
\begin{thmplain}
\, Any normed field is isomorphic either to the field $\mathbb{R}$ of real numbers or to the field $\mathbb{C}$ of complex numbers.
\end{thmplain}

The {\em normed field} means a field $K$ having a subfield $R$ isomorphic to $\mathbb{R}$ and satisfying the following: \,
There is a mapping $\|\cdot\|$ from $K$ to the set of non-negative reals such that
\begin{itemize}
 \item $\|a\| = 0$\, iff\, $a = 0$
 \item $\|ab\| \leqq \|a\|\cdot\|b\|$
 \item $\|a+b\| \leqq \|a\|+\|b\|$
 \item $\|ab\| = |a|\cdot\|b\|$\, when\, $a \in R$\, and\, $b \in K$
\end{itemize}

Using the Gelfand--Tornheim theorem, it can be shown that the only fields with archimedean valuation are isomorphic to subfields of $\mathbb{C}$ and that the valuation is the usual absolute value (modulus) or some positive power of the absolute value.

\begin{thebibliography}{8}
\bibitem{artin}Emil Artin: {\em \PMlinkescapetext{Theory of Algebraic Numbers}}. \,Lecture notes. \,Mathematisches Institut, G\"ottingen (1959).
\end{thebibliography}
%%%%%
%%%%%
\end{document}
