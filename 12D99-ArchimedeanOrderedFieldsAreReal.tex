\documentclass[12pt]{article}
\usepackage{pmmeta}
\pmcanonicalname{ArchimedeanOrderedFieldsAreReal}
\pmcreated{2013-03-22 17:26:22}
\pmmodified{2013-03-22 17:26:22}
\pmowner{rspuzio}{6075}
\pmmodifier{rspuzio}{6075}
\pmtitle{Archimedean ordered fields are real}
\pmrecord{12}{39819}
\pmprivacy{1}
\pmauthor{rspuzio}{6075}
\pmtype{Theorem}
\pmcomment{trigger rebuild}
\pmclassification{msc}{12D99}
\pmrelated{GelfandTornheimTheorem}

% this is the default PlanetMath preamble.  as your knowledge
% of TeX increases, you will probably want to edit this, but
% it should be fine as is for beginners.

% almost certainly you want these
\usepackage{amssymb}
\usepackage{amsmath}
\usepackage{amsfonts}

% used for TeXing text within eps files
%\usepackage{psfrag}
% need this for including graphics (\includegraphics)
%\usepackage{graphicx}
% for neatly defining theorems and propositions
\usepackage{amsthm}
% making logically defined graphics
%%%\usepackage{xypic}

% there are many more packages, add them here as you need them

% define commands here
\newtheorem{dfn}{Definition}
\newtheorem{thm}{Theorem}
\begin{document}
In this entry, we shall show that every Archimedean ordered field is isomorphic to
a subfield of the field of real numbers.  To accomplish this, we shall construct an
isomorphism using Dedekind cuts.

As a preliminary, we agree to some conventions.  Let $\mathbb{F}$ denote an ordered
field with ordering relation $>$.  We will identify the integers with multiples of
the multiplicative identity of fields.  We further assume that $\mathbb{F}$ satisfies 
the Archimedean property: for every element $x$ of $\mathbb{F}$, there exists an 
integer $n$ such that $x < n$.

Since $\mathbb{F}$ is an ordered field, it must have characteristic zero.  Hence,
the subfield generated by the multiplicative identity is isomorphic to the field
of rational numbers.  Following the convention proposed above, we will identify
this subfield with $\mathbb{Q}$.  We use this subfield to construct a map $\rho$
from $\mathbb{F}$ to $\mathbb{R}$:

\begin{dfn}
For every $x \in \mathbb{F}$, we define
\[
\rho (x) = 
\left(
      \{y \in \mathbb{Q} \mid y > x \} , 
      \{y \in \mathbb{Q} \mid y \le x \}
\right)
\]
\end{dfn}

\begin{thm}
For every $x \in \mathbb{F}$, we find that $\rho (x)$ is a Dedekind cut.
\end{thm}

\begin{proof}
Because, for all $y \in \mathbb{Q}$, we have either $y > x$ or $y \le x$, the two
sets of $\rho (x)$ form a partition of $\mathbb{Q}$.  Furthermore, every element
of the latter set is less than every element of the former set.  By the Archimedean
property, there exists an integer $n$ such that $x < n$; hence the former set is
not empty.  Likewise, there exists and integer $m$ such that $-x < m$, or $x > -m$, 
so the latter set is also not empty.
\end{proof}

Having seen that $\rho$ is a bona fide map into the real numbers, we now show that
it is not just any old map, but a monomorphism of fields.

\begin{thm}
The map $\rho \colon \mathbb{F} \to \mathbb{R}$ is a monomorphism.
\end{thm}

\begin{proof}
Let $p$ and $q$ be elements of $\mathbb{F}$; set $(A,B) = \rho(p)$, set $(C,D) = 
\rho (q)$, and set $(E,F) = \rho (p+q)$.  Since  $a > p$ and $b > q$ implies $a + b > p + q$ 
for all rational numbers $a$ and $b$, it follows that $a \in A$ and $b \in C$ implies that
$a + b \in E$.  Likewise, since $a \le p$ and $b \le q$ implies $a + b \le p + q$
for all rational numbers $a$ and $b$, it follows that $a \in B$ and $b \in D$ implies
$a + b \in F$.  Hence, $\rho (p) + \rho (q) = \rho (p + q)$.

Since a rational number is positive if and only if it is greater than $0$,  it
follows that $\rho (0) = 0$.  Together with the fact proven in the last paragraph,
this implies that $\rho (-x) = - \rho (x)$ for all $x \in \mathbb{F}$.

Suppose that $p$ and $q$ are positive elements of $\mathbb{F}$.  As before, set 
$(A,B) = \rho(p)$, set $(C,D) = \rho (q)$, and set $(E,F) = \rho (p \cdot q)$.  Since  
$a > p$ and $b > q$ implies $a \cdot b > p \cdot q$ for all rational numbers $a$ 
and $b$, it follows that $a \in A$ and $b \in C$ implies that $a\cdot b \in E$.
Likewise, since $a \le p$ and $b \le q$ implies $a \cdot b \le p \cdot q$
for all rational numbers $a$ and $b$, it follows that $a \in B$ and $b \in D$ implies
$a \cdot b \in F$.  Hence, $\rho (p) \cdot \rho (q) = \rho (p \cdot q)$.

By using the fact demonstrated previously that $\rho (-x) = -\rho (x)$, we may extend
what was shown above to the statement that $\rho (p \cdot q) = \rho (p) \cdot \rho (q)$
for all $p,q \in \mathbb{F}$.  Thus, $\rho$ is a morphism of fields.  Since $\mathbb{F}$ 
is Archmiedean, if $p \neq q$, there must exist a rational number $r$ between $p$ and $q$,
hence $\rho (p) \neq \rho(q)$, so $\rho$ is a monomorphism.
\end{proof}

Since $\rho$ is a morphism of fields, its image is a subring of $\mathbb{R}$.  Since $\rho$
is a monomorphism, its restriction to this image is an isomorphism, hence $\mathbb{F}$ is
isomorphic to a subfield of $\mathbb{R}$.
%%%%%
%%%%%
\end{document}
