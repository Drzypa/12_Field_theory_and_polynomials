\documentclass[12pt]{article}
\usepackage{pmmeta}
\pmcanonicalname{ComplexPadicNumbers}
\pmcreated{2013-03-22 15:13:44}
\pmmodified{2013-03-22 15:13:44}
\pmowner{alozano}{2414}
\pmmodifier{alozano}{2414}
\pmtitle{complex p-adic numbers}
\pmrecord{6}{36998}
\pmprivacy{1}
\pmauthor{alozano}{2414}
\pmtype{Definition}
\pmcomment{trigger rebuild}
\pmclassification{msc}{12J12}
\pmclassification{msc}{11S99}
\pmsynonym{complex $p$-adic numbers}{ComplexPadicNumbers}

% this is the default PlanetMath preamble.  as your knowledge
% of TeX increases, you will probably want to edit this, but
% it should be fine as is for beginners.

% almost certainly you want these
\usepackage{amssymb}
\usepackage{amsmath}
\usepackage{amsthm}
\usepackage{amsfonts}

% used for TeXing text within eps files
%\usepackage{psfrag}
% need this for including graphics (\includegraphics)
%\usepackage{graphicx}
% for neatly defining theorems and propositions
%\usepackage{amsthm}
% making logically defined graphics
%%%\usepackage{xypic}

% there are many more packages, add them here as you need them

% define commands here

\newtheorem{thm}{Theorem}
\newtheorem*{defn}{Definition}
\newtheorem*{prop}{Proposition}
\newtheorem{lemma}{Lemma}
\newtheorem{cor}{Corollary}

\theoremstyle{definition}
\newtheorem{exa}{Example}

% Some sets
\newcommand{\Nats}{\mathbb{N}}
\newcommand{\Ints}{\mathbb{Z}}
\newcommand{\Reals}{\mathbb{R}}
\newcommand{\Complex}{\mathbb{C}}
\newcommand{\Rats}{\mathbb{Q}}
\newcommand{\Gal}{\operatorname{Gal}}
\newcommand{\Cl}{\operatorname{Cl}}
\begin{document}
First, we review a possible construction of the complex numbers. We start from the rational numbers, $\Rats$, which we consider as a metric space, where the distance is given by the usual absolute value $|\cdot|$, e.g. $|-3/2|=3/2$. As we know, the field of rational numbers is not an algebraically closed field (e.g. $i=\sqrt{-1} \notin \Rats$). Let $\overline{\Rats}$ be a fixed algebraic closure of $\Rats$. The absolute value in $\Rats$ extends uniquely to $\overline{\Rats}$. However, $\overline{\Rats}$ is not complete with respect to $|\cdot|$ (e.g. $e=\sum_{n\geq 0} 1/n!\notin \overline{\Rats}$ because e is transcendental). The completion of $\overline{\Rats}$ with respect to $|\cdot |$ is $\Complex$, the field of complex numbers. 

\subsection*{Construction of $\Complex_p$}

We follow the construction of $\Complex$ above to build $\Complex_p$. Let $p$ be a prime number and let $\Rats_p$ be the \PMlinkname{$p$-adic rationals}{PAdicIntegers} or ($p$-adic numbers). The $p$-adics, $\Rats_p$, are the completion of $\Rats$ with respect to the usual \PMlinkname{$p$-adic valuation}{PAdicValuation} $|\cdot|_p$. Thus, we regard $(\Rats_p, |\cdot|_p)$ as a complete metric space. However, the field $\Rats_p$ is not algebraically closed (e.g. $i=\sqrt{-1}\in \Rats_p$ if and only if $p \equiv 1 \mod 4$). Let $\overline{\Rats}_p$ be a fixed algebraic closure of $\Rats_p$. The $p$-adic valuation $|\cdot|_p$ extends uniquely to $\overline{\Rats}_p$. However:

\begin{prop}
The field $\overline{\Rats}_p$ is not complete with respect to $|\cdot|_p$.
\end{prop}
\begin{proof}
Let $\beta_n$ be defined as:
$$\beta_n=\begin{cases}
e^{2\pi i/n}, \text{ if } (n,p)=1;\\
1 ,\text{ otherwise.}
\end{cases}$$
One can prove that if we define:
$$\alpha=\sum_{n=1}^\infty \beta_n p^n$$
then $\alpha\notin \overline{\Rats}_p$, although $\sum_{n=m}^\infty \beta_n p^n \to 0$ as $m\to \infty$ (see \cite{wash}, p. 48, for details). Thus, $\overline{\Rats}_p$ is not complete with respect to $|\cdot|_p$. 
\end{proof}

\begin{defn}
The field of complex $p$-adic numbers is defined to be the completion of $\overline{\Rats}_p$ with respect to the $p$-adic absolute value $|\cdot|_p$.
\end{defn}

\begin{prop}[Properties of $\Complex_p$]
The field $\Complex_p$ enjoys the following properties:
\begin{enumerate}
\item $\Complex_p$ is algebraically closed.
\item The absolute value $|\cdot|_p$ extends uniquely to $\Complex_p$, which becomes an algebraically closed, complete metric space.
\item $\Complex_p$ is a complete ultrametric field.
\item $\overline{\Rats}_p$ is dense in $\Complex_p$.
\item $\Complex_p$ is isomorphic to $\Complex$ as fields, although they are not isomorphic as topological spaces.
\end{enumerate}
\end{prop}

\begin{thebibliography}{9}
\bibitem{wash} L. C. Washington, {\em Introduction to Cyclotomic Fields},
Springer-Verlag, New York.
\end{thebibliography}
%%%%%
%%%%%
\end{document}
