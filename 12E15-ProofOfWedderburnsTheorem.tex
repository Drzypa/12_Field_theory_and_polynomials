\documentclass[12pt]{article}
\usepackage{pmmeta}
\pmcanonicalname{ProofOfWedderburnsTheorem}
\pmcreated{2013-03-22 13:10:50}
\pmmodified{2013-03-22 13:10:50}
\pmowner{lieven}{1075}
\pmmodifier{lieven}{1075}
\pmtitle{proof of Wedderburn's theorem}
\pmrecord{8}{33627}
\pmprivacy{1}
\pmauthor{lieven}{1075}
\pmtype{Proof}
\pmcomment{trigger rebuild}
\pmclassification{msc}{12E15}

\endmetadata

% this is the default PlanetMath preamble.  as your knowledge
% of TeX increases, you will probably want to edit this, but
% it should be fine as is for beginners.

% almost certainly you want these
\usepackage{amssymb}
\usepackage{amsmath}
\usepackage{amsfonts}

% used for TeXing text within eps files
%\usepackage{psfrag}
% need this for including graphics (\includegraphics)
%\usepackage{graphicx}
% for neatly defining theorems and propositions
%\usepackage{amsthm}
% making logically defined graphics
%%%\usepackage{xypic}

% there are many more packages, add them here as you need them

% define commands here
\begin{document}
We want to show that the multiplication operation in a finite division
ring is abelian.

We denote the centralizer in $D$ of an element $x$ as $C_D(x)$. 

Lemma. The centralizer is a subring.

$0$ and $1$ are obviously elements of $C_D(x)$ and if $y$ and $z$ are,
then $x(-y) = -(xy) = -(yx) = (-y)x$, $x(y+z)=xy+xz=yx+zx=(y+z)x$ and
$x(yz)=(xy)z=(yx)z=y(xz)=y(zx)=(yz)x$, so $-y,y+z$, and $yz$ are also
elements of $C_D(x)$. Moreover, for $y\neq 0$, $xy=yx$ implies
$y^{-1}x=xy^{-1}$, so $y^{-1}$ is also an element of $C_D(x)$.

Now we consider the center of $D$ which we'll call $Z(D)$. This is
also a subring and is in fact the intersection of all centralizers.

$$ Z(D)=\bigcap_{x\in D}C_D(x) $$

$Z(D)$ is an abelian subring of $D$ and is thus a field. We can consider
$D$ and every $C_D(x)$ as vector spaces over $Z(D)$ of dimension $n$
and $n_x$ respectively. Since $D$ can be viewed as a module over
$C_D(x)$ we find that $n_x$ divides $n$. If we put $q:=|Z(D)|$, we see
that $q\geq 2$ since $\{0,1\}\subset Z(D)$, and that $|C_D(x)|=q^{n_x}$
and $|D|=q^n$.

It suffices to show that $n=1$ to prove that multiplication is
abelian, since then $|Z(D)|=|D|$ and so $Z(D)=D$.

We now consider $D^*:=D-\{0\}$ and apply the conjugacy class formula.

$$ |D^*| = |Z(D^*)| + \sum_{x} [D^*:C_{D^*}(x)] $$ 

which gives

$$ q^n-1 = q-1 + \sum_{x} \frac{q^n-1}{q^{n_x}-1}$$.

By Zsigmondy's theorem, there exists a prime $p$ that divides $q^n-1$
but doesn't divide any of the $q^m-1$ for $0<m<n$, except in 2
exceptional cases which will be dealt with separately. Such a prime
$p$ will divide $q^n-1$ and each of the $\frac{q^n-1}{q^{n_x}-1}$. So
it will also divide $q-1$ which can only happen if $n=1$.

We now deal with the 2 exceptional cases. In the first case $n$ equals
$2$, which would \PMlinkescapetext{mean} $D$ is a vector space of dimension 2 over
$Z(D)$, with elements of the form $a+b\alpha$ where $a,b\in
Z(D)$. Such elements clearly commute so $D=Z(D)$ which contradicts
our assumption that $n=2$. In the second case, $n=6$ and $q=2$. The
class equation reduces to $64-1=2-1+\sum_{x} \frac{2^6-1}{2^{n_x}-1}$
where $n_x$ divides 6. This gives $62=63x+21y+9z$ with $x,y$ and $z$
integers, which is impossible since the right hand side is divisible
by 3 and the left hand side isn't.
%%%%%
%%%%%
\end{document}
