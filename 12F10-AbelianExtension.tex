\documentclass[12pt]{article}
\usepackage{pmmeta}
\pmcanonicalname{AbelianExtension}
\pmcreated{2013-03-22 13:09:28}
\pmmodified{2013-03-22 13:09:28}
\pmowner{scanez}{1021}
\pmmodifier{scanez}{1021}
\pmtitle{abelian extension}
\pmrecord{5}{33599}
\pmprivacy{1}
\pmauthor{scanez}{1021}
\pmtype{Definition}
\pmcomment{trigger rebuild}
\pmclassification{msc}{12F10}
\pmrelated{KroneckerWeberTheorem}
\pmrelated{KummerTheory}

\endmetadata

% this is the default PlanetMath preamble.  as your knowledge
% of TeX increases, you will probably want to edit this, but
% it should be fine as is for beginners.

% almost certainly you want these
\usepackage{amssymb}
\usepackage{amsmath}
\usepackage{amsfonts}

% used for TeXing text within eps files
%\usepackage{psfrag}
% need this for including graphics (\includegraphics)
%\usepackage{graphicx}
% for neatly defining theorems and propositions
%\usepackage{amsthm}
% making logically defined graphics
%%%\usepackage{xypic}

% there are many more packages, add them here as you need them

% define commands here
\begin{document}
Let $K$ be a Galois extension of $F$. The extension is said to be an 
{\em abelian extension} if the Galois group $\textrm{Gal$(K/F)$}$ is abelian.

Examples: $\mathbb{Q}(\sqrt{2})/\mathbb{Q}$ has Galois group $\mathbb{Z}/2\mathbb{Z}$ so
$\mathbb{Q}(\sqrt{2})/\mathbb{Q}$ is an abelian extension.

Let $\zeta_n$ be a \PMlinkname{primitive nth root of unity}{RootOfUnity}. Then $\mathbb{Q}(\zeta_n)/\mathbb{Q}$ has
Galois group $(\mathbb{Z}/n\mathbb{Z})^*$ (the group of units of
$\mathbb{Z}/n\mathbb{Z}$) so $\mathbb{Q}(\zeta_n)/\mathbb{Q}$ is abelian.
%%%%%
%%%%%
\end{document}
