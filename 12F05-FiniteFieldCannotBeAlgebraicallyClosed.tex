\documentclass[12pt]{article}
\usepackage{pmmeta}
\pmcanonicalname{FiniteFieldCannotBeAlgebraicallyClosed}
\pmcreated{2013-03-22 16:29:09}
\pmmodified{2013-03-22 16:29:09}
\pmowner{rspuzio}{6075}
\pmmodifier{rspuzio}{6075}
\pmtitle{finite field cannot be algebraically closed}
\pmrecord{6}{38654}
\pmprivacy{1}
\pmauthor{rspuzio}{6075}
\pmtype{Theorem}
\pmcomment{trigger rebuild}
\pmclassification{msc}{12F05}
\pmrelated{AlgebraicClosureOfAFiniteField}

\endmetadata

% this is the default PlanetMath preamble.  as your knowledge
% of TeX increases, you will probably want to edit this, but
% it should be fine as is for beginners.

% almost certainly you want these
\usepackage{amssymb}
\usepackage{amsmath}
\usepackage{amsfonts}

% used for TeXing text within eps files
%\usepackage{psfrag}
% need this for including graphics (\includegraphics)
%\usepackage{graphicx}
% for neatly defining theorems and propositions
\usepackage{amsthm}
% making logically defined graphics
%%%\usepackage{xypic}

\newtheorem*{theorem}{Theorem}
\begin{document}
\begin{theorem}
A finite field cannot be algebraically closed.
\end{theorem}

\begin{proof}
The proof proceeds by the method of contradiction.  Assume that a field
$F$ is both finite and algebraically closed.  Consider the polynomial 
$p(x) = x^2 - x$ as a function from $F$ to $F$.  There are two elements
which any field (in particular, $F$) must have --- the additive identity
$0$ and the multiplicative identity $1$.  The polynomial $p$ maps both of
these elements to $0$.  Since $F$ is finite and the function $p \colon F
\to F$ is not one-to-one, the function cannot map onto $F$ either, so
there must exist an element $a$ of $F$ such that $x^2 - x \not= a$ for
all $x \in F$.  In other words, the polynomial $x^2 - x - a$ has no root
in $F$, so $F$ could not be algebraically closed.
\end{proof}
%%%%%
%%%%%
\end{document}
