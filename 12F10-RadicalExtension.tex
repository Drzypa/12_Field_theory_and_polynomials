\documentclass[12pt]{article}
\usepackage{pmmeta}
\pmcanonicalname{RadicalExtension}
\pmcreated{2013-03-22 12:08:35}
\pmmodified{2013-03-22 12:08:35}
\pmowner{djao}{24}
\pmmodifier{djao}{24}
\pmtitle{radical extension}
\pmrecord{7}{31329}
\pmprivacy{1}
\pmauthor{djao}{24}
\pmtype{Definition}
\pmcomment{trigger rebuild}
\pmclassification{msc}{12F10}
\pmsynonym{radical tower}{RadicalExtension}

\endmetadata

\usepackage{amssymb}
\usepackage{amsmath}
\usepackage{amsfonts}
\usepackage{graphicx}
%%%\usepackage{xypic}
\begin{document}
A {\em radical tower} is a field extension $L/F$ which has a filtration
$$
F = L_0 \subset L_1 \subset \cdots \subset L_n = L
$$
where for each $i$, $0 \leq i < n$, there exists an element $\alpha_i \in L_{i+1}$ and a natural number $n_i$ such that $L_{i+1} = L_i(\alpha_i)$ and $\alpha_i^{n_i} \in L_i$.

A {\em radical extension} is a field extension $K/F$ for which there exists a radical tower $L/F$ with $L \supset K$. The notion of radical extension coincides with the informal concept of solving for the roots of a polynomial by radicals, in the sense that a polynomial over $K$ is solvable by radicals if and only if its splitting field is a radical extension of $F$.
%%%%%
%%%%%
%%%%%
\end{document}
