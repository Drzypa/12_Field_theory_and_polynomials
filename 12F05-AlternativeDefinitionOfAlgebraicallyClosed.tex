\documentclass[12pt]{article}
\usepackage{pmmeta}
\pmcanonicalname{AlternativeDefinitionOfAlgebraicallyClosed}
\pmcreated{2013-03-22 16:53:23}
\pmmodified{2013-03-22 16:53:23}
\pmowner{polarbear}{3475}
\pmmodifier{polarbear}{3475}
\pmtitle{alternative definition of algebraically closed}
\pmrecord{8}{39145}
\pmprivacy{1}
\pmauthor{polarbear}{3475}
\pmtype{Derivation}
\pmcomment{trigger rebuild}
\pmclassification{msc}{12F05}

% this is the default PlanetMath preamble.  as your knowledge
% of TeX increases, you will probably want to edit this, but
% it should be fine as is for beginners.

% almost certainly you want these
\usepackage{amssymb}
\usepackage{amsmath}
\usepackage{amsfonts}

% used for TeXing text within eps files
%\usepackage{psfrag}
% need this for including graphics (\includegraphics)
%\usepackage{graphicx}
% for neatly defining theorems and propositions
\usepackage{amsthm}
% making logically defined graphics
%%%\usepackage{xypic}

% there are many more packages, add them here as you need them

% define commands here
\newtheorem{proposition}{Proposition} 

\begin{document}
\begin{proposition} If $K$ is a field, the following are equivalent:\\
\begin{enumerate}
\item[(1)] $K$ is algebraically closed, i.e. every nonconstant polynomial $f$ in $K[x]$ has a root in $K$.
\item[(2)] Every nonconstant polynomial $f$ in $K[x]$ splits completely over $K$.
\item[(3)] If $L|K$ is an algebraic extension then $L = K$.\end{enumerate}\end{proposition}
\begin{proof}
 If (1) is true then we can prove by induction on degree of $f$ that every nonconstant polynomial $f$ splits completely over $K$. Conversely, (2)$\Rightarrow$ (1) is trivial.\newline
(2)$\Rightarrow$ (3) If $L|K$ is algebraic and $\alpha\in L$, then $\alpha$ is a root of a polynomial $f\in K[x]$. By (2) $f$ splits over $K$, which implies that $\alpha\in K$. It follows that $L=K$.\newline
(3)$\Rightarrow$ (1) Let $f\in K[x]$ and $\alpha$ a root of $f$ (in some extension of $K$). Then $K(\alpha)$ is an algebraic extension of $K$, hence  $\alpha\in K$.
\end{proof} 
\textbf{Examples} 1) The field of real numbers $\mathbb{R}$ is not algebraically closed. Consider the equation $x^2+1=0$. The square of a real number is always positive and cannot be $-1$ so the equation has no roots.\newline
2) The $p$-adic field $\mathbb{Q}_p$ is not algebraically closed because the equation $x^2-p=0$ has no roots. Otherwise $x^2=p$ implies $2v_{p}x = 1$, which is false.

%%%%%
%%%%%
\end{document}
