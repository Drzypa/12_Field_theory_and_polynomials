\documentclass[12pt]{article}
\usepackage{pmmeta}
\pmcanonicalname{ExamplesOfFields}
\pmcreated{2013-03-22 12:50:13}
\pmmodified{2013-03-22 12:50:13}
\pmowner{AxelBoldt}{56}
\pmmodifier{AxelBoldt}{56}
\pmtitle{examples of fields}
\pmrecord{16}{33162}
\pmprivacy{1}
\pmauthor{AxelBoldt}{56}
\pmtype{Example}
\pmcomment{trigger rebuild}
\pmclassification{msc}{12E99}
\pmrelated{NumberField}

% this is the default PlanetMath preamble.  as your knowledge
% of TeX increases, you will probably want to edit this, but
% it should be fine as is for beginners.

% almost certainly you want these
\usepackage{amssymb}
\usepackage{amsmath}
\usepackage{amsfonts}

% used for TeXing text within eps files
%\usepackage{psfrag}
% need this for including graphics (\includegraphics)
%\usepackage{graphicx}
% for neatly defining theorems and propositions
%\usepackage{amsthm}
% making logically defined graphics
%%%\usepackage{xypic} 

% there are many more packages, add them here as you need them

% define commands here
\begin{document}
\PMlinkescapeword{arithmetic}
\PMlinkescapeword{class}
\PMlinkescapeword{contains}
\PMlinkescapeword{domain}
\PMlinkescapeword{field}
\PMlinkescapeword{fields}
\PMlinkescapeword{fractions}
\PMlinkescapeword{prime}
\PMlinkescapeword{sort}

\PMlinkname{Fields}{Field} are typically sets of ``numbers'' in which the arithmetic
  operations of addition, subtraction, multiplication and division are
  defined. The following is a list of examples of fields.

\begin{itemize}

\item The set of all rational numbers $\Bbb{Q}$, all real numbers $\Bbb{R}$ and all
complex numbers $\Bbb{C}$ are the most familiar examples of fields.

\item Slightly more exotic, the hyperreal numbers and the surreal
numbers are fields containing infinitesimal and infinitely large
numbers. (The surreal numbers aren't a field in the strict sense since
they form a proper class and not a set.)

\item The {\em algebraic numbers} form a field; this is the algebraic
closure of $\Bbb{Q}$. In general, every field has an (essentially
unique) algebraic closure.

\item The computable complex numbers (those whose digit sequence can be produced by a Turing machine) form a field. The definable complex numbers (those which can be
precisely specified using a logical formula) form a field containing the computable numbers; arguably, this
field contains all the numbers we can ever talk about.  It is countable.

\item The so-called {\em algebraic number fields} (sometimes just called number fields) arise from $\Bbb{Q}$ by adjoining some (finite number of) algebraic numbers. For instance $\Bbb{Q}(\sqrt{2}) = \{u +
v\sqrt{2} \mid u,v\in\Bbb{Q}\}$ and $\Bbb{Q}(\sqrt[3]{2},i) = \{u + vi
+ w\sqrt[3]{2} + xi\sqrt[3]{2} + y \sqrt[3]{4} + zi \sqrt[3]{4} \mid
u,v,w,x,y,z\in\Bbb{Q}\} = \mathbb{Q}(i\sqrt[3]{2})$ (every separable finite field extension is simple).

\item If $p$ is a prime number, then the $p${\em -adic numbers} form a
field $\Bbb{Q}_p$ which is the completion of the field $\mathbb{Q}$ with respect to the $p$-adic valuation. 

\item If $p$ is a prime number, then the integers modulo $p$ form a
finite field with $p$ elements, typically denoted by $\Bbb{F}_p$. More
generally, for every \PMlinkname{prime}{Prime} power $p^n$ there is one and only one
finite field $\Bbb{F}_{p^n}$ with $p^n$ elements.

\item If $K$ is a field, we can form the field of rational functions
over $K$, denoted by $K(X)$. It consists of quotients of polynomials
in $X$ with coefficients in $K$.

\item  If $V$ is a \PMlinkname{variety}{AffineVariety} over the field $K$, then the function field of $V$, denoted by
$K(V)$, consists of all quotients of polynomial functions defined on $V$.

\item If $U$ is a domain (= connected open set) in $\Bbb{C}$, then the
set of all meromorphic functions on $U$ is a field. More generally, the  meromorphic functions on any Riemann surface form a field.

\item If $X$ is a variety (or scheme) then the rational functions on $X$ form a field.  At each point of $X$, there is also a residue field which contains information about that point.

\item The field of formal Laurent series over the field $K$ in the
variable $X$ consists
of all expressions of the form
$$\sum_{j=-M}^\infty a_j X^j$$
where $M$ is some integer and the coefficients $a_j$ come from $K$.

\item More generally, whenever $R$ is an integral domain, we can form
its field of fractions, a field whose elements are the
fractions of elements of $R$.

\end{itemize}

Many of the fields described above have some sort of additional structure, for example a topology (yielding a topological field), a total order, or a canonical absolute value.
%%%%%
%%%%%
\end{document}
