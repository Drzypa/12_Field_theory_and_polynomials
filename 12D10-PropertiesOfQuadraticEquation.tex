\documentclass[12pt]{article}
\usepackage{pmmeta}
\pmcanonicalname{PropertiesOfQuadraticEquation}
\pmcreated{2015-02-12 9:55:41}
\pmmodified{2015-02-12 9:55:41}
\pmowner{pahio}{2872}
\pmmodifier{pahio}{2872}
\pmtitle{properties of quadratic equation}
\pmrecord{14}{36115}
\pmprivacy{1}
\pmauthor{pahio}{2872}
\pmtype{Result}
\pmcomment{trigger rebuild}
\pmclassification{msc}{12D10}
%\pmkeywords{quadratic equation}
\pmrelated{VietasFormula}
\pmrelated{ValuesOfComplexCosine}
\pmrelated{IntegralBasisOfQuadraticField}

\endmetadata

% this is the default PlanetMath preamble.  as your knowledge
% of TeX increases, you will probably want to edit this, but
% it should be fine as is for beginners.

% almost certainly you want these
\usepackage{amssymb}
\usepackage{amsmath}
\usepackage{amsfonts}

% used for TeXing text within eps files
%\usepackage{psfrag}
% need this for including graphics (\includegraphics)
%\usepackage{graphicx}
% for neatly defining theorems and propositions
 \usepackage{amsthm}
% making logically defined graphics
%%%\usepackage{xypic}

% there are many more packages, add them here as you need them

% define commands here

\theoremstyle{definition}
\newtheorem*{thmplain}{Theorem}
\begin{document}
The quadratic equation
$$ax^2\!+\!bx\!+\!c = 0$$
or
$$x^2\!+\!px+\!q\! = 0$$
with rational, real, \PMlinkname{algebraic}{AlgebraicNumber}  or complex coefficients ($a \neq 0$) has the following properties:
\begin{itemize}
 \item It has in $\mathbb{C}$ two roots (which may be equal), since the complex numbers form an algebraically closed field containing the coefficients.
 \item The sum of the roots is equal to \,$-\frac{b}{a}$,\, i.e.\, $-p$.
 \item The product of the roots is equal to \,$\frac{c}{a}$,\, i.e.\, $q$.
\end{itemize}
\textbf{Corollary.}\, If the leading coefficient and the constant \PMlinkescapetext{term} are equal, then the roots are inverse numbers of each other.

Without solving the equation, the value of any symmetric polynomial 
of the roots can be calculated.\\

\textbf{Example.}\, If one has to \PMlinkescapetext{calculate} 
$x_1^3\!+\!x_2^3$, when $x_1$ and $x_2$ are the roots of the 
equation\, $x^2\!-\!4x\!+\!9 = 0$,\, we have\, 
$x_1\!+\!x_2 = 4$\, and\, $x_1x_2 = 9$.\, Because
$$(x_1\!+\!x_2)^3 = 
x_1^3\!+\!3x_1^2x_2\!+\!3x_1x_2^2\!+\!x_2^3 = 
(x_1^3\!+\!x_2^3)\!+\!3x_1x_2(x_1\!+\!x_2),$$
we obtain
$$x_1^3\!+\!x_2^3 = (x_1\!+\!x_2)^3\!-\!3x_1x_2(x_1\!+\!x_2) = 4^3\!-\!3\cdot 9\cdot 4 = -44.$$

\textbf{Note.}\, If one wants to write easily a quadratic equation 
with rational roots, one could take such one that the sum of the 
coefficients is zero (then one root is always 1).\, For instance, 
the roots of the equation\, $5x^2\!+\!11x\!-\!16 = 0$\, are 1 and 
$-\frac{16}{5}$.
%%%%%
%%%%%
\end{document}
