\documentclass[12pt]{article}
\usepackage{pmmeta}
\pmcanonicalname{LongDivision}
\pmcreated{2013-03-22 15:09:07}
\pmmodified{2013-03-22 15:09:07}
\pmowner{alozano}{2414}
\pmmodifier{alozano}{2414}
\pmtitle{long division}
\pmrecord{7}{36899}
\pmprivacy{1}
\pmauthor{alozano}{2414}
\pmtype{Theorem}
\pmcomment{trigger rebuild}
\pmclassification{msc}{12E99}
\pmclassification{msc}{00A05}
\pmclassification{msc}{11A05}
\pmsynonym{division algorithm}{LongDivision}
\pmrelated{Polynomial}
\pmrelated{PolynomialLongDivision}
\pmrelated{MixedFraction}
\pmdefines{dividend}
\pmdefines{remainder}

\endmetadata

% this is the default PlanetMath preamble.  as your knowledge
% of TeX increases, you will probably want to edit this, but
% it should be fine as is for beginners.

% almost certainly you want these
\usepackage{amssymb}
\usepackage{amsmath}
\usepackage{amsthm}
\usepackage{amsfonts}

% used for TeXing text within eps files
%\usepackage{psfrag}
% need this for including graphics (\includegraphics)
%\usepackage{graphicx}
% for neatly defining theorems and propositions
%\usepackage{amsthm}
% making logically defined graphics
%%%\usepackage{xypic}

% there are many more packages, add them here as you need them

% define commands here

\newtheorem{thm}{Theorem}
\newtheorem{defn}{Definition}
\newtheorem{prop}{Proposition}
\newtheorem{lemma}{Lemma}
\newtheorem{cor}{Corollary}

\theoremstyle{definition}
\newtheorem{exa}{Example}

% Some sets
\newcommand{\Nats}{\mathbb{N}}
\newcommand{\Ints}{\mathbb{Z}}
\newcommand{\Reals}{\mathbb{R}}
\newcommand{\Complex}{\mathbb{C}}
\newcommand{\Rats}{\mathbb{Q}}
\newcommand{\Gal}{\operatorname{Gal}}
\newcommand{\Cl}{\operatorname{Cl}}
\begin{document}
In this entry we treat two cases of long division.

\section{Integers}
\begin{thm}[Integer Long Division]
For every pair of integers $a, b\neq 0$ there exist unique integers $q$ and $r$ such that:
\begin{enumerate}
\item $a=b\cdot q + r,$
\item $0\leq r < |b|$. 
\end{enumerate}
\end{thm}

\begin{exa}
Let $a=10$ and $b=-3$. Then $q=-3$ and $r=1$ correspond to the long division:
$$10=(-3)\cdot(-3)+1.$$
\end{exa}

\begin{defn}
The number $r$ as in the theorem is called the remainder of the division of $a$ by $b$. The numbers $a,\ b$ and $q$ are called the dividend, divisor and quotient respectively. 
\end{defn}

\section{Polynomials}

\begin{thm}[Polynomial Long Division]
Let $R$ be a commutative ring with non-zero unity and let $a(x)$ and $b(x)$ be two polynomials in $R[x]$, where the leading coefficient of $b(x)$ is a unit of $R$. Then there exist unique polynomials $q(x)$ and $r(x)$ in $R[x]$ such that:
\begin{enumerate}
\item $a(x)=b(x)\cdot q(x) + r(x),$
\item $0\leq \deg(r(x)) < \deg b(x)$ or $r(x)=0$. 
\end{enumerate}
\end{thm}

\begin{exa}
Let $R=\Ints$ and let $a(x)=x^3+3$, $b(x)=x^2+1$. Then $q(x)=x$ and $r(x)=-x+3$, so that:
$$x^3+3=x(x^2+1)-x+3.$$
\end{exa}

\begin{exa}
The theorem is not true in general if the leading coefficient of $b(x)$ is not a unit. For example, if $a(x)=x^3+3$ and $b(x)=3x^2+1$ then there are no $q(x)$ and $r(x)$ {\bf with coefficients in} $\Ints$ with the required properties.
\end{exa}
%%%%%
%%%%%
\end{document}
