\documentclass[12pt]{article}
\usepackage{pmmeta}
\pmcanonicalname{CompleteUltrametricField}
\pmcreated{2013-03-22 14:55:37}
\pmmodified{2013-03-22 14:55:37}
\pmowner{pahio}{2872}
\pmmodifier{pahio}{2872}
\pmtitle{complete ultrametric field}
\pmrecord{15}{36615}
\pmprivacy{1}
\pmauthor{pahio}{2872}
\pmtype{Theorem}
\pmcomment{trigger rebuild}
\pmclassification{msc}{12J10}
\pmclassification{msc}{54E35}
%\pmkeywords{convergence}
\pmrelated{Series}
\pmrelated{NecessaryConditionOfConvergence}
\pmrelated{ExtensionOfValuationFromCompleteBaseField}
\pmrelated{PropertiesOfNonArchimedeanValuations}
\pmdefines{ultrametric field}
\pmdefines{non-archimedean field}

% this is the default PlanetMath preamble.  as your knowledge
% of TeX increases, you will probably want to edit this, but
% it should be fine as is for beginners.

% almost certainly you want these
\usepackage{amssymb}
\usepackage{amsmath}
\usepackage{amsfonts}

% used for TeXing text within eps files
%\usepackage{psfrag}
% need this for including graphics (\includegraphics)
%\usepackage{graphicx}
% for neatly defining theorems and propositions
 \usepackage{amsthm}
% making logically defined graphics
%%%\usepackage{xypic}

% there are many more packages, add them here as you need them

% define commands here
\theoremstyle{definition}
\newtheorem*{thmplain}{Theorem}

\begin{document}
A field $K$ equipped with a non-archimedean valuation\, $|\cdot|$\, is called a {\em non-archimedean field} or also an {\em ultrametric field}, since the valuation \PMlinkescapetext{induces} the ultrametric\, $d(x,\,y) := |x\!-\!y|$\, of $K$.\\

\begin{thmplain}
\,Let $(K,\,d)$ be a \PMlinkname{complete}{Complete} ultrametric field.\, A necessary and sufficient condition for the convergence of the {\em series} 
\begin{align}
                 a_1\!+\!a_2\!+\!a_3\!+\ldots
\end{align}
in $K$ is that
\begin{align}
                \lim_{n\to\infty}a_n \;=\; 0.
\end{align}
\end{thmplain}

{\em Proof.}\, Let $\varepsilon$ be any positive number.\, When (1) converges, it satisfies the Cauchy condition and therefore exists a number $m_\varepsilon$ such that surely
 $$|a_{m+1}| \;=\; \left|\sum_{j=1}^{m+1}a_j-\sum_{j=1}^{m}a_j\right| \;<\; \varepsilon$$
for all\, $m \geqq m_\varepsilon$;\, thus (2) is necessary.\, On the contrary, suppose the validity of (2).\, Now one may determine such a great number $n_\varepsilon$ that 
      $$|a_m| \;<\; \varepsilon \qquad \forall m \,\geqq\, n_\varepsilon.$$
No matter how great is the natural number $n$, the ultrametric then guarantees the inequality
  $$|a_m\!+\!a_{m+1}\!+\ldots+\!a_{m+n}| \;\leqq\; 
\max\{|a_m|,\,|a_{m+1}|,\,\ldots,\,|a_{m+n}|\} \;<\; \varepsilon$$
always when\, $m \geqq n_\varepsilon$.\, Thus the partial sums of (1) form a Cauchy sequence, which converges in the complete field.\, Hence the series (1) converges, and (2) is sufficient.

%%%%%
%%%%%
\end{document}
