\documentclass[12pt]{article}
\usepackage{pmmeta}
\pmcanonicalname{SimpleTranscendentalFieldExtension}
\pmcreated{2013-03-22 15:02:20}
\pmmodified{2013-03-22 15:02:20}
\pmowner{pahio}{2872}
\pmmodifier{pahio}{2872}
\pmtitle{simple transcendental field extension}
\pmrecord{11}{36751}
\pmprivacy{1}
\pmauthor{pahio}{2872}
\pmtype{Corollary}
\pmcomment{trigger rebuild}
\pmclassification{msc}{12F99}
\pmsynonym{simple infinite field extension}{SimpleTranscendentalFieldExtension}
\pmrelated{FunctionField}

% this is the default PlanetMath preamble.  as your knowledge
% of TeX increases, you will probably want to edit this, but
% it should be fine as is for beginners.

% almost certainly you want these
\usepackage{amssymb}
\usepackage{amsmath}
\usepackage{amsfonts}

% used for TeXing text within eps files
%\usepackage{psfrag}
% need this for including graphics (\includegraphics)
%\usepackage{graphicx}
% for neatly defining theorems and propositions
%\usepackage{amsthm}
% making logically defined graphics
%%%\usepackage{xypic}

% there are many more packages, add them here as you need them

% define commands here
\begin{document}
The extension field $K(\alpha)$ of a base field $K$, where $\alpha$ is a transcendental element with respect to $K$, is a \emph{\PMlinkname{simple}{SimpleFieldExtension} transcendental extension of} $K$.\, All such extension fields are isomorphic to the field $K(X)$ of rational functions in one indeterminate $X$ over $K$, and thus to each other.

\textbf{Example.}\, The subfields $\mathbb{Q}(\pi)$ and $\mathbb{Q}(e)$ of $\mathbb{R}$, where $\pi$ is \PMlinkname{Ludolph's constant}{Pi} and $e$ Napier's constant, are isomorphic.
%%%%%
%%%%%
\end{document}
