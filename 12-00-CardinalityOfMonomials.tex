\documentclass[12pt]{article}
\usepackage{pmmeta}
\pmcanonicalname{CardinalityOfMonomials}
\pmcreated{2013-03-22 16:34:42}
\pmmodified{2013-03-22 16:34:42}
\pmowner{rspuzio}{6075}
\pmmodifier{rspuzio}{6075}
\pmtitle{cardinality of monomials}
\pmrecord{10}{38769}
\pmprivacy{1}
\pmauthor{rspuzio}{6075}
\pmtype{Theorem}
\pmcomment{trigger rebuild}
\pmclassification{msc}{12-00}

\endmetadata

% this is the default PlanetMath preamble.  as your knowledge
% of TeX increases, you will probably want to edit this, but
% it should be fine as is for beginners.

% almost certainly you want these
\usepackage{amssymb}
\usepackage{amsmath}
\usepackage{amsfonts}

% used for TeXing text within eps files
%\usepackage{psfrag}
% need this for including graphics (\includegraphics)
%\usepackage{graphicx}
% for neatly defining theorems and propositions
\usepackage{amsthm}
% making logically defined graphics
%%%\usepackage{xypic}

% there are many more packages, add them here as you need them

% define commands here


\newtheorem{theorem}{Theorem}
\begin{document}
\begin{theorem}
If $S$ is a finite set of variable symbols, then the number of monomials of 
degree $n$ constructed from these symbols is ${n + m - 1 \choose n}$, where 
$m$ is the cardinality of $S$.
\end{theorem}

\begin{proof}
The proof proceeds by inducion on the cardinality of $S$.  If $S$ has but one
element, then there is but one monomial of degree $n$, namely the sole element
of $S$ raised to the $n$-th power.  Since ${n + 1 - 1 \choose n} = 1$, the 
conclusion holds when $m = 1$.

Suppose, then, that the result holds whenver $m < M$ for some $M$.  Let $S$ be
a set with exactly $M$ elements and let $x$ be an element of $S$.  A monomial
of degree $n$ constructed from elements of $S$ can be expressed as the product
of a power of $x$ and a monomial constructed from the elements of $S \setminus
\{x\}$.  By the induction hypothesis, the number of monomials of degree $k$ 
constructed from elements of $S \setminus \{x\}$ is ${k + M - 2 \choose k}$.
Summing over the possible powers to which $x$ may be raised, the number of
monomials of degree $n$ constructed from the elements of $S$ is as follows:
 \[ \sum_{k=0}^n {k + M - 2 \choose k} = {k + M - 1 \choose k} \]
\end{proof}

\begin{theorem}
If $S$ is an infinite set of variable symbols, then the number of monomials 
of degree $n$ constructed from these symbols equals the cardinality of $S$.
\end{theorem}

%%%%%
%%%%%
\end{document}
