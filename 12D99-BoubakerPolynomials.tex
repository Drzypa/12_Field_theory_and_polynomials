\documentclass[12pt]{article}
\usepackage{pmmeta}
\pmcanonicalname{BoubakerPolynomials}
\pmcreated{2013-03-22 19:16:08}
\pmmodified{2013-03-22 19:16:08}
\pmowner{Zhangaini}{20741}
\pmmodifier{Zhangaini}{20741}
\pmtitle{Boubaker Polynomials}
\pmrecord{13}{42200}
\pmprivacy{1}
\pmauthor{Zhangaini}{20741}
\pmtype{Definition}
\pmcomment{trigger rebuild}
\pmclassification{msc}{12D99}
%\pmkeywords{polynomials}

\endmetadata

\usepackage{amssymb}
\usepackage{amsmath}
\usepackage{amsfonts}
\usepackage{pstricks}
\usepackage{psfrag}
\usepackage{graphicx}
\usepackage{amsthm}
%%\usepackage{xypic}


\begin{document}
\PMlinkescapephrase{occur in}
\PMlinkescapeword{right}

The Boubaker polynomials($B_n (X)$) are a sequence with integer coefficients. They were established in an applied physics study. The monomial expression of the Boubaker polynomials is :

$$B_n(X)=\sum_{p = 0}^{\xi(n)} (-1)^p \frac{(n-4p)}{(n-p)}\binom{n-p}{p}(X)^{n-2p}$$

where :
$$\xi(n)=\left \lfloor \frac{n}{2} \right \rfloor =\frac{2n+((-1)^n - 1)}{4}$$
(The symbol :$\lfloor . \rfloor$ designates the \htmladdnormallink{floor function}{http://planetmath.org/encyclopedia/GreatestIntegerLessThanOrEqualTo.html}.)

The Boubaker Polynomials Expansion Scheme BPES has been used by several applied physics and engineering studies. Agida et al. used this protocol for establishing an analytical method for solving Love integral equation in the case of a rational kernel.

O. B. Awojoyogbe et al. took profit from the similarities between the hemodynamic flow system inside some organic tissues and the BPES definition system, in order to express the tissue response to magnetic fields excitation.
Kumar combined the Boubaker Polynomials Expansion Scheme (BPES) analyses and array analyses for determining the normalized field created conjointly by two similar circular coaxial conducting disks separated by a pre-fixed distance. On an other hand, J. Ghanouchi et al. used the BPES to discuss the intriguing paradox of establishment of non-Gaussian isothermal generative lines beneath a plate surface targeted by a Gaussian beam.

The works carried out by S. Slama et al. proposed solutions to the heat transfer problem inside different welding and annealing systems. These works used the BPES as a guide to solve heat discrete conservation equations during cooling phases, and yielded consistent cooling velocities profile. T. Ghrib et al. used the BPES in order to establish a first order correlation between the Vickers microhardness and the thermal diffusivity of treated steel alloys.

In the last years, S. Lazzez et al. and D. H. Zhang et al. investigated semiconductor micro layers physical properties using the BPES. More recently, A. Yildirim et al. proposed analytical solutions to the KleinâGordon equation in a pulsed stationary regime using modified variational iteration method (MVIM) and BPES. Gengâs standard second-order boundary value problem (BVP) was also investigated by D.H. Zang et al. using the BPES.

In a different filed like animal biology and medical sciences, Dubey et al. proposed, in a study commented and corrected by Milgram, an analytical method for the identification of predatorâprey populations time-dependent evolution in a Lotka-Volterra predatorâprey model which took into account the concept of accelerated-predator-satiety, O. B. Awojoyogbe et al. proposed also a mathematical formulation for the NMR diffusion equation derived from the Bloch NMR flow in lower heart coronary artery. This formulation was totally based on the properties of the Boubaker polynomials expansion scheme BPES.


References


Belhadj, A., Onyango O., Rozibaeva, N., Boubaker Polynomials Expansion Scheme-Related Heat Transfer Investigation Inside Keyhole Model , J. Thermophys. Heat Transf. 23, 639-640 (2009a)

Belhadj, A., Bessrour, J., Bouhafs, M., Barrallier, L., Experimental and theoretical cooling velocity profile inside laser welded metals using keyhole approximation and Boubaker polynomials expansion, J. of Thermal Analysis and

Dubey B., Zhao T.G., Jonsson M., Rahmanov H., A solution to the accelerated-predator-satiety Lotka-Volterra predator-prey problem using Boubaker polynomial expansion scheme, J. of Theoretical Biol. 264, (1),1542010-14 (2010)
Fridjine, S., Amlouk, M., A new parameter: An ABACUS for optimizig functional materials using the Boubaker polynomials expansion scheme, Modern Phys. Lett. B 23, 2179-2182 (2009)

Oyodum, O.D., Awojoyogbe, O.B., Dada, M., Magnuson, J., On the earliest definition of the Boubaker polynomials, Eur. Phys. J.-App. Phys. 46, 21201-21203 (2009)

Tabatabaei, S., Zhao, T., Awojoyogbe, O., Moses, F., Cut-off cooling velocity profiling inside a keyhole model using the Boubaker polynomials expansion scheme, Heat Mass Transf. 45, 1247-1251 (2009)

Zhang D. H., On the earliest definition of the Boubaker polynomials", The Eur. Phys. J. Appl. Physics 50, (1), 11201-03 (2010)

Zhao T. G., Naing L., Yue W. X., Some new features of the Boubaker polynomials expansion scheme BPES, Mat. Zametki 87(2), 175-178 (2010)

Zhang D. H., Li F. W. Boubaker Polynomials Expansion Scheme(BPES) optimisation of copper tin sulfide, Materials Letters 64, (6), 778-780 (2010)

Zhang D. H., Li F. W. A Boubaker Polynomials Expansion Scheme BPES-related analytical solution to Williams-Brinkmann stagnation point flow equation at a blunt body, Ir. Journal of App. Phys. Lett. IJAPLett. 2,25-28 (2009)

Luzon A. and M. A.Moron, 2010. Recurrence relations for polynomial sequences via Riordan matrices, Linear Algebra and its Applications 433(7):1422-1446

Barry, P. and A. Hennessy, 2010. Meixner-Type results for Riordan arrays and associated integer sequences, section 6: The Boubaker polynomials, Journal of Integer Sequences, 13:1-34

Agida M. and A. S. Kumar, 2010. A Boubaker Polynomials Expansion Scheme solution to random Love equation in the case of a rational kernel, El. Journal of Theoretical Physics 7 (24):319-326

Kumar, A. S., 2010, An analytical solution to applied mathematics-related Love's equation using the Boubaker Polynomials Expansion Scheme, Journal of the Franklin Institute. 347:1755-1761

A. Yildirim, S. T. Mohyud-Din, D. H. Zhang, Analytical solutions to the pulsed Klein-Gordon equation using Modified Variational Iteration Method (MVIM) and Boubaker Polynomials Expansion Scheme (BPES), Computers and Mathematics with Applications (2010); DOI:10.1016/j.camwa.2009.12.026 .

Milgram, A., 2011. The stability of the Boubaker polynomials expansion scheme (BPES)-based solution to Lotka-Volterra problem, J. of Theoretical Biology, 271( 1, 21): 157-158


More to come\dots
%%%%%
%%%%%
\end{document}
