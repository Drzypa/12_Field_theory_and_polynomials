\documentclass[12pt]{article}
\usepackage{pmmeta}
\pmcanonicalname{GausssLemmaII}
\pmcreated{2013-03-22 13:07:52}
\pmmodified{2013-03-22 13:07:52}
\pmowner{bshanks}{153}
\pmmodifier{bshanks}{153}
\pmtitle{Gauss's lemma II}
\pmrecord{18}{33567}
\pmprivacy{1}
\pmauthor{bshanks}{153}
\pmtype{Theorem}
\pmcomment{trigger rebuild}
\pmclassification{msc}{12E05}
\pmsynonym{Gauss' lemma II}{GausssLemmaII}
\pmrelated{GausssLemmaI}
\pmrelated{EisensteinCriterion}
\pmrelated{ProofOfEisensteinCriterion}
\pmrelated{PrimeFactorsOfXn1}
\pmrelated{AlternativeProofThatSqrt2IsIrrational}
\pmdefines{primitive polynomial}

\endmetadata

\usepackage{amssymb}
\usepackage{amsmath}
\usepackage{amsfonts}
\begin{document}
\PMlinkescapeword{completes}
\PMlinkescapeword{name}
\PMlinkescapeword{primitive}
\PMlinkescapeword{proposition}

\textbf{Definition.}\, A polynomial $P=a_nx^n+\cdots+a_0$ over an integral domain
$D$ is said to be \emph{primitive} if its coefficients are not all divisible
by any element of $D$ other than a unit.

\textbf{Proposition (Gauss).}\, Let $D$ be a unique factorization domain and $F$ its field of fractions. \,
If a polynomial $P\in D[x]$ is reducible in $F[x]$, then it is reducible in $D[x]$.

\textbf{Remark.}\,The above statement is often used in its contrapositive form.\, For an example of this usage, see \PMlinkname{this entry}{AlternativeProofThatSqrt2IsIrrational}.

\emph{Proof.}\, A primitive polynomial in $D[x]$ is by definition divisible by a non invertible constant polynomial, and therefore reducible in $D[x]$ (unless it is itself constant). There is therefore nothing to prove unless $P$ (which is not constant) is primitive.\, By assumption there exist non-constant $S,\,T \in F[x]$ such that\, $P=ST$.\, There are elements $a,\,b\in F$ such that $aS$ and $bT$ are in $D[x]$ and are primitive (first multiply by a nonzero element of $D$ to chase any denominators, then divide by the gcd of the resulting coefficients in $D$).\, Then $aSbT = abP$ is primitive by Gauss's lemma I, but $P$ is primitive as well, so $ab$ is a unit of $D$ and\, $P=(ab)^{-1}(aS)(bT)$\, is a nontrivial decomposition of $P$ in $D[X]$.\, This completes the proof.\\

\textbf{Remark.}\, Another result with the same name is Gauss' lemma on quadratic residues.\\

From the above proposition and its proof one may infer the

\textbf{Theorem.}\, If a primitive polynomial of $D[x]$ is divisible in $F[x]$, then it splits in $D[x]$ into primitive prime factors.\, These are uniquely determined up to unit factors of $D$.
%%%%%
%%%%%
\end{document}
