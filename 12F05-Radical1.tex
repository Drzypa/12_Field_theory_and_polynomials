\documentclass[12pt]{article}
\usepackage{pmmeta}
\pmcanonicalname{Radical1}
\pmcreated{2013-03-22 16:55:36}
\pmmodified{2013-03-22 16:55:36}
\pmowner{Wkbj79}{1863}
\pmmodifier{Wkbj79}{1863}
\pmtitle{radical}
\pmrecord{9}{39190}
\pmprivacy{1}
\pmauthor{Wkbj79}{1863}
\pmtype{Definition}
\pmcomment{trigger rebuild}
\pmclassification{msc}{12F05}
\pmclassification{msc}{12F10}
\pmrelated{RadicalExtension}
\pmrelated{NthRoot}
\pmrelated{SolvableByRadicals}
\pmrelated{Radical6}

\usepackage{amssymb}
\usepackage{amsmath}
\usepackage{amsfonts}

\usepackage{psfrag}
\usepackage{graphicx}
\usepackage{amsthm}
%%\usepackage{xypic}

\begin{document}
Let $F$ be a field and $\alpha$ be \PMlinkname{algebraic}{Algebraic} over $F$.  Then $\alpha$ is a \emph{radical} over $F$ if there exists a positive integer $n$ with $\alpha^n \in F$.

Note that, if $K/F$ is a field extension and $\alpha$ is a radical over $F$, then $\alpha$ is automatically a radical over $K$.

Following are some examples of radicals:

\begin{enumerate}

\item All numbers of the form $\displaystyle \sqrt[n]{\frac{a}{b}}$, where $n$ is a positive integer and $a$ and $b$ are integers with $b \neq 0$ are radicals over $\mathbb{Q}$.

\item The number $\sqrt[4]{2}$ is a radical over $\mathbb{Q}(\sqrt{2})$ since $(\sqrt[4]{2})^2=\sqrt{2} \in \mathbb{Q}(\sqrt{2})$.

\end{enumerate}
%%%%%
%%%%%
\end{document}
