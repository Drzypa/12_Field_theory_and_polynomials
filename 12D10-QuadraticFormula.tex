\documentclass[12pt]{article}
\usepackage{pmmeta}
\pmcanonicalname{QuadraticFormula}
\pmcreated{2013-03-22 11:46:15}
\pmmodified{2013-03-22 11:46:15}
\pmowner{yark}{2760}
\pmmodifier{yark}{2760}
\pmtitle{quadratic formula}
\pmrecord{13}{30227}
\pmprivacy{1}
\pmauthor{yark}{2760}
\pmtype{Theorem}
\pmcomment{trigger rebuild}
\pmclassification{msc}{12D10}
\pmclassification{msc}{26A99}
\pmclassification{msc}{26A24}
\pmclassification{msc}{26A09}
\pmclassification{msc}{26A06}
\pmclassification{msc}{26-01}
\pmclassification{msc}{11-00}
%\pmkeywords{Algebra}
%\pmkeywords{Polynomial}
\pmrelated{DerivationOfQuadraticFormula}
\pmrelated{QuadraticInequality}
\pmrelated{QuadraticEquationInMathbbC}
\pmrelated{ConjugatedRootsOfEquation2}
\pmrelated{QuadraticCongruence}

\endmetadata

\usepackage{graphicx}
\usepackage{bbm}
\newcommand{\Z}{\mathbbmss{Z}}
\newcommand{\C}{\mathbbmss{C}}
\newcommand{\R}{\mathbbmss{R}}
\newcommand{\Q}{\mathbbmss{Q}}
\newcommand{\mathbb}[1]{\mathbbmss{#1}}
\newcommand{\figura}[1]{\begin{center}\includegraphics{#1}\end{center}}
\newcommand{\figuraex}[2]{\begin{center}\includegraphics[#2]{#1}\end{center}}
\begin{document}
The roots of the quadratic equation
\[
  ax^2+bx+c=0\qquad{a,b,c\in\R,a\neq 0}
\]
are given by the formula
\[
  x=\frac{-b\pm\sqrt{b^2-4ac}}{2a}.
\]

The number $\Delta=b^2-4ac$ is called the \emph{discriminant} of the equation.
If $\Delta>0$, there are two different real roots,
if $\Delta=0$ there is a single real root,
and if $\Delta<0$ there are no real roots (but two different complex roots).

Let's work a few examples.

First, consider $2x^2-14x+24=0$.
Here $a=2$, $b=-14$, and $c=24$.
Substituting in the formula gives us
\[
  x=\frac{14\pm \sqrt{(-14)^2-4\cdot2\cdot24}}{2\cdot 2}
   =\frac{14\pm\sqrt{4}}{4}
   =\frac{14\pm2}{4}
   =\frac{7\pm1}{2}.
\]
So we have two solutions (depending on whether we take the sign $+$ or $-$):
$x=\frac{8}{2}=4$ and $x=\frac{6}{2}=3$.

Now we will solve $x^2-x-1=0$.
Here $a=1$, $b=-1$, and $c=-1$, so
\[
  x=\frac{1\pm\sqrt{(-1)^2-4(1)(-1)}}{2}
   =\frac{1\pm{\sqrt{5}}}{2},
\]
and the solutions are $x=\frac{1+\sqrt{5}}{2}$ and $x=\frac{1-\sqrt{5}}{2}$.
%%%%%
%%%%%
%%%%%
%%%%%
\end{document}
