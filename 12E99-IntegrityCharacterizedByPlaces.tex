\documentclass[12pt]{article}
\usepackage{pmmeta}
\pmcanonicalname{IntegrityCharacterizedByPlaces}
\pmcreated{2013-03-22 14:56:57}
\pmmodified{2013-03-22 14:56:57}
\pmowner{pahio}{2872}
\pmmodifier{pahio}{2872}
\pmtitle{integrity characterized by places}
\pmrecord{11}{36643}
\pmprivacy{1}
\pmauthor{pahio}{2872}
\pmtype{Theorem}
\pmcomment{trigger rebuild}
\pmclassification{msc}{12E99}
\pmclassification{msc}{13B21}
%\pmkeywords{integral over a ring}
\pmrelated{Integral}
\pmrelated{PlaceOfField}

\endmetadata

% this is the default PlanetMath preamble.  as your knowledge
% of TeX increases, you will probably want to edit this, but
% it should be fine as is for beginners.

% almost certainly you want these
\usepackage{amssymb}
\usepackage{amsmath}
\usepackage{amsfonts}

% used for TeXing text within eps files
%\usepackage{psfrag}
% need this for including graphics (\includegraphics)
%\usepackage{graphicx}
% for neatly defining theorems and propositions
 \usepackage{amsthm}
% making logically defined graphics
%%%\usepackage{xypic}

% there are many more packages, add them here as you need them

% define commands here

\theoremstyle{definition}
\newtheorem*{thmplain}{Theorem}
\begin{document}
\begin{thmplain}
 \, Let $R$ be a subring of the field $K$, \,$1\in R$. \,An element $\alpha$ of the field is integral over $R$ if and only if all \PMlinkname{places}{PlaceOfField} $\varphi$ of $K$ satisfy the implication
     $$\varphi \mathrm{\,\,is\,finite\,in\,}R\,\,\,\Rightarrow
          \,\,\varphi(\alpha)\mathrm{\,is\,finite}.$$
\end{thmplain}

\textbf{\PMlinkescapetext{Corollary} 1.} \,Let $R$ be a subring of the field $K$, \,$1\in R$. \,The integral closure of $R$ in $K$ is the intersection of all valuation domains in $K$ which contain the ring $R$. \,The integral closure is integrally closed in the field $K$.


\textbf{\PMlinkescapetext{Corollary} 2.}  \,Every valuation domain is integrally closed in its field of fractions.
%%%%%
%%%%%
\end{document}
