\documentclass[12pt]{article}
\usepackage{pmmeta}
\pmcanonicalname{CommensurableNumbers}
\pmcreated{2013-03-22 18:11:14}
\pmmodified{2013-03-22 18:11:14}
\pmowner{pahio}{2872}
\pmmodifier{pahio}{2872}
\pmtitle{commensurable numbers}
\pmrecord{13}{40760}
\pmprivacy{1}
\pmauthor{pahio}{2872}
\pmtype{Definition}
\pmcomment{trigger rebuild}
\pmclassification{msc}{12D99}
\pmclassification{msc}{03E02}
\pmrelated{RationalAndIrrational}
\pmrelated{CommensurableSubgroups}
\pmdefines{commensurable}
\pmdefines{incommensurable}
\pmdefines{commensurability}

\endmetadata

% this is the default PlanetMath preamble.  as your knowledge
% of TeX increases, you will probably want to edit this, but
% it should be fine as is for beginners.

% almost certainly you want these
\usepackage{amssymb}
\usepackage{amsmath}
\usepackage{amsfonts}

% used for TeXing text within eps files
%\usepackage{psfrag}
% need this for including graphics (\includegraphics)
%\usepackage{graphicx}
% for neatly defining theorems and propositions
 \usepackage{amsthm}
% making logically defined graphics
%%%\usepackage{xypic}

% there are many more packages, add them here as you need them

% define commands here

\theoremstyle{definition}
\newtheorem*{thmplain}{Theorem}

\begin{document}
Two positive real numbers $a$ and $b$ are {\em commensurable}, iff there exists a positive real number $u$ such that
\begin{align}
a \;=\; mu, \quad b \;=\; nu
\end{align}
with some positive integers $m$ and $n$.\, If the positive numbers $a$ and $b$ are not commensurable, they are {\em incommensurable}.\\

\textbf{Theorem.}\, The positive numbers $a$ and $b$ are commensurable if and only if their ratio is a rational number 
$\displaystyle\frac{m}{n}$\, ($m,\,n \in \mathbb{Z}$).\\

{\em Proof.}\, The equations (1) imply the \PMlinkname{proportion}{ProportionEquation}
\begin{align}
\frac{a}{b} \;=\; \frac{m}{n}.
\end{align}
Conversely, if (2) is valid with\, $m,\,n \in \mathbb{Z}$,\, then we can write
$$a \;=\; m\!\cdot\!\frac{b}{n}, \quad b \;=\; n\!\cdot\!\frac{b}{n},$$
which means that $a$ and $b$ are multiples of $\displaystyle\frac{b}{n}$ and thus commensurable.\, Q.E.D.\\

\textbf{Example.}\, The lengths of the side and the diagonal of \PMlinkid{square}{1086} are always incommensurable.

\subsection{Commensurability as relation}

\begin{itemize}

\item The commensurability is an equivalence relation in the set $\mathbb{R}_+$ of the positive reals:\, the reflexivity and the symmetry are trivial;\, if\, $a\!:\!b = r$\, and\, $b\!:\!c = s$,\, then\, $a\!:\!c = (a\!:\!b)(b\!:\!c) = rs$,\, whence one obtains the transitivity.\\

\item The equivalence classes of the commensurability are of the form
$$[\varrho] \;:=\; \{r\varrho\,\vdots\;\; r \in \mathbb{Q}_+\}.$$

\item One of the equivalence classes is the set\, $[1] = \mathbb{Q}_+$\, of the positive rationals, all others consist of positive irrational numbers.

\item If one sets\; $[\varrho]\!\cdot\![\sigma] := [\varrho\sigma]$,\, the equivalence classes form with respect to this binary operation an Abelian group.

\end{itemize}

%%%%%
%%%%%
\end{document}
