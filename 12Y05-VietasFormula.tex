\documentclass[12pt]{article}
\usepackage{pmmeta}
\pmcanonicalname{VietasFormula}
\pmcreated{2013-03-22 15:21:55}
\pmmodified{2013-03-22 15:21:55}
\pmowner{neapol1s}{9480}
\pmmodifier{neapol1s}{9480}
\pmtitle{Vieta's formula}
\pmrecord{9}{37190}
\pmprivacy{1}
\pmauthor{neapol1s}{9480}
\pmtype{Theorem}
\pmcomment{trigger rebuild}
\pmclassification{msc}{12Y05}
\pmrelated{PropertiesOfQuadraticEquation}

% this is the default PlanetMath preamble.  as your knowledge
% of TeX increases, you will probably want to edit this, but
% it should be fine as is for beginners.

% almost certainly you want these
\usepackage{amssymb}
\usepackage{amsmath}
\usepackage{amsfonts}

% used for TeXing text within eps files
%\usepackage{psfrag}
% need this for including graphics (\includegraphics)
%\usepackage{graphicx}
% for neatly defining theorems and propositions
%\usepackage{amsthm}
% making logically defined graphics
%%%\usepackage{xypic}

% there are many more packages, add them here as you need them

% define commands here
\begin{document}
Suppose $P(x)$ is a polynomial of degree $n$ with roots $r_1, r_2, \ldots, r_n$ (not necessarily distinct). For $1\leq k\leq n$, define $S_k$ by
\[S_k = \sum\limits_{1\leq\alpha_{1} < \alpha_{2} < \ldots\alpha_k\leq n} r_{\alpha_1}r_{\alpha_2}\ldots r_{\alpha_k}\]
For example, 
\[S_1 = r_1 + r_2 + r_3 + \ldots + r_n\]
\[S_2 = r_1r_2 + r_1r_3 + r_1r_4 + r_2r_3 + \ldots + r_{n-1}r_{n}\]
Then writing $P(x)$ as 
\[P(x) = a_nx^n + a_{n-1}x^{n-1} + \ldots a_{1}x + a_{0},\]
we find that 
\[S_i = (-1)^{i}\frac{a_{n-i}}{a_n}\]

For example, if $P(x)$ is a polynomial of degree 1, then $P(x) = a_1x + a_0$ and clearly $r_1 = -\frac{a_0}{a_1}$.

If $P(x)$ is a polynomial of degree 2, then $P(x) = a_2x^2 + a_1x + a_0$ and  $r_1 + r_2 = -\frac{a_1}{a_2}$ and $r_1r_2 = \frac{a_0}{a_2}$. Notice that both of these formulas can be determined from the quadratic formula. 

More intrestingly, if $P(x) = a_3x^3 + a_2x^2 + a_1x + a_0$, then $r_1 + r_2 + r_3 = -\frac{a_2}{a_3}$, $r_1r_2 + r_2r_3 + r_3r_1 = \frac{a_1}{a_3}$, and $r_1r_2r_3 = -\frac{a_0}{a_3}$.
%%%%%
%%%%%
\end{document}
