\documentclass[12pt]{article}
\usepackage{pmmeta}
\pmcanonicalname{WeightedHomogeneousPolynomial}
\pmcreated{2013-03-22 15:21:18}
\pmmodified{2013-03-22 15:21:18}
\pmowner{jirka}{4157}
\pmmodifier{jirka}{4157}
\pmtitle{weighted homogeneous polynomial}
\pmrecord{5}{37177}
\pmprivacy{1}
\pmauthor{jirka}{4157}
\pmtype{Definition}
\pmcomment{trigger rebuild}
\pmclassification{msc}{12-00}

% this is the default PlanetMath preamble.  as your knowledge
% of TeX increases, you will probably want to edit this, but
% it should be fine as is for beginners.

% almost certainly you want these
\usepackage{amssymb}
\usepackage{amsmath}
\usepackage{amsfonts}

% used for TeXing text within eps files
%\usepackage{psfrag}
% need this for including graphics (\includegraphics)
%\usepackage{graphicx}
% for neatly defining theorems and propositions
\usepackage{amsthm}
% making logically defined graphics
%%%\usepackage{xypic}

% there are many more packages, add them here as you need them

% define commands here
\theoremstyle{theorem}
\newtheorem*{thm}{Theorem}
\newtheorem*{lemma}{Lemma}
\newtheorem*{conj}{Conjecture}
\newtheorem*{cor}{Corollary}
\newtheorem*{example}{Example}
\newtheorem*{prop}{Proposition}
\theoremstyle{definition}
\newtheorem*{defn}{Definition}
\theoremstyle{remark}
\newtheorem*{rmk}{Remark}
\begin{document}
Let ${\mathbb{F}}$ be either the real or complex numbers.

\begin{defn}
Let $p \colon {\mathbb{F}}^n \to {\mathbb{F}}$ be a polynomial in $n$ variables 
and take integers $d_1, d_2, \ldots, d_n$.
The polynomial $p$ is said to be
{\em weighted homogeneous of degree $k$} if for all $t > 0$ we have
\begin{equation*}
p(t^{d_1} x_1,t^{d_2} x_2,\ldots,t^{d_n} x_n) = t^k p(x_1,x_2,\ldots,x_n) .
\end{equation*}
The $d_1,\ldots,d_n$ are called the {\em weights} of the variables $x_1,\ldots,x_n$.  
\end{defn}

Note that if $d_1 = d_2 = \ldots = d_n = 1$ then this definition is the standard homogeneous polynomial.
%%%%%
%%%%%
\end{document}
