\documentclass[12pt]{article}
\usepackage{pmmeta}
\pmcanonicalname{RationalRootTheorem}
\pmcreated{2013-03-22 11:46:18}
\pmmodified{2013-03-22 11:46:18}
\pmowner{drini}{3}
\pmmodifier{drini}{3}
\pmtitle{rational root theorem}
\pmrecord{13}{30228}
\pmprivacy{1}
\pmauthor{drini}{3}
\pmtype{Theorem}
\pmcomment{trigger rebuild}
\pmclassification{msc}{12D10}
\pmclassification{msc}{12D05}
\pmclassification{msc}{26A99}
\pmclassification{msc}{26A24}
\pmclassification{msc}{26A09}
\pmclassification{msc}{26A06}
\pmclassification{msc}{26-01}
\pmclassification{msc}{11-00}
%\pmkeywords{polynomial}
\pmrelated{FactorTheorem}

\endmetadata

\usepackage{amssymb}
\usepackage{amsmath}
\usepackage{amsfonts}
\usepackage{graphicx}
%%%%\usepackage{xypic}
\begin{document}
\PMlinkescapeword{states}
\PMlinkescapeword{domain}
\PMlinkescapeword{base}
Consider the polynomial
$$p(x)\;=\; a_nx^n + a_{n-1}x^{n-1}+\cdots+a_1x+a_0$$
where all the coefficients $a_i$ are integers.

If $p(x)$ has a rational zero $u/v$ where\, $\gcd(u,\,v)=1$,\, then\,
$u\mid a_0$\, and\, $v\mid a_n$.\, Thus, for finding all rational zeros of $p(x)$, it suffices to perform a finite number of tests.\\

The theorem is related to the result about monic polynomials whose coefficients belong to a unique factorization domain. Such theorem then states that any root in the fraction field is also in the base domain.
%%%%%
%%%%%
%%%%%
%%%%%
\end{document}
