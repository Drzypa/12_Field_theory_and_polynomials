\documentclass[12pt]{article}
\usepackage{pmmeta}
\pmcanonicalname{InertialDegree}
\pmcreated{2013-03-22 12:38:17}
\pmmodified{2013-03-22 12:38:17}
\pmowner{djao}{24}
\pmmodifier{djao}{24}
\pmtitle{inertial degree}
\pmrecord{6}{32902}
\pmprivacy{1}
\pmauthor{djao}{24}
\pmtype{Definition}
\pmcomment{trigger rebuild}
\pmclassification{msc}{12F99}
\pmclassification{msc}{13B02}
\pmclassification{msc}{11S15}
\pmsynonym{residue degree}{InertialDegree}
\pmrelated{Ramify}
\pmrelated{DecompositionGroup}
\pmdefines{inert}

% this is the default PlanetMath preamble.  as your knowledge
% of TeX increases, you will probably want to edit this, but
% it should be fine as is for beginners.

% almost certainly you want these
\usepackage{amssymb}
\usepackage{amsmath}
\usepackage{amsfonts}

% used for TeXing text within eps files
%\usepackage{psfrag}
% need this for including graphics (\includegraphics)
%\usepackage{graphicx}
% for neatly defining theorems and propositions
%\usepackage{amsthm}
% making logically defined graphics
%%%\usepackage{xypic} 

% there are many more packages, add them here as you need them

% define commands here
\newcommand{\p}{\mathfrak{p}}
\newcommand{\m}{\mathfrak{m}}
\newcommand{\M}{\mathfrak{M}}
\renewcommand{\P}{\mathfrak{P}}
\newcommand{\lra}{\longrightarrow}
\renewcommand{\O}{\mathcal{O}}
\newcommand{\Z}{\mathbb{Z}}
\newcommand{\Q}{\mathbb{Q}}
\begin{document}
Let $\iota\colon A \to B$ be a ring homomorphism. Let $\P \subset B$ be a prime ideal, with $\p := \iota^{-1}(\P) \subset A$. The algebra map $\iota$ induces an $A/\p$ module structure on the ring $B/\P$. If the dimension of $B/\P$ as an $A/\p$ module exists, then it is called the \emph{inertial degree} of $\P$ over $A$.

A particular case of special importance in number theory is when $L/K$ is a field extension and $\iota\colon \O_K \to \O_L$ is the inclusion map of the ring of integers. In this case, the domain $\O_K/\p$ is a field, so $\dim_{\O_K/\p}\O_L/\P$ is guaranteed to exist, and the inertial degree of $\P$ over $\O_K$ is denoted $f(\P/\p)$. We have the formula
$$
\sum_{\P \mid \p} e(\P/\p) f(\P/\p) = [L:K],
$$
where $e(\P/\p)$ is the ramification index of $\P$ over $\p$ and the sum is taken over all prime ideals $\P$ of $\O_L$ dividing $\p\O_L$. The prime $\P$ (and also the prime $\p$) is said to be \emph{inert} if $f(\P/\p) = [L:K]$.

{\bf Example:}

Let $\iota\colon \Z \to \Z[i]$ be the inclusion of the integers into the Gaussian integers. A prime $p$ in $\Z$ may or may not factor in $\Z[i]$; if it does factor, then it must factor as $p = (x+yi)(x-yi)$ for some integers $x,y$. Thus a prime $p$ factors into two primes if it equals $x^2 + y^2$, and remains prime in $\Z[i]$ otherwise. There are then three categories of primes in $\Z[i]$:
\begin{enumerate}
\item The prime 2 factors as $(1+i)(1-i)$, and the principal ideals generated by $(1+i)$ and $(1-i)$ are equal in $\Z[i]$, so the ramification index of $(1+i)$ over $\Z$ is two. The ring $Z[i]/(1+i)$ is isomorphic to $\Z/2$, so the inertial degree $f((1+i)/(2))$ is one.
\item For primes $p \equiv 1 \bmod{4}$, the prime $p \in \Z$ factors into the product of the two primes $(x+yi)(x-yi)$, with ramification index and inertial degree one.
\item For primes $p \equiv 3 \pmod{4}$, the prime $p$ remains prime in $\Z[i]$ and $\Z[i]/(p)$ is a two dimensional field extension of $\Z/p$, so the inertial degree is two and the ramification index is one.
\end{enumerate}
In all cases, the sum of the products of the inertial degree and ramification index is equal to 2, which is the dimension of the corresponding extension $\Q(i)/\Q$ of number fields.

\section{Local interpretations \& generalizations}

For any extension $\iota\colon A \to B$ of Dedekind domains, the inertial degree of the prime $\P \subset B$ over the prime $\p := \iota^{-1}(\P) \subset A$ is equal to the inertial degree of $\P B_\P$ over $\p A_\p$ in the localizations at $\P$ and $\p$. Moreover, the same is true even if we pass to completions of the local rings $B_\P$ and $A_\p$ at $\P$ and $\p$. The preservation of inertial degree and ramification indices with respect to localization is one of the reasons why the technique of localization is a useful tool in the study of such domains.

As in the case of ramification indices, it is possible to define the notion of inertial degree in the more general setting of locally ringed spaces. However, the generalizations of inertial degree are not as widely used because in algebraic geometry one usually works with a fixed base field, which makes all the residue fields at the points equal to the same field.
%%%%%
%%%%%
\end{document}
