\documentclass[12pt]{article}
\usepackage{pmmeta}
\pmcanonicalname{SplittingField}
\pmcreated{2013-03-22 12:08:01}
\pmmodified{2013-03-22 12:08:01}
\pmowner{djao}{24}
\pmmodifier{djao}{24}
\pmtitle{splitting field}
\pmrecord{7}{31303}
\pmprivacy{1}
\pmauthor{djao}{24}
\pmtype{Definition}
\pmcomment{trigger rebuild}
\pmclassification{msc}{12F05}
\pmrelated{NormalExtension}

\endmetadata

\usepackage{amssymb}
\usepackage{amsmath}
\usepackage{amsfonts}
\usepackage{graphicx}
%%%\usepackage{xypic}
\begin{document}
Let $f \in F[x]$ be a polynomial over a field $F$. A {\em splitting field} for $f$ is a field extension $K$ of $F$ such that
\begin{enumerate}
\item $f$ splits (factors into a product of linear factors) in $K[x]$,
\item $K$ is the smallest field with this property (any sub-extension field of $K$ which satisfies the first property is equal to $K$).
\end{enumerate}
{\bf Theorem:} Any polynomial over any field has a splitting field, and any two such splitting fields are isomorphic. A splitting field is always a normal extension of the ground field.
%%%%%
%%%%%
%%%%%
\end{document}
