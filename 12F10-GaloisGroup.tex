\documentclass[12pt]{article}
\usepackage{pmmeta}
\pmcanonicalname{GaloisGroup}
\pmcreated{2013-03-22 12:08:19}
\pmmodified{2013-03-22 12:08:19}
\pmowner{djao}{24}
\pmmodifier{djao}{24}
\pmtitle{Galois group}
\pmrecord{8}{31317}
\pmprivacy{1}
\pmauthor{djao}{24}
\pmtype{Definition}
\pmcomment{trigger rebuild}
\pmclassification{msc}{12F10}
\pmrelated{FundamentalTheoremOfGaloisTheory}
\pmrelated{InfiniteGaloisTheory}

\endmetadata

\usepackage{amssymb}
\usepackage{amsmath}
\usepackage{amsfonts}
\usepackage{graphicx}
%%%\usepackage{xypic}
\begin{document}
The \emph{Galois group} $\operatorname{Gal}(K/F)$ of a field extension $K/F$ is the group of all field automorphisms $\sigma\colon K \to K$ of $K$ which fix $F$ (i.e., $\sigma(x) = x$ for all $x \in F$). The group operation is given by composition: for two automorphisms $\sigma_1, \sigma_2 \in \operatorname{Gal}(K/F)$, given by $\sigma_1\colon K \to K$ and $\sigma_2\colon K \to K$, the product $\sigma_1 \cdot \sigma_2 \in \operatorname{Gal}(K/F)$ is the composite of the two maps $\sigma_1 \circ \sigma_2\colon K \to K$.

The \emph{Galois group} of a polynomial $f(x) \in F[x]$ is defined to be the Galois group of the splitting field of $f(x)$ over $F$.
%%%%%
%%%%%
%%%%%
\end{document}
