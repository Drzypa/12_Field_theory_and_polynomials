\documentclass[12pt]{article}
\usepackage{pmmeta}
\pmcanonicalname{HilbertsIrreducibilityTheorem}
\pmcreated{2013-03-22 15:14:16}
\pmmodified{2013-03-22 15:14:16}
\pmowner{alozano}{2414}
\pmmodifier{alozano}{2414}
\pmtitle{Hilbert's irreducibility theorem}
\pmrecord{5}{37010}
\pmprivacy{1}
\pmauthor{alozano}{2414}
\pmtype{Theorem}
\pmcomment{trigger rebuild}
\pmclassification{msc}{12E25}
\pmsynonym{Hilbertian}{HilbertsIrreducibilityTheorem}
\pmdefines{Hilbert property}
\pmdefines{Hilbertian field}

\endmetadata

% this is the default PlanetMath preamble.  as your knowledge
% of TeX increases, you will probably want to edit this, but
% it should be fine as is for beginners.

% almost certainly you want these
\usepackage{amssymb}
\usepackage{amsmath}
\usepackage{amsthm}
\usepackage{amsfonts}

% used for TeXing text within eps files
%\usepackage{psfrag}
% need this for including graphics (\includegraphics)
%\usepackage{graphicx}
% for neatly defining theorems and propositions
%\usepackage{amsthm}
% making logically defined graphics
%%%\usepackage{xypic}

% there are many more packages, add them here as you need them

% define commands here

\newtheorem*{thm}{Theorem}
\newtheorem{defn}{Definition}
\newtheorem{prop}{Proposition}
\newtheorem{lemma}{Lemma}
\newtheorem{cor}{Corollary}

\theoremstyle{definition}
\newtheorem{exa}{Example}

% Some sets
\newcommand{\Nats}{\mathbb{N}}
\newcommand{\Ints}{\mathbb{Z}}
\newcommand{\Reals}{\mathbb{R}}
\newcommand{\Complex}{\mathbb{C}}
\newcommand{\Rats}{\mathbb{Q}}
\newcommand{\Gal}{\operatorname{Gal}}
\newcommand{\Cl}{\operatorname{Cl}}
\begin{document}
In this entry, $K$ is a field of characteristic zero and $V$ is an irreducible algebraic variety over $K$.

\begin{defn}
A variety $V$ satisfies the Hilbert property over $K$ if $V(K)$ is not a thin algebraic set.
\end{defn}

\begin{defn}
A field $K$ is said to be Hilbertian if there exists an irreducible variety $V/K$ of $\dim V \geq 1$ which has the Hilbert property.
\end{defn}

\begin{thm}[Hilbert's irreducibility theorem]
A number field $K$ is Hilbertian. In particular, for every $n$, the affine space $\mathbb{A}^n(K)$ has the Hilbert property over $K$.
\end{thm}

However, the field of real numbers $\Reals$ and the field of $p$-adic rationals $\Rats_p$ are not Hilbertian.

\begin{thebibliography}{9}
\bibitem{serre} J.-P. Serre, {\em Topics in Galois Theory},
Research Notes in Mathematics, Jones and Barlett Publishers, London.
\end{thebibliography}
%%%%%
%%%%%
\end{document}
