\documentclass[12pt]{article}
\usepackage{pmmeta}
\pmcanonicalname{EuclideanField}
\pmcreated{2013-03-22 14:22:39}
\pmmodified{2013-03-22 14:22:39}
\pmowner{CWoo}{3771}
\pmmodifier{CWoo}{3771}
\pmtitle{Euclidean field}
\pmrecord{34}{35869}
\pmprivacy{1}
\pmauthor{CWoo}{3771}
\pmtype{Definition}
\pmcomment{trigger rebuild}
\pmclassification{msc}{12D15}
\pmrelated{ConstructibleNumbers}
\pmrelated{EuclideanNumberField}
\pmdefines{Euclidean}

\endmetadata

% this is the default PlanetMath preamble.  as your knowledge
% of TeX increases, you will probably want to edit this, but
% it should be fine as is for beginners.

% almost certainly you want these
\usepackage{amssymb}
\usepackage{amsmath}
\usepackage{amsfonts}

% used for TeXing text within eps files
%\usepackage{psfrag}
% need this for including graphics (\includegraphics)
%\usepackage{graphicx}
% for neatly defining theorems and propositions
%\usepackage{amsthm}
% making logically defined graphics
%%%\usepackage{xypic}

% there are many more packages, add them here as you need them

% define commands here

\begin{document}
\PMlinkescapeword{close}
\PMlinkescapeword{constructible}
\PMlinkescapeword{Euclidean}
\PMlinkescapeword{length}
\PMlinkescapeword{level}
\PMlinkescapeword{measure}
\PMlinkescapeword{open}

An ordered field $F$ is \emph{Euclidean} if every non-negative element $a$ ($a\geq0$) is a square in $F$ (there exists $b\in F$ such that $b^2=a$).

\section{Examples}
\begin{itemize}
\item
$\mathbb{R}$ is Euclidean.  
\item$\mathbb{Q}$ is not Euclidean because $2$ is not a square in $\mathbb{Q}$ (\PMlinkname{i.e.}{Ie}, $\pm\sqrt{2}\notin \mathbb{Q}$).
\item  $\mathbb{C}$ is not a Euclidean field because \PMlinkname{$\mathbb{C}$ is not an ordered field}{MathbbCIsNotAnOrderedField}.
\item
The \PMlinkname{field of real constructible numbers}{ConstructibleNumbers} is Euclidean.
\end{itemize}


A Euclidean field is an ordered Pythagorean field.
 
There are ordered fields that are Pythagorean but not Euclidean.
%%%%%
%%%%%
\end{document}
