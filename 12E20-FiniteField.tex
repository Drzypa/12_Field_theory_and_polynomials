\documentclass[12pt]{article}
\usepackage{pmmeta}
\pmcanonicalname{FiniteField}
\pmcreated{2013-03-22 12:37:50}
\pmmodified{2013-03-22 12:37:50}
\pmowner{yark}{2760}
\pmmodifier{yark}{2760}
\pmtitle{finite field}
\pmrecord{16}{32893}
\pmprivacy{1}
\pmauthor{yark}{2760}
\pmtype{Definition}
\pmcomment{trigger rebuild}
\pmclassification{msc}{12E20}
\pmclassification{msc}{11T99}
\pmsynonym{Galois field}{FiniteField}
\pmrelated{AlgebraicClosureOfAFiniteField}
\pmrelated{IrreduciblePolynomialsOverFiniteField}

\usepackage{amssymb}
\usepackage{amsmath}
\usepackage{amsfonts}
\usepackage{amsthm}

\newcommand{\p}{{\mathfrak{p}}}
\newcommand{\C}{\mathbb{C}}
\newcommand{\Z}{\mathbb{Z}}
\newcommand{\N}{\mathbb{N}}
\newcommand{\R}{\mathbb{R}}
\newcommand{\F}{\mathbb{F}}
\newcommand{\lra}{\longrightarrow}
\renewcommand{\div}{\mid}
\newcommand{\id}{\operatorname{id}}
\newcommand{\Frob}{\operatorname{Frob}}
\newcommand{\Gal}{\operatorname{Gal}}

\newtheorem{theorem}{Theorem}[section]
\newtheorem{proposition}[theorem]{Proposition}
\newtheorem{lemma}[theorem]{Lemma}
\newtheorem{corollary}[theorem]{Corollary}

\theoremstyle{definition}
\newtheorem{definition}[theorem]{Definition}
\newtheorem{example}[theorem]{Example}
\begin{document}
\PMlinkescapeword{order}

A \emph{finite field} (also called a \emph{Galois field}) is a field that has finitely many elements.
The number of elements in a finite field is sometimes called the \emph{order} of the field.
We will present some basic facts about finite fields.

\section{Size of a finite field}

\begin{theorem}
A finite field $F$ has positive characteristic $p > 0$ for some prime $p$. The
cardinality of $F$ is $p^n$ where $n := [F:\F_p]$ and $\F_p$ denotes
the prime subfield of $F$.
\end{theorem}
\begin{proof}
The characteristic of $F$ is positive because otherwise the additive
subgroup generated by $1$ would be an infinite subset of
$F$. Accordingly, the prime subfield $\F_p$ of $F$ is isomorphic to
the field $\Z/p\Z$ of integers mod $p$. The integer $p$ is prime since otherwise $\Z/p\Z$ would have zero divisors. Since the field $F$ is an
$n$--dimensional vector space over $\F_p$ for some finite $n$, it is set--isomorphic to
$\F_p^n$ and thus has cardinality $p^n$.
\end{proof}

\section{Existence of finite fields}

Now that we know every finite field has $p^n$ elements, it is natural
to ask which of these actually arise as cardinalities of finite
fields. It turns out that for each prime $p$ and each natural number
$n$, there is essentially exactly one finite field of size $p^n$.

\begin{lemma}\label{fermat}
In any field $F$ with $m$ elements, the equation $x^m = x$ is satisfied by all elements $x$ of $F$.
\end{lemma}
\begin{proof}
The result is clearly true if $x = 0$. We may therefore assume $x$ is not zero. By definition of field, the set $F^\times$ of nonzero elements of $F$ forms a group under multiplication. This set has $m-1$ elements, and by Lagrange's theorem $x^{m-1} = 1$ for any $x \in F^\times$, so $x^m = x$ follows.
\end{proof}

\begin{theorem}\label{existence}
For each prime $p > 0$ and each natural number $n \in \N$, there
exists a finite field of cardinality $p^n$, and any two such are
isomorphic.
\end{theorem}
\begin{proof}
For $n=1$, the finite field $\F_p := \Z/p\Z$ has $p$ elements, and any
two such are isomorphic by the map sending $1$ to $1$.

In general, the polynomial $f(X) := X^{p^n} - X \in \F_p[X]$ has
derivative $-1$ and thus is separable over $\F_p$. We claim that the
splitting field $F$ of this polynomial is a finite field of size
$p^n$. The field $F$ certainly contains the set $S$ of roots of
$f(X)$. However, the set $S$ is closed under the field operations, so
$S$ is itself a field. Since splitting fields are minimal by
definition, the containment $S \subset F$ means that $S = F$. Finally,
$S$ has $p^n$ elements since $f(X)$ is separable, so $F$ is a field of
size $p^n$.

For the uniqueness part, any other field $F'$ of size $p^n$ contains a
subfield isomorphic to $\F_p$. Moreover, $F'$ equals the splitting field of
the polynomial $X^{p^n} - X$ over $\F_p$, since by Lemma~\ref{fermat} every element of $F'$ is a root of this polynomial, and all $p^n$ possible roots of the polynomial are accounted for in this way. By the uniqueness of
splitting fields up to isomorphism, the two fields $F$ and $F'$ are
isomorphic.
\end{proof}

Note: The proof of Theorem~\ref{existence} given here, while standard
because of its efficiency, relies on more abstract algebra than is
strictly necessary. The reader may find a more concrete presentation
of this and many other results about finite fields
in~\cite[Ch. 7]{ir}.

\begin{corollary}\label{normal}
Every finite field $F$ is a normal extension of its prime subfield
$\F_p$.
\end{corollary}
\begin{proof}
This follows from the fact that field extensions obtained from
splitting fields are normal extensions.
\end{proof}

\section{Units in a finite field}

Henceforth, in light of Theorem~\ref{existence}, we will write $\F_q$
for the unique (up to isomorphism) finite field of cardinality $q =
p^n$. A fundamental step in the investigation of finite fields is the
observation that their multiplicative groups are cyclic:

\begin{theorem}\label{cyclic}
The multiplicative group $\F_q^*$ consisting of nonzero elements of
the finite field $\F_q$ is a cyclic group.
\end{theorem}
\begin{proof}
We begin with the formula
\begin{equation}\label{sum}
\sum_{d \div k} \phi(d) = k,
\end{equation}
where $\phi$ denotes the Euler totient function. It is proved as
follows. For every divisor $d$ of $k$, the cyclic group $C_k$ of size
$k$ has exactly one cyclic subgroup $C_d$ of size $d$. Let $G_d$ be
the subset of $C_d$ consisting of elements of $C_d$ which have the
maximum possible \PMlinkname{order}{OrderGroup} of $d$. Since every element of $C_k$ has
maximal order in the subgroup of $C_k$ that it generates, we see that
the sets $G_d$ partition the set $C_k$, so that
$$
\sum_{d \div k} |G_d| = |C_k| = k.
$$
The identity~\eqref{sum} then follows from the observation that the
cyclic subgroup $C_d$ has exactly $\phi(d)$ elements of maximal order
$d$.

We now prove the theorem. Let $k = q-1$, and for each divisor $d$ of
$k$, let $\psi(d)$ be the number of elements of $\F_q^*$ of order
$d$. We claim that $\psi(d)$ is either zero or $\phi(d)$. Indeed, if
it is nonzero, then let $x \in \F_q^*$ be an element of order $d$, and
let $G_x$ be the subgroup of $\F_q^*$ generated by $x$. Then $G_x$ has
size $d$ and every element of $G_x$ is a root of the polynomial $x^d -
1$. But this polynomial cannot have more than $d$ roots in a field, so
every root of $x^d - 1$ must be an element of $G_x$. In particular,
every element of order $d$ must be in $G_x$ already, and we see that
$G_x$ only has $\phi(d)$ elements of order $d$.

We have proved that $\psi(d) \leq \phi(d)$ for all $d \div q-1$. If
$\psi(q-1)$ were 0, then we would have
$$
\sum_{d \div q-1} \psi(d) < \sum_{d \div q-1} \phi(d) = q-1,
$$
which is impossible since the first sum must equal $q-1$ (because
every element of $\F_q^*$ has order equal to some divisor $d$ of
$q-1$).
\end{proof}

A more constructive proof of Theorem~\ref{cyclic}, which actually
exhibits a generator for the cyclic group, may be found
in~\cite[Ch. 16]{stewart}.

\begin{corollary}
Every extension of finite fields is a primitive extension.
\end{corollary}
\begin{proof}
By Theorem~\ref{cyclic}, the multiplicative group of the extension field is cyclic. Any generator of the multiplicative group of the extension field also algebraically generates the extension field over the base field.
\end{proof}

\section{Automorphisms of a finite field}

Observe that, since a splitting field for $X^{q^m} - X$ over $\F_p$
contains all the roots of $X^q - X$, it follows that the field
$\F_{q^m}$ contains a subfield isomorphic to $\F_q$. We will show
later (Theorem~\ref{subfields}) that this is the only way that extensions of
finite fields can arise. For now we will construct the Galois group of
the field extension $\F_{q^m}/\F_q$, which is normal by
Corollary~\ref{normal}.

\begin{theorem}
The Galois group of the field extension $\F_{q^m}/\F_q$ is a cyclic
group of size $m$ generated by the $q^{\rm th}$ power Frobenius map
$\Frob_q$.
\end{theorem}
\begin{proof}
The fact that $\Frob_q$ is an element of $\Gal(\F_{q^m}/\F_q)$, and
that $(\Frob_q)^m = \Frob_{q^m}$ is the identity on $\F_{q^m}$, is
obvious. Since the extension $\F_{q^m}/\F_q$ is normal and of degree
$m$, the group $\Gal(\F_{q^m}/\F_q)$ must have size $m$, and we will
be done if we can show that $(\Frob_q)^k$, for $k = 0, 1, \ldots,
m-1$, are distinct elements of $\Gal(\F_{q^m}/\F_q)$.

It is enough to show that none of $(\Frob_q)^k$, for $k = 1, 2,
\ldots, m-1$, is the identity map on $\F_{q^m}$, for then we will have
shown that $\Frob_q$ is of order exactly equal to $m$. But, if any
such $(\Frob_q)^k$ were the identity map, then the polynomial $X^{q^k}
- X$ would have $q^m$ distinct roots in $\F_{q^m}$, which is
impossible in a field since $q^k < q^m$.
\end{proof}

We can now use the Galois correspondence between subgroups of the
Galois group and intermediate fields of a field extension to
immediately classify all the intermediate fields in the extension
$\F_{q^m}/\F_q$.

\begin{theorem}\label{subfields}
The field extension $\F_{q^m}/\F_q$ contains exactly one intermediate
field isomorphic to $\F_{q^d}$, for each divisor $d$ of $m$, and no
others. In particular, the subfields of $\F_{p^n}$ are precisely the
fields $\F_{p^d}$ for $d \div n$.
\end{theorem}
\begin{proof}
By the fundamental theorem of Galois theory, each intermediate field
of $\F_{q^m}/\F_q$ corresponds to a subgroup of
$\Gal(\F_{q^m}/\F_q)$. The latter is a cyclic group of order $m$, so
its subgroups are exactly the cyclic groups generated by
$(\Frob_q)^d$, one for each $d \div m$. The fixed field of
$(\Frob_q)^d$ is the set of roots of $X^{q^d} - X$, which forms a
subfield of $\F_{q^m}$ isomorphic to $\F_{q^d}$, so the result
follows.

The subfields of $\F_{p^n}$ can be obtained by applying the above
considerations to the extension $\F_{p^n}/\F_p$.
\end{proof}


\begin{thebibliography}{9}
\bibitem{ir} Kenneth Ireland \& Michael Rosen, {\em A Classical
Introduction to Modern Number Theory, Second Edition},
Springer--Verlag, 1990 (GTM {\bf 84}).
\bibitem{stewart} Ian Stewart, {\em Galois Theory, Second Edition},
Chapman \& Hall, 1989.
\end{thebibliography}
%%%%%
%%%%%
\end{document}
