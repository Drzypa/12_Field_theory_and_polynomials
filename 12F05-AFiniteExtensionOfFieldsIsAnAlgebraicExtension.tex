\documentclass[12pt]{article}
\usepackage{pmmeta}
\pmcanonicalname{AFiniteExtensionOfFieldsIsAnAlgebraicExtension}
\pmcreated{2013-03-22 13:57:30}
\pmmodified{2013-03-22 13:57:30}
\pmowner{alozano}{2414}
\pmmodifier{alozano}{2414}
\pmtitle{a finite extension of fields is an algebraic extension}
\pmrecord{6}{34725}
\pmprivacy{1}
\pmauthor{alozano}{2414}
\pmtype{Theorem}
\pmcomment{trigger rebuild}
\pmclassification{msc}{12F05}
%\pmkeywords{algebraic}
%\pmkeywords{polynomial}
%\pmkeywords{finite}
\pmrelated{Algebraic}
\pmrelated{AlgebraicExtension}
\pmrelated{ProofOfTranscendentalRootTheorem}

\endmetadata

% this is the default PlanetMath preamble.  as your knowledge
% of TeX increases, you will probably want to edit this, but
% it should be fine as is for beginners.

% almost certainly you want these
\usepackage{amssymb}
\usepackage{amsmath}
\usepackage{amsthm}
\usepackage{amsfonts}

% used for TeXing text within eps files
%\usepackage{psfrag}
% need this for including graphics (\includegraphics)
%\usepackage{graphicx}
% for neatly defining theorems and propositions
%\usepackage{amsthm}
% making logically defined graphics
%%%\usepackage{xypic}

% there are many more packages, add them here as you need them

% define commands here

\newtheorem{thm}{Theorem}
\newtheorem{defn}{Definition}
\newtheorem{prop}{Proposition}
\newtheorem{lemma}{Lemma}
\newtheorem{cor}{Corollary}

% Some sets
\newcommand{\Nats}{\mathbb{N}}
\newcommand{\Ints}{\mathbb{Z}}
\newcommand{\Reals}{\mathbb{R}}
\newcommand{\Complex}{\mathbb{C}}
\newcommand{\Rats}{\mathbb{Q}}
\begin{document}
\begin{thm}
Let $L/K$ be a finite field extension. Then $L/K$ is an algebraic
extension.
\end{thm}

\begin{proof}
In order to prove that $L/K$ is an algebraic extension, we need to show that any element
$\alpha\in L$ is algebraic, i.e., there exists a non-zero
polynomial $p(x)\in K[x]$ such that $p(\alpha)=0$.

Recall that $L/K$ is a finite extension of fields, by definition,
it means that $L$ is a finite dimensional vector space over $K$.
Let the dimension be
$$[L\colon K]=n$$
for some $n\in \Nats$.

Consider the following set of ``vectors'' in $L$:
$$\mathcal{S}=\{ 1, \alpha, \alpha^2,\alpha^3,\ldots,\alpha^n\}$$
Note that the cardinality of $S$ is $n+1$, one more than the
dimension of the vector space. Therefore, the elements of $S$ must
be linearly dependent over $K$, otherwise the dimension of $S$
would be greater than $n$. Hence, there exist $k_i\in K,\ 0\leq i
\leq n$, not all zero, such that
$$k_0+k_1\alpha+k_2\alpha^2+k_3\alpha^3+\ldots+k_n\alpha^n=0$$
Thus, if we define
$$p(X)=k_0+k_1X+k_2X^2+k_3X^3+\ldots+k_nX^n$$
then $p(X)\in K[X]$ and $p(\alpha)=0$, as desired.

\end{proof}

{\bf NOTE: } The converse is not true. See the entry ``algebraic
extension'' for details.
%%%%%
%%%%%
\end{document}
