\documentclass[12pt]{article}
\usepackage{pmmeta}
\pmcanonicalname{ThinAlgebraicSet}
\pmcreated{2013-03-22 15:14:13}
\pmmodified{2013-03-22 15:14:13}
\pmowner{alozano}{2414}
\pmmodifier{alozano}{2414}
\pmtitle{thin algebraic set}
\pmrecord{5}{37009}
\pmprivacy{1}
\pmauthor{alozano}{2414}
\pmtype{Definition}
\pmcomment{trigger rebuild}
\pmclassification{msc}{12E25}
\pmsynonym{thin set}{ThinAlgebraicSet}
\pmsynonym{mince set}{ThinAlgebraicSet}

\endmetadata

% this is the default PlanetMath preamble.  as your knowledge
% of TeX increases, you will probably want to edit this, but
% it should be fine as is for beginners.

% almost certainly you want these
\usepackage{amssymb}
\usepackage{amsmath}
\usepackage{amsthm}
\usepackage{amsfonts}

% used for TeXing text within eps files
%\usepackage{psfrag}
% need this for including graphics (\includegraphics)
%\usepackage{graphicx}
% for neatly defining theorems and propositions
%\usepackage{amsthm}
% making logically defined graphics
%%%\usepackage{xypic}

% there are many more packages, add them here as you need them

% define commands here

\newtheorem{thm}{Theorem}
\newtheorem{defn}{Definition}
\newtheorem{prop}{Proposition}
\newtheorem{lemma}{Lemma}
\newtheorem{cor}{Corollary}

\theoremstyle{definition}
\newtheorem*{exa}{Example}

% Some sets
\newcommand{\Nats}{\mathbb{N}}
\newcommand{\Ints}{\mathbb{Z}}
\newcommand{\Reals}{\mathbb{R}}
\newcommand{\Complex}{\mathbb{C}}
\newcommand{\Rats}{\mathbb{Q}}
\newcommand{\Gal}{\operatorname{Gal}}
\newcommand{\Cl}{\operatorname{Cl}}
\begin{document}
\begin{defn}
Let $V$ be an irreducible algebraic variety (we assume it to be integral and quasi-projective) over a field $K$ with characteristic zero. We regard $V$ as a topological space with the usual Zariski topology.
\begin{enumerate}
\item A subset $A\subset V(K)$ is said to be of type $C_1$ if there is a closed subset $W\subset V$, with $W\neq V$, such that $A\subset W(K)$. In other words, $A$ is not dense in $V$ (with respect to the Zariski topology).

\item A subset $A\subset V(K)$ is said to be of type $C_2$ if there is an irreducible variety $V'$ of the same dimension as $V$, and a (generically) surjective algebraic morphism $\phi\colon V'\to V$ of degree $\geq 2$, with $A\subset \phi(V'(K))$
\end{enumerate}
\end{defn}

\begin{exa}
Let $K$ be a field and let $V(K)=\mathbb{A}(K)=\mathbb{A}^1(K)=K$ be the $1$-dimensional affine space. Then, the only Zariski-closed subsets of $V$ are finite subsets of points. Thus, the only subsets of type $C_1$ are subsets formed by a finite number of points. 

Let $V'(K)=\mathbb{A}(K)$ be affine space and define:
$$\phi\colon V' \to V$$
by $\phi(k)=k^2$. Then $\deg(\phi)=2$. Thus, the subset:
$$A=\{k^2: k \in \mathbb{A}(K)\}$$
, i.e. $A$ is the subset of perfect squares in $K$, is a subset of type $C_2$.
\end{exa}

\begin{defn}
A subset $A$ of an irreducible variety $V/K$ is said to be a thin algebraic set (or thin set, or ``mince'' set) if it is a union of a finite number of subsets of type $C_1$ and type $C_2$.
\end{defn}

\begin{thebibliography}{9}
\bibitem{serre} J.-P. Serre, {\em Topics in Galois Theory},
Research Notes in Mathematics, Jones and Barlett Publishers, London.
\end{thebibliography}
%%%%%
%%%%%
\end{document}
