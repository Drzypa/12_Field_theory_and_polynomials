\documentclass[12pt]{article}
\usepackage{pmmeta}
\pmcanonicalname{Monomial}
\pmcreated{2013-03-22 12:34:32}
\pmmodified{2013-03-22 12:34:32}
\pmowner{bbukh}{348}
\pmmodifier{bbukh}{348}
\pmtitle{monomial}
\pmrecord{5}{32824}
\pmprivacy{1}
\pmauthor{bbukh}{348}
\pmtype{Definition}
\pmcomment{trigger rebuild}
\pmclassification{msc}{12-00}
\pmdefines{degree of a monomial}

\endmetadata

\usepackage{amssymb}
\usepackage{amsmath}
\usepackage{amsfonts}
\begin{document}
A \emph{monomial} is a product of non-negative powers of variables.  It may also include an optional coefficient (which is sometimes ignored when discussing particular properties of monomials).  A polynomial can be thought of as a sum over a set of monomials.

For example, the following are monomials.

$$
\begin{array}{ccc}
1 & x & x^2y \\
\\
xyz & 3x^4y^2z^3 & -z
\end{array}
$$

If there are $n$ variables from which a monomial may be formed, then
a monomial may be represented without its coefficient as a vector of $n$
naturals.  Each position in this vector would correspond to a particular
variable, and the value of the element at each position would correspond
to the power of that variable in the monomial.  For instance, the monomial
$x^2yz^3$ formed from the set of variables $\left\{ w, x, y, z \right\}$
would be represented as $\begin{pmatrix}0&2&1&3\end{pmatrix}^T$.  A constant would be a zero vector.

Given this representation, we may define a few more concepts.  First, the
\emph{degree of a monomial} is the sum of the elements of its vector representation.  Thus, the degree of $x^2yz^3$ is $0 + 2 + 1 + 3 = 6$,
and the degree of a constant is 0.  If a polynomial is represented as a sum
over a set of monomials, then the degree of a polynomial can be defined as the
degree of the monomial of largest degree belonging to that polynomial.
%%%%%
%%%%%
\end{document}
