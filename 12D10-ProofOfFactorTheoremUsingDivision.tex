\documentclass[12pt]{article}
\usepackage{pmmeta}
\pmcanonicalname{ProofOfFactorTheoremUsingDivision}
\pmcreated{2013-03-22 15:08:58}
\pmmodified{2013-03-22 15:08:58}
\pmowner{alozano}{2414}
\pmmodifier{alozano}{2414}
\pmtitle{proof of factor theorem using division}
\pmrecord{8}{36896}
\pmprivacy{1}
\pmauthor{alozano}{2414}
\pmtype{Proof}
\pmcomment{trigger rebuild}
\pmclassification{msc}{12D10}
\pmclassification{msc}{12D05}

% this is the default PlanetMath preamble.  as your knowledge
% of TeX increases, you will probably want to edit this, but
% it should be fine as is for beginners.

% almost certainly you want these
\usepackage{amssymb}
\usepackage{amsmath}
\usepackage{amsthm}
\usepackage{amsfonts}

% used for TeXing text within eps files
%\usepackage{psfrag}
% need this for including graphics (\includegraphics)
%\usepackage{graphicx}
% for neatly defining theorems and propositions
%\usepackage{amsthm}
% making logically defined graphics
%%%\usepackage{xypic}

% there are many more packages, add them here as you need them

% define commands here

\newtheorem{thm}{Theorem}
\newtheorem{defn}{Definition}
\newtheorem{prop}{Proposition}
\newtheorem*{lemma}{Lemma}
\newtheorem{cor}{Corollary}

\theoremstyle{definition}
\newtheorem{exa}{Example}

% Some sets
\newcommand{\Nats}{\mathbb{N}}
\newcommand{\Ints}{\mathbb{Z}}
\newcommand{\Reals}{\mathbb{R}}
\newcommand{\Complex}{\mathbb{C}}
\newcommand{\Rats}{\mathbb{Q}}
\newcommand{\Gal}{\operatorname{Gal}}
\newcommand{\Cl}{\operatorname{Cl}}
\begin{document}
\begin{lemma}[cf. factor theorem]
Let $R$ be a commutative ring with identity and let $p(x)\in R[x]$ be a polynomial with coefficients in $R$. The element $a\in R$ is a root of $p(x)$ if and only if $(x-a)$ divides $p(x)$.
\end{lemma}
\begin{proof}
Let $p(x)$ be a polynomial in $R[x]$ and let $a$ be an element of $R$.
\begin{enumerate}
\item First we assume that $(x-a)$ divides $p(x)$. Therefore, there is a polynomial $q(x)\in R[x]$ such that $p(x)=(x-a)\cdot q(x)$. Hence, $p(a)=(a-a)\cdot q(a)=0$ and $a$ is a root of $p(x)$.\\

\item Assume that $a$ is a root of $p(x)$, i.e. $p(a)=0$. Since $x-a$ is a monic polynomial, we can perform the \PMlinkname{polynomial long division}{LongDivision} of $p(x)$ by $(x-a)$. Thus, there exist polynomials $q(x)$ and $r(x)$ such that:
$$p(x)=(x-a)\cdot q(x) + r(x)$$
and the degree of $r(x)$ is less than the degree of $x-a$ (so $r(x)$ is just a constant). Moreover, $0=p(a)=0+r(a)=r(a)=r(x)$. Therefore $p(x)=(x-a)\cdot q(x)$ and $(x-a)$ divides $p(x)$.
\end{enumerate}
\end{proof}
%%%%%
%%%%%
\end{document}
