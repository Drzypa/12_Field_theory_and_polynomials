\documentclass[12pt]{article}
\usepackage{pmmeta}
\pmcanonicalname{GaloistheoreticDerivationOfTheQuarticFormula}
\pmcreated{2013-03-22 12:34:35}
\pmmodified{2013-03-22 12:34:35}
\pmowner{djao}{24}
\pmmodifier{djao}{24}
\pmtitle{Galois-theoretic derivation of the quartic formula}
\pmrecord{9}{32825}
\pmprivacy{1}
\pmauthor{djao}{24}
\pmtype{Proof}
\pmcomment{trigger rebuild}
\pmclassification{msc}{12D10}
\pmrelated{GaloisTheoreticDerivationOfTheCubicFormula}
\pmdefines{resolvent cubic}

\endmetadata

% this is the default PlanetMath preamble.  as your knowledge
% of TeX increases, you will probably want to edit this, but
% it should be fine as is for beginners.

% almost certainly you want these
\usepackage{amssymb}
\usepackage{amsmath}
\usepackage{amsfonts}

% used for TeXing text within eps files
%\usepackage{psfrag}
% need this for including graphics (\includegraphics)
%\usepackage{graphicx}
% for neatly defining theorems and propositions
%\usepackage{amsthm}
% making logically defined graphics
%%%\usepackage{xypic} 

% there are many more packages, add them here as you need them

% define commands here
\usepackage[matrix,arrow]{xypic}
\newcommand{\C}{\mathbb{C}}
\newcommand{\Z}{\mathbb{Z}}
\newcommand{\Gal}{\operatorname{Gal}}
\begin{document}
Let $x^4 + ax^3 + bx^2 + cx + d$ be a general polynomial with four roots
$r_1,r_2,r_3,r_4$, so $(x-r_1)(x-r_2)(x-r_3)(x-r_4) = x^4 + ax^3 +
bx^2 + cx + d$. The goal is to exhibit the field extension
$\C(r_1,r_2,r_3,r_4)/\C(a,b,c,d)$ as a radical extension, thereby
expressing $r_1,r_2,r_3,r_4$ in terms of $a,b,c,d$ by radicals.

Write $N$ for $\C(r_1,r_2,r_3,r_4)$ and $F$ for $\C(a,b,c,d)$. The
Galois group $\Gal(N/F)$ is the symmetric group $S_4$, the permutation group on the four elements $\{r_1,r_2,r_3,r_4\}$, which has a composition series
$$
1 \lhd \Z/2 \lhd V_4 \lhd A_4 \lhd S_4,
$$
where:
\begin{itemize}
\item $A_4$ is the alternating group in $S_4$, consisting of the even
  permutations.
\item $V_4 = \{1, (12)(34), (13)(24), (14)(23)\}$ is the Klein four-group.
\item $\Z/2$ is the two--element subgroup $\{1, (12)(34)\}$ of $V_4$.
\end{itemize}

Under the Galois correspondence, each of these subgroups corresponds
to an intermediate field of the extension $N/F$. We denote these fixed
fields by  (in increasing order) $K$, $L$, and $M$.

We thus have a tower of field extensions, and corresponding
automorphism groups:
$$
\xymatrix{
\text{Subgroup} & \text{Fixed field} \\
1 & N \ar@{-}[d] \\
\Z/2 & M \ar@{-}[d] \\
V & L \ar@{-}[d] \\
A_4 & K \ar@{-}[d] \\
S_4 & F
}
$$

By Galois theory, or Kummer theory, each field in this diagram is a radical
extension of the one below it, and our job is done if we explicitly
find what the radical extension is in each case.

We start with $K/F$. The index of $A_4$ in $S_4$ is two, so $K/F$ is a
degree two extension. We have to find an element of $K$ that is not in
$F$. The easiest such element to take is the element $\Delta$ obtained by taking the products of the differences of the roots, namely,
$$
\Delta := \prod_{1 \leq i < j \leq 4} (r_i - r_j) = (r_1-r_2)
(r_1-r_3) (r_1-r_4) (r_2-r_3) (r_2-r_4) (r_3-r_4).
$$
Observe that $\Delta$ is fixed by any even permutation of the roots
$r_i$, but that $\sigma(\Delta) = -\Delta$ for any odd permutation
$\sigma$. Accordingly, $\Delta^2$ is actually
fixed by all of $S_4$, so:
\begin{itemize}
\item $\Delta \in K$, but $\Delta \notin F$.
\item $\Delta^2 \in F$.
\item $K = F[\Delta]$ = $F[\sqrt{\Delta^2}]$, thus exhibiting $K/F$ as
  a radical extension.
\end{itemize}

The element $\Delta^2 \in F$ is called the {\em discriminant} of the polynomial. An explicit formula for $\Delta^2$ can be found using the reduction algorithm for symmetric polynomials, and, although it is not needed for our purposes, we list it here for reference:
\begin{eqnarray*}
\Delta^2 & = & 256 d^3 - d^2 (27 a^4 - 144 a^2 b + 128 b^2 + 192 a c) - \\
& & c^2 (27c^2 - 18 abc + 4a^3 c + 4 b^3 - a^2 b^2) - \\
& & 2d (abc (9a^2 - 40 b) - 2b^3 (a^2-4b) - 3c^2 (a^2-24b)).
\end{eqnarray*}

Next up is the extension $L/K$, which has degree 3 since
$[A_4:V_4]=3$. We have to find an element of $N$ which is fixed by
$V_4$ but not by $A_4$. Luckily, the form of $V_4$ almost cries out
that the following elements be used:
\begin{eqnarray*}
t_1 & := & (r_1+r_2)(r_3+r_4) \\
t_2 & := & (r_1+r_3)(r_2+r_4) \\
t_3 & := & (r_1+r_4)(r_2+r_3)
\end{eqnarray*}

These three elements of $N$ are fixed by everything in $V_4$, but not
by everything in $A_4$. They are therefore elements of $L$ that are
not in $K$. Moreover, every permutation in $S_4$ permutes the set
$\{t_1, t_2, t_3\}$, so the cubic polynomial
$$
\Phi(x) := (x-t_1) (x-t_2) (x-t_3)
$$
actually has coefficients in $F$!  In fancier language, the cubic
polynomial $\Phi(x)$ defines a cubic extension $E$ of $F$ which is
linearly disjoint from $K$, with the composite extension $EK$ equal to
$L$. The polynomial $\Phi(x)$ is called the {\em resolvent cubic} of
the quartic polynomial $x^4 + ax^3 + bx^2 + cx + d$. The coefficients
of $\Phi(x)$ can be found fairly easily using (again) the reduction
algorithm for symmetric polynomials, which yields
\begin{equation}\label{resolvent}
\Phi(x) = x^3 - 2b x^2 + (b^2+ac-4d) x + (c^2+a^2d - abc).
\end{equation}
Using the cubic formula, one can find radical expressions for the
three roots of this polynomial, which are $t_1$, $t_2$, and $t_3$, and
henceforth we assume radical expressions for these three quantities
are known. We also have $L = K[t_1]$, which in light of what we just
said, exhibits $L/K$ as an explicit radical extension.

The remaining extensions are easier and the reader who has followed to
this point should have no trouble with the rest. For the degree two
extension $M/L$, we require an element of $M$ that is not in $L$; one
convenient such element is $r_1+r_2$, which is a root of the quadratic
polynomial
\begin{equation}\label{step}
(x-(r_1+r_2)) (x-(r_3+r_4)) = x^2 + a x + t_1 \in L[x]
\end{equation}
and therefore equals $(-a + \sqrt{a^2 - 4 t_1})/2$. Hence $M =
L[r_1+r_2] = L[(-a + \sqrt{a^2 - 4 t_1})/2]$ is a radical extension of
$L$.

Finally, for the extension $N/M$, an element of $N$ that is not in $M$
is of course $r_1$, which is a root of the quadratic polynomial
\begin{equation}\label{final}
(x-r_1)(x-r_2) = x^2 - (r_1+r_2) x + r_1 r_2.
\end{equation}
Now, $r_1+r_2$ is known from the previous paragraph, so it remains to
find an expression for $r_1 r_2$. Note that $r_1 r_2$ is fixed by
$(12)(34)$, so it is in $M$ but not in $L$. To find it, use the
equation $(t_2 + t_3 - t_1)/2 = r_1 r_2 + r_3 r_4$, which gives
$$
(x - r_1 r_2) (x - r_3 r_4) = x^2 - \frac{(t_2 + t_3 - t_1)}{2} x + d
$$
and, upon solving for $r_1 r_2$ with the quadratic formula, yields
\begin{eqnarray}
\label{post}
r_1 r_2 & = & \frac{(t_2 + t_3 - t_1) + \sqrt{(t_2 + t_3 - t_1)^2 - 16d}}{4} \\
\label{post2}
r_3 r_4 & = & \frac{(t_2 + t_3 - t_1) - \sqrt{(t_2 + t_3 - t_1)^2 - 16d}}{4}
\end{eqnarray}
We can then use this expression, combined with Equation~\eqref{final},
to solve for $r_1$ using the quadratic formula. Perhaps, at this
point, our poor reader needs a summary of the procedure, so we give one
here:
\begin{enumerate}
\item Find $t_1$, $t_2$, and $t_3$ by solving the resolvent cubic
  (Equation~\eqref{resolvent}) using the cubic formula,
\item From Equation~\eqref{step}, obtain
\begin{eqnarray*}
r_1+r_2 & = & \frac{(-a + \sqrt{a^2 - 4 t_1})}{2} \\
r_3+r_4 & = & \frac{(-a - \sqrt{a^2 - 4 t_1})}{2}
\end{eqnarray*}
\item Using Equation~\eqref{final}, write
\begin{eqnarray*}
r_1 & = & \frac{(r_1+r_2) + \sqrt{(r_1+r_2)^2 - 4 (r_1 r_2)}}{2} \\
r_2 & = & \frac{(r_1+r_2) - \sqrt{(r_1+r_2)^2 - 4 (r_1 r_2)}}{2} \\
r_3 & = & \frac{(r_3+r_4) + \sqrt{(r_3+r_4)^2 - 4 (r_3 r_4)}}{2} \\
r_4 & = & \frac{(r_3+r_4) - \sqrt{(r_3+r_4)^2 - 4 (r_3 r_4)}}{2} \\
\end{eqnarray*}
where the expressions $r_1+r_2$ and $r_3+r_4$ are derived in the previous step, and the expressions $r_1 r_2$ and $r_3 r_4$ come from Equation~\eqref{post} and~\eqref{post2}.
\item Now the roots $r_1,r_2,r_3,r_4$ of the quartic polynomial $x^4 + ax^3 + bx^2 + cx + d$ have been found, and we are done!
\end{enumerate}
%%%%%
%%%%%
\end{document}
