\documentclass[12pt]{article}
\usepackage{pmmeta}
\pmcanonicalname{RulesOfCalculusForDerivativeOfFormalPowerSeries}
\pmcreated{2013-03-22 18:22:35}
\pmmodified{2013-03-22 18:22:35}
\pmowner{rspuzio}{6075}
\pmmodifier{rspuzio}{6075}
\pmtitle{rules of calculus for derivative of formal power series}
\pmrecord{8}{41018}
\pmprivacy{1}
\pmauthor{rspuzio}{6075}
\pmtype{Derivation}
\pmcomment{trigger rebuild}
\pmclassification{msc}{12E05}
\pmclassification{msc}{11C08}
\pmclassification{msc}{13P05}
\pmrelated{InvertibleFormalPowerSeries}

\endmetadata

% this is the default PlanetMath preamble.  as your knowledge
% of TeX increases, you will probably want to edit this, but
% it should be fine as is for beginners.

% almost certainly you want these
\usepackage{amssymb}
\usepackage{amsmath}
\usepackage{amsfonts}

% used for TeXing text within eps files
%\usepackage{psfrag}
% need this for including graphics (\includegraphics)
%\usepackage{graphicx}
% for neatly defining theorems and propositions
\usepackage{amsthm}
% making logically defined graphics
%%%\usepackage{xypic}

% there are many more packages, add them here as you need them

% define commands here
\newtheorem{thm}{Theorem}
\begin{document}
In this entry, we will show that the rules of calculus
hold for derivatives of formal power series.  While
this could be verified directly in a manner analogous 
to what was done for polynomials in the parent entry,
we will take a different tack, deriving the results
for power series from the corresponding results for
polynomials.  The basis for our approach is the 
observation that the ring of formal power series can
be expressed as a limit of quotients of the ring of
polynomials:
\[
 A[[x]]  = \lim_{k \to \infty} A[x] / \langle x^k \rangle
\]
Thus, we will proceed in two steps, first extending the 
derivative operation to the quotient rings and showing that 
its properties still hold there, then extending it to the 
limit and showing that its properties hold there as well.

We begin by noting that the derivative is well-defined 
as a map from $A[x] / \langle x^{k+1} \rangle$ to 
$A[x] / \langle x^k \rangle$ for all integers $k \ge 0$.  

\begin{thm}
Suppose that $A$ is a commutative ring, $k$ is a non-negative
integer, and that $p$ and $q$ are elements of $A[x]$ such
that $p \equiv q$ modulo $x^{k+1}$.  Then $p' \equiv q'$
modulo $x^k$.
\end{thm}

\begin{proof}
By definition of congruence, $p(x) = q(x) + x^{k+1} r(x)$ for 
some polynomial $r \in A[x]$.  Taking derivaitves, $p'(x) = 
q'(x) + x^k (k r(x) + x r'(x))$, so $p'$ and $q'$ are 
equivalent modulo $x^k$.
\end{proof}

It is easy to verify that the sum and product rules hold
in this new context:

\begin{thm}
If $A$ is a commutative ring, $k$ is a non-negative integer,
and $f,g$ are elements of $A[x] / \langle x^{k+1} \rangle$,
then $(f+g)' = f' + g'$.
\end{thm}

\begin{proof}
Let $p,q$ be representatives of the equivalence classes
$f,g$.  Then we have $(p+q)' = p' + q'$ by the corresponding
theorem for polynomials.  Hence, by definition of quotient,
we have $(f + g)' = f' + g'$.
\end{proof}

\begin{thm}
If $A$ is a commutative ring, $k$ is a non-negative integer,
and $f,g$ are elements of $A[x] / \langle x^{k+1} \rangle$,
then $(f \cdot g)' = f' \cdot g + f \cdot g'$.
\end{thm}

\begin{proof}
Let $p,q$ be representatives of the equivalence classes
$f,g$.  Then we have $(p \cdot q)' = p' \cdot q + p \cdot q'$ 
by the corresponding theorem for polynomials.  Hence, by 
definition of quotient, we have 
$(f \cdot g)' = f' \cdot g + f \cdot g'$.
\end{proof}

When considering the chain rule, we need to note that composition
does not always pass to the quotient, so we need to restrict the
operands to obtain a well-defined operation.  In particular, we
will consider the following two cases:

\begin{thm}
If $A$ is a commutative ring, $p,q,r$ is a n element of $A[x]$,
and $q \equiv r$ modulo $x^k$ for some integer $k > 0$, then
$p \circ q \equiv p \circ r$ modulo $x^k$.
\end{thm}

\begin{thm}
If $A$ is a commutative ring, $k$ is a non-negative integer, and
$p,q,P,Q$ are elements of $A[x]$ such that $p \equiv P$ modulo 
$x^k$, $q \equiv Q$ modulo $x^k$ and $p(0) = 0$, then $p \circ q
\equiv P \circ Q$ modulo $x^k$.
\end{thm}

[More to come]
%%%%%
%%%%%
\end{document}
