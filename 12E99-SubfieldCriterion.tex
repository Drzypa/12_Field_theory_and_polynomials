\documentclass[12pt]{article}
\usepackage{pmmeta}
\pmcanonicalname{SubfieldCriterion}
\pmcreated{2013-03-22 16:26:34}
\pmmodified{2013-03-22 16:26:34}
\pmowner{pahio}{2872}
\pmmodifier{pahio}{2872}
\pmtitle{subfield criterion}
\pmrecord{7}{38598}
\pmprivacy{1}
\pmauthor{pahio}{2872}
\pmtype{Theorem}
\pmcomment{trigger rebuild}
\pmclassification{msc}{12E99}
\pmclassification{msc}{12E15}
\pmrelated{FieldOfAlgebraicNumbers}

\endmetadata

% this is the default PlanetMath preamble.  as your knowledge
% of TeX increases, you will probably want to edit this, but
% it should be fine as is for beginners.

% almost certainly you want these
\usepackage{amssymb}
\usepackage{amsmath}
\usepackage{amsfonts}

% used for TeXing text within eps files
%\usepackage{psfrag}
% need this for including graphics (\includegraphics)
%\usepackage{graphicx}
% for neatly defining theorems and propositions
 \usepackage{amsthm}
% making logically defined graphics
%%%\usepackage{xypic}

% there are many more packages, add them here as you need them

% define commands here

\theoremstyle{definition}
\newtheorem*{thmplain}{Theorem}

\begin{document}
Let $K$ be a skew field and $S$ its subset.\, For $S$ to be a subfield of $K$, it's necessary and sufficient that the following three conditions are fulfilled:
\begin{enumerate}
\item $S$ \PMlinkescapetext{contains} a non-zero element of $K$.
\item $a\!-\!b \in S$ always when\, $a,\,b \in S$.
\item $ab^{-1} \in S$ always when\, $a,\,b \in S$\, and\, $b \neq 0$.
\end{enumerate}

{\em Proof.}\, Because the conditions are fulfilled in every skew field, they are necessary.\, For proving the sufficience, suppose now that the subset $S$ \PMlinkescapetext{satisfies} these conditions.\, The condition 1 guarantees that $S$ is not empty and the condition 2 that\, $(S,\, +)$\, is an subgroup of\, $(K,\, +)$;\, thus all the required properties of addition for a skew field hold in $S$.\, If $b$ is a non-zero element of $S$, then, according to the condition 3, we have\, $0 \neq 1 = bb^{-1} \in S$.\, Moreover,\, $a\!\cdot\!1 = 1\!\cdot a = a \in S$\, for all\, $a \in S \subseteq K$.\, The laws of multiplication (associativity and left and \PMlinkescapetext{right} distributivity over addition) hold in $S$ since they hold in whole $K$.\, So $S$ fulfils all the postulates for a skew field.
%%%%%
%%%%%
\end{document}
