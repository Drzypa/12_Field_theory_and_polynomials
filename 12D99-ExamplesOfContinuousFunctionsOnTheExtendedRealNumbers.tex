\documentclass[12pt]{article}
\usepackage{pmmeta}
\pmcanonicalname{ExamplesOfContinuousFunctionsOnTheExtendedRealNumbers}
\pmcreated{2013-03-22 16:59:34}
\pmmodified{2013-03-22 16:59:34}
\pmowner{Wkbj79}{1863}
\pmmodifier{Wkbj79}{1863}
\pmtitle{examples of continuous functions on the extended real numbers}
\pmrecord{9}{39273}
\pmprivacy{1}
\pmauthor{Wkbj79}{1863}
\pmtype{Example}
\pmcomment{trigger rebuild}
\pmclassification{msc}{12D99}
\pmclassification{msc}{28-00}

\endmetadata

\usepackage{amssymb}
\usepackage{amsmath}
\usepackage{amsfonts}

\usepackage{psfrag}
\usepackage{graphicx}
\usepackage{amsthm}
%%\usepackage{xypic}

\begin{document}
Within this entry, $\overline{\mathbb{R}}$ will be used to refer to the extended real numbers.

Examples of continuous functions on $\overline{\mathbb{R}}$ include:

\begin{itemize}

\item Polynomial functions:  Let $f \in \mathbb{R}[x]$ with $\displaystyle f(x)=\sum_{j=0}^n a_nx^n$ for some $n \in \mathbb{N}$ and $a_0, \ldots , a_n \in \mathbb{R}$ with $a_n \neq 0$ if $n \neq 0$.  Then $\overline{f}$ is defined in the following manner:

\begin{enumerate}

\item If $n=0$, then $\overline{f}(x)=a_0$ for all $x \in \overline{\mathbb{R}}$.

\item If $n$ is odd and $a_n>0$, then $\displaystyle \overline{f}(x)=\begin{cases}
f(x) & \text{ if } x \in \mathbb{R} \\
x & \text{ if } x \notin \mathbb{R}. \end{cases}$

\item If $n$ is odd and $a_n<0$, then $\displaystyle \overline{f}(x)=\begin{cases}
f(x) & \text{ if } x \in \mathbb{R} \\
-x & \text{ if } x \notin \mathbb{R}.  \end{cases}$

\item If $n \neq 0$ is even and $a_n>0$, then $\displaystyle \overline{f}(x)=\begin{cases}
f(x) & \text{ if } x \in \mathbb{R} \\
\infty & \text{ if } x \notin \mathbb{R}.  \end{cases}$

\item If $n \neq 0$ is even and $a_n<0$, then $\displaystyle \overline{f}(x)=\begin{cases}
f(x) & \text{ if } x \in \mathbb{R} \\
-\infty & \text{ if } x \notin \mathbb{R}.  \end{cases}$

\end{enumerate}

\item Exponential functions:  Let $f(x)=a^x$ for some $a \in \mathbb{R}$ with $a>0$ and $a \neq 1$.  Then $\overline{f}$ is defined in the following manner:

\begin{enumerate}

\item If $a<1$, then $\displaystyle \overline{f}(x)=\begin{cases}
f(x) & \text{ if } x \in \mathbb{R} \\
0 & \text{ if } x=\infty \\
\infty & \text{ if } x=-\infty.  \end{cases}$

\item If $a>1$, then $\displaystyle \overline{f}(x)=\begin{cases}
f(x) & \text{ if } x \in \mathbb{R} \\
\infty & \text{ if } x=\infty \\
0 & \text{ if } x=-\infty.  \end{cases}$

\end{enumerate}

\item Miscellaneous \PMlinkescapetext{functions}

\begin{enumerate}

\item Let $f(x)=\arctan x$.  Then $\overline{f}$ is defined by $\displaystyle \overline{f}(x)=\begin{cases}
f(x) & \text{ if } x \in \mathbb{R} \\
& \\
\displaystyle \frac{\pi}{2} & \text{ if } x=\infty \\
& \\
\displaystyle -\frac{\pi}{2} & \text{ if } x=-\infty.  \end{cases}$

\item Let $f(x)=\tanh x$.  Then $\overline{f}$ is defined by $\displaystyle \overline{f}(x)=\begin{cases}
f(x) & \text{ if } x \in \mathbb{R} \\
1 & \text{ if } x=\infty \\
-1 & \text{ if } x=-\infty.  \end{cases}$

\end{enumerate}
\end{itemize}

Of course, not every function $f$ that is continuous on $\mathbb{R}$ extends to a continuous function on $\overline{\mathbb{R}}$.  Common examples of these include the real functions $x \mapsto \sin x$ and $x \mapsto \cos x$.  (It is proven that these are continuous on $\mathbb{R}$ in the entry continuity of sine and cosine.)

On the other hand, there are some continuous functions $\overline{f} \colon \overline{\mathbb{R}} \to \overline{\mathbb{R}}$ that have no analogous function $f \colon \mathbb{R} \to \mathbb{R}$.  For example, consider

\begin{center}
$\overline{f}(x)=\begin{cases}
\displaystyle \frac{1}{x^2} & \text{ if } x \in \mathbb{R} \setminus \{0\} \\
\infty & \text{ if } x=0 \\
0 & \text{ if } x=\pm \infty. \end{cases}$
\end{center}
%%%%%
%%%%%
\end{document}
