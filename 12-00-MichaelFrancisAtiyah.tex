\documentclass[12pt]{article}
\usepackage{pmmeta}
\pmcanonicalname{MichaelFrancisAtiyah}
\pmcreated{2013-03-22 18:26:53}
\pmmodified{2013-03-22 18:26:53}
\pmowner{bci1}{20947}
\pmmodifier{bci1}{20947}
\pmtitle{Michael Francis Atiyah}
\pmrecord{16}{41109}
\pmprivacy{1}
\pmauthor{bci1}{20947}
\pmtype{Biography}
\pmcomment{trigger rebuild}
\pmclassification{msc}{12-00}
\pmclassification{msc}{11S70}
\pmclassification{msc}{11R70}
\pmclassification{msc}{00-02}
%\pmkeywords{K-theory}
%\pmkeywords{Atiyah's biography}
%\pmkeywords{Honors}
%\pmkeywords{British top mathematician}

% this is the default PlanetMath preamble.  as your knowledge
% of TeX increases, you will probably want to edit this, but
% it should be fine as is for beginners.

% almost certainly you want these
\usepackage{amssymb}
\usepackage{amsmath}
\usepackage{amsfonts}

% used for TeXing text within eps files
%\usepackage{psfrag}
% need this for including graphics (\includegraphics)
%\usepackage{graphicx}
% for neatly defining theorems and propositions
%\usepackage{amsthm}
% making logically defined graphics
%%%\usepackage{xypic}

% there are many more packages, add them here as you need them

% define commands here
\usepackage{amsmath, amssymb, amsfonts, amsthm, amscd, latexsym}
%%\usepackage{xypic}
\usepackage[mathscr]{eucal}

\setlength{\textwidth}{6.5in}
%\setlength{\textwidth}{16cm}
\setlength{\textheight}{9.0in}
%\setlength{\textheight}{24cm}

\hoffset=-.75in     %%ps format
%\hoffset=-1.0in     %%hp format
\voffset=-.4in

\theoremstyle{plain}
\newtheorem{lemma}{Lemma}[section]
\newtheorem{proposition}{Proposition}[section]
\newtheorem{theorem}{Theorem}[section]
\newtheorem{corollary}{Corollary}[section]

\theoremstyle{definition}
\newtheorem{definition}{Definition}[section]
\newtheorem{example}{Example}[section]
%\theoremstyle{remark}
\newtheorem{remark}{Remark}[section]
\newtheorem*{notation}{Notation}
\newtheorem*{claim}{Claim}

\renewcommand{\thefootnote}{\ensuremath{\fnsymbol{footnote%%@
}}}
\numberwithin{equation}{section}

\newcommand{\Ad}{{\rm Ad}}
\newcommand{\Aut}{{\rm Aut}}
\newcommand{\Cl}{{\rm Cl}}
\newcommand{\Co}{{\rm Co}}
\newcommand{\DES}{{\rm DES}}
\newcommand{\Diff}{{\rm Diff}}
\newcommand{\Dom}{{\rm Dom}}
\newcommand{\Hol}{{\rm Hol}}
\newcommand{\Mon}{{\rm Mon}}
\newcommand{\Hom}{{\rm Hom}}
\newcommand{\Ker}{{\rm Ker}}
\newcommand{\Ind}{{\rm Ind}}
\newcommand{\IM}{{\rm Im}}
\newcommand{\Is}{{\rm Is}}
\newcommand{\ID}{{\rm id}}
\newcommand{\GL}{{\rm GL}}
\newcommand{\Iso}{{\rm Iso}}
\newcommand{\Sem}{{\rm Sem}}
\newcommand{\St}{{\rm St}}
\newcommand{\Sym}{{\rm Sym}}
\newcommand{\SU}{{\rm SU}}
\newcommand{\Tor}{{\rm Tor}}
\newcommand{\U}{{\rm U}}

\newcommand{\A}{\mathcal A}
\newcommand{\Ce}{\mathcal C}
\newcommand{\D}{\mathcal D}
\newcommand{\E}{\mathcal E}
\newcommand{\F}{\mathcal F}
\newcommand{\G}{\mathcal G}
\newcommand{\Q}{\mathcal Q}
\newcommand{\R}{\mathcal R}
\newcommand{\cS}{\mathcal S}
\newcommand{\cU}{\mathcal U}
\newcommand{\W}{\mathcal W}

\newcommand{\bA}{\mathbb{A}}
\newcommand{\bB}{\mathbb{B}}
\newcommand{\bC}{\mathbb{C}}
\newcommand{\bD}{\mathbb{D}}
\newcommand{\bE}{\mathbb{E}}
\newcommand{\bF}{\mathbb{F}}
\newcommand{\bG}{\mathbb{G}}
\newcommand{\bK}{\mathbb{K}}
\newcommand{\bM}{\mathbb{M}}
\newcommand{\bN}{\mathbb{N}}
\newcommand{\bO}{\mathbb{O}}
\newcommand{\bP}{\mathbb{P}}
\newcommand{\bR}{\mathbb{R}}
\newcommand{\bV}{\mathbb{V}}
\newcommand{\bZ}{\mathbb{Z}}

\newcommand{\bfE}{\mathbf{E}}
\newcommand{\bfX}{\mathbf{X}}
\newcommand{\bfY}{\mathbf{Y}}
\newcommand{\bfZ}{\mathbf{Z}}

\renewcommand{\O}{\Omega}
\renewcommand{\o}{\omega}
\newcommand{\vp}{\varphi}
\newcommand{\vep}{\varepsilon}

\newcommand{\diag}{{\rm diag}}
\newcommand{\grp}{{\mathbb G}}
\newcommand{\dgrp}{{\mathbb D}}
\newcommand{\desp}{{\mathbb D^{\rm{es}}}}
\newcommand{\Geod}{{\rm Geod}}
\newcommand{\geod}{{\rm geod}}
\newcommand{\hgr}{{\mathbb H}}
\newcommand{\mgr}{{\mathbb M}}
\newcommand{\ob}{{\rm Ob}}
\newcommand{\obg}{{\rm Ob(\mathbb G)}}
\newcommand{\obgp}{{\rm Ob(\mathbb G')}}
\newcommand{\obh}{{\rm Ob(\mathbb H)}}
\newcommand{\Osmooth}{{\Omega^{\infty}(X,*)}}
\newcommand{\ghomotop}{{\rho_2^{\square}}}
\newcommand{\gcalp}{{\mathbb G(\mathcal P)}}

\newcommand{\rf}{{R_{\mathcal F}}}
\newcommand{\glob}{{\rm glob}}
\newcommand{\loc}{{\rm loc}}
\newcommand{\TOP}{{\rm TOP}}

\newcommand{\wti}{\widetilde}
\newcommand{\what}{\widehat}

\renewcommand{\a}{\alpha}
\newcommand{\be}{\beta}
\newcommand{\ga}{\gamma}
\newcommand{\Ga}{\Gamma}
\newcommand{\de}{\delta}
\newcommand{\del}{\partial}
\newcommand{\ka}{\kappa}
\newcommand{\si}{\sigma}
\newcommand{\ta}{\tau}
\newcommand{\med}{\medbreak}
\newcommand{\medn}{\medbreak \noindent}
\newcommand{\bign}{\bigbreak \noindent}
\newcommand{\lra}{{\longrightarrow}}
\newcommand{\ra}{{\rightarrow}}
\newcommand{\rat}{{\rightarrowtail}}
\newcommand{\oset}[1]{\overset {#1}{\ra}}
\newcommand{\osetl}[1]{\overset {#1}{\lra}}
\newcommand{\hr}{{\hookrightarrow}}
\begin{document}
\section{Michael Francis Atiyah}
British mathematician, \\
Born:  April 22nd, 1929 in London, UK

\subsection{Education:}
\begin{itemize}
\item High school education was partly at Victoria College, in Cairo, Egypt,
and partly in Manchester, UK,  at the Manchester Grammar School
\item BA, at Trinity College, in Cambridge, UK 
\item MA (doctorate) at Trinity College, in Cambridge, UK in 1954
\item Commonwealth Fellow at the Institute for Advanced Study in Princeton, USA, in 1955
\item College Lecturer in Cambridge, UK, in 1957 
\item Active Fellow of Pembroke College in Cambridge, UK, 1958-1961.
\item Reader at the University of Oxford and Fellow of St Catherine's College. 
\item Savilian Chair of Geometry at Oxford, 1963-1969
\item Professor of Mathematics at the Institute for Advanced Study in Princeton, USA in 1969
\item Royal Society Research Professor at Oxford in 1972
\end{itemize}

\subsection{Major Fields of Research: Topology, $K$-Theory and TQFT Foundations}
Quotations from the listed articles on this page:

\begin{itemize}
\item ``{\em Michael Atiyah has contributed to a wide range of topics in mathematics centering around the interaction between geometry and analysis. His first major contribution (in collaboration with F. Hirzebruch) was the development of a new and powerful technique in topology ($K$-theory) which led to the solution of many outstanding difficult problems. Subsequently (in collaboration with I. M. Singer) he established an important theorem dealing with the number of solutions of elliptic differential equations. This `index theorem' had antecedents in algebraic geometry and led to important new links between differential geometry, topology and analysis. Combined with considerations of symmetry it led (jointly with Raoul Bott) to a new and refined fixed point theorem' with wide applicability.};'' 
\item ``The K-theory and the index theorem are studied in Atiyah's book entitled `{\em $K$-theory}' (1967, reprinted 1989) and his joint work with G. B. Segal, ``The Index of Elliptic Operators I-V'' in {\em Annals of Mathematics}, volumes 88 and 93 (1968, 1971). Atiyah also described his work on the index theorem in ``The index of elliptic operators'' given as an {\em American Mathematical Society Colloquium Lecture} in 1973.''
\item The index theorem could be interpreted in terms of quantum theory and has proved a useful tool for theoretical physicists. Beyond these linear problems, gauge theories involved deep and interesting nonlinear differential equations. In particular, the Yang-Mills equations have turned out to be particularly fruitful for mathematicians. Atiyah initiated much of the early work in this field and his student Simon Donaldson went on to make spectacular use of these ideas in 4-dimensional geometry. More recently Atiyah has been influential in stressing the role of topology in quantum field theory''...(more specifically in TQFT)...`` and in bringing the work of theoretical physicists, notably E Witten, to the attention of the mathematical community.'' 
\item ``{\em The theories of superspace and supergravity and the string theory of fundamental particles, that involve the theory of Riemann surfaces in novel and unexpected ways, were all areas of theoretical physics which developed using''} ...some of the ideas which Atiyah introduced.
\end{itemize}

\subsection{Honours awarded to Michael Francis Atiyah:}
\begin{itemize}
\item Foreign member of several academies including those of:  the United States, Sweden, Germany, France, Ireland, India, Australia, China, Russia and the Ukraine.
\item Fellow of the Royal Society of London in 1962 at the age of 32
\item British Mathematical Colloquium morning speaker 1957, 1962  
\item LMS Berwick Prize winner 1961 
\item Speaker at International Congress 1966  
\item Fields Medal 1966
\item Royal Medal of the Society in 1968
\item American Mathematical Society Colloquium Lecturer in 1973
\item Royal Society's Bakerian Lecture on Global geometry in 1975
\item President of the London Mathematical Society in 1974-76
\item Honorary Fellow of the Edinburgh Maths Society 1979
\item De Morgan Medal in 1980
\item Feltrinelli Prize from the Accademia Nazionale dei Lincei in 1981
\item He was \emph{Knighted by Queen Elizabeth II in 1983}
\item Fellow of the Royal Society of Edinburgh 1985   
\item King Faisal International Prize for Science in 1987
\item Copley Medal in 1988 
\item AMS Gibbs Lecturer 1991  
\item Member of the Order of Merit in 1992
\item Hedrick lecturer 1993  
\item President of the Royal Society, 1990 to 1995
\item Abel Prize 2004 
\item Honorary Degrees from: Bonn, Warwick, Durham, St Andrews, Dublin, Chicago, Edinburgh, Cambridge, Essex, London, Sussex, Ghent, Reading, Helsinki, Leicester, Rutgers, Salamanca, Montreal, Waterloo, Wales, Queen's-Kingston, Keele, Birmingham, Lebanon and the Open University.
\end{itemize}



Suggested further reading: an article by \PMlinkexternal{J. J. O'Connor and E F Robertson}{http://www-groups.dcs.st-and.ac.uk/~history/Biographies/Atiyah.html} 


\subsection{Other Articles about M.F. Atiyah:}

M Atiyah, Address of the president, Sir Michael Atiyah, given at the anniversary meeting on 29 November 1991, Notes and Records Roy. Soc. London 46 (1) (1992), 155-169. 
M Atiyah, Address of the President, Sir Michael Atiyah, O. M., given at the anniversary meeting on 30 November 1994, Notes and Records Roy. Soc. London 49 (1) (1995), 141-151. 
Michael F Atiyah, in M Atiyah and D Iagolnitzer (eds.), Fields Medalists Lectures (Singapore, 1997), 113-114. 
H Cartan, L'oeuvre de Michael F Atiyah, Proceedings of the International Congress of Mathematicians, Moscow, 1966 (Moscow, 1968). 
E Getzler, The Atiyah-Bott fixed point formula, in Raoul Bott: collected papers 2 (Boston, MA, 1994), xxxi-xxxiii. 
R Minio, An interview with Michael Atiyah (Czech), Pokroky Mat. Fyz. Astronom. 31 (3) (1986), 154-168. 
R Minio, An interview with Michael Atiyah (Slovenian), Obzornik Mat. Fiz. 31 (5-6) (1984), 129-142. 
R Minio, An interview with Michael Atiyah, Math. Intelligencer 6 (1) (1984), 9-19. 
%%%%%
%%%%%
\end{document}
