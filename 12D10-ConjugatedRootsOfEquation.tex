\documentclass[12pt]{article}
\usepackage{pmmeta}
\pmcanonicalname{ConjugatedRootsOfEquation}
\pmcreated{2013-03-22 17:36:51}
\pmmodified{2013-03-22 17:36:51}
\pmowner{pahio}{2872}
\pmmodifier{pahio}{2872}
\pmtitle{conjugated roots of equation}
\pmrecord{7}{40032}
\pmprivacy{1}
\pmauthor{pahio}{2872}
\pmtype{Topic}
\pmcomment{trigger rebuild}
\pmclassification{msc}{12D10}
\pmclassification{msc}{30-00}
\pmclassification{msc}{12D99}
\pmsynonym{roots of algebraic equation with real coefficients}{ConjugatedRootsOfEquation}
\pmrelated{PartialFractionsOfExpressions}
\pmrelated{QuadraticFormula}
\pmrelated{ExampleOfSolvingACubicEquation}

% this is the default PlanetMath preamble.  as your knowledge
% of TeX increases, you will probably want to edit this, but
% it should be fine as is for beginners.

% almost certainly you want these
\usepackage{amssymb}
\usepackage{amsmath}
\usepackage{amsfonts}

% used for TeXing text within eps files
%\usepackage{psfrag}
% need this for including graphics (\includegraphics)
%\usepackage{graphicx}
% for neatly defining theorems and propositions
 \usepackage{amsthm}
% making logically defined graphics
%%%\usepackage{xypic}

% there are many more packages, add them here as you need them

% define commands here

\theoremstyle{definition}
\newtheorem*{thmplain}{Theorem}

\begin{document}
The rules
$$\overline{w_1+w_2} = \overline{w_1}+\overline{w_2} \quad\mbox{and}\quad \overline{w_1w_2} = \overline{w_1}\,\overline{w_2},$$
concerning the complex conjugates of the sum and product of two complex numbers, may be by induction generalised for arbitrary number of complex numbers $w_k$.  Since the complex conjugate of a real number is the same real number, we may write
$$\overline{a_kz^k} = a_k\overline{z}^k$$
for real numbers $a_k\,\, (k = 0,\,1,\,2,\,\ldots)$.  Thus, for a polynomial \,$P(x) := a_0x^n+a_1x^{n-1}+\ldots+a_n$\, we obtain
$$\overline{P(z)} = \overline{a_0z^n+a_1z^{n-1}+\ldots+a_n} = 
{a_0\overline{z}^n+a_1\overline{z}^{n-1}+\ldots+a_n} = P(\overline{z}).$$
I.e., the values of a polynomial with real coefficients computed at a complex number and its complex conjugate are 
complex conjugates of each other.

If especially the value of a polynomial with real coefficients vanishes at some complex number $z$, it vanishes also at $\overline{z}$.\, So the roots of an algebraic equation
$$P(x) = 0$$
with real coefficients are pairwise complex conjugate numbers.\\

\textbf{Example.}  The roots of the binomial equation
$$x^3\!-\!1 = 0$$
are\, $x = 1$,\, $x = \frac{-1\pm{i}\sqrt{3}}{2}$,\, the third roots of unity.
%%%%%
%%%%%
\end{document}
