\documentclass[12pt]{article}
\usepackage{pmmeta}
\pmcanonicalname{SomeValuesCharacterisingI}
\pmcreated{2013-03-22 18:31:29}
\pmmodified{2013-03-22 18:31:29}
\pmowner{pahio}{2872}
\pmmodifier{pahio}{2872}
\pmtitle{some values characterising i}
\pmrecord{7}{41222}
\pmprivacy{1}
\pmauthor{pahio}{2872}
\pmtype{Result}
\pmcomment{trigger rebuild}
\pmclassification{msc}{12D99}
%\pmkeywords{imaginary unit}
\pmrelated{GeneralPower}
\pmrelated{ComplexSineAndCosine}
\pmrelated{ComplexLogarithm}

\endmetadata

% this is the default PlanetMath preamble.  as your knowledge
% of TeX increases, you will probably want to edit this, but
% it should be fine as is for beginners.

% almost certainly you want these
\usepackage{amssymb}
\usepackage{amsmath}
\usepackage{amsfonts}

% used for TeXing text within eps files
%\usepackage{psfrag}
% need this for including graphics (\includegraphics)
%\usepackage{graphicx}
% for neatly defining theorems and propositions
 \usepackage{amsthm}
% making logically defined graphics
%%%\usepackage{xypic}

% there are many more packages, add them here as you need them

% define commands here

\theoremstyle{definition}
\newtheorem*{thmplain}{Theorem}

\begin{document}
\begin{itemize}
\item $\displaystyle i^{\,2} \;=\; -1$
\item $|i| \;=\; 1$
\item $\displaystyle\arg{i} \;=\; 2n\pi\!+\!\frac{\pi}{2}$ \quad ($n \in \mathbb{Z}$)
\item $\displaystyle\bar{i} \;=\; -i$
\item $\displaystyle\frac{1}{i} \;=\; -i$
\item $\displaystyle\sqrt{i} \;=\; \pm\frac{1\!+\!i}{\sqrt{2}}$
\item $\displaystyle (-1)^i \;=\; e^{(2n+1)\pi}$\;\, ($n \in \mathbb{Z}$)
\item $\displaystyle i^{\,i} \;=\; e^{2n\pi-\frac{\pi}{2}}$ \quad ($n \in \mathbb{Z}$)
\item $\displaystyle \cos{i} \;=\; \frac{1}{2}\left(e+\frac{1}{e}\right) \;\approx\; 1.54308$
\item $\displaystyle \sin{i} \;=\; \frac{i}{2}\left(e-\frac{1}{e}\right) \;\approx\; 1.17520\,i$
\item $\displaystyle \cosh{i} \;=\; \cos1 \;\approx\; 0.54030$
\item $\displaystyle \sinh{i} \;=\; i\,\sin1 \;\approx\; 0.84147\,i$
\item $\displaystyle e^i \;=\; \cos1+i\,\sin1$
\item $\displaystyle \log{i} = \left(2n\pi\!+\!\frac{\pi}{2}\right)i$ \quad ($n \in \mathbb{Z}$)

\end{itemize}
%%%%%
%%%%%
\end{document}
