\documentclass[12pt]{article}
\usepackage{pmmeta}
\pmcanonicalname{ExamplesOfTotallyRealFields}
\pmcreated{2013-03-22 13:55:05}
\pmmodified{2013-03-22 13:55:05}
\pmowner{alozano}{2414}
\pmmodifier{alozano}{2414}
\pmtitle{examples of totally real fields}
\pmrecord{6}{34675}
\pmprivacy{1}
\pmauthor{alozano}{2414}
\pmtype{Example}
\pmcomment{trigger rebuild}
\pmclassification{msc}{12D99}
%\pmkeywords{totally}
%\pmkeywords{real}
%\pmkeywords{imaginary}
%\pmkeywords{complex multiplication}
\pmrelated{TotallyRealAndImaginaryFields}
\pmrelated{NumberField}
\pmdefines{examples of totally imaginary fields}
\pmdefines{examples of CM-fields}

\endmetadata

% this is the default PlanetMath preamble.  as your knowledge
% of TeX increases, you will probably want to edit this, but
% it should be fine as is for beginners.

% almost certainly you want these
\usepackage{amssymb}
\usepackage{amsmath}
\usepackage{amsthm}
\usepackage{amsfonts}

% used for TeXing text within eps files
%\usepackage{psfrag}
% need this for including graphics (\includegraphics)
%\usepackage{graphicx}
% for neatly defining theorems and propositions
%\usepackage{amsthm}
% making logically defined graphics
%%%\usepackage{xypic}

% there are many more packages, add them here as you need them

% define commands here

\newtheorem{thm}{Theorem}
\newtheorem{defn}{Definition}
\newtheorem{prop}{Proposition}
\newtheorem{lemma}{Lemma}
\newtheorem{cor}{Corollary}

% Some sets
\newcommand{\Nats}{\mathbb{N}}
\newcommand{\Ints}{\mathbb{Z}}
\newcommand{\Reals}{\mathbb{R}}
\newcommand{\Complex}{\mathbb{C}}
\newcommand{\Rats}{\mathbb{Q}}
\begin{document}
Here we present examples of totally real fields, totally imaginary
fields and CM-fields.

{\bf Examples}:
\begin{enumerate}
\item Let $K=\Rats(\sqrt{d})$ with $d$ a square-free {\bf
positive} integer. Then
$$\Sigma_K=\{ \operatorname{Id}_K, \sigma\}$$
where $\operatorname{Id}_K\colon K \hookrightarrow \Complex $ is
the identity map ($\operatorname{Id}_K(k)=k$, for all $k\in K$),
whereas
$$\sigma\colon K \hookrightarrow \Complex,\quad
\sigma(a+b\sqrt{d})=a-b\sqrt{d}$$ Since $\sqrt{d}\in \Reals$ it
follows that $K$ is a totally real field.

\item Similarly, let $K=\Rats(\sqrt{d})$ with $d$ a square-free
{\bf negative} integer. Then
$$\Sigma_K=\{ \operatorname{Id}_K, \sigma\}$$
where $\operatorname{Id}_K\colon K \hookrightarrow \Complex $ is
the identity map ($\operatorname{Id}_K(k)=k$, for all $k\in K$),
whereas
$$\sigma\colon K \hookrightarrow \Complex,\quad
\sigma(a+b\sqrt{d})=a-b\sqrt{d}$$ Since $\sqrt{d}\in \Complex$ and
it is not in $\Reals$, it follows that $K$ is a totally imaginary
field.

\item Let $\zeta_n, n\geq 3$, be a primitive $n^{th}$ root of
unity and let $L=\Rats(\zeta_n)$, a cyclotomic extension. Note
that the only roots of unity that are real are $\pm 1$. If
$\psi\colon L \hookrightarrow \Complex $ is an embedding, then
$\psi(\zeta_n)$ must be a conjugate of $\zeta_n$, i.e. one of $$\{
\zeta_n^a \mid a \in (\Ints/n\Ints)^{\times}\}$$ but those are all
imaginary. Thus $\psi(L)\nsubseteq \Reals$. Hence $L$ is a totally
imaginary field.

\item In fact, $L$ as in $(3)$ is a CM-field. Indeed, the maximal
real subfield of $L$ is $$F=\Rats(\zeta_n + \zeta_n^{-1})$$ Notice
that the minimal polynomial of $\zeta_n$ over $F$ is
$$X^2-(\zeta_n+\zeta_n^{-1})X+1$$
so we obtain $L$ from $F$ by adjoining the square root of the
discriminant of this polynomial which is
$$\zeta_n^2+\zeta_n^{-2}-2= 2\cos(\frac{4\pi}{n})-2 < 0$$
and any other conjugate is
$$\zeta_n^{2a}+\zeta_n^{-2a}-2=2\cos(\frac{4a\pi}{n})-2 < 0, a\in
(\Ints/n\Ints)^{\times}$$ Hence, $L$ is a CM-field.

\item Notice that any quadratic imaginary number field is
obviously a CM-field.
\end{enumerate}
%%%%%
%%%%%
\end{document}
