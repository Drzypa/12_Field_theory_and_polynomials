\documentclass[12pt]{article}
\usepackage{pmmeta}
\pmcanonicalname{RealAndComplexEmbeddings}
\pmcreated{2013-03-22 13:54:43}
\pmmodified{2013-03-22 13:54:43}
\pmowner{alozano}{2414}
\pmmodifier{alozano}{2414}
\pmtitle{real and complex embeddings}
\pmrecord{4}{34666}
\pmprivacy{1}
\pmauthor{alozano}{2414}
\pmtype{Definition}
\pmcomment{trigger rebuild}
\pmclassification{msc}{12D99}
\pmrelated{GaloisGroup}
\pmrelated{TotallyRealAndImaginaryFields}
\pmrelated{RamificationOfArchimedeanPlaces}
\pmdefines{real embedding}
\pmdefines{complex embedding}

\endmetadata

% this is the default PlanetMath preamble.  as your knowledge
% of TeX increases, you will probably want to edit this, but
% it should be fine as is for beginners.

% almost certainly you want these
\usepackage{amssymb}
\usepackage{amsmath}
\usepackage{amsthm}
\usepackage{amsfonts}

% used for TeXing text within eps files
%\usepackage{psfrag}
% need this for including graphics (\includegraphics)
%\usepackage{graphicx}
% for neatly defining theorems and propositions
%\usepackage{amsthm}
% making logically defined graphics
%%%\usepackage{xypic}

% there are many more packages, add them here as you need them

% define commands here

\newtheorem{thm}{Theorem}
\newtheorem{defn}{Definition}
\newtheorem{prop}{Proposition}
\newtheorem{lemma}{Lemma}
\newtheorem{cor}{Corollary}

% Some sets
\newcommand{\Nats}{\mathbb{N}}
\newcommand{\Ints}{\mathbb{Z}}
\newcommand{\Reals}{\mathbb{R}}
\newcommand{\Complex}{\mathbb{C}}
\newcommand{\Rats}{\mathbb{Q}}
\begin{document}
Let $L$ be a subfield of $\Complex$.

\begin{defn}\quad
\begin{enumerate}
\item A \emph{real embedding} of $L$ is an injective field
homomorphism
$$ \sigma\colon L \hookrightarrow \Reals $$

\item A (non-real) \emph{complex embedding} of $L$ is an injective
field homomorphism
$$ \tau\colon L \hookrightarrow \Complex $$
such that $\tau(L)\nsubseteq \Reals$.

\item We denote $\Sigma_L$ the set of all embeddings, real and
complex, of $L$ in $\Complex$ (note that all of them must fix
$\Rats$, since they are field homomorphisms).
\end{enumerate}
\end{defn}

Note that if $\sigma$ is a real embedding then
$\bar{\sigma}=\sigma$, where $\overline{\cdot}$ denotes the
complex conjugation automorphism: $$ \overline{\cdot}\colon
\Complex \to \Complex,\quad \overline{(a+bi)}=a-bi$$ On the other
hand, if $\tau$ is a complex embedding, then $\bar{\tau}$ is
another complex embedding, so the complex embeddings always come
in pairs $\{\tau,\bar{\tau}\}$.

Let $K\subseteq L$ be another subfield of $\Complex$. Moreover,
assume that $[L:K]$ is finite (this is the dimension of $L$ as a
vector space over $K$). We are interested in the embeddings of $L$
that fix $K$ pointwise, i.e. embeddings $\psi\colon L
\hookrightarrow \Complex$ such that
$$\psi(k)=k,\quad \forall k\in K$$

\begin{thm}
For any embedding $\psi$ of $K$ in $\Complex$, there are exactly
$[L:K]$ embeddings of $L$ such that they extend $\psi$. In other
words, if $\varphi$ is one of them, then
$$\varphi(k)=\psi(k),\quad \forall k\in K$$
Thus, by taking $\psi=\operatorname{Id}_K$, there are exactly
$[L:K]$ embeddings of $L$ which fix $K$ pointwise.
\end{thm}

Hence, by the theorem, we know that the order of $\Sigma_L$ is
$[L:\Rats]$. The number $[L:\Rats]$ is usually decomposed as
$$[L:\Rats]=r_1+2r_2$$
where $r_1$ is the number of embeddings which are real, and $2r_2$
is the number of embeddings which are complex (non-real). Notice
that by the remark above this number is always even, so $r_2$ is
an integer.

{\bf Remark}: Let $\psi$ be an embedding of $L$ in $\Complex$.
Since $\psi$ is injective, we have $\psi(L)\cong L$, so we can
regard $\psi$ as an automorphism of $L$. When $L/\mathbb{Q}$ is a
Galois extension, we can prove that $\Sigma_L\cong
\operatorname{Gal}(L/\Rats)$, and hence proving in a different way
the fact that
$$\mid \Sigma_L \mid = [L:\Rats]= \mid
\operatorname{Gal}(L/\Rats)\mid$$
%%%%%
%%%%%
\end{document}
