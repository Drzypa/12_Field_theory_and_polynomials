\documentclass[12pt]{article}
\usepackage{pmmeta}
\pmcanonicalname{DegreeOfAnAlgebraicNumber}
\pmcreated{2013-03-22 17:50:05}
\pmmodified{2013-03-22 17:50:05}
\pmowner{Wkbj79}{1863}
\pmmodifier{Wkbj79}{1863}
\pmtitle{degree of an algebraic number}
\pmrecord{4}{40305}
\pmprivacy{1}
\pmauthor{Wkbj79}{1863}
\pmtype{Definition}
\pmcomment{trigger rebuild}
\pmclassification{msc}{12E05}
\pmclassification{msc}{12F05}
\pmclassification{msc}{11C08}
\pmclassification{msc}{11R04}
\pmrelated{AlgebraicNumber}
\pmrelated{Degree8}
\pmrelated{MinimalPolynomial}
\pmrelated{TheoryOfAlgebraicNumbers}

\endmetadata

\usepackage{amssymb}
\usepackage{amsmath}
\usepackage{amsfonts}
\usepackage{pstricks}
\usepackage{psfrag}
\usepackage{graphicx}
\usepackage{amsthm}
%%\usepackage{xypic}

\begin{document}
\PMlinkescapeword{degree}

Let $\alpha$ be an algebraic number.  The \emph{degree} of $\alpha$ is the \PMlinkname{degree}{Degree8} of the minimal polynomial for $\alpha$ over $\mathbb{Q}$.

In a \PMlinkescapetext{similar} manner to polynomials, the degree of $\alpha$ may be denoted $\deg\alpha$.

For example, since $x^3-2$ is the minimal polynomial for $\sqrt[3]{2}$ over $\mathbb{Q}$, we have $\deg\sqrt[3]{2}=3$.
%%%%%
%%%%%
\end{document}
