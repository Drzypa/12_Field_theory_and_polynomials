\documentclass[12pt]{article}
\usepackage{pmmeta}
\pmcanonicalname{TschirnhausTransformations}
\pmcreated{2013-03-22 13:50:12}
\pmmodified{2013-03-22 13:50:12}
\pmowner{mathcam}{2727}
\pmmodifier{mathcam}{2727}
\pmtitle{Tschirnhaus transformations}
\pmrecord{11}{34572}
\pmprivacy{1}
\pmauthor{mathcam}{2727}
\pmtype{Definition}
\pmcomment{trigger rebuild}
\pmclassification{msc}{12E05}
\pmsynonym{Tschirnhausen Transformation}{TschirnhausTransformations}
%\pmkeywords{reduction}
%\pmkeywords{polynomial}
%\pmkeywords{resultant}
\pmrelated{QuadraticResolvent}
\pmrelated{EulersDerivationOfTheQuarticFormula}

% this is the default PlanetMath preamble.  as your knowledge
% of TeX increases, you will probably want to edit this, but
% it should be fine as is for beginners.

% almost certainly you want these
\usepackage{amssymb}
\usepackage{amsmath}
\usepackage{amsfonts}

% used for TeXing text within eps files
%\usepackage{psfrag}
% need this for including graphics (\includegraphics)
%\usepackage{graphicx}
% for neatly defining theorems and propositions
%\usepackage{amsthm}
% making logically defined graphics
%%%\usepackage{xypic}

% there are many more packages, add them here as you need them
\begin{document}
A polynomial transformation which transforms a polynomial to another with certain zero-coefficients is called a
\emph{Tschirnhaus Transformation}. It is thus an invertible transformation of the form   $x \mapsto g(x)/h(x)$ where $g,h$ are polynomials over the base field $K$ (or some subfield of the splitting field of the polynomial being transformed). If $\gcd(h(x),f(x)) = 1$ then the Tschirnhaus transformation becomes a polynomial transformation mod f.

Specifically, it concerns a substitution that reduces finding the roots of the polynomial
$$
\textmd{p} = T^n + a_1T^{n-1} + ... + a_n = \prod_{i=1}^n
(T-r_i)\in k[T]
$$
to finding the roots of another \textmd{q} - with less parameters
- and solving an auxiliary polynomial equation \textmd{s}, with
$\deg(s)<\deg(p \cap q).$

Historically, the transformation was applied to reduce the general quintic equation, to simpler resolvents. Examples due to Hermite and Klein are
respectively: The principal resolvent
$$
K(X):=X^5+a_0X^2+a_1X+a_3
$$
and the Bring-Jerrard form
$$
K(X):=X^5+a_1X+a_2
$$
Tschirnhaus transformations are also used when computing Galois
groups to remove repeated roots in resolvent polynomials. Almost any transformation will work but it is
extremely hard to find an efficient algorithm that can be proved
to work.
%%%%%
%%%%%
\end{document}
