\documentclass[12pt]{article}
\usepackage{pmmeta}
\pmcanonicalname{ExistenceOfTheMinimalPolynomial}
\pmcreated{2013-03-22 13:57:24}
\pmmodified{2013-03-22 13:57:24}
\pmowner{alozano}{2414}
\pmmodifier{alozano}{2414}
\pmtitle{existence of the minimal polynomial}
\pmrecord{7}{34723}
\pmprivacy{1}
\pmauthor{alozano}{2414}
\pmtype{Theorem}
\pmcomment{trigger rebuild}
\pmclassification{msc}{12F05}
%\pmkeywords{minimal polynomial}
%\pmkeywords{root of polynomial}
\pmrelated{FiniteExtension}
\pmrelated{Algebraic}

% this is the default PlanetMath preamble.  as your knowledge
% of TeX increases, you will probably want to edit this, but
% it should be fine as is for beginners.

% almost certainly you want these
\usepackage{amssymb}
\usepackage{amsmath}
\usepackage{amsthm}
\usepackage{amsfonts}

% used for TeXing text within eps files
%\usepackage{psfrag}
% need this for including graphics (\includegraphics)
%\usepackage{graphicx}
% for neatly defining theorems and propositions
%\usepackage{amsthm}
% making logically defined graphics
%%%\usepackage{xypic}

% there are many more packages, add them here as you need them

% define commands here

\newtheorem{thm}{Theorem}
\newtheorem{defn}{Definition}
\newtheorem{prop}{Proposition}
\newtheorem{lemma}{Lemma}
\newtheorem{cor}{Corollary}

% Some sets
\newcommand{\Nats}{\mathbb{N}}
\newcommand{\Ints}{\mathbb{Z}}
\newcommand{\Reals}{\mathbb{R}}
\newcommand{\Complex}{\mathbb{C}}
\newcommand{\Rats}{\mathbb{Q}}
\begin{document}
\begin{prop}
Let $K/L$ be a finite extension of fields and let $k\in K$. There
exists a unique polynomial $m_{k}(x)\in L[x]$ such that:
\begin{enumerate}
\item $m_{k}(x)$ is a monic polynomial;

\item $m_{k}(k)=0$;

\item If $p(x)\in L[x]$ is another polynomial such that $p(k)=0$,
then $m_{k}(x)$ divides $p(x)$.
\end{enumerate}
\end{prop}
\begin{proof}
We start by defining the following map: $$\psi\colon L[x] \to K$$
$$\psi(p(x))=p(k)$$
Note that this map is clearly a ring homomorphism. For all
$p(x),q(x) \in L[x]$:
\begin{itemize}
\item $\psi(p(x)+q(x))=p(k)+q(k)=\psi(p(x))+\psi(q(x))$

\item $\psi(p(x)\cdot q(x))=p(k)\cdot
q(k)=\psi(p(x))\cdot\psi(q(x))$
\end{itemize}
Thus, the kernel of $\psi$ is an ideal of $L[x]$:
$$\operatorname{Ker}(\psi)=\{ p(x)\in L[x] \mid p(k)=0 \}$$
Note that the kernel is a {\bf non-zero} ideal. This fact relies
on the fact that $K/L$ is a finite extension of fields, and
therefore it is an algebraic extension, so every element of $K$ is
a root of a non-zero polynomial $p(x)$ with coefficients in $L$,
this is, $p(x)\in \operatorname{Ker}(\psi)$.

Moreover, the ring of polynomials $L[x]$ is a principal ideal
domain (see example of PID).
Therefore, the kernel of $\psi$ is a principal ideal, generated by
some polynomial $m(x)$:
$$\operatorname{Ker}(\psi)=(m(x))$$
Note that the only units in $L[x]$ are the constant polynomials,
hence if $m'(x)$ is another generator of
$\operatorname{Ker}(\psi)$ then
$$m'(x)=l\cdot m(x), \quad l\neq 0,\quad l\in L$$
Let $\alpha$ be the leading coefficient of $m(x)$. We define
$m_{k}(x)=\alpha^{-1}m(x)$, so that the leading coefficient of
$m_{k}$ is $1$. Also note that by the previous remark, $m_{k}$ is
the unique generator of $\operatorname{Ker}(\psi)$ which is monic.

By construction, $m_{k}(k)=0$, since $m_{k}$ belongs to the kernel
of $\psi$, so it satisfies $(2)$.

Finally, if $p(x)$ is any polynomial such that $p(k)=0$, then
$p(x) \in \operatorname{Ker}(\psi)$. Since $m_{k}$ generates this
ideal, we know that $m_{k}$ must divide $p(x)$ (this is property
$(3)$).

For the uniqueness, note that any polynomial satisfying $(2)$ and
$(3)$ must be a generator of $\operatorname{Ker}(\psi)$, and, as
we pointed out, there is a unique monic generator, namely
$m_{k}(x)$.

\end{proof}
%%%%%
%%%%%
\end{document}
