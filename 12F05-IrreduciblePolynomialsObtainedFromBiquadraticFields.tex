\documentclass[12pt]{article}
\usepackage{pmmeta}
\pmcanonicalname{IrreduciblePolynomialsObtainedFromBiquadraticFields}
\pmcreated{2013-03-22 17:54:22}
\pmmodified{2013-03-22 17:54:22}
\pmowner{Wkbj79}{1863}
\pmmodifier{Wkbj79}{1863}
\pmtitle{irreducible polynomials obtained from biquadratic fields}
\pmrecord{5}{40398}
\pmprivacy{1}
\pmauthor{Wkbj79}{1863}
\pmtype{Corollary}
\pmcomment{trigger rebuild}
\pmclassification{msc}{12F05}
\pmclassification{msc}{12E05}
\pmclassification{msc}{11R16}
\pmrelated{ExamplesOfMinimalPolynomials}
\pmrelated{BiquadraticEquation2}

\usepackage{amssymb}
\usepackage{amsmath}
\usepackage{amsfonts}
\usepackage{pstricks}
\usepackage{psfrag}
\usepackage{graphicx}
\usepackage{amsthm}
%%\usepackage{xypic}

\newtheorem*{cor*}{Corollary}
\begin{document}
\begin{cor*}
Let $m$ and $n$ be distinct squarefree integers, neither of which is equal to $1$.  Then the polynomial
\[
x^4-2(m+n)x^2+(m-n)^2
\]
is \PMlinkname{irreducible}{IrreduciblePolynomial2} (over $\mathbb{Q}$).
\end{cor*}

\begin{proof}
By the theorem stated in the \PMlinkname{parent entry}{PrimitiveElementOfBiquadraticField}, $\sqrt{m}+\sqrt{n}$ is an algebraic number of \PMlinkname{degree}{DegreeOfAnAlgebraicNumber} $4$.  Thus, a polynomial of degree $4$ that has $\sqrt{m}+\sqrt{n}$ as a root must be \PMlinkescapetext{irreducible} over $\mathbb{Q}$.  We set out to construct such a polynomial.

\begin{center}
$\begin{array}{rl}
x & =\sqrt{m}+\sqrt{n} \\
x-\sqrt{m} & =\sqrt{n} \\
(x-\sqrt{m})^2 & =n \\
x^2-2\sqrt{m}\, x+m & =n \\
x^2+m-n & =2\sqrt{m}\, x \\
(x^2+m-n)^2 & =4mx^2 \\
x^4+(2m-2n)x^2+(m-n)^2 & =4mx^2 \\
x^4+(2m-2n-4m)x^2+(m-n)^2 & =0 \\
x^4-2(m+n)x^2+(m-n)^2 & =0 \qedhere
\end{array}$
\end{center}
\end{proof}
%%%%%
%%%%%
\end{document}
