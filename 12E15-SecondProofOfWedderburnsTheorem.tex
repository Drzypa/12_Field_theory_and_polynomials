\documentclass[12pt]{article}
\usepackage{pmmeta}
\pmcanonicalname{SecondProofOfWedderburnsTheorem}
\pmcreated{2013-03-22 13:34:39}
\pmmodified{2013-03-22 13:34:39}
\pmowner{Mathprof}{13753}
\pmmodifier{Mathprof}{13753}
\pmtitle{second proof of Wedderburn's theorem}
\pmrecord{17}{34198}
\pmprivacy{1}
\pmauthor{Mathprof}{13753}
\pmtype{Proof}
\pmcomment{trigger rebuild}
\pmclassification{msc}{12E15}
%\pmkeywords{cyclotomic polynomial}

\usepackage{amssymb}
\usepackage{amsmath}
\usepackage{amsfonts}

\newcommand {\cnums}{\mathbb{C}}
\newcommand {\znums}{\mathbb{Z}}
\begin{document}
We can prove Wedderburn's theorem,without using Zsigmondy's theorem on the conjugacy class formula of the first proof;
let $G_n$ set of n-th roots of unity and $P_n$ set of n-th primitive
roots of unity and $\Phi_d(q)$ the d-th cyclotomic polynomial.\\
It results
\begin{itemize}
    \item $ \Phi_n(q)=\prod_{\xi \in P_n}(q- \xi)$
    \item $ p(q)=q^n-1=\prod _{\xi \in G_n}
 (q- \xi)=\prod_{d\mid n}\Phi_d(q) $
    \item $ \Phi_n(q)\in \znums [q] \;$, it has multiplicative identity and $\Phi_n(q)\mid q^n-1$
    \item $ \Phi_n(q) \mid \frac{q^n-1 }{q^d-1} \;$with $ d \mid n, d<n$
\end{itemize}
by conjugacy class formula, we have:
$$q^n-1=q-1+\sum_x \frac{q^n-1}{q^{n_x}-1} $$
by last two previous properties, it results:
$$ \Phi_n(q) \mid q^n-1 \;,\; \Phi_n(q) \mid \frac{q^n-1}{q^{n_x}-1}
\Rightarrow \Phi_n(q) \mid q-1$$
\\because $\Phi_n(q)$
divides the left and each addend of $ \sum_x \frac{q^n-1}{q^{n_x}-1} $  
of the right member of the conjugacy class formula.
\\By third property
$$q>1 \;,\; \Phi_n(x)\in \znums[x]
\Rightarrow \Phi_n(q)\in \znums \Rightarrow |\Phi_n(q)| \mid q-1
\Rightarrow |\Phi_n(q)|\leqslant q-1 $$
\\If, for $n>1$,we have $|\Phi_n(q)|>q-1 $, then $n=1$ and the theorem is proved.
\\We know that
\\ $$ |\Phi_n(q)|=\prod_{\xi \in P_n} |q - \xi|\;,\;with\; q- \xi\in \cnums $$
\\by the triangle inequality in $\cnums$
$$ |q-\xi|\geqslant||q|-|\xi||=|q-1|$$ as $\xi$ is a primitive root of unity,
 besides $$|q-\xi|=|q-1| \Leftrightarrow \xi=1$$
but $$n>1 \Rightarrow \xi \neq 1$$ therefore, we have
$$|q-\xi|>|q-1|=q-1 \Rightarrow |\Phi_n(q)|>q-1$$
%%%%%
%%%%%
\end{document}
