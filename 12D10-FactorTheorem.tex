\documentclass[12pt]{article}
\usepackage{pmmeta}
\pmcanonicalname{FactorTheorem}
\pmcreated{2013-03-22 12:17:24}
\pmmodified{2013-03-22 12:17:24}
\pmowner{drini}{3}
\pmmodifier{drini}{3}
\pmtitle{factor theorem}
\pmrecord{10}{31810}
\pmprivacy{1}
\pmauthor{drini}{3}
\pmtype{Theorem}
\pmcomment{trigger rebuild}
\pmclassification{msc}{12D10}
\pmclassification{msc}{12D05}
\pmsynonym{root theorem}{FactorTheorem}
\pmrelated{Polynomial}
\pmrelated{RationalRootTheorem}
\pmrelated{Root}
\pmrelated{APolynomialOfDegreeNOverAFieldHasAtMostNRoots}

\endmetadata

%\usepackage{graphicx}
%%%\usepackage{xypic} 
\usepackage{bbm}
\newcommand{\Z}{\mathbbmss{Z}}
\newcommand{\C}{\mathbbmss{C}}
\newcommand{\R}{\mathbbmss{R}}
\newcommand{\Q}{\mathbbmss{Q}}
\newcommand{\mathbb}[1]{\mathbbmss{#1}}
\begin{document}
If $f(x)$ is a polynomial over a ring with identity, then $x-a$ is a factor if and only if $a$ is a root (that is, $f(a)=0$).

This theorem is of great help for finding factorizations of higher degree polynomials. As example, let us think that we need to factor the polynomial $p(x)=x^3+3x^2-33x-35$. With some help of the rational root theorem we can find that $x=-1$ is a root (that is, $p(-1)=0$), so we know $(x+1)$ must be a factor of the polynomial. We can write then
$$p(x)=(x+1)q(x)$$
where the polynomial $q(x)$ can be found using long or synthetic division of $p(x)$ between $x-1$. In our case $q(x)=x^2+2x-35$ which can be easily factored as $(x-5)(x+7)$. We conclude that
$$p(x)=(x+1)(x-5)(x+7).$$
%%%%%
%%%%%
\end{document}
