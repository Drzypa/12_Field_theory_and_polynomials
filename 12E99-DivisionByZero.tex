\documentclass[12pt]{article}
\usepackage{pmmeta}
\pmcanonicalname{DivisionByZero}
\pmcreated{2013-03-22 16:35:53}
\pmmodified{2013-03-22 16:35:53}
\pmowner{Algeboy}{12884}
\pmmodifier{Algeboy}{12884}
\pmtitle{division by zero}
\pmrecord{10}{38794}
\pmprivacy{1}
\pmauthor{Algeboy}{12884}
\pmtype{Example}
\pmcomment{trigger rebuild}
\pmclassification{msc}{12E99}
\pmclassification{msc}{00A05}
\pmrelated{Field}

\usepackage{latexsym}
\usepackage{amssymb}
\usepackage{amsmath}
\usepackage{amsfonts}
\usepackage{amsthm}

%%\usepackage{xypic}

%-----------------------------------------------------

%       Standard theoremlike environments.

%       Stolen directly from AMSLaTeX sample

%-----------------------------------------------------

%% \theoremstyle{plain} %% This is the default

\newtheorem{thm}{Theorem}

\newtheorem{coro}[thm]{Corollary}

\newtheorem{lem}[thm]{Lemma}

\newtheorem{lemma}[thm]{Lemma}

\newtheorem{prop}[thm]{Proposition}

\newtheorem{conjecture}[thm]{Conjecture}

\newtheorem{conj}[thm]{Conjecture}

\newtheorem{defn}[thm]{Definition}

\newtheorem{remark}[thm]{Remark}

\newtheorem{ex}[thm]{Example}



%\countstyle[equation]{thm}



%--------------------------------------------------

%       Item references.

%--------------------------------------------------


\newcommand{\exref}[1]{Example-\ref{#1}}

\newcommand{\thmref}[1]{Theorem-\ref{#1}}

\newcommand{\defref}[1]{Definition-\ref{#1}}

\newcommand{\eqnref}[1]{(\ref{#1})}

\newcommand{\secref}[1]{Section-\ref{#1}}

\newcommand{\lemref}[1]{Lemma-\ref{#1}}

\newcommand{\propref}[1]{Prop\-o\-si\-tion-\ref{#1}}

\newcommand{\corref}[1]{Cor\-ol\-lary-\ref{#1}}

\newcommand{\figref}[1]{Fig\-ure-\ref{#1}}

\newcommand{\conjref}[1]{Conjecture-\ref{#1}}


% Normal subgroup or equal.

\providecommand{\normaleq}{\unlhd}

% Normal subgroup.

\providecommand{\normal}{\lhd}

\providecommand{\rnormal}{\rhd}
% Divides, does not divide.

\providecommand{\divides}{\mid}

\providecommand{\ndivides}{\nmid}


\providecommand{\union}{\cup}

\providecommand{\bigunion}{\bigcup}

\providecommand{\intersect}{\cap}

\providecommand{\bigintersect}{\bigcap}










\begin{document}
Suppose that we construct a number system where we can divide by zero.  What are the consequences?

Dividing by $x$ means there exists an inverse $\displaystyle \frac{1}{x}$ 
so that $\displaystyle x\cdot\frac{1}{x}=1$.  So if we could divide by $0$ 
then we would have a number we would call $\displaystyle \frac{1}{0}$.
We know $0\cdot x=0$ for any $x$, so $\displaystyle 0\cdot \frac{1}{0}=0$.  However $\displaystyle 0\cdot\frac{1}{0}=1$ because $\displaystyle x\cdot\frac{1}{x}=1$ for all $x$.  Therefore $0=1$.

The problem is worse than forcing $0=1$.  Indeed, $1\cdot x=x$ for all $x$, so
\[x=1\cdot x=0\cdot x=0.\]
So indeed every number is then 0.

\emph{Therefore, in a number system where we can add, subtract, multiply and divide (a field), we do not allow division by 0 because it would force the number system to only contain 0.}

Hence, most calculators, from the the very basic to the full-fledged computer algebra system, generate some kind of an error message when asked to divide by 0. In fact, such an operation on a basic calculator causes the device to become inoperable until either the error clear key is pressed or the device is reset. A computer algebra system like Mathematica will complain about encountering ``the infinite expression $\frac{1}{0}$'' but deliver the result ``\verb=ComplexInfinity='' and remain operable.

But of course the limitations of machinery alone are not enough to forbid something from reality. So there are various thought experiments that can also be used to explain the omission of division by zero, for instance:
\begin{quote}
How could you slice a pizza into slices that have 0 width?
\end{quote}
Since this would correspond to division by zero, it seems intellectually impossible.  There is however an unfortunate side effect from such thinking.
For example:
\begin{quote}
How could you slice a pizza into slices of width -7, since width is always a positive
quantity?
Furthermore, how could you actually slice a pizza into slices of width $\sqrt{7}$ or $e^\pi$, and so on?  Yet we can actually divide by each of these
numbers.
\end{quote}
This is an example of how mathematical abstraction benefits the understanding
of a problem.

%%%%%
%%%%%
\end{document}
