\documentclass[12pt]{article}
\usepackage{pmmeta}
\pmcanonicalname{ProofOfFactorTheorem}
\pmcreated{2013-03-22 12:39:54}
\pmmodified{2013-03-22 12:39:54}
\pmowner{Wkbj79}{1863}
\pmmodifier{Wkbj79}{1863}
\pmtitle{proof of factor theorem}
\pmrecord{8}{32937}
\pmprivacy{1}
\pmauthor{Wkbj79}{1863}
\pmtype{Proof}
\pmcomment{trigger rebuild}
\pmclassification{msc}{12D05}
\pmclassification{msc}{12D10}

% this is the default PlanetMath preamble.  as your knowledge
% of TeX increases, you will probably want to edit this, but
% it should be fine as is for beginners.

% almost certainly you want these
\usepackage{amssymb}
\usepackage{amsmath}
\usepackage{amsfonts}

% used for TeXing text within eps files
%\usepackage{psfrag}
% need this for including graphics (\includegraphics)
%\usepackage{graphicx}
% for neatly defining theorems and propositions
%\usepackage{amsthm}
% making logically defined graphics
%%%\usepackage{xypic}

% there are many more packages, add them here as you need them

% define commands here
\begin{document}
Suppose that $f(x)$ is a polynomial with real or complex coefficients of degree $n-1$. Since $f$ is a polynomial, it is infinitely differentiable.  Therefore, $f$ has a Taylor expansion about $a$. Since $f^{(n)}(x)=0$, the \PMlinkescapetext{expansion} terminates after the $n-1^{\text{th}}$ term. Also, the $n^{\text{th}}$ remainder of the Taylor series vanishes; \PMlinkname{i.e.}{Ie}, $\displaystyle R_n(x)=\frac{f^{(n)}(y)}{n!}x^n=0$.  Thus, the function is equal to its Taylor series.  Hence,

\begin{center}
$\begin{array}{rl}
f(x) & \displaystyle =\sum_{k=0}^{n-1}\frac{f^{(k)}(a)}{k!}(x-a)^k \\
& \\
& \displaystyle =f(a)+\sum_{k=1}^{n-1}\frac{f^{(k)}(a)}{k!}(x-a)^k \\
& \\
& \displaystyle =f(a)+(x-a)\sum_{k=1}^{n-1}\frac{f^{(k)}(a)}{k!}(x-a)^{k-1} \\
& \\
& \displaystyle =f(a)+(x-a)\sum_{k=0}^{n-2}\frac{f^{(k+1)}(a)}{(k+1)!}(x-a)^k. \end{array}$
\end{center}

If $f(a)=0$, then $\displaystyle f(x)=(x-a)\sum_{k=0}^{n-2}\frac{f^{(k+1)}(a)}{(k+1)!}(x-a)^k$.  Thus, $f(x)=(x-a)g(x)$, where $g(x)$ is the polynomial $\displaystyle \sum_{k=0}^{n-2}\frac{f^{(k+1)}(a)}{(k+1)!}(x-a)^k$. Hence, $x-a$ is a factor of $f(x)$.

Conversely, if $x-a$ is a factor of $f(x)$, then $f(x)=(x-a)g(x)$ for some polynomial $g(x)$.  Hence, $f(a)=(a-a)g(a)=0$.

It follows that $x-a$ is a factor of $f(x)$ if and only if $f(a)=0$.
%%%%%
%%%%%
\end{document}
