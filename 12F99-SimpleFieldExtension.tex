\documentclass[12pt]{article}
\usepackage{pmmeta}
\pmcanonicalname{SimpleFieldExtension}
\pmcreated{2013-03-22 14:23:06}
\pmmodified{2013-03-22 14:23:06}
\pmowner{pahio}{2872}
\pmmodifier{pahio}{2872}
\pmtitle{simple field extension}
\pmrecord{25}{35878}
\pmprivacy{1}
\pmauthor{pahio}{2872}
\pmtype{Definition}
\pmcomment{trigger rebuild}
\pmclassification{msc}{12F99}
%\pmkeywords{adjunction}
\pmrelated{PrimitiveElementTheorem}
\pmrelated{CanonicalFormOfElementOfNumberField}
\pmdefines{primitive element}

% this is the default PlanetMath preamble.  as your knowledge
% of TeX increases, you will probably want to edit this, but
% it should be fine as is for beginners.

% almost certainly you want these
\usepackage{amssymb}
\usepackage{amsmath}
\usepackage{amsfonts}

% used for TeXing text within eps files
%\usepackage{psfrag}
% need this for including graphics (\includegraphics)
%\usepackage{graphicx}
% for neatly defining theorems and propositions
%\usepackage{amsthm}
% making logically defined graphics
%%%\usepackage{xypic}

% there are many more packages, add them here as you need them

% define commands here
\begin{document}
Let $K(\alpha)$ be obtained from the field $K$ via the \PMlinkescapetext{simple adjunction} of the element $\alpha$, which is called the \emph{primitive element} of the field extension $K(\alpha)/K$.\, We shall settle the \PMlinkescapetext{structure types} of the field $K(\alpha)$.

We consider the substitution homomorphism \,$\varphi: K[X] \rightarrow K[\alpha]$, where
            $$\sum a_{\nu}X^\nu \mapsto \sum a_{\nu}\alpha^\nu.$$
According to the ring homomorphism theorem, the image ring $K[\alpha]$ is isomorphic with the residue class ring $K[X]/\frak{p}$, where $\frak{p}$ is the ideal of polynomials having $\alpha$ as their zero. \,Because $K[\alpha]$ is, as subring of the field $K(\alpha)$, an integral domain, then also $K[X]/\frak{p}$ has no zero divisors, and hence $\frak{p}$ is a prime ideal. \,It must be principal, for $K[X]$ is a principal ideal ring.

There are two possibilities:

\begin{enumerate}
\item  $\frak{p} = (p(X))$, where $p(X)$ is an irreducible polynomial with \,$p(\alpha) = 0$. \,Because every non-zero prime ideal of $K[X]$ is maximal, the isomorphic image $K[X]/(p(X))$ of $K[\alpha]$ is a field, and it must give the \PMlinkescapetext{structure} of \,$K(\alpha) = K[\alpha]$. \,We say that $\alpha$ is {\em algebraic with respect to} $K$ (or {\em over} $K$). \,In this case, we have a finite field extension \,$K(\alpha)/K$.

\item  $\frak{p} = (0)$. \,This means that the homomorphism $\varphi$ is an isomorphism between $K[X]$ and $K[\alpha]$, i.e. all expressions $\sum a_{\nu}\alpha^\nu$ behave as the polynomials $\sum a_{\nu}X^\nu$. \,Now, $K[\alpha]$ is no field because $K[X]$ is not such, but the isomorphy of the rings implies the isomorphy of the corresponding fields of fractions. \,Thus the simple extension field $K(\alpha)$ is isomorphic with the field $K(X)$ of rational functions in one indeterminate $X$. \,We say that $\alpha$ is {\em \PMlinkname{transcendental}{Algebraic} with respect to} $K$ (or {\em over} $K$).  \,This time we have a simple infinite field extension \,$K(\alpha)/K$. 
\end{enumerate}
%%%%%
%%%%%
\end{document}
