\documentclass[12pt]{article}
\usepackage{pmmeta}
\pmcanonicalname{ExamplesOfTraceAndNorm}
\pmcreated{2013-03-22 15:55:45}
\pmmodified{2013-03-22 15:55:45}
\pmowner{polarbear}{3475}
\pmmodifier{polarbear}{3475}
\pmtitle{examples of trace and norm}
\pmrecord{16}{37937}
\pmprivacy{1}
\pmauthor{polarbear}{3475}
\pmtype{Example}
\pmcomment{trigger rebuild}
\pmclassification{msc}{12F05}

% this is the default PlanetMath preamble.  as your knowledge
% of TeX increases, you will probably want to edit this, but
% it should be fine as is for beginners.

% almost certainly you want these
\usepackage{amssymb}
\usepackage{amsmath}
\usepackage{amsfonts}

% used for TeXing text within eps files
%\usepackage{psfrag}
% need this for including graphics (\includegraphics)
%\usepackage{graphicx}
% for neatly defining theorems and propositions
%\usepackage{amsthm}
% making logically defined graphics
%%%\usepackage{xypic}

% there are many more packages, add them here as you need them

% define commands here

\begin{document}
Let $\omega$ be a complex root of unity different than 1. Then $\omega$ and $\omega^2$ are the conjugate roots of the minimal polynomial $x^2 + x +1$.
 Since $\mathbb{Q}(\omega)$ is the splitting field of $x^2 + x +1$, it is Galois over $\mathbb{Q}$. Moreover the Galois group $Gal(\mathbb{Q}(\omega)/\mathbb{Q}))$ is formed by the identity and the automorphism $g(\omega) = \omega^2$
 The elements of $\mathbb{Q}(\omega)$ have the form $a + b\omega$, $a,b\in \mathbb{Q}$.
 Then we obtain
\[N_{\mathbb{Q}(\omega)}^{\mathbb{Q}}(a + b\omega) = (a +b\omega)(a +b\omega^2) = a^2 - ab + b^2,
Tr_{\mathbb{Q}(\omega)}^{\mathbb{Q}}(a + b\omega) = (a +b\omega) + (a + b\omega^2) = 2a - b \]
%%%%%
%%%%%
\end{document}
