\documentclass[12pt]{article}
\usepackage{pmmeta}
\pmcanonicalname{IrreducibilityOfBinomialsWithUnityCoefficients}
\pmcreated{2013-03-22 15:13:08}
\pmmodified{2013-03-22 15:13:08}
\pmowner{pahio}{2872}
\pmmodifier{pahio}{2872}
\pmtitle{irreducibility of binomials with unity coefficients}
\pmrecord{14}{36982}
\pmprivacy{1}
\pmauthor{pahio}{2872}
\pmtype{Result}
\pmcomment{trigger rebuild}
\pmclassification{msc}{12D05}
\pmclassification{msc}{13F15}
\pmrelated{FactoringASumOrDifferenceOfTwoCubes}
\pmrelated{PrimeFaxtorsOfXn1}
\pmrelated{PrimeFactorsOfXn1}
\pmrelated{ExpressibleInClosedForm}

\endmetadata

% this is the default PlanetMath preamble.  as your knowledge
% of TeX increases, you will probably want to edit this, but
% it should be fine as is for beginners.

% almost certainly you want these
\usepackage{amssymb}
\usepackage{amsmath}
\usepackage{amsfonts}

% used for TeXing text within eps files
%\usepackage{psfrag}
% need this for including graphics (\includegraphics)
%\usepackage{graphicx}
% for neatly defining theorems and propositions
 \usepackage{amsthm}
% making logically defined graphics
%%%\usepackage{xypic}

% there are many more packages, add them here as you need them

% define commands here

\theoremstyle{definition}
\newtheorem*{thmplain}{Theorem}
\begin{document}
Let $n$ be a positive integer.\, We consider the possible factorization of the binomial $x^n\!+\!1$.

\begin{itemize}
 \item If $n$ has no odd prime factors, then the binomial $x^n\!+\!1$ is \PMlinkname{irreducible}{Irreducible Polynomial}.\, Thus, $x\!+\!1$, $x^2\!+\!1$, $x^4\!+\!1$, $x^8\!+\!1$ and so on are irreducible polynomials (i.e. \PMlinkescapetext{irreducible} in the field $\mathbb{Q}$ of their coefficients).\, N.B., only $x\!+\!1$ and $x^2\!+\!1$ are \PMlinkescapetext{irreducible} in the field $\mathbb{R}$; e.g. one has\, $x^4\!+\!1 = (x^2\!-\!x\sqrt{2}\!+\!1)(x^2\!+\!x\sqrt{2}\!+\!1)$.
 \item If $n$ is an odd number, then $x^n\!+\!1$ is always divisible by $x\!+\!1$:
  \begin{align}
         x^n+1 = (x+1)(x^{n-1}-x^{n-2}+x^{n-3}-+\cdots-x+1)
  \end{align}
This \PMlinkescapetext{formula} is usable when $n$ is an odd prime number, e.g.
              $$x^5+1 = (x+1)(x^4-x^3+x^2-x+1).$$
 \item When $n$ is not a prime number but has an odd prime factor $p$, say\,
 $n = mp$,\, then we write\, $x^n\!+\!1 = (x^m)^p\!+\!1$\, and apply the idea of (1); for example:
   $$x^{12}+1 = (x^4)^3+1 = (x^4+1)[(x^4)^2-x^4+1] = (x^4+1)(x^8-x^4+1)$$
\end{itemize}

There are similar results for the binomial $x^n\!+\!y^n$, and the \PMlinkescapetext{formula} corresponding to (1) is
  \begin{align}
         x^n+y^n = (x+y)(x^{n-1}-x^{n-2}y+x^{n-3}y^2-+\cdots-xy^{n-2}+y^n),
  \end{align}
which may be verified by performing the multiplication on the right hand \PMlinkescapetext{side}.
%%%%%
%%%%%
\end{document}
