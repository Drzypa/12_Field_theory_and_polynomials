\documentclass[12pt]{article}
\usepackage{pmmeta}
\pmcanonicalname{ProofOfCasusIrreducibilisForRealFields}
\pmcreated{2013-03-22 17:43:08}
\pmmodified{2013-03-22 17:43:08}
\pmowner{rm50}{10146}
\pmmodifier{rm50}{10146}
\pmtitle{proof of casus irreducibilis for real fields}
\pmrecord{8}{40164}
\pmprivacy{1}
\pmauthor{rm50}{10146}
\pmtype{Theorem}
\pmcomment{trigger rebuild}
\pmclassification{msc}{12F10}

% this is the default PlanetMath preamble.  as your knowledge
% of TeX increases, you will probably want to edit this, but
% it should be fine as is for beginners.

% almost certainly you want these
\usepackage{amssymb}
\usepackage{amsmath}
\usepackage{amsfonts}

% used for TeXing text within eps files
%\usepackage{psfrag}
% need this for including graphics (\includegraphics)
%\usepackage{graphicx}
% for neatly defining theorems and propositions
\usepackage{amsthm}
% making logically defined graphics
%%%\usepackage{xypic}

% there are many more packages, add them here as you need them

% define commands here
\DeclareMathOperator{\Gal}{Gal}
\newtheorem{thm}{Theorem}
\newcommand{\Reals}{\mathbb{R}}
\newcommand{\Rats}{\mathbb{Q}}
\begin{document}
The classical statement of the \emph{casus irreducibilis} is that if $f(x)$ is an irreducible cubic polynomial with rational coefficients and three real roots, then the roots of $f(x)$ are not expressible using real radicals. One example of such a polynomial is $x^3-3x+1$, whose roots are $2\cos(2\pi/9), 2\cos(8\pi/9), 2\cos(14\pi/9)$.

This article generalizes the classical case to include all polynomials whose degree is not a power of $2$, and also generalizes the base field to be any real extension of $\Rats$:

\begin{thm} Let $F\subset \Reals$ be a field, and assume $f(x)\in F[x]$ is an irreducible polynomial whose splitting field $L$ is real with $F\subset L\subset \Reals$. Then the following are equivalent:
\begin{enumerate}
\item Some root of $f(x)$ is expressible by real radicals over $F$;
\item All roots of $f(x)$ are expressible by real radicals over $F$ using only square roots;
\item $F\subset L$ is a radical extension;
\item $[L:F]$ is  a power of $2$.
\end{enumerate}
\end{thm}
\textbf{Proof. } That $(2)\Rightarrow (1)$ is obvious, and $(3)\Rightarrow (1)$ since $F\subset L$ is radical, and is real since $L\subset \Reals$. $(4)$ implies that $G=\Gal(L/F)$ has order a power of $2$. Since $G$ is a $2$-group, it has a nontrivial center (this follows directly from the class equation, or look \PMlinkname{here}{ANontrivialNormalSubgroupOfAFinitePGroupGAndTheCenterOfGHaveNontrivialIntersection}) and thus has a normal subgroup $H$ of order $2$, which corresponds to a subfield $M$ of $L$ Galois over $F$ with $[L:M]=2$. But then $\Gal(M/F)$ is also a $2$-group, so inductively we see that we can write
\[F=K_0\subset K_1\subset \ldots \subset K_{m-1}=M \subset K_m=L\]
where $[K_i:K_{i-1}]=2$. Thus each $K_i$ is obtained from $K_{i-1}$ by adjoining a square root; it must be a real square root since $L\subset \Reals$. This shows that $(4)\Rightarrow (2)$ and $(3)$.

The meat of the proof is in showing that $(1)\Rightarrow (4)$. Let the roots of $f(x)$ be $\alpha_1,\ldots,\alpha_m$, and assume, by renumbering if necessary, that $\alpha=\alpha_1$ lies in a real radical extension $K$ of $F$ but that $[L:F]$ is not a power of $2$. Choose an odd prime $p$ dividing $[L:F]=\lvert G\rvert$, and choose an element $\tau\in G$ of order $p$. Then $\tau$ is not the identity, so for some $i$, $\tau(\alpha_i)\neq \alpha_i$. Also, since $f(x)$ is irreducible, $G$ acts transitively on the roots of $f(x)$, so for some $\nu\in G$, $\nu(\alpha)=\alpha_i$. Then $\sigma=\nu^{-1}\tau\nu$ does not fix $\alpha$, since
\[\nu^{-1}\tau\nu(\alpha)=\nu^{-1}\tau(\alpha_i)\neq \nu^{-1}(\alpha_i)=\alpha\]
Let $N=L^{\sigma}$ be the fixed field of $\sigma$. Then $L$ is Galois over $N$, and clearly $[L:N]=p$. But Galois subfields of real radical extensions are at most quadratic, so $L$ cannot lie in a real radical extension of $N$.

However, $\alpha\notin N, \alpha\in L$, and $[L:N]$ is prime. Thus $L=N(\alpha)\subset NK$ (since $\alpha\in K$). Additionally, since $F\subset F(\alpha)\subset K$ is a real radical extension of $F$, we have also that $NK$ is a real radical extension of $NF=N$. So $L$ lies in the real radical extension $NK$ of $N$. But this is a contradiction and thus $[L:F]$ must be a power of $2$.

One consequence of this theorem is the fact that if $f(x)\in F[x]$ has degree not a power of $2$, then if $f(x)$ has all real roots, those roots are not expressible in terms of real radicals. If $\deg f=3$, we recover the original \emph{casus irreducibilis}.

\begin{thebibliography}{10}
\bibitem{bib:andrews}
D.A.~Cox, \emph{Galois Theory}, Wiley-Interscience, 2004.
\end{thebibliography}
%%%%%
%%%%%
\end{document}
