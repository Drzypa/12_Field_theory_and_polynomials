\documentclass[12pt]{article}
\usepackage{pmmeta}
\pmcanonicalname{ExclusionOfIntegerRoot}
\pmcreated{2013-03-22 19:08:21}
\pmmodified{2013-03-22 19:08:21}
\pmowner{pahio}{2872}
\pmmodifier{pahio}{2872}
\pmtitle{exclusion of integer root}
\pmrecord{5}{42038}
\pmprivacy{1}
\pmauthor{pahio}{2872}
\pmtype{Theorem}
\pmcomment{trigger rebuild}
\pmclassification{msc}{12D10}
\pmclassification{msc}{12D05}
\pmrelated{DivisibilityInRings}

% this is the default PlanetMath preamble.  as your knowledge
% of TeX increases, you will probably want to edit this, but
% it should be fine as is for beginners.

% almost certainly you want these
\usepackage{amssymb}
\usepackage{amsmath}
\usepackage{amsfonts}

% used for TeXing text within eps files
%\usepackage{psfrag}
% need this for including graphics (\includegraphics)
%\usepackage{graphicx}
% for neatly defining theorems and propositions
 \usepackage{amsthm}
% making logically defined graphics
%%%\usepackage{xypic}

% there are many more packages, add them here as you need them

% define commands here

\theoremstyle{definition}
\newtheorem*{thmplain}{Theorem}

\begin{document}
\textbf{Theorem.}\, The equation
$$p(x) \;:=\; a_nx^n+a_{n-1}x^{n-1}+\ldots+a_0 \;=\; 0$$
with integer coefficients $a_i$ has no integer \PMlinkname{roots}{Equation}, if $p(0)$ and $p(1)$ are odd.\\

\emph{Proof.}\, Make the antithesis, that there is an integer $x_0$ such that\, $p(x_0) = 0$.\, This $x_0$ cannot be even, because else all terms of $p(x_0)$ except $a_0$ were even and thus the whole sum could not have the even value 0.\, Consequently, $x_0$ and also its \PMlinkname{powers}{GeneralAssociativity} have to be odd.\, Since 
$$2 \mid 0 = p(x_0) \quad \textrm{and} \quad 2 \nmid p(0) = a_0,$$
there must be among the coefficients $a_n,\,a_{n-1},\,\ldots,\,a_1$ an odd amount of odd numbers.\, This means that
$$2 \mid a_n\!+\!a_{n-1}\!+\!\ldots\!+\!a_1\!+\!a_0 \;=\; p(1).$$
This however contradicts the assumption on the parity of $p(1)$, whence the antithesis is wrong and the theorem \PMlinkescapetext{right}.
%%%%%
%%%%%
\end{document}
