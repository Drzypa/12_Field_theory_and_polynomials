\documentclass[12pt]{article}
\usepackage{pmmeta}
\pmcanonicalname{ContinuousFunctionsOnTheExtendedRealNumbers}
\pmcreated{2013-03-22 16:59:31}
\pmmodified{2013-03-22 16:59:31}
\pmowner{Wkbj79}{1863}
\pmmodifier{Wkbj79}{1863}
\pmtitle{continuous functions on the extended real numbers}
\pmrecord{10}{39272}
\pmprivacy{1}
\pmauthor{Wkbj79}{1863}
\pmtype{Theorem}
\pmcomment{trigger rebuild}
\pmclassification{msc}{12D99}
\pmclassification{msc}{28-00}

\usepackage{amssymb}
\usepackage{amsmath}
\usepackage{amsfonts}

\usepackage{psfrag}
\usepackage{graphicx}
\usepackage{amsthm}
%%\usepackage{xypic}

\newtheorem{thm*}{Theorem}

\begin{document}
Within this entry, $\overline{\mathbb{R}}$ will be used to refer to the extended real numbers.

\begin{thm*}
Let $f \colon \mathbb{R} \to \mathbb{R}$ be a function.  Then $\overline{f} \colon \overline{\mathbb{R}} \to \overline{\mathbb{R}}$ defined by

\begin{center}
$\overline{f}(x)=\begin{cases}
f(x) & \text{ if } x \in \mathbb{R} \\
A & \text{ if } x=\infty \\
B & \text{ if } x=-\infty \end{cases}$
\end{center}

is continuous if and only if $f$ is continuous such that $\displaystyle \lim_{x \to \infty} f(x)=A$ and $\displaystyle \lim_{x \to -\infty} f(x)=B$ for some $A,B \in \overline{\mathbb{R}}$.
\end{thm*}

\begin{proof}
Note that $\overline{f}$ is continuous if and only if $\displaystyle \lim_{x \to c} \overline{f}(x)=\overline{f}(c)$ for all $c \in \overline{\mathbb{R}}$.  By defintion of $\overline{f}$ and the topology of $\overline{\mathbb{R}}$, $\displaystyle \lim_{x \to c} \overline{f}(x)=\displaystyle \lim_{x \to c} f(x)$ for all $c \in \overline{\mathbb{R}}$.  Thus, $\overline{f}$ is continuous if and only if $\displaystyle \lim_{x \to c} f(x)=\overline{f}(c)$ for all $c \in \overline{\mathbb{R}}$.  The latter condition is \PMlinkname{equivalent}{Equivalent3} to the hypotheses that $f$ is continuous on $\mathbb{R}$, $\displaystyle \lim_{x \to \infty}f(x)=A$, and $\displaystyle \lim_{x \to -\infty}f(x)=B$.
\end{proof}

Note that, without the universal assumption that $f$ is a function from $\mathbb{R}$ to $\mathbb{R}$, necessity holds, but sufficiency does not.  As a counterexample to sufficiency, consider the function $\overline{f} \colon \mathbb{R} \to \mathbb{R}$ defined by

\begin{center}
$\overline{f}(x)=\begin{cases}
\displaystyle \frac{1}{x^2} & \text{ if } x \in \mathbb{R} \setminus \{0\} \\
\infty & \text{ if } x=0 \\
0 & \text{ if } x=\pm \infty. \end{cases}$
\end{center}
%%%%%
%%%%%
\end{document}
