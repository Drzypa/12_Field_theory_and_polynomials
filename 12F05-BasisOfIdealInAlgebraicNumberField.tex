\documentclass[12pt]{article}
\usepackage{pmmeta}
\pmcanonicalname{BasisOfIdealInAlgebraicNumberField}
\pmcreated{2013-03-22 17:51:15}
\pmmodified{2013-03-22 17:51:15}
\pmowner{pahio}{2872}
\pmmodifier{pahio}{2872}
\pmtitle{basis of ideal in algebraic number field}
\pmrecord{9}{40329}
\pmprivacy{1}
\pmauthor{pahio}{2872}
\pmtype{Theorem}
\pmcomment{trigger rebuild}
\pmclassification{msc}{12F05}
\pmclassification{msc}{11R04}
\pmclassification{msc}{06B10}
\pmsynonym{basis of ideal in number field}{BasisOfIdealInAlgebraicNumberField}
\pmrelated{IntegralBasis}
\pmrelated{IdealNorm}
\pmrelated{AlgebraicNumberTheory}
\pmdefines{basis of ideal}
\pmdefines{ideal basis}
\pmdefines{discriminant of the ideal}

\endmetadata

% this is the default PlanetMath preamble.  as your knowledge
% of TeX increases, you will probably want to edit this, but
% it should be fine as is for beginners.

% almost certainly you want these
\usepackage{amssymb}
\usepackage{amsmath}
\usepackage{amsfonts}

% used for TeXing text within eps files
%\usepackage{psfrag}
% need this for including graphics (\includegraphics)
%\usepackage{graphicx}
% for neatly defining theorems and propositions
 \usepackage{amsthm}
% making logically defined graphics
%%%\usepackage{xypic}

% there are many more packages, add them here as you need them

% define commands here

\theoremstyle{definition}
\newtheorem*{thmplain}{Theorem}

\begin{document}
\textbf{Theorem.}\, Let $\mathcal{O}_K$ be the maximal order of the algebraic number field $K$ of degree $n$.\, Every ideal $\mathfrak{a}$  of $\mathcal{O}_K$ has a {\em basis}, i.e. there are in $\mathfrak{a}$ the linearly independent numbers\, $\alpha_1,\,\alpha_2,\,\ldots,\,\alpha_n$\, such that the numbers
$$m_1\alpha_1+m_2\alpha_2+\ldots+m_n\alpha_n,$$
where the $m_i$'s run all rational integers, form precisely all numbers of $\mathfrak{a}$.\, One has also
$$\mathfrak{a} = (\alpha_1,\,\alpha_2,\,\ldots,\,\alpha_n),$$
i.e. the basis of the ideal can be taken for the system of generators of the ideal.\\

Since\, $\{\alpha_1,\,\alpha_2,\,\ldots,\,\alpha_n\}$\, is a basis of the field extension $K/\mathbb{Q}$, any element of $\mathfrak{a}$ is uniquely expressible in the form $m_1\alpha_1+m_2\alpha_2+\ldots+m_n\alpha_n$.

It may be proven that all bases of an ideal $\mathfrak{a}$ have the same discriminant $\Delta(\alpha_1,\,\alpha_2,\,\ldots,\,\alpha_n)$, which is an integer; it is called the {\em discriminant of the ideal}.\, The discriminant of the ideal has the minimality property, that if $\beta_1,\,\beta_2,\,\ldots,\,\beta_n$ are some elements of $\mathfrak{a}$, then
$$\Delta(\beta_1,\,\beta_2,\,\ldots,\,\beta_n) \geqq \Delta(\alpha_1,\,\alpha_2,\,\ldots,\,\alpha_n)
\quad \mbox{or} \quad \Delta(\beta_1,\,\beta_2,\,\ldots,\,\beta_n) = 0$$
But if\, $\Delta(\beta_1,\,\beta_2,\,\ldots,\,\beta_n) = \Delta(\alpha_1,\,\alpha_2,\,\ldots,\,\alpha_n)$,\, then also the $\beta_i$'s form a basis of the ideal $\mathfrak{a}$.\\

\textbf{Example.}\, The integers of the quadratic field $\mathbb{Q}(\sqrt{2})$ are $l+m\sqrt{2}$ with\, $l,\,m \in \mathbb{Z}$.\, Determine a basis $\{\alpha_1,\,\alpha_2\}$\, and the discriminant of the ideal 
a)\, $(6\!-\!6\sqrt{2},\,9\!+\!6\sqrt{2})$,\, b) $(1\!-\!2\sqrt{2})$.

a) The ideal may be seen to be the principal ideal $(3)$, since the both generators are of the form\, $(l+m\sqrt{2})\cdot3$\, and on the other side,\, $3 = 0\cdot(6-6\sqrt{2})+(3-2\sqrt{2})(9+6\sqrt{2})$.\, Accordingly, any element of the ideal are of the form
$$(m_1+m_2\sqrt{2})\cdot3 = m_1\cdot3+m_2\cdot3\sqrt{2}$$
where $m_1$ and $m_2$ are rational integers.\, Thus we can infer that 
$$\alpha_1 = 3, \quad \alpha_2 = 3\sqrt{2}$$
is a basis of the ideal concerned.\, So its discriminant is
$$\Delta(\alpha_1,\,\alpha_2) = 
\left|\begin{matrix}
3 & 3\sqrt{2}\\
3 & -3\sqrt{2}
\end{matrix}\right|^2 = 648.$$
b) All elements of the ideal $(1-2\sqrt{2})$ have the form
\begin{align}
\alpha = (a+b\sqrt{2})(1-2\sqrt{2}) = (a-4b)+(b-2a)\sqrt{2} \quad 
\mbox{with\, } a,\,b \in\mathbb{Z}.
\end{align}
Especially the rational integers of the ideal satisfy\, $b-2a = 0$,\, when\, $b = 2a$\, and thus\, $\alpha = a-4\cdot2a = -7a$.\,  This means that in the presentation\, $\alpha = m_1\alpha_1+m_2\alpha_2$\, we can assume $\alpha_1$ to be $7$.\, Now the rational portion $a-4b$ in the form (1) of $\alpha$ should be splitted into two parts so that the first would be always divisible by 7 and the second by $b-2a$, i.e.\, $a-4b = 7m_1+(b-2a)x$; this equation may be written also as
$$(2x+1)a-(x+4)b = 7m_1.$$
By experimenting, one finds the simplest value\, $x = 3$,\, another would be\, $x = 10$.\, The first of these yields
$$\alpha = 7(a-b)+(b-2a)(3+\sqrt{2}) = m_1\cdot7+m_2(3+\sqrt{2}),$$
i.e. we have the basis
$$\alpha_1 = 7, \quad \alpha_2 = 3+\sqrt{2}.$$
The second alternative\, $x = 10$\, similarly would give
$$\alpha_1 = 7, \quad \alpha_2 = 10+\sqrt{2}.$$
For both alternatives,\, $\Delta(\alpha_1,\,\alpha_2) = 392$.




%%%%%
%%%%%
\end{document}
