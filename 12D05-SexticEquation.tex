\documentclass[12pt]{article}
\usepackage{pmmeta}
\pmcanonicalname{SexticEquation}
\pmcreated{2013-03-22 13:55:29}
\pmmodified{2013-03-22 13:55:29}
\pmowner{mathcam}{2727}
\pmmodifier{mathcam}{2727}
\pmtitle{sextic equation}
\pmrecord{12}{34682}
\pmprivacy{1}
\pmauthor{mathcam}{2727}
\pmtype{Definition}
\pmcomment{trigger rebuild}
\pmclassification{msc}{12D05}
\pmsynonym{sextic polynomial}{SexticEquation}
%\pmkeywords{sextic}
%\pmkeywords{sixth degree polynomial}

% this is the default PlanetMath preamble.  as your knowledge
% of TeX increases, you will probably want to edit this, but
% it should be fine as is for beginners.

% almost certainly you want these
\usepackage{amssymb}
\usepackage{amsmath}
\usepackage{amsfonts}
\usepackage{amsthm}

% used for TeXing text within eps files
%\usepackage{psfrag}
% need this for including graphics (\includegraphics)
%\usepackage{graphicx}
% for neatly defining theorems and propositions
%\usepackage{amsthm}
% making logically defined graphics
%%%\usepackage{xypic}

% there are many more packages, add them here as you need them

% define commands here

\newcommand{\mc}{\mathcal}
\newcommand{\mb}{\mathbb}
\newcommand{\mf}{\mathfrak}
\newcommand{\ol}{\overline}
\newcommand{\ra}{\rightarrow}
\newcommand{\la}{\leftarrow}
\newcommand{\La}{\Leftarrow}
\newcommand{\Ra}{\Rightarrow}
\newcommand{\nor}{\vartriangleleft}
\newcommand{\Gal}{\text{Gal}}
\newcommand{\GL}{\text{GL}}
\newcommand{\Z}{\mb{Z}}
\newcommand{\R}{\mb{R}}
\newcommand{\Q}{\mb{Q}}
\newcommand{\C}{\mb{C}}
\newcommand{\<}{\langle}
\renewcommand{\>}{\rangle}
\begin{document}
The sextic Equation is the univariate polynomial of the sixth degree:
\begin{align*}
x^6+ax^5+bx^4+cx^3+dx^2+ex+f = 0.
\end{align*}
Joubert showed in 1861 that this polynomial can be reduced without any form of accessory irrationalities to the 3-parameterized resolvent:
\begin{align*}
x^6+ax^4+bx^2+cx+c = 0.
\end{align*}
 
This polynomial was studied in great detail by Felix Klein and Robert Fricke in the 19th century and it is diectly related to the algebraic aspect of Hilbert's 13th Problem. Its solution has been reduced (by Klein) to the solution of the so-called Valentiner Form problem, a ternary form problem which seeks the ratios of the variables involved in the invariant system of the Valentiner Group of order 360. It can also be solved with a class of generalized hypergeometric series, by Birkeland's approach to algebraic equations. Scott Crass has given an explicit solution to the Valentiner problem by purely iterational methods, see
%%%%%
%%%%%
\end{document}
