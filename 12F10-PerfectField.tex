\documentclass[12pt]{article}
\usepackage{pmmeta}
\pmcanonicalname{PerfectField}
\pmcreated{2013-03-22 13:08:23}
\pmmodified{2013-03-22 13:08:23}
\pmowner{sleske}{997}
\pmmodifier{sleske}{997}
\pmtitle{perfect field}
\pmrecord{11}{33577}
\pmprivacy{1}
\pmauthor{sleske}{997}
\pmtype{Definition}
\pmcomment{trigger rebuild}
\pmclassification{msc}{12F10}
\pmrelated{SeparablePolynomial}
\pmrelated{ExtensionField}
\pmdefines{perfect}
\pmdefines{perfect ring}

\endmetadata

% this is the default PlanetMath preamble.  as your knowledge
% of TeX increases, you will probably want to edit this, but
% it should be fine as is for beginners.

% almost certainly you want these
\usepackage{amssymb}
\usepackage{amsmath}
\usepackage{amsfonts}

% used for TeXing text within eps files
%\usepackage{psfrag}
% need this for including graphics (\includegraphics)
%\usepackage{graphicx}
% for neatly defining theorems and propositions
%\usepackage{amsthm}
% making logically defined graphics
%%%\usepackage{xypic}

% there are many more packages, add them here as you need them

% define commands here
\begin{document}
A \emph{perfect field} is a field $K$ such that every algebraic extension field $L/K$ is separable over $K$.

All fields of characteristic 0 are perfect, so in particular the fields $\mathbb R$, $\mathbb C$ and $\mathbb Q$ are perfect.  If $K$ is a field of characteristic $p$ (with $p$ a prime number), then $K$ is perfect if and only if the Frobenius endomorphism $F$ on $K$, defined by
$$
F(x)=x^p\quad(x\in K),
$$
is an automorphism of $K$.  Since the Frobenius map is always injective, it is sufficient to check whether $F$ is surjective.  In particular, all finite fields are perfect (any injective endomorphism is also surjective).  Moreover, any field whose characteristic is nonzero that is \PMlinkname{algebraic}{AlgebraicExtension} over its prime subfield is perfect.  Thus, the only fields that are not perfect are those whose characteristic is nonzero and are transcendental over their prime subfield.

Similarly, a ring $R$ of characteristic $p$ is perfect if the endomorphism $x\mapsto x^p$ of $R$ is an {\it automorphism} (i.e., is surjective).
%%%%%
%%%%%
\end{document}
