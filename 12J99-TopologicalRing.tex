\documentclass[12pt]{article}
\usepackage{pmmeta}
\pmcanonicalname{TopologicalRing}
\pmcreated{2013-03-22 12:45:59}
\pmmodified{2013-03-22 12:45:59}
\pmowner{djao}{24}
\pmmodifier{djao}{24}
\pmtitle{topological ring}
\pmrecord{6}{33076}
\pmprivacy{1}
\pmauthor{djao}{24}
\pmtype{Definition}
\pmcomment{trigger rebuild}
\pmclassification{msc}{12J99}
\pmclassification{msc}{13J99}
\pmclassification{msc}{54H13}
\pmrelated{TopologicalGroup}
\pmrelated{TopologicalVectorSpace}
\pmdefines{topological field}
\pmdefines{topological division ring}

% this is the default PlanetMath preamble.  as your knowledge
% of TeX increases, you will probably want to edit this, but
% it should be fine as is for beginners.

% almost certainly you want these
\usepackage{amssymb}
\usepackage{amsmath}
\usepackage{amsfonts}

% used for TeXing text within eps files
%\usepackage{psfrag}
% need this for including graphics (\includegraphics)
%\usepackage{graphicx}
% for neatly defining theorems and propositions
%\usepackage{amsthm}
% making logically defined graphics
%%%\usepackage{xypic} 

% there are many more packages, add them here as you need them

% define commands here
\begin{document}
A ring $R$ which is a topological space is called a \emph{topological ring} if the addition, multiplication, and the additive inverse functions are continuous functions from $R \times R$ to $R$.

A \emph{topological division ring} is a topological ring such that the multiplicative inverse function is continuous away from $0$.  A \emph{topological field} is a topological division ring that is a field.

\textbf{Remark}.  It is easy to see that if $R$ contains the multiplicative identity $1$, then $R$ is a topological ring iff addition and multiplication are continuous.  This is true because the additive inverse of an element can be written as the product of the element and $-1$.  However, if $R$ does not contain $1$, it is necessary to impose the continuity condition on the additive inverse operation.

%%%%%
%%%%%
\end{document}
