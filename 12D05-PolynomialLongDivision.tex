\documentclass[12pt]{article}
\usepackage{pmmeta}
\pmcanonicalname{PolynomialLongDivision}
\pmcreated{2013-03-22 14:19:59}
\pmmodified{2013-03-22 14:19:59}
\pmowner{rm50}{10146}
\pmmodifier{rm50}{10146}
\pmtitle{polynomial long division}
\pmrecord{7}{35803}
\pmprivacy{1}
\pmauthor{rm50}{10146}
\pmtype{Definition}
\pmcomment{trigger rebuild}
\pmclassification{msc}{12D05}
\pmrelated{LongDivision}

% this is the default PlanetMath preamble.  as your knowledge
% of TeX increases, you will probably want to edit this, but
% it should be fine as is for beginners.

% almost certainly you want these
\usepackage{amssymb}
\usepackage{amsmath}
\usepackage{amsfonts}

% used for TeXing text within eps files
%\usepackage{psfrag}
% need this for including graphics (\includegraphics)
\usepackage{graphicx}
% for neatly defining theorems and propositions
%\usepackage{amsthm}
% making logically defined graphics
%%%\usepackage{xypic}

% there are many more packages, add them here as you need them

% define commands here
\begin{document}
Given two polynomials $a(x)$ and $b(x)$ \emph{polynomial (long) division} is a method for calculating $a(x)/b(x)$ that is, finding the polynomials $q(x)$ and $r(x)$ such that $a(x)=b(x)q(x)+r(x)$.

Here is an example to show the method.Let $a(x)=x^4-2x^3+5$ and $b(x)=x^2+3x-2$.
The method looks very similar to integer division since a polynomial $\sum_{i=0}^{n} c_ix^i$ is somewhat similar to an integer $\sum_{i=0}^{n} c_i  10^i$

In the initial setting we only write the coefficients, notice that $a(x)=x^4-2x^3+0x^2+0x+5$. It will then be

\includegraphics{pd.eps}

In the next step we se that $1/1=1$ and we multiply 1 3 -2 with 1 and then subtract the result. 

\includegraphics{pd1.eps}


Then we move down the next number, in this case a zero, and $-5/1=-5$ so we get -5, and multiply by -5 and subtract

\includegraphics{pd2.eps}

as a final result we get


\includegraphics{pd3.eps}


The result is $q(x)=1\ -5\  17$, which translates to $q(x)=x^2-5x+17$ and $r(x)=-61x+39$.

It is also possible to write the entire polynomial, that is, writing all the $x^i$'s. Like this

\includegraphics{pd4.eps}
%%%%%
%%%%%
\end{document}
