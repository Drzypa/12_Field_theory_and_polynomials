\documentclass[12pt]{article}
\usepackage{pmmeta}
\pmcanonicalname{ProofOfCharacterizationOfPerfectFields}
\pmcreated{2013-03-22 14:47:44}
\pmmodified{2013-03-22 14:47:44}
\pmowner{mclase}{549}
\pmmodifier{mclase}{549}
\pmtitle{proof of characterization of perfect fields}
\pmrecord{6}{36448}
\pmprivacy{1}
\pmauthor{mclase}{549}
\pmtype{Proof}
\pmcomment{trigger rebuild}
\pmclassification{msc}{12F10}

% this is the default PlanetMath preamble.  as your knowledge
% of TeX increases, you will probably want to edit this, but
% it should be fine as is for beginners.

% almost certainly you want these
\usepackage{amssymb}
\usepackage{amsmath}
\usepackage{amsfonts}

% used for TeXing text within eps files
%\usepackage{psfrag}
% need this for including graphics (\includegraphics)
%\usepackage{graphicx}
% for neatly defining theorems and propositions
%\usepackage{amsthm}
% making logically defined graphics
%%%\usepackage{xypic}

% there are many more packages, add them here as you need them

% define commands here

\newtheorem{proposition}{Proposition}
\newcommand{\cchar}{\operatorname{char}}
\begin{document}
\begin{proposition}
The following are equivalent:
\begin{enumerate}
\item[(a)] Every algebraic extension of $K$ is separable.
\item[(b)] Either $\cchar K = 0$ or $\cchar K = p$ and the Frobenius map is surjective.
\end{enumerate}
\end{proposition}

{\bf Proof.}
Suppose (a) and not (b).  Then we must have $\cchar K = p > 0$, and there must be $a \in K$
with no $p$-th root in $K$.  Let $L$ be a splitting field over $K$ for the polynomial $X^p - a$,
and let $\alpha \in L$ be a root of this polynomial.
Then $(X - \alpha)^p = X^p - \alpha^p = X^p - a$, which has coefficients in $K$.
This means that the minimum polynomial for $\alpha$ over $K$ must be a divisor of
$(X - \alpha)^p$ and so must have repeated roots.  This is not possible since $L$ is separable
over $K$.

Conversely, suppose (b) and not (a).
Let $\alpha$ be an element which is algebraic over $K$ but not separable.
Then its minimum polynomial $f$ must have a repeated root, and by replacing $\alpha$ by this
root if necessary, we may assume that $\alpha$ is a repeated root of $f$.
Now, $f'$ has coefficients in $K$ and also has $\alpha$ as a root.  Since it is of lower
degree than $f$, this is not possible unless $f' = 0$,
whence $\cchar K = p > 0$ and $f$ has the form:
$$f = x^{pn} + a_{n-1}x^{p(n-1)} + \dots + a_1 x^p + a_0.$$
with $a_o \neq 0$.
By (b), we may choose elements $b_i \in K$, for $0 \le i \le n-1$ such that ${b_i}^p = a_i$.
Then we may write $f$ as:
$$f = ( x^n + b_{n-1} x^{n-1} + \dots + b_1 x  + b_0)^p.$$
Since $f(\alpha) = 0$ and since the Frobenius map $x \mapsto x^p$ is injective, we see that
$$\alpha^n + b_{n-1} \alpha^{n-1} + \dots + b_1 \alpha + b_0 = 0$$
But then $\alpha$ is a root of the polynomial
$$x^n + b_{n-1} x^{n-1} + \dots + b_1 x  + b_0$$
which has coefficients in $K$, is non-zero (since $b_o \neq 0$), and has lower degree than $f$.
This contradicts the choice of $f$ as the minimum polynomial of $\alpha$.
%%%%%
%%%%%
\end{document}
