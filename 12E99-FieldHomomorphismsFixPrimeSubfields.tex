\documentclass[12pt]{article}
\usepackage{pmmeta}
\pmcanonicalname{FieldHomomorphismsFixPrimeSubfields}
\pmcreated{2013-03-22 16:19:54}
\pmmodified{2013-03-22 16:19:54}
\pmowner{Wkbj79}{1863}
\pmmodifier{Wkbj79}{1863}
\pmtitle{field homomorphisms fix prime subfields}
\pmrecord{10}{38461}
\pmprivacy{1}
\pmauthor{Wkbj79}{1863}
\pmtype{Theorem}
\pmcomment{trigger rebuild}
\pmclassification{msc}{12E99}

\usepackage{amssymb}
\usepackage{amsmath}
\usepackage{amsfonts}

\usepackage{psfrag}
\usepackage{graphicx}
\usepackage{amsthm}
%%\usepackage{xypic}

\newtheorem*{thm*}{Theorem}

\begin{document}
\begin{thm*}
Let $F$ and $K$ be fields having the same prime subfield $L$ and $\varphi \colon F \to K$ be a field homomorphism.  Then $\varphi$ fixes $L$.
\end{thm*}

\begin{proof}
Without loss of generality, it will be assumed that $L$ is either $\mathbb{Q}$ or $\mathbb{Z}/c\mathbb{Z}$.

Since $\varphi$ is a field homomorphism, $\varphi(0)=0$, $\varphi(1)=1$, and, for every $x \in F$, $\varphi(-x)=-\varphi(x)$.

Let $n \in \mathbb{Z}$ and $c$ be the characteristic of $F$.  Then

\begin{center}
\begin{tabular}{rl}
$\varphi(n)$ & $\equiv \varphi(\operatorname{sign}(n)|n|) \operatorname{mod} c$, where $\operatorname{sign}$ denotes the signum function \\
& $\displaystyle \equiv \operatorname{sign}(n)\varphi(|n|) \operatorname{mod} c$ \\
& $\displaystyle \equiv \operatorname{sign}(n)\varphi\left(\sum_{j=1}^{|n|} 1\right) \operatorname{mod} c$ \\
& $\displaystyle \equiv \operatorname{sign}(n)\sum_{j=1}^{|n|} \varphi(1) \operatorname{mod} c$ \\
& $\displaystyle \equiv \operatorname{sign}(n)\sum_{j=1}^{|n|} 1 \operatorname{mod} c$ \\
& $\equiv \operatorname{sign}(n)|n| \operatorname{mod} c$ \\
& $\equiv n \operatorname{mod} c$. \end{tabular}
\end{center}

This \PMlinkescapetext{completes} the proof in the case that $c$ is prime.

Now consider $c=0$.  Let $x \in \mathbb{Q}$.  Then there exist $a,b \in \mathbb{Z}$ with $b>0$ such that $\displaystyle x=\frac{a}{b}$.  Thus, $\displaystyle b\varphi(x)=\sum_{j=1}^b\varphi\left(\frac{a}{b}\right)=\varphi\left(\sum_{j=1}^b \frac{a}{b}\right)=\varphi(a)=a$.  Therefore, $\displaystyle \varphi(x)=\frac{a}{b}=x$.  Hence, $\varphi$ fixes $\mathbb{Q}$.
\end{proof}
%%%%%
%%%%%
\end{document}
