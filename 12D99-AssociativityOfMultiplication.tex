\documentclass[12pt]{article}
\usepackage{pmmeta}
\pmcanonicalname{AssociativityOfMultiplication}
\pmcreated{2013-03-22 15:09:22}
\pmmodified{2013-03-22 15:09:22}
\pmowner{pahio}{2872}
\pmmodifier{pahio}{2872}
\pmtitle{associativity of multiplication}
\pmrecord{5}{36904}
\pmprivacy{1}
\pmauthor{pahio}{2872}
\pmtype{Application}
\pmcomment{trigger rebuild}
\pmclassification{msc}{12D99}
\pmclassification{msc}{00A35}

\endmetadata

% this is the default PlanetMath preamble.  as your knowledge
% of TeX increases, you will probably want to edit this, but
% it should be fine as is for beginners.

% almost certainly you want these
\usepackage{amssymb}
\usepackage{amsmath}
\usepackage{amsfonts}

% used for TeXing text within eps files
%\usepackage{psfrag}
% need this for including graphics (\includegraphics)
%\usepackage{graphicx}
% for neatly defining theorems and propositions
%\usepackage{amsthm}
% making logically defined graphics
%%%\usepackage{xypic}

% there are many more packages, add them here as you need them

% define commands here
\begin{document}
It's important to know the following interpretation of the associative law
\begin{align}
a\cdot(b\cdot c) = (a\cdot b)\cdot c
\end{align}
of multiplication in arithmetics and elementary algebra:
 
{\em A product ($b\cdot c$) is multiplied by a number ($a$) so that only one \PMlinkescapetext{factor} ($b$) of the product is multiplied by that number.}

This rule is sometimes violated even in high school \PMlinkescapetext{level. \,A pupil may calculate} e.g. like
$$10 \cdot 2.5 \cdot 0.3 = 25 \cdot 3 = 75,$$
which is wrong. \,Or when solving an equation like
$$x\cdot\frac{2x-1}{3} = 1$$
one would like to multiply both sides by 3 for removing the denominator, getting perhaps
$$3x(2x-1) = 3;$$
then the both \PMlinkescapetext{factors} of left side have incorrectly been multiplied by 3.

The reason of such mistakes is very likely that one confuses the associative law with the distributive law; \,cf. (1) with this latter
\begin{align}
a\cdot(b+c) = a\cdot b+a\cdot c,
\end{align}
which \PMlinkescapetext{contains} two different operations, multiplication and addition;  both {\em addends must be multiplied separately}.
%%%%%
%%%%%
\end{document}
