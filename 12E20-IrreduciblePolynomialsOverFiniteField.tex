\documentclass[12pt]{article}
\usepackage{pmmeta}
\pmcanonicalname{IrreduciblePolynomialsOverFiniteField}
\pmcreated{2013-03-22 17:43:14}
\pmmodified{2013-03-22 17:43:14}
\pmowner{pahio}{2872}
\pmmodifier{pahio}{2872}
\pmtitle{irreducible polynomials over finite field}
\pmrecord{7}{40166}
\pmprivacy{1}
\pmauthor{pahio}{2872}
\pmtype{Theorem}
\pmcomment{trigger rebuild}
\pmclassification{msc}{12E20}
\pmclassification{msc}{11T99}
\pmrelated{FiniteField}

% this is the default PlanetMath preamble.  as your knowledge
% of TeX increases, you will probably want to edit this, but
% it should be fine as is for beginners.

% almost certainly you want these
\usepackage{amssymb}
\usepackage{amsmath}
\usepackage{amsfonts}

% used for TeXing text within eps files
%\usepackage{psfrag}
% need this for including graphics (\includegraphics)
%\usepackage{graphicx}
% for neatly defining theorems and propositions
 \usepackage{amsthm}
% making logically defined graphics
%%%\usepackage{xypic}

% there are many more packages, add them here as you need them

% define commands here

\theoremstyle{definition}
\newtheorem*{thmplain}{Theorem}

\begin{document}
\textbf{Theorem.}\, Over a finite field $F$, there exist irreducible polynomials of any degree.\\

{\em Proof.}\, Let $n$ be a positive integer, $p$ be the characteristic of $F$, $\mathbb{F}_p$ be the prime subfield, and $p^r$ be the \PMlinkname{order}{FiniteField} of the field $F$.\, Since $p^r\!-\!1$ is a divisor of $p^{rn}\!-\!1$, the zeros of the polynomial $X^{p^r}\!-\!X$ form in\, $G := \mathbb{F}_{p^{rn}}$\, a subfield isomorphic to $F$.\, Thus, one can regard $F$ as a subfield of $G$.\, Because
$$[G\!:\!F] = \frac{[G\!:\!\mathbb{F}_p]}{[F\!:\!\mathbb{F}_p]} = \frac{rn}{r} = n,$$
the minimal polynomial of a primitive element of the field extension $G/F$ is an irreducible polynomial of degree $n$ in the ring $F[X].$
%%%%%
%%%%%
\end{document}
