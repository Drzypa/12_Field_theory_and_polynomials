\documentclass[12pt]{article}
\usepackage{pmmeta}
\pmcanonicalname{ProofOfFundamentalTheoremOfGaloisTheory}
\pmcreated{2013-03-22 14:26:38}
\pmmodified{2013-03-22 14:26:38}
\pmowner{pbruin}{1001}
\pmmodifier{pbruin}{1001}
\pmtitle{proof of fundamental theorem of Galois theory}
\pmrecord{5}{35958}
\pmprivacy{1}
\pmauthor{pbruin}{1001}
\pmtype{Proof}
\pmcomment{trigger rebuild}
\pmclassification{msc}{12F10}
\pmclassification{msc}{11R32}
\pmclassification{msc}{11S20}
\pmclassification{msc}{13B05}

% this is the default PlanetMath preamble.  as your knowledge
% of TeX increases, you will probably want to edit this, but
% it should be fine as is for beginners.

% almost certainly you want these
\usepackage{amssymb}
\usepackage{amsmath}
\usepackage{amsfonts}

% used for TeXing text within eps files
%\usepackage{psfrag}
% need this for including graphics (\includegraphics)
%\usepackage{graphicx}
% for neatly defining theorems and propositions
\usepackage{amsthm}
% making logically defined graphics
%%%\usepackage{xypic}

% there are many more packages, add them here as you need them

% define commands here
\DeclareMathOperator{\Gal}{Gal}
\newtheorem{lemma}{Lemma}

\begin{document}
The theorem is a consequence of the following lemmas, roughly
corresponding to the various assertions in the theorem.  We assume
$L/F$ to be a finite-dimensional Galois extension of fields with
Galois group
\[
G=\Gal(L/F).
\]
The first two lemmas establish the correspondence between subgroups of
$G$ and extension fields of $F$ contained in $L$.

\begin{lemma}
\label{lemma1}
Let $K$ be an extension field of $F$ contained in $L$.  Then $L$ is
Galois over $K$, and $\Gal(L/K)$ is a subgroup of $G$.
\end{lemma}

\begin{proof}
Note that $L/F$ is normal and separable because it is a Galois
extension; it remains to prove that $L/K$ is also normal and
separable.  Since $L$ is normal and finite over $F$, it is the
splitting field of a polynomial $f\in F[X]$ over $F$.  Now $L$ is also
the splitting field of $f$ over $K$ (because $F\subset K\subset L$),
so $L/K$ is normal.

To see that $L/K$ is also separable, suppose $\alpha\in L$, and let
$f^\alpha_F\in F[X]$ be its minimal polynomial over $F$.  Then the
minimal polynomial $f^\alpha_K$ of $\alpha$ over $K$ divides
$f^\alpha_F$, which has no double roots in its splitting field by the
separability of $L/F$.  Therefore $f^\alpha_K$ has no double roots in
its splitting field for any $\alpha\in L$, so $L$ is separable over
$K$.

The assertion that $\Gal(L/K)$ is a subgroup of $G$ is clear from the
fact that $K\supset F$.
\end{proof}

\begin{lemma}
\label{lemma2}
The function $\phi$ from the set of extension fields of $F$ contained
in $L$ to the set of subgroups of $G$ defined by
\[
\phi(K)=\Gal(L/K)
\]
is an inclusion-reversing bijection.  The inverse is given by
\[
\phi^{-1}(H)=L^H,
\]
where $L^H$ is the fixed field of $H$ in $L$.
\end{lemma}

\begin{proof}
The definition of $\phi$ makes sense because of Lemma \ref{lemma1}.
The \PMlinkescapetext{identities}
\[
\phi^{-1}\circ\phi(K)=K\quad\hbox{and}\quad
\phi\circ\phi^{-1}(H)=H
\]
for all subgroups $H\subset G$ and all fields $K$ with $F\subset
K\subset L$ follow from the properties of the Galois group.  The fixed
field of $\Gal(L/K)$ is precisely $K$; on the other hand, since $L^H$
is the fixed field of $H$ in $L$, $H$ is the Galois group of $L/L^H$.

For extensions $K$ and $K'$ of $F$ with $F\subset K\subset K'\subset
L$, we have
\[
\sigma\in\Gal(L/K')\iff\sigma\in\Gal(L/K),
\]
so $\phi(K)\supset\phi(K')$.  This shows that $\phi$ is
inclusion-reversing.
\end{proof}

The following lemmas show that normal subextensions of $L/F$ are
Galois extensions and that their Galois groups are quotient groups of
$G$.

\begin{lemma}
\label{lemma3}
Let $H$ be a subgroup of $G$.  Then the following are equivalent:
\begin{enumerate}
\item $L^H$ is normal over $F$.
\item $\sigma\left(L^H\right)=L^H$ for all $\sigma\in G$.
\item $\sigma H\sigma^{-1}=H$ for all $\sigma\in G$.
\end{enumerate}
In particular, $L^H$ is normal over $F$ if and only if $H$ is a normal
subgroup of $G$.
\end{lemma}

\begin{proof}
$1\Rightarrow2$: Since for all $\sigma\in G$ and $\alpha\in L^H$,
$\sigma(\alpha)$ is a zero of the minimal polynomial of $\alpha$ over
$F$, we have $\sigma(\alpha)\in L^H$ by the
\PMlinkescapetext{normality} of $L^H/F$.

$2\Rightarrow3$: For all $\sigma\in G,\tau\in H$ the equality
\[
\sigma\tau\sigma^{-1}(x)=\sigma\sigma^{-1}(x)=x
\]
holds for all $x\in L^H$ (from the assumption it follows that
$\sigma^{-1}(x)\in L^H$, which is fixed by $\tau$).  This implies that
\[
\sigma\tau\sigma^{-1}\in\Gal(L/L^H)=H
\]
for all $\sigma\in G,\tau\in H$.

$3\Rightarrow1$: Let $\alpha\in L^H$, and let $f$ be the minimal
polynomial of $\alpha$ over $F$.  Since $L/F$ is normal, $f$ splits
into linear factors in $L[X]$.  Suppose $\alpha'\in L$ is another zero
of $f$, and let $\sigma\in G$ be such that $\sigma(\alpha')=\alpha$
(such a $\sigma$ always exists).  By assumption, for all $\tau\in H$
we have $\tau':=\sigma\tau\sigma^{-1}\in H$, so that
\[
\tau(\alpha')=\sigma^{-1}\tau'\sigma(\alpha')
=\sigma^{-1}\tau'(\alpha)=\sigma^{-1}(\alpha)=\alpha'.
\]
This shows that $\alpha'$ lies in $L^H$ as well, so $f$ splits in
$L^H[X]$.  We conclude that $L^H$ is normal over $F$.
\end{proof}

\begin{lemma}
Let $H$ be a normal subgroup of $G$.  Then $L^H$ is a Galois extension
of $F$, and the homomorphism
\begin{eqnarray*}
r\colon G&\to&\Gal(L^H/F) \\
\sigma&\mapsto&\sigma\vert_{L^H}
\end{eqnarray*}
induces a natural identification
\[
\Gal(L^H/F)\cong G/H.
\]
\end{lemma}

\begin{proof}
By Lemma \ref{lemma3}, $L^H$ is normal over $F$, and because a
subextension of a separable extension is separable, $L^H/F$ is a
Galois extension.

The map $r$ is well-defined by the implication $1\Rightarrow2$ from
Lemma \ref{lemma3}.  It is surjective since every automorphism of
$L^H$ that fixes $F$ can be extended to an automorphism of $L$ (if
$L\ne L^H$, for example, we can choose an $\alpha\in L\setminus L^H$
such that $L=L^H(\alpha)$ using the primitive element theorem, and we
can extend $\sigma\in\Gal(L^H/F)$ to $L$ by putting
$\sigma(\alpha)=\alpha$).  The kernel of $r$ is clearly equal to $H$,
so the first isomorphism theorem gives the claimed identification.
\end{proof}

%%%%%
%%%%%
\end{document}
