\documentclass[12pt]{article}
\usepackage{pmmeta}
\pmcanonicalname{LinearlyDisjoint}
\pmcreated{2013-03-22 14:19:28}
\pmmodified{2013-03-22 14:19:28}
\pmowner{CWoo}{3771}
\pmmodifier{CWoo}{3771}
\pmtitle{linearly disjoint}
\pmrecord{7}{35793}
\pmprivacy{1}
\pmauthor{CWoo}{3771}
\pmtype{Definition}
\pmcomment{trigger rebuild}
\pmclassification{msc}{12F20}

\endmetadata

% this is the default PlanetMath preamble.  as your knowledge
% of TeX increases, you will probably want to edit this, but
% it should be fine as is for beginners.

% almost certainly you want these
\usepackage{amssymb}
\usepackage{amsmath}
\usepackage{amsfonts}

% used for TeXing text within eps files
%\usepackage{psfrag}
% need this for including graphics (\includegraphics)
%\usepackage{graphicx}
% for neatly defining theorems and propositions
%\usepackage{amsthm}
% making logically defined graphics
%%%\usepackage{xypic}

% there are many more packages, add them here as you need them

% define commands here
\begin{document}
Let $E$ and $F$ be subfields of $L$, each containing a field $K$.  $E$ is said to be \emph{linearly disjoint} from $F$ over $K$ if every subset of $E$ linearly independent over $K$ is also linearly independent over $F$.

\textbf{Remark.}  If $E$ is linearly disjoint from $F$ over $K$, then $F$ is linearly disjoint from $E$ over $K$.  Then one can speak of $E$ and $F$ being linearly disjoint over $K$ without causing any confusions.
%%%%%
%%%%%
\end{document}
