\documentclass[12pt]{article}
\usepackage{pmmeta}
\pmcanonicalname{CasusIrreducibilis}
\pmcreated{2013-03-22 15:21:00}
\pmmodified{2013-03-22 15:21:00}
\pmowner{pahio}{2872}
\pmmodifier{pahio}{2872}
\pmtitle{casus irreducibilis}
\pmrecord{12}{37171}
\pmprivacy{1}
\pmauthor{pahio}{2872}
\pmtype{Theorem}
\pmcomment{trigger rebuild}
\pmclassification{msc}{12F10}
%\pmkeywords{irreducible polynomial}
%\pmkeywords{roots real}
\pmrelated{RadicalExtension}
\pmrelated{CardanosFormulae}
\pmrelated{TakingSquareRootAlgebraically}
\pmrelated{EulersDerivationOfTheQuarticFormula}

\endmetadata

% this is the default PlanetMath preamble.  as your knowledge
% of TeX increases, you will probably want to edit this, but
% it should be fine as is for beginners.

% almost certainly you want these
\usepackage{amssymb}
\usepackage{amsmath}
\usepackage{amsfonts}

% used for TeXing text within eps files
%\usepackage{psfrag}
% need this for including graphics (\includegraphics)
%\usepackage{graphicx}
% for neatly defining theorems and propositions
 \usepackage{amsthm}
% making logically defined graphics
%%%\usepackage{xypic}

% there are many more packages, add them here as you need them

% define commands here

\theoremstyle{definition}
\newtheorem*{thmplain}{Theorem}
\begin{document}
\PMlinkescapeword{irreducible}

Let the polynomial
$$P(x) := x^n+a_1x^{n-1}+\ldots+a_n$$
with complex coefficients $a_j$ be \PMlinkname{irreducible}{IrreduciblePolynomial2}, i.e. irreducible in the field  $\mathbb{Q}(a_1,\,\ldots,\,a_n)$\, of its coefficients.\, If the equation \,$P(x) = 0$\, can be \PMlinkname{solved algebraically}{AlgebraicallySolvable} and if all of its roots are real, then no root may be expressed with the numbers $a_j$ using mere real \PMlinkname{radicals}{NthRoot} unless the \PMlinkname{degree}{AlgebraicEquation} $n$ of the equation is an \PMlinkname{integer power}{GeneralAssociativity} of 2.

\begin{thebibliography}{9}
\bibitem{K.V.} {\sc K. V\"ais\"al\"a}: {\em Lukuteorian ja korkeamman algebran alkeet}.\, Tiedekirjasto No. 17.\quad  Kustannusosakeyhti\"o Otava, Helsinki (1950).
\end{thebibliography}
%%%%%
%%%%%
\end{document}
