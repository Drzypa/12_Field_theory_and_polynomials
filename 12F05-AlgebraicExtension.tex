\documentclass[12pt]{article}
\usepackage{pmmeta}
\pmcanonicalname{AlgebraicExtension}
\pmcreated{2013-03-22 13:57:27}
\pmmodified{2013-03-22 13:57:27}
\pmowner{alozano}{2414}
\pmmodifier{alozano}{2414}
\pmtitle{algebraic extension}
\pmrecord{7}{34724}
\pmprivacy{1}
\pmauthor{alozano}{2414}
\pmtype{Definition}
\pmcomment{trigger rebuild}
\pmclassification{msc}{12F05}
\pmsynonym{algebraic field extension}{AlgebraicExtension}
%\pmkeywords{algebraic}
%\pmkeywords{root of polynomial}
\pmrelated{Algebraic}
\pmrelated{FiniteExtension}
\pmrelated{AFiniteExtensionOfFieldsIsAnAlgebraicExtension}
\pmrelated{ProofOfTranscendentalRootTheorem}
\pmrelated{EquivalentConditionsForNormalityOfAFieldExtension}
\pmdefines{examples of field extension}
\pmdefines{transcendental extension}

% this is the default PlanetMath preamble.  as your knowledge
% of TeX increases, you will probably want to edit this, but
% it should be fine as is for beginners.

% almost certainly you want these
\usepackage{amssymb}
\usepackage{amsmath}
\usepackage{amsthm}
\usepackage{amsfonts}

% used for TeXing text within eps files
%\usepackage{psfrag}
% need this for including graphics (\includegraphics)
%\usepackage{graphicx}
% for neatly defining theorems and propositions
%\usepackage{amsthm}
% making logically defined graphics
%%%\usepackage{xypic}

% there are many more packages, add them here as you need them

% define commands here

\newtheorem{thm}{Theorem}
\newtheorem{defn}{Definition}
\newtheorem{prop}{Proposition}
\newtheorem{lemma}{Lemma}
\newtheorem{cor}{Corollary}

% Some sets
\newcommand{\Nats}{\mathbb{N}}
\newcommand{\Ints}{\mathbb{Z}}
\newcommand{\Reals}{\mathbb{R}}
\newcommand{\Complex}{\mathbb{C}}
\newcommand{\Rats}{\mathbb{Q}}
\begin{document}
\begin{defn}
Let $L/K$ be an extension of fields. $L/K$ is said to be an
\emph{algebraic extension} of fields if every element of $L$ is
algebraic over $K$. If $L/K$ is not algebraic then we say that it is a transcendental extension of fields.
\end{defn}

{\bf Examples: }
\begin{enumerate}
\item Let $L=\Rats(\sqrt{2})$. The extension $L/\Rats$ is an
algebraic extension. Indeed, any element $\alpha\in L$ is of the
form
$$\alpha=q+t\sqrt{2}\in L$$
for some $q,t\in\Rats$. Then $\alpha\in L$ is a root of
$$X^2-2qX+q^2-2t^2=0$$

\item The field extension $\Reals/ \Rats$ is not an algebraic
extension. For example, $\pi\in \Reals$ is a transcendental number
over $\Rats$ (see pi). So $\Reals/\Rats$ is a transcendental extension of fields.

\item Let $K$ be a field and denote by $\overline{K}$ the
algebraic closure of $K$. Then the extension $\overline{K}/K$ is
algebraic.

\item In general, a finite extension of fields is an algebraic
extension. However, the converse is not true. The extension
$\overline{\Rats}/\Rats$ is \emph{far} from finite.

\item The extension $\Rats(\pi)/\Rats$ is transcendental because $\pi$ is a transcendental number, i.e. $\pi$ is not the root of any polynomial $p(x)\in \Rats[x]$.
\end{enumerate}


%%%%%
%%%%%
\end{document}
