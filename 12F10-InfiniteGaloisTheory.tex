\documentclass[12pt]{article}
\usepackage{pmmeta}
\pmcanonicalname{InfiniteGaloisTheory}
\pmcreated{2013-03-22 12:39:06}
\pmmodified{2013-03-22 12:39:06}
\pmowner{djao}{24}
\pmmodifier{djao}{24}
\pmtitle{infinite Galois theory}
\pmrecord{7}{32918}
\pmprivacy{1}
\pmauthor{djao}{24}
\pmtype{Topic}
\pmcomment{trigger rebuild}
\pmclassification{msc}{12F10}
\pmclassification{msc}{13B05}
\pmrelated{FundamentalTheoremOfGaloisTheory}
\pmrelated{GaloisGroup}
\pmrelated{InverseLimit}
\pmdefines{Krull topology}

% this is the default PlanetMath preamble.  as your knowledge
% of TeX increases, you will probably want to edit this, but
% it should be fine as is for beginners.

% almost certainly you want these
\usepackage{amssymb}
\usepackage{amsmath}
\usepackage{amsfonts}

% used for TeXing text within eps files
%\usepackage{psfrag}
% need this for including graphics (\includegraphics)
%\usepackage{graphicx}
% for neatly defining theorems and propositions
\usepackage{amsthm}
% making logically defined graphics
%%%\usepackage{xypic} 

% there are many more packages, add them here as you need them

% define commands here

\newcommand{\p}{{\mathfrak{p}}}
\newcommand{\m}{{\mathfrak{m}}}
\newcommand{\M}{{\mathfrak{M}}}
\renewcommand{\P}{{\mathfrak{P}}}
\newcommand{\C}{\mathbb{C}}
\newcommand{\R}{\mathbb{R}}
\newcommand{\Z}{\mathbb{Z}}
\newcommand{\Q}{\mathbb{Q}}
\newcommand{\N}{\mathbb{N}}
\renewcommand{\H}{\mathcal{H}}
\newcommand{\A}{\mathcal{A}}
\renewcommand{\c}{\mathcal{C}}
\renewcommand{\O}{\mathcal{O}}
\newcommand{\D}{\mathcal{D}}
\newcommand{\lra}{\longrightarrow}
\renewcommand{\div}{\mid}
\newcommand{\res}{\operatorname{res}}
\newcommand{\Spec}{\operatorname{Spec}}
\newcommand{\Gal}{\operatorname{Gal}}
\newcommand{\id}{\operatorname{id}}
\newcommand{\diff}{\operatorname{diff}}
\newcommand{\incl}{\operatorname{incl}}
\newcommand{\Hom}{\operatorname{Hom}}
\renewcommand{\Re}{\operatorname{Re}}
\newcommand{\intersect}{\cap}
\newcommand{\union}{\cup}
\newcommand{\bigintersect}{\bigcap}
\newcommand{\bigunion}{\bigcup}
\newcommand{\ilim}{\,\underset{\longleftarrow}{\lim}\,}

\newtheorem{theorem}{Theorem}
\newtheorem{proposition}[theorem]{Proposition}
\newtheorem{lemma}[theorem]{Lemma}
\newtheorem{corollary}[theorem]{Corollary}

\theoremstyle{definition}
\newtheorem{definition}[theorem]{Definition}
\newtheorem{example}[theorem]{Example}
\begin{document}
Let $L/F$ be a Galois extension, not necessarily finite dimensional.

\section{Topology on the Galois group}

Recall that the \emph{Galois group} $G := \Gal(L/F)$ of $L/F$ is the
group of all field automorphisms $\sigma: L \lra L$ that restrict to
the identity map on $F$, under the group operation of composition. In
the case where the extension $L/F$ is infinite dimensional, the group
$G$ comes equipped with a natural topology, which plays a key role in
the statement of the Galois correspondence.

We define a subset $U$ of $G$ to be open if, for each $\sigma \in U$,
there exists an intermediate field $K \subset L$ such that
\begin{itemize}
\item The degree $[K:F]$ is finite,
\item If $\sigma'$ is another element of $G$, and the restrictions
  $\sigma|_K$ and $\sigma'|_K$ are equal, then $\sigma' \in U$.
\end{itemize}

The resulting collection of open sets forms a topology on $G$, called
the \emph{Krull topology}, and $G$ is a topological group under the
Krull topology.  Another way to define the topology is to state that
the subgroups $\Gal(L/K)$ for finite extensions $K/F$ form a neighborhood
basis for $\Gal(L/F)$ at the identity.

\section{Inverse limit structure}

In this section we exhibit the group $G$ as a projective limit of an
inverse system of finite groups. This construction shows that the
Galois group $G$ is actually a profinite group.

Let $\A$ denote the set of finite normal extensions $K$ of $F$ which
are contained in $L$. The set $\A$ is a partially ordered set under
the inclusion relation. Form the inverse limit
$$
\Gamma := \ilim \Gal(K/F) \subset \prod_{K \in \A} \Gal(K/F)
$$
consisting, as usual, of the set of all $(\sigma_K) \in \prod_K
\Gal(K/F)$ such that $\sigma_{K'}|_K = \sigma_K$ for all $K,K' \in \A$
with $K \subset K'$. We make $\Gamma$ into a topological space by
putting the discrete topology on each finite set $\Gal(K/F)$ and
giving $\Gamma$ the subspace topology induced by the product topology
on $\prod_K \Gal(K/F)$. The group $\Gamma$ is a closed subset of the
compact group $\prod_K \Gal(K/F)$, and is therefore compact.

Let
$$
\phi: G \lra \prod_{K \in \A} \Gal(K/F)
$$
be the group homomorphism which sends an element $\sigma \in G$ to the
element $(\sigma_K)$ of $\prod_K \Gal(K/F)$ whose $K$--th coordinate
is the automorphism $\sigma|_K \in \Gal(K/F)$. Then the function
$\phi$ has image equal to $\Gamma$ and in fact is a homeomorphism
between $G$ and $\Gamma$. Since $\Gamma$ is profinite, it follows that
$G$ is profinite as well.

\section{The Galois correspondence}

\begin{theorem}[Galois correspondence for infinite extensions]
Let $G$, $L$, $F$ be as before. For every closed subgroup $H$ of $G$,
  let $L^H$ denote the fixed field of $H$. The correspondence
$$
K \mapsto \Gal(L/K),
$$
defined for all intermediate field extensions $F \subset K \subset L$,
is an inclusion reversing bijection between the set of all
intermediate extensions $K$ and the set of all closed subgroups of
$G$. Its inverse is the correspondence
$$
H \mapsto L^H,
$$
defined for all closed subgroups $H$ of $G$. The extension $K/F$ is
normal if and only if $\Gal(L/K)$ is a normal subgroup of $G$, and in
this case the restriction map
$$
G \lra \Gal(K/F)
$$
has kernel $\Gal(L/K)$.
\end{theorem}

\begin{theorem}[Galois correspondence for finite subextensions]
Let $G$, $L$, $F$ be as before.
\begin{itemize}
\item Every open subgroup $H \subset G$ is closed and has finite index
  in $G$.
\item If $H \subset G$ is an open subgroup, then the field extension
  $L^H/F$ is finite.
\item For every intermediate field $K$ with $[K:F]$ finite, the Galois
  group $\Gal(L/K)$ is an open subgroup of $G$.
\end{itemize}
\end{theorem}
%%%%%
%%%%%
\end{document}
