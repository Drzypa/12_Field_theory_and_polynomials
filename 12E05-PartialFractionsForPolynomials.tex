\documentclass[12pt]{article}
\usepackage{pmmeta}
\pmcanonicalname{PartialFractionsForPolynomials}
\pmcreated{2013-03-22 15:40:22}
\pmmodified{2013-03-22 15:40:22}
\pmowner{stevecheng}{10074}
\pmmodifier{stevecheng}{10074}
\pmtitle{partial fractions for polynomials}
\pmrecord{6}{37613}
\pmprivacy{1}
\pmauthor{stevecheng}{10074}
\pmtype{Result}
\pmcomment{trigger rebuild}
\pmclassification{msc}{12E05}
\pmsynonym{partial fraction decomposition of rational functions}{PartialFractionsForPolynomials}
\pmsynonym{partial fractions for rational functions}{PartialFractionsForPolynomials}
%\pmkeywords{partial fraction decomposition}
\pmrelated{PartialFractionsOfExpressions}
\pmrelated{ALectureOnThePartialFractionDecompositionMethod}

\endmetadata

\usepackage{amssymb}
\usepackage{amsmath}
\usepackage{amsfonts}
\usepackage{amsthm}
\usepackage{enumerate}

% used for TeXing text within eps files
%\usepackage{psfrag}
% need this for including graphics (\includegraphics)
%\usepackage{graphicx}
% making logically defined graphics
%%%\usepackage{xypic}

% define commands here
\providecommand{\defnterm}[1]{\emph{#1}}

\newtheorem{thm}{Theorem}
\begin{document}
This entry precisely states and proves 
the existence and uniqueness of partial fraction decompositions
of ratios of polynomials of a single variable, with coefficients over a field.  

The theory is used for, for example,
the method of \PMlinkname{partial fraction decomposition for integrating 
rational functions over the reals}{ALectureOnThePartialFractionDecompositionMethod}.

The proofs involve fairly elementary algebra only. Although we refer
to Euclidean domains in our proofs, the reader who is not familiar with abstract 
algebra may simply read that as ``set of polynomials''
(which is one particular Euclidean domain).

Also note that the proofs themselves furnish a method for actually computing
the partial fraction decomposition, as a finite-time algorithm,
provided the irreducible factorization of the denominator is known.
It is not an efficient way to find the partial fraction decomposition; usually 
one uses instead the method of making substitutions into the polynomials, 
to derive linear constraints on the coefficients.
But what is important is that the existence proofs here 
\emph{justify} the substitution method.  The uniqueness property proved here
might also simplify some calculations: it shows that we never have
to consider multiple solutions for the coefficients in the decomposition.


\begin{thm}
\label{thm:powers-uniq}
Let $p$ and $q \neq 0$ be polynomials over a field,
and $n$ be any positive integer.
Then there exist unique polynomials 
$\alpha_1, \dotsc, \alpha_n, \beta$ such that
\begin{equation}
\label{eq:powers-uniq}
\frac{p}{q^n} = \beta + \frac{\alpha_1}{q} + \frac{\alpha_2}{q^2} + \dotsb + \frac{\alpha_n}{q^n}\,, \quad
\deg \alpha_j < \deg q\,.
\end{equation}

\begin{proof}
Existence has already been proven as a special case of partial fractions in Euclidean domains; 
we now prove uniqueness.  
Suppose equation \eqref{eq:powers-uniq} has been given.
Multiplying by $q^n$ and rearranging, 
\[
p = \beta q^n + r_1 \,, \quad r_1 = \alpha_1 q^{n-1} + \dotsb + \alpha_n\,, \quad \deg r_1 < \deg q^n\,.
\]
But according to the division algorithm for polynomials (also known as long division), the quotient and remainder polynomial
after a division (by $q^n$ in this case) are unique.  
So $\beta$ must be uniquely determined.
Then we can rearrange:
\[
p - \beta \, q^n = \alpha_1 q^{n-1} + r_2\,, \quad r_2 = 
\alpha_2 q^{n-2} + \dotsb + \alpha_n\,, \quad \deg r_2 < \deg q^{n-1}\,.
\]
By uniqueness of division again (by $q^{n-1}$), $\alpha_1$ is determined.
Repeating this process, we see that all the polynomials $\alpha_j$ and $\beta$
are uniquely determined.
\end{proof}
\end{thm}


\begin{thm}
\label{thm:partial-fractions-uniq}
Let $p$ and $q \neq 0$ be polynomials over a field.
Let $q = \phi_1^{n_1} \, \phi_2^{n_2} \, \dotsb \, \phi_k^{n_k}$
be the factorization of $q$ to irreducible factors $\phi_i$
(which is unique except for the ordering and constant factors).
Then there exist unique polynomials $\alpha_{ij}, \beta$
such
that
\begin{equation}
\label{eq:partial-fractions-uniq}
\frac{p}{q} = \beta + \sum_{i=1}^k \sum_{j=1}^{n_i} \frac{\alpha_{ij}}{\phi_i^j}\,, \quad 
\deg \alpha_{ij} < \deg \phi_i\,.
\end{equation}
\begin{proof}
Existence has already been proven as a special case of
partial fractions in Euclidean domains; we now prove uniqueness.  
Suppose equation \eqref{eq:partial-fractions-uniq} has been given.
First, multiply the equation by $q$:
\[
p = \beta \, q + \sum_{i,j} \alpha_{ij} \, \frac{q}{\phi_i^j}\,.
\]
The polynomial sum on the far right of this equation
has degree $< q$, because
each summand has degree 
$\deg (\alpha_{ij} \, q/\phi_i^j) < \deg \phi_i + \deg q - j \cdot \deg \phi_i \leq \deg q$.
So the polynomial sum is the remainder of a division of $p$ by $q$.
Then the quotient polynomial $\beta$ is uniquely determined.

Now suppose $s_i$ and $s'_i$ are polynomials of degree $< \phi_i^{n_i}$,
such that
\begin{equation}
\label{eq:pf-s1-unique}
\sum_{i=1}^k \frac{s_i}{\phi_i^{n_i}} = \sum_{i=1}^k \frac{s'_i}{\phi_i^{n_i}}\,.
\end{equation}
We claim that $s_i = s'_i$.  Let $q_1 = \phi_1^{n_1}$ and $q_2 = q/q_1$,
and write
\[
\frac{s_1}{q_1} + \frac{u}{q_2} = 
\sum_{i=1}^k \frac{s_i}{\phi_i^{n_i}} =
\sum_{i=1}^k \frac{s'_i}{\phi_i^{n_i}} = \frac{s'_1}{q_1} + \frac{u'}{q_2}\,,
\]
for some polynomials $u$ and $u'$.
Rearranging, we get:
\[
(s_1 - s'_1) \, q_2 = (u' - u) \, q_1\,.
\]
In particular, $q_1$ divides the left side.
Since $q_1 = \phi_1^{n_1}$ is relatively prime from $q_2$, it must divide 
the factor $(s_1 - s'_1)$.  But $\deg (s_1 - s'_1) < \deg q_1$,
hence $s_1 - s'_1$ must be the zero polynomial. That is, $s_1 = s'_1$.

So we can cancel the term $s_1/\phi_1^{n_1} = s'_1/\phi_1^{n_1}$ 
on both sides of 
equation \eqref{eq:pf-s1-unique}.
  And we could repeat the argument,
and show that $s_2$ and $s'_2$ are the same, 
$s_3$ and $s'_3$ are the same, and so on.
Therefore, we have shown that
the polynomials $s_i$ in the following expression
\[
\frac{p}{q} - \beta = \sum_{i=1}^k \frac{s_i}{\phi_i^{n_i}}\,, \quad \deg s_i < \deg \phi_i^{n_i}
\]
are unique.  In particular, $s_i$ is
the following numerator that results when the fractions $\alpha_{ij}/\phi_i^j$
are put under a common denominator $\phi^i_{n_i}$:
\[
s_i = \sum_{j=1}^{n_i} \alpha_{ij} \, \phi_i^{n_i-j}\,.
\]
But by the uniqueness part of Theorem \ref{thm:powers-uniq},
the decomposition
\[
\frac{s_i}{\phi_i^{n_i}} = \beta_i + \sum_{j=1}^{n_i} \frac{\alpha_{ij}}{\phi_i^j}\,, \quad \deg \alpha_{ij} < \deg \phi_i
\]
uniquely determines $\alpha_{ij}$.
(Note that the proof of Theorem \ref{thm:powers-uniq} shows that
$\beta_i = 0$, as $\deg s_i < \deg \phi_i^{n_i}$.)
\end{proof}
\end{thm}
%%%%%
%%%%%
\end{document}
