\documentclass[12pt]{article}
\usepackage{pmmeta}
\pmcanonicalname{Norm}
\pmcreated{2013-03-22 12:18:02}
\pmmodified{2013-03-22 12:18:02}
\pmowner{djao}{24}
\pmmodifier{djao}{24}
\pmtitle{norm}
\pmrecord{5}{31846}
\pmprivacy{1}
\pmauthor{djao}{24}
\pmtype{Definition}
\pmcomment{trigger rebuild}
\pmclassification{msc}{12F05}

% this is the default PlanetMath preamble.  as your knowledge
% of TeX increases, you will probably want to edit this, but
% it should be fine as is for beginners.

% almost certainly you want these
\usepackage{amssymb}
\usepackage{amsmath}
\usepackage{amsfonts}

% used for TeXing text within eps files
%\usepackage{psfrag}
% need this for including graphics (\includegraphics)
%\usepackage{graphicx}
% for neatly defining theorems and propositions
%\usepackage{amsthm}
% making logically defined graphics
%%%\usepackage{xypic} 

% there are many more packages, add them here as you need them

% define commands here
\begin{document}
Let $K/F$ be a Galois extension, and let $x \in K$. The {\em norm} $\operatorname{N}_F^K(x)$ of $x$ is defined to be the product of all the elements of the orbit of $x$ under the group action of the Galois group $\operatorname{Gal}(K/F)$ on $K$; taken with multiplicities if $K/F$ is a finite extension.

In the case where $K/F$ is a finite extension, the norm of $x$ can be defined to be the determinant of the linear transformation $[x]: K \to K$ given by $[x](k) := xk$, where $K$ is regarded as a vector space over $F$. This definition does not require that $K/F$ be Galois, or even that $K$ be a field---for instance, it remains valid when $K$ is a division ring (although $F$ does have to be a field, in order for determinant to be defined). Of course, for finite Galois extensions $K/F$, this definition agrees with the previous one, and moreover the formula
$$
\operatorname{N}_F^K(x) := \prod_{\sigma \in \operatorname{Gal}(K/F)} \sigma(x)
$$
holds.

The norm of $x$ is always an element of $F$, since any element of $\operatorname{Gal}(K/F)$ permutes the orbit of $x$ and thus fixes $\operatorname{N}_F^K(x)$.
%%%%%
%%%%%
\end{document}
