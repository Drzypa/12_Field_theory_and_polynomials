\documentclass[12pt]{article}
\usepackage{pmmeta}
\pmcanonicalname{HomogeneousPolynomial}
\pmcreated{2013-03-22 13:21:11}
\pmmodified{2013-03-22 13:21:11}
\pmowner{jgade}{861}
\pmmodifier{jgade}{861}
\pmtitle{homogeneous polynomial}
\pmrecord{11}{33872}
\pmprivacy{1}
\pmauthor{jgade}{861}
\pmtype{Definition}
\pmcomment{trigger rebuild}
\pmclassification{msc}{12-00}

\usepackage{amsmath,amssymb,amsthm}
\DeclareMathOperator{\ord}{ord}
\begin{document}
A polynomial $P(x_1, \cdots, x_n)$ of degree $k$ is called homogeneous if 
$P(cx_1, \cdots, cx_n) = c^{k}P(x_1, \cdots, x_n)$ for all constants $c$.

An equivalent definition is that all terms of the polynomial have the same degree (i.e. $k$).

Observe that a polynomial $P$ is homogeneous iff $\deg P = \ord P$.

As an important example of homogeneous polynomials one can mention the symmetric polynomials.
%%%%%
%%%%%
\end{document}
