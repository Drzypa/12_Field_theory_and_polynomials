\documentclass[12pt]{article}
\usepackage{pmmeta}
\pmcanonicalname{Characteristic}
\pmcreated{2013-03-22 12:05:01}
\pmmodified{2013-03-22 12:05:01}
\pmowner{Mathprof}{13753}
\pmmodifier{Mathprof}{13753}
\pmtitle{characteristic}
\pmrecord{16}{31160}
\pmprivacy{1}
\pmauthor{Mathprof}{13753}
\pmtype{Definition}
\pmcomment{trigger rebuild}
\pmclassification{msc}{12E99}

\usepackage{amssymb}
\usepackage{amsmath}
\usepackage{amsfonts}
\usepackage{graphicx}
%%%\usepackage{xypic}

\DeclareMathOperator{\cha}{Char}
\begin{document}
Let $(F,+,\cdot)$ be a field.  The \emph{characteristic} $\cha(F)$ of $F$ is commonly given by one of three equivalent definitions:

\begin{itemize}
\item
if there is some positive integer $n$ for which the result of adding any element to itself $n$ times yields $0$, then the characteristic of the field is the least such $n$.  Otherwise, $\cha(F)$ is defined to be $0$.
\item
if $f:\mathbb{Z}\to F$ is defined by $f(n) = n\cdot 1$ then $\cha(F)$ is the least strictly positive generator of $\operatorname{ker}(f)$ if $\operatorname{ker}(f)\neq \{ 0\}$; otherwise it is $0$.
\item
if $K$ is the prime subfield of $F$, then $\cha(F)$ is the size of $K$ if this is finite, and $0$ otherwise.
\end{itemize}

Note that the first definition also applies to arbitrary rings, and not just to fields.

The characteristic of a field (or more generally an integral domain) is always prime.  For if the characteristic of $F$ were composite, say $mn$ for $m,n>1$, then in particular $mn$ would equal zero.  Then either $m$ would be zero or $n$ would be zero, so the characteristic of $F$ would actually be smaller than $mn$, contradicting the minimality condition.
%%%%%
%%%%%
%%%%%
\end{document}
