\documentclass[12pt]{article}
\usepackage{pmmeta}
\pmcanonicalname{Monic1}
\pmcreated{2013-03-22 12:18:42}
\pmmodified{2013-03-22 12:18:42}
\pmowner{akrowne}{2}
\pmmodifier{akrowne}{2}
\pmtitle{monic}
\pmrecord{9}{31899}
\pmprivacy{1}
\pmauthor{akrowne}{2}
\pmtype{Definition}
\pmcomment{trigger rebuild}
\pmclassification{msc}{12E10}
\pmsynonym{monic polynomial}{Monic1}
\pmrelated{EisensteinCriterion}
\pmrelated{IrreduciblePolynomial2}
\pmrelated{AlgebraicInteger}

\usepackage{amssymb}
\usepackage{amsmath}
\usepackage{amsfonts}

%\usepackage{psfrag}
%\usepackage{graphicx}
%%%\usepackage{xypic}
\begin{document}
A \emph{monic polynomial} is a polynomial with a leading coefficient of 1.  That is, if $P_n(x)$ is a polynomial of degree $n$ in the variable $x$, then the coefficient of $x^n$ in $P_n(x)$ is 1.

For example, $x^5+3x^3-10x^2+1$ is a monic 5th-degree polynomial.  $3x^2+2z-5$ is a 2nd-degree polynomial which is not monic.
%%%%%
%%%%%
\end{document}
